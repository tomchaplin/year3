\documentclass[11pt]{article}

%{{{ Packages
\usepackage[margin=1in]{geometry}
\usepackage{enumitem}
\usepackage{amsfonts}
\usepackage{amssymb}
\usepackage{amsmath}
\usepackage{amsthm}
\usepackage{mathdots}
\usepackage[dvipsnames]{xcolor}
\usepackage[framemethod=TikZ]{mdframed}
\usepackage{microtype}
\usepackage{silence}
\WarningFilter{mdframed}{You got a bad break}
\setlength{\parindent}{0pt}
%}}}
%{{{ Custom commands
% Nice maths commands
\newcommand{\defeq}{:=}
\newcommand{\eqdef}{=:}
\newcommand{\abs}[1]{\left|#1\right|}
\newcommand{\norm}[1]{\left|\left|#1\right|\right|}
%\renewcommand{\dots}{...}
\newcommand{\msrspc}{\ensuremath{(X,\mathcal{B},\mu)}}
\DeclareMathOperator{\inv}{\text{inv}}
\DeclareMathOperator{\ord}{\text{ord}}
\DeclareMathOperator{\ind}{\text{ind}}
\DeclareMathOperator{\res}{\text{res}}
\DeclareMathOperator{\Int}{\text{Int}}
\DeclareMathOperator{\Ext}{\text{Ext}}
\newcommand{\relmiddle}[1]{\mathrel{}\middle#1\mathrel{}}
\newcommand{\rmv}{\relmiddle|}
\newcommand{\stcmp}{^{\mathsf{c}}}
\newcommand\restr[2]{{% we make the whole thing an ordinary symbol
  \left.\kern-\nulldelimiterspace % automatically resize the bar with \right
  #1 % the function
  \vphantom{\big|} % pretend it's a little taller at normal size
  \right|_{#2} % this is the delimiter
  }}
\newcommand{\contr}{\Rightarrow\Leftarrow}

% Spaces
\newcommand{\ktor}{\mathbb{T}^k}
\newcommand{\R}{\mathbb{R}}
\newcommand{\C}{\mathbb{C}}
\newcommand{\Z}{\mathbb{Z}}
\newcommand{\N}{\mathbb{N}}

% Derivatives
\newcommand*{\pd}[3][]{\ensuremath{\frac{\partial^{#1} {#2}}{\partial {#3}^{#1}}}}
\newcommand{\grad}{\bigtriangledown}

% Vectors
\newcommand{\mv}[1]{\textbf{#1}}

%}}}
%{{{ Enviornments
% Definitions environment
\newenvironment{defin}
	{\begin{mdframed}[backgroundcolor=white, roundcorner=5pt, linewidth=1pt]}
	{\end{mdframed}}
\newcommand{\mdf}[1]{{\color{red} #1}}

% Important notes environment
\newenvironment{note}
	{\begin{mdframed}[backgroundcolor=white, linecolor=red, roundcorner=5pt, linewidth=1pt]\bfseries{Note:}\normalfont}
	{\end{mdframed}}

% Examples enviornmnet
\definecolor{mylg}{rgb}{0.9,0.9,0.9}
\newenvironment{eg}
	{\begin{mdframed}[backgroundcolor=myld,roundcorner=5pt,linewidth=0pt]\bfseries{Example:}}
	{\end{mdframed}}

% Theorem enviornment
\newtheorem{theorem}{Theorem}[section]
\newtheorem{cor}[theorem]{Corollary}
\newtheorem{prop}[theorem]{Prop}
\newtheorem{lemma}[theorem]{Lemma}
%}}}
%{{{ Document metadata
\title{Complex Analysis Notes}
\author{Thomas Chaplin}
\date{}
%}}}

\begin{document}
\maketitle

\section{Complex Algebra}
\begin{defin}
	\begin{itemize}
		\item The \mdf{principal value of the argument} is the unique $\theta\in (-\pi, \pi]$.	
			This is a continuous function on $\C$ without any half-line (including $0$). 
		\item $\xi + i \eta$ is the \mdf{logarithm} of $re^{i\theta}$ id
			\[
				\xi = \log(r) \quad\quad \eta = \theta + 2\pi n \;\; n\in \Z
			\]
		\item The \mdf{principal value of the logarithm} corresponds to $n=0$.
		\item We say that $\xi+ i \eta$ is an element of $z_0^{z_1}$ if
			\[
				\xi + i \eta \in e^{z_1\log(z_0)}
			\]
		\item The \mdf{extended complex plane} is $\hat{\C}\defeq \C \cup \left\{\infty\right\}$.
		\item We can extended inversion to the $\hat{\C}$ by setting
		\[
			\frac{1}{0}\defeq\infty \quad\quad \frac{1}{\infty}\defeq 0
		\]
	\end{itemize}
\end{defin}

\subsection{Riemann Sphere}
To represent the complex plane, we use stereographic projection of $S^2\setminus\left\{\text{north pole}\right\}$ into $\C$ and then send the north pole to $\infty$.
\begin{align*}
	\pi:S^2\setminus\left\{(0,0,1)\right\}&\to \C \\
	(x_1, x_2, x_3) &\mapsto \frac{x_1}{1-x_3}+i\frac{x_2}{1-x_3}
\end{align*}

\begin{lemma}
A \mdf{circle on $S^2$} is the intersection of $S^2$ with some plane.
The image of every non-vanishing circle on $S^2$, under $\pi$ is a line or circle in $\C$.
\end{lemma}

In this proof we notice that circles through the north pole go to lines and circles not through the north pole go to circles.
So we can define $\pi(\text{north pole})\defeq\infty$ and see that $\pi(S^2)=\hat{\C}$.

We can use this to define a metric on $\hat{\C}$.
\[
	\forall z, w\in \C \quad\quad d(z, w)\defeq\norm{\pi^{-1}(z) - \pi^{-1}w}
\]
where $\norm{\cdot}$ is the Euclidean norm on $S^2$.

\begin{note}
We can compute everything in this definition in terms of complex algebra to find
\begin{align*}
	d(z, w) &= \frac{2\abs{z-w}}{\sqrt{1+\abs{z}^2}+\sqrt{1+\abs{w}^2}}\\	
	d(z, \infty) &= \frac{2}{\sqrt{1+\abs{z}^2}}\\	
\end{align*}
\end{note}
When doing complex algebra we stick to the following conventions
\begin{itemize}
	\item $\infty + z = z + \infty = \infty \quad\quad\forall z\in\C$
	\item $\infty \cdot z = z \cdot \infty = \infty \quad\quad\forall z\in\hat{\C}\setminus\left\{0\right\}$
	\item $\frac{z}{\infty}=0 \quad\quad\forall z\in\C$
	\item $\frac{z}{0}=\infty \quad\quad\forall z\in\hat{\C}\setminus\left\{0\right\}$
\end{itemize}
\section{Mobius Transformations}

Given $a, b, c, d\in \C$ such that $ad-bc\neq 0$ we can define a \mdf{Mobius transformation}
\[
	f(z)\defeq\frac{az+b}{cz+d}\quad\quad\forall z\in\C\setminus\left\{-\frac{d}{c}\right\}
\]
We can extend this to $\hat{\C}$ by defining $\hat{f}(-\frac{d}{c})=\infty$ and $\hat{f}(\infty)=\frac{a}{c}$.

Notice we can multiply $a, b, c, d$ by any non-zero complex number and recover the same function.
We say that $f$ is \mdf{normalised} if $ad-bc=1$.

It can be noticed that composing two Mobius transformations yields another Mobius transformation.
We can calculate the coefficients of the transformation by multiplying the corresponding matrices
\[
\begin{pmatrix}
	a & b \\
	c & d
\end{pmatrix}
\]
\begin{lemma}
Extended Mobius transforms are invertible and their inverse is another Mobius transform.
\end{lemma}

\subsection{Decomposing Mobius transformations}
Let $\inv$ be the inversion map $z\mapsto\frac{1}{z}$.
\begin{lemma}
Let $\mathcal{C}$ be a circle or a line then $\inv(\mathcal{C})$ is a circle or a line.
\end{lemma}
\begin{proof}
Worth going over.
\end{proof}

\begin{defin}
	The \mdf{elementary Mobius transformations} are
	
	\begin{tabular}{rll}
		\textit{(a)} & \mdf{Inversion}: & $\inv(z)=\frac{1}{z}$ \\
		\textit{(b)} & \mdf{Translation}: & $z\mapsto z+b$\\
		\textit{(c)} & \mdf{Rotation}: & $z\mapsto az$ for $a=e^{i\theta}$\\
		\textit{(d)} & \mdf{Expansion/Contraction}: & $z \mapsto rz$ for $z\in\R, z>0$
	\end{tabular}
\end{defin}

\begin{lemma}
Every Mobius transformation can be written as a composition of elementary Mobius transformations.
\end{lemma}

\begin{proof}
\textbf{Case 1:} $c\neq 0$

We can write 
\[
\frac{az+b}{cz+d}=\frac{a}{c}+\frac{b-\frac{ad}{c}}{cz+d}
\]
\textbf{Case 2:} $c=0$

$c=0$ and $ad-bc\neq 0$ $\implies$ $d\neq 0$ and hence we can write 
\[
\frac{az+b}{cz+d}=\frac{a}{d}z + \frac{b}{d}
\]

In both cases these transformations can be easily decomposed.
\end{proof}

\begin{theorem}
The image of a circle or line in $\hat{\C}$ under a Mobius transformation is another circle or line.
\end{theorem}

\begin{theorem}
Given 3 distinct points $z_1, z_2, z_3\in\hat{\C}$ and three other distinct points $w_1, w_2, w_3\in\hat{\C}$ there exists a unique Mobius transform $f$ with $f(z_i)=w_i$ for all $i$.
\end{theorem}

\begin{proof}
\textbf{Existence: }
We define two helper functions, assuming that none of the points are $\infty$
\[
	S(z)\defeq\frac{z-z_2}{z-z_3}\cdot\frac{z_1-z_3}{z_1-z_2}
\]
and if any $z_i$ is $\infty$ then we simply remove any term containing that $z_i$.
Notice
\[
	S(z_1) = 1 \quad S(z_2) = 0 \quad S(z_3) = \infty
\]
We define $T$ in the same way but replacing each $z_i$ with $w_i$.
Then we can notice that defining $f\defeq T^{-1}S$ yields a function with the desired properties.

\textbf{Uniqueness: }
It suffices to check the cases when $w_1=1$, $w_2=0$ and $w_3=\infty$ because we can always compose with $T$.
Then we can just pick two suitable Mobius transformations $f_1$ and $f_2$, then show that $g\defeq f_1\circ f_2^{-1}$ is the identity Mobius transformation.
\end{proof}

\begin{note}
Look up the cross ratio.
\begin{itemize}
	\item A non-identity Mobius transform has at most two fixed points because
		\[
			z=\frac{az+b}{cz+d}\iff 0 =cz^2 + (d-a)z -b
		\]
\end{itemize}
\end{note}

\section{Complex Differentiability}

\begin{defin}
	Given $D\subseteq \C$ open, a function $f:D\to \C$ is \mdf{complex differentiable at $z_0\in\C$} if
	\[
		f'(z_0)\defeq\lim_{z\to z_0}\frac{f(z)-f(z_0)}{z-z_0}\quad\text{exists}
	\]
\end{defin}

\begin{note}
This definition of $f'$ can be restated as 
\[
	\forall \epsilon >0 \;\; \exists\delta >0 \;\; s.t. \;\; \abs{z- z_0} < \delta \implies \abs{f(z)-f(z_0)- f'(z_0)(z- z_0)} \leq \epsilon \abs{z- z_0}
\]
\end{note}

\begin{prop}
$f:D\to\C$ complex differentiable at $z_0\in D$ $implies$ $f$ is continuous at $z_0$.
\end{prop}

The complex derivative also satisfies all of the usual algebra of derivative functions from real analysis, including the chain rule
\[
	(g\circ f)'(z_0) = g'(f(z_0))f'(z_0)
\]

\begin{theorem}[Cauchy-Riemann Equations]
The following are equivalent, given $f:D\to\C$ and $z_0=x_0 + i y_0\in D$
\begin{enumerate}[label=(\alph*)]
	\item $f$ is $\C$-differentiable at $z_0$.
	\item $f$ is $\R$-differentiable at $(x_0,y_0)$ and $df(z_0)$ is complex linear
	\item $f$ is $\R$-differentiable at $(x_0,y_0)$ and the CR equations hold:
		\[
			u_x=v_y \quad\quad u_y = -v_x
		\]
\end{enumerate}

\begin{proof}
\textit{(i)}$\iff$\textit{(ii)} is somewhat immediate.
Consider the alternative definition given in the notes.
We see that being $\C$-differentiable is equivalent to the existence of a complex number $\xi$ such that
\[
	\lim_{h\to 0}\frac{f(z_0+h)-f(z_0)-\xi\cdot h}{h}=0
\]
We can view thus view the derivative as a $\C$-linear function $h\mapsto \xi\cdot h$.
This is equivalent to the definition of $\R$-differentiability with the additional requirement that the map is $\C$-linear.
In practice this means that the Jacobian matrix is some real number multiplied by a rotation matrix.

This explains $\textit{(ii)}\iff\textit{(iii)}$ as well because the Jacobian must be given by
\[
	r
	\begin{pmatrix}
		\cos(\theta) & -\sin(\theta)	\\
		\sin(\theta) & \cos(\theta)
	\end{pmatrix}
\]
Alternatively, writing out the Jacobian we see that the derivative as a $\C$-linear map
\[
	M(h)=
\begin{pmatrix}
	u_x & u_y \\
	v_x & v_y \\
\end{pmatrix}
	\begin{pmatrix}
	h_x \\
	h_y
	\end{pmatrix}
\]
and then the condition $M(ih)=iM(h)\; \forall h\in \C$ is equivalent to the Cauchy-Riemann equations.
\end{proof}
\end{theorem}

\begin{theorem}[Power Series Expansion]
Given a sequence $(a_k)_{k\in\N_0}$ with $a_k\in\C$, consider the power series
\begin{equation}
	\sum_{k=0}^{\infty}a_k z^k	
	\label{eq:seriesexpansion1}
\end{equation}
\begin{enumerate}[label=(\alph*)]
	\item There exists a \mdf{radius of convergence} $r\in [0,\infty]$ such that for all $z$ with $\abs{z}<r$ the series (\ref{eq:seriesexpansion1}) converges, and for all $z$ with $\abs{z}=r' > r$ the series (\ref{eq:seriesexpansion1}) does not converge.
	\item The series
		\begin{equation}
			\sum_{k=1}^{\infty}k a_k z^{k-1}
			\label{eq:seriesexpansion2}
		\end{equation}
		head the same radius of convergence.
	\item $f(z)\defeq\sum_{k=0}^{\infty}a_k z^k$ is holomorphic on $\mathcal{B}_r(0)=\left\{\abs{z}< r\right\}$.
\end{enumerate}
\end{theorem}

\begin{proof}
\begin{enumerate}[label=(\alph*)]
	\item If we convergence for some $z_0$ then $(a_kz_0^k)\to 0$ is bounded by $C>0$ say and hence for all $z$ with $\abs{z}<\abs{z_0}$ we have
		\[
			\sum_{k=0}^{\infty}\abs{a_kz^k}=\sum_{k=0}^{\infty}\abs{a_k z_0^k}\frac{\abs{z}^k}{\abs{z_0}^k}\leq C \frac{1}{\abs{z_0}-\abs{z}}
		\]
		and hence we get convergence.
		
		The radius of convergence is therefore $\sup\left\{\eta \geq 0 \rmv \exists z \text{ with }\abs{z}=\eta \;\; s.t. \;\; (\ref{eq:seriesexpansion1}) \text{ converges}\right\}$
	\item We now consider (\ref{eq:seriesexpansion2}).
		Suppose $\abs{z}<\hat{r} < r$, then we have
		\[
			\sum_{k=1}^{\infty}\abs{ka_k z^{k-1}} \leq \frac{1}{\hat{r}}\sum_{k=1}^{\infty}\underbrace{k\left(\frac{\abs{z}^{k-1}}{\hat{r}^{k-1}}\right)}_{\to 0}\underbrace{\abs{a_k\hat{r}^k}}_{\text{convergent}}
		\]
		and hence the sum converges.
		Likewise the sum diverges wherever the other one does.
	\item Confusing proof.
\end{enumerate}
\end{proof}


\section{Cauchy's Collection of Complex Corollaries}

\subsection{Complex Integration}
\begin{defin}
	Let $f:D\to\C$ be continuous and $\gamma$ a smooth curve with $\Gamma=\gamma[a,b]\subseteq D$
	\[
		\int_\Gamma f(z)\;dz=\int_\gamma f(z)\;dz \defeq \int_a^bf(\gamma(t))\dot{\gamma}(t)\; dt
	\]
	The length of a curve is defined to be
	\[
		L(\gamma)\defeq\int_a^b\abs{\dot{\gamma}}\; dt
	\]

	Two curves $\gamma:[a,b]\to\C$ and $\lambda:[c, d]\to\C$ are \mdf{smoothly equivalent parametrisations} for $\Gamma$ if there is a smooth  function $\rho:[a,b]\to[c, d]$ such that
	\begin{enumerate}[label=(\roman*)]
		\item $\dot{\rho}(t)\neq0 \quad \forall t$.
		\item $\rho^{-1}\in\mathcal{C}^1$ and is never zero.
		\item $\gamma = \lambda \circ \rho$.
		\item $\rho(a)=c$ and $\rho(b)=d$.
	\end{enumerate}
\end{defin}
\begin{lemma}
The complex line integral is invariant under change of parametrisation.
\end{lemma}
\begin{lemma}
If $\gamma$ and $\lambda$ are smoothly equivalent then $L(\gamma)=L(\lambda)$.
\end{lemma}

\begin{lemma}
$f:D\to \C$ holomorphic and $\gamma\in\mathcal{C}^1([a, b])$ such that $\Gamma\subseteq D$ then
\begin{align*}
	\int_\gamma f'(z)\;dz &=\int_a^b f'(\gamma(t))\dot{\gamma}(t)\;dt	\\
						  &=\int_a^b \frac{d}{dt}f(\gamma(t))\;dt \\
						  &=f(\gamma(b)) = f(\gamma(a))
\end{align*}
\end{lemma}

\subsection{My First Sony Cauchy's Theorem}
\begin{theorem}[Goursat's Theorem]
Take $D\subseteq \C$ open and $f:D\to\C$ holomorphic.
Take a rectangle $Q\subseteq D$ such that $Q\cup \partial Q = \overline{Q}\subseteq D$.
Take a $\mathcal{C}^1$ parametrisation $\gamma:[a, b]\to\C$ such that $\gamma[a, b]=\partial Q$ and $\gamma$ circles around $Q$ exactly once in the positive direction.
Then
\[
	\int_\gamma f(z)\; dz =0
\]
\end{theorem}

\begin{proof}
We split the proof into a number of steps:
\begin{enumerate}
	\item $f\equiv 1$.

	This proof follows easily from the FTC.
	\item $f(z)=z$.

	This proof also follows easily from the FTC because
	\[
		\int\gamma(t)\dot{\gamma}(t)\;dt=\frac{1}{2}\int_a^b \frac{d}{dt}\left(\gamma(t)\right)^2\; dt = \frac{1}{2}\left[\gamma(b)^2 - \gamma(a)^2\right]
	\]
	\item $f$ holomorphic in $D$.

	Divide into rectangles, this is a very long proof in Lecture 9.
\end{enumerate}
\end{proof}

\begin{cor}[Cauchy's Theorem for images of rectangles]
	Given $D\subseteq\C$ open and $f:D\to\C$ holomorphic such that $\overline{Q}\subseteq D$.
	Suppose $\phi:\overline{Q}\to D$ is $\mathcal{C}^1$.
Let $\gamma$ be a $\mathcal{C}^1$ parametrisation of $\partial Q$ then
\[
	\int_{\phi\circ\gamma} f(z)\;dz =0
\]
\end{cor}

\begin{defin}
We say $D\subseteq \C$ is
\begin{itemize}
	\item a \mdf{region} if it is non-empty and connected.
	\item \mdf{polygonally connected} if between every two points are joined by a path consisting of a finite collection of straight lines all contained within $D$.
\end{itemize}
A \mdf{contour} is a simple closed curve.
\end{defin}

\begin{theorem}
Given a non-empty open set $D\subseteq C$
\[
	D\text{ is a region}\quad\iff\quad D\text{ is polygonally connected}
\]
\end{theorem}

\begin{theorem}[Jordan Curve Theorem]
Let $\gamma$ be a contour and $\Gamma=\gamma[a, b]$ then $\stcmp{\gamma}$ consists of
\[
	I(\gamma)\cup O(\gamma)
\]
where $I(\gamma)$ is bounded and $O(\gamma)$ is unbounded and the two regions are disjoint.
\end{theorem}

\begin{note}
Jordan Curve Theorem $\implies$ Cauchy's theorem for contours
\end{note}

\subsection{Cauchy's Integral Formula}
\begin{theorem}[Cauchy's Integral Formula]
Given $D\subseteq \C$ open and $f:D\to\C$ holomorphic, suppose that $\overline{\mathcal{B}_r(a)}\subseteq D$ for some $a\in D$ and $r>0$.
Then for all $z_0\in\mathcal{B}_r(a)$
\[
	f(z_0)=\frac{1}{2\pi i}\int_{\partial\mathcal{B}_r(a)}\frac{f(\xi)}{\xi-z_0}d\xi
\]
\end{theorem}

\begin{proof}
Not too difficult, worth going over (Lecture 11)
\end{proof}

\subsection{Applications}

\begin{theorem}[Taylor's Theorem]
Given $D\subseteq C$ open and polygonally connected and $f:D\to C$ holomorphic.
Assume $\exists R>0$ and $z_0\in D$ such that $\overline{\mathcal{B}_R(z_0)}\subseteq D$ then for all $z\in\mathcal{B}_R(z_0)$ we have
\[
	f(z)=\sum_{k=0}^{\infty}a_k(z-z_0)^k\quad\text{with}\quad a_k=\frac{1}{2\pi i}\int_{\partial \mathcal{B}_r(z_0)} \frac{f(\xi)}{(\xi - z_0)^{k+1}}d\xi
\]
\end{theorem}

\begin{cor}
	Every homolomorphic function on $D$ is in fact $C^\infty(D)$.
\end{cor}

\begin{cor}
$D\subseteq \C$ is open and polygonally collected and $f:D\to\C$ then the following are equivalent:
\begin{enumerate}[label=(\roman*)]
	\item $f$ is holomorphic in $D$.
	\item $f$ is real differentiable on $D$ and the CR equations hold.
	\item $f$ can be expressed in a power series.
\end{enumerate}
\end{cor}

\begin{cor}
	Suppose $f(z)=\sum_{k\in\N}a_k z^k$ is holomorphic on $\mathcal{B}_R(0)$ for some $R>0$ and suppose $f$ is bounded in that ball, say by M.
	Then for all $k\in\N$
	\[
		\abs{a_k}\leq \frac{M}{R^k}
	\]
	where $R$ is the radius of convergence.
\end{cor}

\begin{theorem}[Liouville's Theorem]
Any entire, bounded function is constant.
\end{theorem}

\begin{proof}
Pick $z_0\in\C$ and $M>0$ such that $\abs{f(z)}\leq M \; \forall z\in\C$.
Define
\[
m(f,R,z_0)\defeq\max_{z\in\partial\mathcal{B}_R(z_0)}\abs{f(z)}
\]
Then by Taylor's theorem we see that
\[
	\abs{f^{(n)}(z_0)}\leq\frac{n!}{R^n}m(f, R, z_0) \leq \frac{n!}{R^n}M
\]
and in particular $\abs{f'(z_0)}\leq\frac{M}{R}\to 0$ as $R\to\infty$.
Hence $f'(z_0)=0$.
\end{proof}

\begin{cor}[Fundamental Theorem of Algebra]
	Every non-constant polynomial has at least one zero in $\C$.
\end{cor}

\begin{proof}
Take some polynomial $P(z)\defeq a_nz^n + \dots + a_1 z + a_0$ such that $a_n\neq 0$.
Then for any $\epsilon >0$ there is a radius $R$ such that $\forall\abs{z}> R$ we have
\[
	(1-\epsilon)\abs{a_n}\abs{z}^n\leq\abs{P(z)}\leq(1+\epsilon)\abs{a_n}\abs{z_n}^n
\]
Suppose $P(z)$ has no zeros in $\C$ then $\frac{1}{P(z)}$ is complex differentiable in $\C$ and there is an $R>)$ such that for all $\abs{z}>R$
\[
	\frac{1}{2}\abs{a_n}{z_n}\leq\abs{P(z)}\implies\abs{\frac{1}{P(z)}}\leq\frac{2}{\abs{a_n}\abs{z^n}}\leq\frac{2}{\abs{a_n}R^n}
\]
and hence $\frac{1}{P}$ is bounded on $\left\{\abs{z}>R\right\}$ and it is obviously bounded inside by compactness.
Hence by Liouvilles's Theorem $P$ is constant.
This is a contradiction.
\end{proof}

\begin{theorem}[Morea's Theorem]
Given a region $D\subseteq\C$ and a $f:D\to\C$ continuous, suppose given any triangle $T$ with $T\cup\partial T\subseteq D$ we have $\int_{\partial T}f(z)\;dz=0$.
Then $f$ is holomorphic in $D$.
\end{theorem}

\begin{theorem}[Schwarz Reflection Principle]
Suppose $D$ is open in $\overline{H^+}\defeq\left\{z\in\C \rmv \mathcal{I}(z)\geq 0\right\}$ and $f:D\to\C$ is continuous on $D$ and holomorphic on $D^\circ$.
Then
\[
	\widetilde{f}(z)\defeq
	\begin{cases}
		f(z) & z\in D \\
		\overline{f(\overline{z})} & z\in\widetilde{D}
	\end{cases}
\]
where $\widetilde{D}$ is the complex conjugate of $D$, is well-defined and holomorphic on $D\cup\widetilde{D}$.
\end{theorem}

\begin{proof}
By composition of reflections we can easily sow that $\widetilde{f}$ is holomorphic on $\widetilde{D}^\circ$ so that only the lines remain.
We show that the integral of $f$ over any triangle with one edge on the line is $0$ by a continuity argument, approaching from both sides.
\end{proof}

\section{Zeros of Holomorphic Functions}
\begin{defin}
	Given $D\subseteq\C$ open and connected and $f:D\to\C$ holomorphic, the \mdf{order} of any zero $z_0\in D$ is
	\[
		\mdf{\ord(f,z_0)}\defeq\inf\left\{k\in\N \rmv f^{(k)}(z_0)\neq 0\right\}\in\N\cup\left\{\infty\right\}
	\]
	We say that $f:D\to\C$ is a \mdf{conformal mapping} if $f$ is holomorphic in $D$ and it's derivative is non-vanishing on $D$.

	We say that $f$ is \mdf{biholomorphic} if $f$ is a conformal mapping such that $f^{-1}$ exists and is also conformal.
\end{defin}

\begin{prop}
Given $D\subseteq \C$ open and connected and $f:D\to\C$ holomorphic, suppose we have a zero $z_0\in D$ of order $k\in\N$.
Then there is a neighbourhood $U_0$ of $z_0$ and a holomorphic function $h:U_0\to V_0$ such that $h(z_0)=0$, $\ord(f,z_0)=1$ and
\[
	f(z)=(h(z))^k\quad\forall z\in U_0
\]
\end{prop}

\begin{proof}
WLOG we may assume that $z_0=0$, then we apply Taylor's theorem to write $f$ as
\[
	f(z)=\sum_{n=k}^{\infty}c_nz^n
\]
because $\ord(f,z_0)=k$ and hence the first $k$ terms vanish.
For simplicity we can also assume that $c_k=1$.
Hence we can write
\[
	f(z)=z^k\left(1+\underbrace{\sum_{n=k+1}^{\infty}c_nz^{n-k}}_{\eqdef g(z)}\right)=\left(\underbrace{z\sqrt[k]{1+ g(z)}}_{\text{\eqdef h(z)}}\right)^k
\]
Note that $g$ is holomorphic and $g(0)=0$, and $h(0)=0$.
Moreover, $h'(0)=\sqrt[k]{1+g(0)}+0\left(\sqrt[k]{1+g(z)}\right)' = 1 \neq 0$.
Hence $\ord(h, 0)=1$.
Read up on making it holomorpic (Lecture 14).
\end{proof}

\begin{note}
	This implies that all zeros of finite order are isolated.
\end{note}

\begin{theorem}
If $\ord(f, z_0)=k\in\N$ for $f:D\to\C$ holomophic then $\forall\epsilon >0$ the exists a $U_\epsilon\subseteq D$ with $z_0\in U_\epsilon$ such that $f(U_\epsilon)=\mathcal{B}_\epsilon(0)$ and $\restr{f}{U_\epsilon}$ takes every $w$ with $0<\abs{w}<\epsilon$ exactly $k$ times and $0$ for $z_0$.
\end{theorem}

\begin{proof}
Without loss of generality we may assume that $z_0=0$.

If $f(z)=z^k$ then any $w=re^{i\theta}$ has exactly $k$ roots.

In the general case we can write $f(z)=(h(z))^k$ for $h:U\to V$ holomorphic such that $h(0)=0$ and $h'(0)\neq 0$.
Moreover, $h$ is locally biholomorphic around a neighbourhood of $0$.
Choose $\epsilon>0$ sufficiently small that
\[
	A\defeq\left\{\xi\in\C \rmv \abs{\xi}\leq\sqrt[k]{\epsilon}\right\}\subseteq V
\]
Then define $U_\epsilon\defeq h^{-1}(A)$.
This set has the desired properties because the original roots of $z\mapsto z^k$ lie in $A$.
\end{proof}

\begin{note}
Every bijective holomorphic function is biholomorphic.
\end{note}

\begin{theorem}[Identity Theorem]
Given $D\subseteq\C$ open and connected with $f_1, f_2:D\to\C$ holomorphic, assume that $\left\{f_1=f_2\right\}$ has at least one accumulation point in $D$.
Then $f_1=f_2$ on $D$.
\end{theorem}

\begin{proof}
Define $g\defeq f_1 - f_2$ and let $z_0$ be one of the accumulation points.
Then $z_0$ is a zero of infinite order for $g$.
Apparently this is a proof?
\end{proof}

\begin{theorem}[Open Mapping Theorem]
Given $D\subseteq\C$ open and connected and $f:D\to\C$ holomorphic and non-constant, $f(D)$ is open and connected.
\end{theorem}

\begin{theorem}[Maximum Modulus Principle]
Given $D\subseteq\C$ open and connected and $f:D\to\C$ holomorphic and non-constant, $\abs{f}$ does not have any maxima.
\end{theorem}

\begin{lemma}[Schwarz Lemma]
Suppose $f:\Delta\to\Delta$ is holomorphic such that $f(0)=0$ then
\begin{enumerate}[label=(\roman*)]
	\item $\abs{f(z)}\leq\abs{z}$ for all $z\in\Delta$.
	\item $\abs{f'(0)}\leq 1$.
	\item If for some $z\in\Delta\setminus\left\{0\right\}$ we have $\abs{f(z)}=\abs{z}$ or $\abs{f'(z)}=1$ then $\exists\theta\in\R$ such that $f(\widetilde{z})=e^{i\theta}\widetilde{z}$ for all $\widetilde{z}\in\Delta$.
\end{enumerate}
\end{lemma}

\section{Singularities}

\begin{defin}
	Given $D\subseteq \C$ open and connected and $f\in\mathcal{H}(D)$,
	\begin{itemize}
		\item $f$ has an \mdf{isolated singularity} at $z_0\not\in D$ if there is an $\epsilon >0$ such that $f$ is defined on $\mathcal{B}_\epsilon(z_0)\setminus\left\{z_0\right\}$.
		\item $z_0\in D$ is a \mdf{regular point} if $f$ is complex differentiable at $z_0\in D$.
	\end{itemize}
	Given an isolated singularity $z_0$ then it has \mdf{order}
	\[
		\mdf{\ord(f,z_0)}\defeq-\inf\left\{n\in\Z \rmv \lim_{z\to z_0} (z-z_0)^n f(z) \text{ exists and }<\infty\right\}
	\]
	then we say $z_0$ is a
	\begin{itemize}
		\item \mdf{removable singularity} if $\ord(f,z_0)\geq 0$.
		\item \mdf{pole of order $n\in\N$} if $\ord(f, z_0)=-n\in(-\infty, -1]$.
		\item \mdf{essential singularity} if $\ord(f, z_0)=-\infty$.
	\end{itemize}
	Let $S\subseteq D$ be a discrete set, then a holomorphic function $f:D\setminus S\to\C$ is called \mdf{meromorphic on $D$} if none of the isolated singularities in $S$ are essential.
\end{defin}

\begin{prop}
Let $\mathcal{Z}_f$ and $\mathcal{P}_f$ be the set of zeros and poles respectively of $f:D\to\C$ meromorphic, $f\neq 0$.
Then neither set has an accumulation point in $D$.
\end{prop}

\begin{proof}
Certainly any pole of $f$ is an isolated singularity and hence cannot be an accumulation point of $\mathcal{P}_f$.
Any other $z\in D$ where $f$ is holomorphic cannot be an accumulation point of poles either.

Suppose now that $\mathcal{Z}_f$ has an accumulation point at $z_0\in D$ then $z_0$ cannot be a pole otherwise we'd be able to write
\[
	f(z)=\frac{g(z)}{(z-z_0)^m}\quad\text{with }m\in\N, g(z_0)\neq 0
\]
and hence for some $\epsilon>0$ we have that $f(z)\neq 0$ for all $0\leq\abs{z-z_0}<\epsilon$ which means that $z_0$ is not an accumulation point.

So any accumulation point must be a complex differentiable point.
We are left to show that $D\setminus\mathcal{P}_f$ see open and connected because then the identity theorem tells us that $f\equiv 0$ because $\left\{ f=0\right\}$ has an accumulation point in $D \setminus \mathcal{P}_f$.
\end{proof}

\begin{lemma}
Suppose $D\subseteq\C$ is open and connected.
Suppose $M\subseteq D$ has no accumulation point in D.
Then $D\setminus M$ is open and connected.
\end{lemma}

\begin{proof}
Openness is immediate, for connectedness just draw a picture.
\end{proof}

\subsection{Laurent Series}
\begin{defin}
	A \mdf{Laurent series} is a series of the form
	\[
		\sum_{k\in\Z}a_k(z-z_0)^k
	\]
	such that the positive terms converge inside some ball around $z_0$ and the negative terms converge outside sum larger ball around $z_0$.
	Hence the Laurent series converges in an annulus around the point $z_0$.
\end{defin}

\begin{theorem}[Cauchy's Theorem for annuli]
Given $0\leq R_1 < R_2 < \infty$ and $D\subseteq\C$ open and connected such that $\overline{A}\defeq\overline{A(R_1, R_2, z_0)}\subseteq D$, for any $z\in A$ we have
\[
	f(z) = \frac{1}{2\pi i}\int_{\partial \mathcal{B}_{R_2}(z_0)}\frac{f(\xi)}{\xi-z_0}d\xi - \frac{1}{2\pi i}\int_{\partial \mathcal{B}_{R_1}(z_0)}\frac{f(\xi)}{\xi-z_0}d\xi
\]
\end{theorem}

\begin{proof}
Do normal Cauchy on a small ball contained in the annulus then do some appropriate contour integration.
\end{proof}

\begin{theorem}[Laurent's Theorem]
Given $f$ holomorphic on a neighbourhood of an annulus $A=A(R_1, R_2, z_0)$ and any $z\in A$,
\[
	f(z)=\sum_{k\in\Z}a_k(z-z_0)^k
\]
where for all $\rho\in [R_1, R_2]$ we can write
\[
	a_k=\frac{1}{2\pi i}\int_{\partial \mathcal{B}_{\rho}(z_0)}\frac{f(\xi)}{\left(\xi-z_0\right)^{k+1}}d\xi
\]
\end{theorem}

\begin{cor}
	Under the same assumption, if $f$ is bounded on $\left\{\abs{z-z_0}=\rho\right\}$ for sum $\rho\in[R_1, R_2]$ then
	\[
		\abs{a_k}\leq\frac{M}{\rho^k}\quad\text{for all }k\in\Z
	\]
\end{cor}

\subsection{Classification of Singularities}
\begin{theorem}[Riemann's removable singularity theorem]
Gain an isolated singularity $z_0$ for a function $f\in\mathcal{H}(D\setminus\left\{z_0\right\})$, assume that $\abs{f}$ is bounded in a neighbourhood of $z_0$.
Then there is a holomorphic function $\widetilde{f}\in\mathcal{H}(D)$ which extend to $f$.
Moreover, $z_0$ was a removable singularity.
\end{theorem}

\begin{proof}
In the neighbourhood we can expand in a Laurent series
\[
	f(z)=\sum_{k\in\Z}a_k(z-z_0)^k
\]
and then for all sufficiently small $\rho>0$ we have $\abs{a_k}\leq\frac{M}{\rho^k}$ for all $k\in\Z$.
Taking $\rho\to 0$ we see that $a_k=0$ for all $k<0$.
So we can extend $f$ by taking $\widetilde{f}(z_0)=a_0$.
\end{proof}

\begin{cor}
Given $f:D\to\C$ which is holomorphic except for an isolated singularity at $z_0$, the following are equivalent:
\begin{enumerate}[label=(\roman*)]
	\item $f$ has a pole at $z_0$.
	\item At least coefficient of negative order in the Laurent series around $z_0$ in non-zero, but at most finitely many.
	\item $\lim_{z\to z_0}\abs{f(z)}=+\infty$.
\end{enumerate}
\end{cor}

\begin{theorem}[Casorati-Weirstrass]
Given $f\in\mathcal{H}(D\setminus\left\{z_0\right\})$ with an isolated, essential singularity at $z_0$, for all $\epsilon>0$ the set $f(\mathcal{B}_\epsilon(z_0)\setminus\left\{z_0\right\})$ is dense in $\C$.
\end{theorem}

\begin{proof}
Suppose it's not dense.
Then $\exists\delta >0$ and $w\in\C$ such that $\abs{f(z)-w}>\delta$ for all $z\in\mathcal{B}_\epsilon(z_0)\setminus\left\{z_0\right\}$.
So we can define a function
\[
	g(z)\defeq\frac{1}{f(z)-w}
\]
and it may be extended to a function which is holomorphic on $\mathcal{B}_\epsilon(z_0)$ because the bottom is always bigger than $\delta$ hence $g$ is bounded.
So we can write $f(z)=\frac{1}{g(z)}+w$ which clearly doesn't have an essential singularity $\contr$.
\end{proof}

\begin{note}
	Suppose $z_0$ is an essential singularity for $f$, then certainly $\lim_{z\to z_0}f(z)$ cannot have any finite value $c\in \C$.
	Suppose $\lim_{z\to z_0} f(z) = \infty$ then there would be some $\epsilon$ ball around $z_0$ such that $\abs{z-z_0}< \epsilon \implies \abs{f(z)} > 1$.
	Hence the image of this ball could not be dense.
	This shows that essential singularities exhibit no limiting behaviour.
\end{note}

\section{Residual Theory}

\subsection{Winding Numbers}
\begin{defin}
	Given $z_0, z_1\in\C\setminus\left\{0\right\}$ such that $\frac{z_0}{\abs{z_0}}\neq\frac{-z_1}{\abs{z_1}}$ then there is a unique $\theta\in (-\pi, pi)$ such that
	\[
		\frac{z_0}{\abs{z_0}}e^{i\theta}=\frac{z_1}{\abs{z_1}}
	\]
	then we define $\mdf{\measuredangle(z_0,z_1)}\defeq\theta$.
	
	We say $\gamma[t_0, t_1]$ is a \mdf{half-plane curve} if the image of $\gamma$ is contained entirely in one half plane.
	Then we define $\mdf{\measuredangle\gamma}\defeq\measuredangle(\gamma(t_0),\gamma(t_1))$.

	Given any other piecewise $\mathcal{C}^1$ curve we can break it up into a sequence of half-plane curves and then sum the angles to define $\measuredangle\gamma$.
\end{defin}

\begin{lemma}
Given a closed $\mathcal{C}^1$ curve $\gamma;[t_0, t_1]\to\C\setminus\left\{0\right\}$, there exists a unique integer \mdf{$\inf(\gamma,0)$} called the \mdf{index} or \mdf{winding number} of $\gamma$ around $0$ such that $\measuredangle\gamma=2\pi \ind(\gamma, 0)$.
\end{lemma}

\begin{prop}
Given a closed $\mathcal{C}^1$ curve $\gamma$ with $a\not\in\gamma[t_0, t_1]$ we have
\[
	\ind(\gamma, a)=\frac{1}{2\pi i}\int_\gamma\frac{dz}{z-a}
\]
\end{prop}

\begin{proof}
Assume $a=0$, $t_0=0$ and $t_1=1$ then we can split up into the half-plane decomposition: 
\[
0=\tau_0 \leq \dots \leq \tau_n =1
\]
Let $\alpha_i$ be the straight line connected $\gamma(\tau_i)$ with $\frac{\gamma(\tau_i)}{\abs{\gamma(\tau_i)}}$ and let $\beta_i$ be the counter-clockwise path along the unit circle connecting $\frac{\gamma(\tau_{i-1}}{\abs{\gamma(\tau_{i-1}}}$ with $\frac{\gamma(\tau_i)}{\abs{\gamma(\tau_i)}}$.
Then we can see
\[
	\int_{\gamma[\tau_{i-1}, \tau_i]}\frac{dz}{z}=\int_{\alpha_{i-1}}\frac{dz}{z}+\int_{\beta_i}\frac{dz}{z}-\int_{\alpha_i}\frac{dz}{z}\quad\forall i
\]
Then summing over $i$ and noting $\alpha_0=\alpha_n$ obtains the result.
\end{proof}

\begin{defin}
	A \mdf{cycle} is a formal linear combination of closed curves $\gamma=\sum_{i=1}^{n}\alpha_i\gamma_i$ with $\alpha_i\in\Z$ then we define
	\[
		\ind(\gamma,a)=\sum_{i=1}^{n}\alpha_i\ind(\gamma_i, a)
	\]
	Then $\gamma$ is called \mdf{homologous to $0$ in $D$} if for every $a\in\C\setminus D$ we have
	\[
		\ind(\gamma, a)=0
	\]
\end{defin}

\begin{theorem}[Cauchy's Theorem (homotopy version)]
Let $D\subseteq C$ be open and connected and $\gamma$ a $\mathcal{C}^1$ cycle that is homologous to $0$ in $D$.
Then $\forall f\in\mathcal{H}(D)$
\[
	\int_\gamma f(z)\;dz=0
\]
\end{theorem}

\subsection{Residual Theorem}

\begin{defin}
	Given $f$ holomorphic with an isolated singularity at $z_0$, the \mdf{residue of $f$ at $z_0$} is defined to be
	\[
		\mdf{\res(f,z_0)}\defeq\frac{1}{2\pi i}\int_{\partial \mathcal{B}_\epsilon(z_0)}f(\xi)\;d\xi
	\]
\end{defin}

There are a number of convenient ways of calculating the residue.
\begin{enumerate}
	\item For a Laurent series
	\[
		f(z)=\sum_{k\in\Z}a_k(z-z_0)^k
	\]
	we have $\res(f, z_0)=a_{-1}$.
\item Suppose that $f$ has a pole of order $n$ at $z_0$.
	Define $g(z)\defeq(z-z_0)^nf(z)$ then $g$ is holomorphic at $z_0$ and in fact
	\[
		\res(f,z_0)=a_{-1}=\frac{g^{(n-1)}(z_0)}{(n-1)!}=\lim_{z\to z_0}\left[\frac{1}{(n-1)!}\frac{d^{n-1}}{dz^{n-1}}\left((z-z_0)^nf(z)\right)\right]
	\]
\item If $f(z)=\frac{h(z)}{k(z)}$ and $k(z)$ has a simple zero at $z_0$ then
	\[
		\res(f,z_0)=\lim_{z\to z_0} \frac{h(z)}{\left(\frac{k(z)-k(z_0)}{z-z_0}\right)}=\frac{h(z_0)}{k'(z_0)}
	\]
\end{enumerate}

\begin{defin}
	Given $\gamma;[t_0, t_1]\to\C$ a closed curve with image $\Gamma$
	\begin{align*}
		\mdf{\Int(\gamma)}&\defeq\left\{z\in\C\setminus\Gamma \rmv \ind(\gamma, z)\neq 0\right\} \\
		\mdf{\Ext(\gamma)}&\defeq\left\{z\in\C\setminus\Gamma \rmv \ind(\gamma, z) = 0\right\}
	\end{align*}
\end{defin}

\begin{lemma}
\begin{enumerate}[label=(\alph*)]
	\item $\left[a\mapsto\ind(\gamma, a)\right]$ is locally constant in $\C\setminus\Gamma$.
	\item $\Int(\gamma)$ is bounded.
	\item $\Ext(\gamma)$ is non-empty and unbounded.
\end{enumerate}
\end{lemma}

\begin{theorem}[The Residue Theorem]
Given $D\subseteq\C$ open and connected and $f\in\mathcal{H}(D\setminus S)$ for some discrete set $S$ of isolated singularities, suppose $\gamma$ is a closed $\mathcal{C}^1$ curve homologous to $0$ in $D$ such that $\Gamma\cap S = \emptyset$ and $\Gamma\subseteq D$.
Then $\gamma$ winds around at most a finite number of singularities in $S$ and
\[
	\int_\gamma f(z)\;dz = 2\pi i \sum_{a\in S}\ind(\gamma, a)\res(f, a)
\]
\end{theorem}

\begin{proof}
First things first, $A\defeq\left\{a \in S \rmv \ind(\gamma, a)\neq 0 \right\}$ is bounded.
Assume that $\gamma$ winds around infinitely many points in $S$ then $A$ is infinite.
Hence there is a sequence of points $(a_n)$ in $A$ such that $a_n\to a$.

\textbf{Case 1:} $a\in\Gamma$

\textbf{Case 2:} $\ind(\gamma, a)\neq 0$

Both lead to a contradiction apparently.

Now suppose $a_1, \dots, a_N$ are the points around with $\gamma$ winds and define $\alpha_i\defeq\ind(\gamma,a_i)$.

Choose $\epsilon >0$ small such that $\overline{\mathcal{B}_\epsilon(a_i)}\cap\Gamma=\emptyset$ for all $i$.
Then define $\gamma_i(t)\defeq a_i+\epsilon e^{i 2 \pi t}$ for $t\in [0, 1]$.
Let $\beta_i$ be some little paths joining $\gamma$ to $\gamma_i$ and then concatenate all the $\gamma_i$s and $\beta_i$s with $\gamma$ in some appropriate way to form $\widetilde{\gamma}$
Then 
\[
	\int_{\widetilde{\gamma}}=0\implies\int_\gamma f(z)dz = 2\pi i \sum_{a\in S}\ind(\gamma, a)\res(f, a)
\]
\end{proof}

\begin{theorem}[Argument Principle]
Given $D\subseteq\C$ open and connected and $f$ meromorphic on $D$, let $A\subseteq D$ be open with boundary $\partial A$ being a closed $\mathcal{C}^1$ curve $\gamma$.
Assuming $\partial A \subseteq D$ and $\Gamma\cap\mathcal{P}(f)=\Gamma\cap\mathcal{Z}(f)=\emptyset$, then
\[
	\frac{1}{2\pi i}\int_\gamma \frac{f'(z)}{f(z)}dz = \mathcal{Z}_A(f)-\mathcal{P}_A(f)
\]
where
\[
	\mathcal{Z}_A(f)\defeq\sum_{z\in f^{-1}(0)\cap A}\ord(f, z) \quad \text{and} \quad \mathcal{P}_A(f)\defeq\sum_{z\in\mathcal{P}(f)\cap A}\abs{\ord(f, z)}
\]
\end{theorem}

\begin{proof}
The proof uses the residual theorem to calculate the integral on the left hand side.
The isolated singularities of $\frac{f'}{f}$ are the poles and zeros of $f$.
We need to understand the residue of $\frac{f'}{f}$ (the \mdf{logarithmic derivative}) at these points.

To this end, let $z_0$ be a zero or a pole of $f$.
Then for some $n\in \Z$ and function $g$ which is holomorphic in neighbourhood of $z_0$ with $g(z_0)\neq 0$ we can write
\[
	f(z)=(z-z_0)^ng(z)
\]
Then we can calculate the logarithmic derivative as follows
\[
	\frac{f'(z)}{f(z)}=\frac{1}{f(z)}\left( n(z-z_0)^{n-1}g(z) + (z-z_0)^n g'(z) \right) = \frac{n}{z-z_0} + \frac{g'(z)}{g(z)}
\]
Since $g(z_0)\neq 0$ we see that $\frac{f'}{f}$ has a simple pole at $z_0$ and hence $\res\left(\frac{f'}{f}, z_0\right)=n$.

So when we integrate $\frac{f'}{f}$ we sum $\ord(f, z_0)$ for all zeros and poles of $f$ with a factor of $2\pi i$ out front.
This counts zeros positively and poles negatively, each with multiplicity respecting the order.
\end{proof}

\section{Rouches Theorem}

\begin{theorem}[Rouches Theorem]
Given $D\subseteq\C$ open and connected and $\gamma$ a closed $\mathcal{C}^1$ curve with $\Gamma=im(\gamma)\subseteq D$.
Suppose $f,g\in\mathcal{H}(D)$ satisfy
\begin{equation}
	\abs{f(\xi)-g(\xi)} < \abs{g(\xi)}\quad\forall\xi\in\Gamma	
	\label{eq:roche}
\end{equation}
Then $f$ and $g$ have the same number of zeros in $\Int(\gamma)$.
\end{theorem}

\begin{proof}
Thanks to the strict inequality in \textit{(\ref{eq:roche})}, there is a neighbourhood $U\supseteq \Gamma$ such that $h\defeq\frac{f}{g}$ is holomorphic in $U$.
Then again by \textit{(\ref{eq:roche})}, we see
\[
	\abs{h(z)-1}=\abs{\frac{f(z)}{g(z)}-1}<1 \quad \forall z\in U
\]
and hence $h(U)\subseteq \mathcal{B}_1(1) \subseteq \C\setminus\left\{z\in\C \rmv \mathcal{R}(z)\leq 0, \; \mathcal{I}(z)=0\right\}$ and so $\log(h)$ is well-defined on $U$ and we can write
\[
	(\log(h))'=\frac{h'}{h}=\frac{f'}{f}-\frac{g'}{g}
\]
then since $h$ is holomorphic on $U$ we can integrate its derivative over $\gamma$ and get
\[
	0=\frac{1}{2\pi i}\int_\gamma\frac{f'(\xi)}{f(\xi)}\;d\xi - \frac{1}{2\pi i}\int_\gamma\frac{g'(\xi)}{g(\xi)}\;d\xi
\]
\end{proof}

\begin{note}
This can alternatively be restated as follows:

Given $h,w\in\mathcal{H}(D)$, such that we can write $h=f+g$ and $w=f$, suppose that
\[
	\abs{g(\xi)} < \abs{f(\xi)} \quad \forall\xi\in\Gamma
\]
then $f+g$ and $f$ have the same number of zeros in $\Int(\gamma)$.
\end{note}

\begin{proof}
Define the meromorphic function
\[
	F(z)\defeq \frac{f(z) + g(z)}{f(z)} = 1 + \frac{g(z)}{f(z)}
\]
Note that we want to show that $Z_A(F) = P_A(F)$.
By the argument principle we have that $Z_A(F) - P_A(F)$ is the winding number of $F\circ \gamma$ about $0$ since
\[
	\int_\gamma \frac{F'(z)}{F(z)}dz = \int_{F\circ \gamma} \frac{1}{z}dz
\]
But by assumption we have that for all $z$ in the image of $\gamma$ $\abs{F(z)-1 } < 1$.
So the image of $\gamma$ under $F$ is contained 
\end{proof}

\section{Functional Convergence}
\begin{defin}
Suppose $f_n:D\to\C$ where $D\subseteq\C$ is open.
We say that $f_n$ \mdf{converges locally uniformly} to $f$ as $n\to\infty$ if
\[
	\forall\text{compact }K\subseteq D, \quad \restr{f_n}{K}\to\restr{f}{K}\;\text{uniformly}
\]
\end{defin}
\begin{theorem}[Weirstrass Convergence Theorem]
Given $D\subseteq\C$ open and connected and a sequence $(f_n)$ of holomorphic functions on $D$ such that $f_n$ converges locally uniformly to $f$, then $f\in\mathcal{H}(D)$.
\end{theorem}

\begin{proof}
Pick $z_0\in D$ and $\delta>0$ sufficiently small that $\overline{\mathcal{B}_\delta(z_0)}\subseteq D$.
We will prove that $f$ is holomorphic on all of this ball by use of Morea's Theorem.
So take $\gamma$ closed curve in $\mathcal{B}_\delta(z_0)$.
We can write
\[
	\int_\gamma f = \int_\gamma f_n + \int_\gamma (f - f_n)
\]
But $f(z)-f_n(z)\to 0$ uniformly on $\gamma$ and hence $\int_\gamma (f-f_n) \to 0$ as $n\to\infty$.
Moreover, $f_n$ is holomorphic on $D$ and hence the integral over $\gamma$ is $0$ and thus $\int_\gamma f_n=0$.
We may conclude $\int_\gamma f=0$ and so by Morea's theorem $f$ is holomorphic in the ball.
\end{proof}

\begin{theorem}
Given $D\subseteq\C$ open and connected and a sequence of functions $f_n\in\mathcal{H}(D)$, if $f_n$ converges locally uniformly to $f$ then the derivatives also converge locally uniformly.
\end{theorem}

\begin{theorem}[Hurwitz Theorem]
$D\subseteq\C$ open and connected, $f_n:D\to\C$ and $f_n\in\mathcal{H}(D)$.
Suppose that $f_n$ converges locally uniformly to $f$ and that none of the $f_n$ have more than some $k\in\N$ zeros.
Then either $f$ is constant or has at most $k$ zeros.
\end{theorem}

\begin{proof}
Suppose that $f$ is not constant.
Suppose that $f$ has $K$ zeros of multiplicity $m_1, \dots, m_k$ at distinct $z_1, \dots, z_K\in D$.
Fix $\delta >0$ such that for all no other $z_j$ lies in the $\delta$ ball around $z_i$.
Define
\[
	\epsilon\defeq\inf_{i=1, \dots, k}\inf_{\xi\in\partial\mathcal{B}_\delta(z_i)}\abs{f(\xi)} >0
\]
By compactness we have that $\epsilon >0$.
Then by locally uniform convergence we know there is a $n_0$ such that
\[
	\sup_{i=1, \dots, k}\sup_{\xi\in\partial\mathcal{B}_\delta(z_i)}\abs{f_n(\xi)-f(\xi)} < \frac{\epsilon}{2}\quad\forall n> n_0
\]
Rouche's theorem tells us that $f$ has exactly $m_i$ zeros in $\mathcal{B}_\delta(x_i)$ since
\[
	\abs{f_n(\xi)-f(\xi)} < \abs{f(\xi)} \quad \forall\xi\in\mathcal{B}_\delta(z_i)
\]
\end{proof}

In words $\epsilon$ was the `minimum' value of $f$ on the boundary of these balls.
Then thanks to locally uniform convergence we were able to ensure that the difference between $f_n$ and $f$ was always less than this minimum value and thus we could use Rouche's theorem.

\section{Special Functions}

\subsection{The Gamma Function}
\begin{defin}
	To begin we define the function for $\mathcal{R}(z)> 0$
	\[
		\mdf{\Gamma(z)}\defeq\int_0^\infty t^{z-1}e^{-t}dt
	\]
\end{defin}

\begin{theorem}
The Gamma-integral defines a holomorphic function and the $k$'th derivative is
\[
	\Gamma^{(k)}(z)=\int_0^\infty t^{z-1}(\log t)^k e^{-t}dt
\]
\end{theorem}
One can easily show directly from the integral that $\Gamma$ satisfies the functional relationship
\[
	\Gamma(z+1)=z\Gamma(z)
\]
which we can use to extend $\Gamma$ to the rest of $\C$.
Given any $z\in\C$ we just choose $n\in\N$ such that $\mathcal{R}(z+n)>0$ and then define
\[
	\Gamma(z)\defeq\frac{\Gamma(z+n)}{z(z+1)\dots(z+n)}
\]
which is well-defined and does not depend on $n$.
Of course this gives us $\infty$ at the negative integers and so
\begin{theorem}
The gamma function extends via. the functional relationship to a meromorphic function with poles at $z=0, -1, -2 \dots$, each of order $1$ and satisfies
\[
	\res(\Gamma, -n)=\frac{(-1)^n}{n!}
\]
\end{theorem}
\subsection{Infinite Products}
\begin{defin}
	We say $\prod_{k=1}^\infty  w_k$ \mdf{converges} if
	\begin{itemize}
		\item only finitely many $w_n=0$
		\item The sequence of partial products converge with non-zero limit
	\end{itemize}
	Note that if the first few terms are $0$ then we just ignore them.

	Then $\prod_{k=1}^\infty w_K$ \mdf{converges absolutely} if $\exists n_0\in\N$ such that
	\[
		\sum_{k=n_0}^{\infty}\log w_k
	\]
	converges.
\end{defin}

\begin{note}
	\[
		w_n=\frac{\prod_{k=1}^n w_n}{\prod_{k=1}^{n-1} w_n}\to 1 \quad \text{as} \quad n\to\infty
	\]
\end{note}


\begin{prop}
Defining $C^-\defeq\C\setminus\left\{y=0, x\leq 0\right\}$
\begin{enumerate}[label=(\alph*)]
	\item $\prod_{n=1}^\infty w_n$ with $w_n\in\C^-\;\forall n\in\N$ converges $\iff$ $\sum_{n=1}^{\infty}\log w_n$ converges.
	\item $\prod_{n=0}^\infty(1+a_n)$ converges absolutely $\iff$ $\sum_{k=1}^\infty\abs{a_k}$ converges
\end{enumerate}
\end{prop}

We can use infinite products to further characterise the Gamma function.

\begin{lemma}
The infinite product
\[
	H(z)\defeq\prod_{n=1}^\infty \left(1+\frac{z}{n}\right)e^{-\frac{z}{n}}
\]
converges and absolutely and defines an entire function.
\end{lemma}

\begin{cor}
	\begin{align*}
		G_N(z)&\defeq z e^{-z\log(N)}\prod_{n=1}^N\left(1+\frac{z}{n}\right)	\\
			  &= ze^{-z\left(\log N - \sum_{n=1}^{N}\frac{1}{n}\right)}\prod_{n=1}^N\left(1+\frac{z}{n}\right)e^{-\frac{z}{n}}
	\end{align*}
\end{cor}

\begin{theorem}
For any $z\in\C\setminus\left\{0, -1, -2, \dots \right\}$ we can write
\[
	\frac{1}{\Gamma(z)}=G(z)=\lim_{N\to\infty}G_N(z)
\]
\end{theorem}

\subsection{The Zeta Function}
\begin{defin}
Given any $z\in\C$ with $\mathcal{R}(z)>1$ we can define the \mdf{Riemann zeta function} by
\[
	\mdf{\zeta(z)}\defeq\sum_{n=1}^{\infty}\frac{1}{n^z}
\]
\end{defin}

\begin{theorem}
We can express this as an infinite product over the primes $\mathbb{P}=\left\{2, 3, 5, 7, \dots\right\}$ by
\[
	\frac{1}{\zeta(z)}= \prod_{p\in\mathbb{P}}\left(1-\frac{1}{p^z}\right)
\]
\end{theorem}

\begin{note}
	The $\zeta$-function does not have any zeros in $\left\{\mathcal{R}(z)>1\right\}$.
\end{note}

\begin{lemma}
We can also relate the $\zeta$-function to the $\Gamma$-function by
\[
	\zeta(z)=\frac{1}{\Gamma(z)}\int_0^\infty\frac{t^{z-1}e^{-t}}{1-e^{-t}}dt
\]
\end{lemma}

\begin{note}
The $\zeta$-function can also be characterised by the Hankel contour but we really didn't do much on this.
It might be worth reading over.
\end{note}

\section{Riemann Mapping Theorem}

\begin{defin}
	Two open set $U, V\subseteq\C$ are said to be \mdf{conformally equivalent} if there exists $\phi:U \to V$ such that
	\begin{itemize}
		\item $\phi$ is holomorphic.
		\item $\phi$ is a bijection.
		\item The inverse map $\phi^{-1}$ is holomorphic.
	\end{itemize}
	If $\phi$ can be given by $z\mapsto az+b$ for some complex $a, b$ then we say $U$ and $V$ are \mdf{congruent}.

	Given $D\subseteq\C$ and loops $\gamma_1, \gamma_2:[t_0, t_1]\to D$ then they are \mdf{homotopic} if there exists a continuous mapping $h:[0, 1]\times[t_0, t_1]\to D$ such that for all $t\in[t_0, t_1]$ and $s\in[0, 1]$ we have
	\begin{itemize}
		\item $h(0, t)= \gamma_1(t)$
		\item $h(1, t)=\gamma_2(t)$
		\item $h(s, t_0)=\gamma_1(t_0)=\gamma_2(t_0)$
		\item $h(s, t_1)= \gamma_1(t_1)=\gamma_2(t_1)$
	\end{itemize}
	Then we define $h_\tau(t)\defeq(\tau, t)$ so that $h_0=\gamma_1$ and $h_1=\gamma_2$.

	Finally, a connected set $D\subseteq\C$ is called \mdf{simply connected} if any two continuous curves with the same base point are homo topic to one another. 
\end{defin}
\begin{note}
	$\text{Congruent}\implies\text{Conformally equivalent}\implies\text{Homeomorphic}$
\end{note}

\begin{theorem}
Given $\gamma_1, \gamma_2$ homotopic and $f:D\to\C$ holomorphic
\[
	\int_{ \gamma_1 }f(z) dz = \int_{\gamma_2 }f(z)dz
\]
\end{theorem}

\begin{theorem}[Cauchy's theorem for simply connected set]
Given a closed curve $\gamma$ in $D$ an open and simply connected domain along with any $f\in\mathcal{H}(D)$,
\[
	\int_\gamma f(z)dz=0
\]
\end{theorem}

\begin{proof}
Using the previous theorem, denote by $e$ the constant path at any point along $\gamma$ then
$\int_\gamma f(z)dz = \int_e f(z)dz =0$
\end{proof}

\begin{lemma}
Given $D\subseteq \C$ open and simply connected, $f:D\to\C\setminus\left\{0\right\}$ holomorphic there exists a holomorphic function $g:D\to \C$ such that
\[
	f(z)=e^{g(z)}\quad\forall z\in D
\]
moreover, that $g$ is unique up to an additive constant $2\pi n$ for $n\in \Z$.
\end{lemma}

\begin{proof}
Choose $z_0\in D$ then $f(z_0)\neq 0$ so there exists $w_o\in\C$ such that $f(z_0)=e^{w_0}$.
Now given any other $z\in D$ choose a path $\gamma$ in $D$ which connects $z_0$ to $z$.
Now define
\[
	g(z)\defeq w_0+\int_{f\circ\gamma}\frac{1}{z}dz
\]
This is well-defined because our set is simply connected.
Now we can see that in fact
\[
	g(z)=w_0+\int_\gamma\frac{f'(z)}{f(z)}dz
\]
and so $g$ is holomorphic with derivative $g'(z)\frac{f'(z)}{f(z)}$.
Hence we have that
\[
	\left(f(z)e^{-g(z)}\right)'=f'(z)e^{-g(z)}-f(z)e^{-g(z)}\frac{f'(z)}{f(z)}=0
\]
and so $f(z)e^{-g(z)}$ is a constant say $\alpha$ and moreover
\[
	\alpha=f(z_0)e^{-g(z_0)}=1
\]
\end{proof}

\begin{theorem}[Riemann Mapping Theorem]
Given $D\subseteq \C$ open and simply connected such that $D\neq\emptyset$ and $D\neq\C$, then $D$ is conformally equivalent to $\Delta=\left\{\abs{z}<1\right\}$.
\end{theorem}
\begin{proof}
Bit of a mess.
\end{proof}

\[
	\mathbb{T}\mathbb{H}\mathbb{E}\quad\mathbb{E}\mathbb{N}\mathbb{D}
\]

\end{document}
