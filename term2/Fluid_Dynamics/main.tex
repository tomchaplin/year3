\documentclass[11pt]{article}

%{{{ Document metadata
\title{Fluid Dynamics Notes}
\author{Thomas Chaplin}
\date{}
\setlength{\parindent}{0pt}
%\vfuzz=30pt
%\hfuzz=30pt
%}}}
%{{{ Packages
\usepackage[margin=1in]{geometry}
\usepackage{enumitem}
\usepackage{amsfonts}
\usepackage{amssymb}
\usepackage{amsmath}
\usepackage{amsthm}
\usepackage{mathdots}
\usepackage[dvipsnames]{xcolor}
\usepackage[framemethod=TikZ]{mdframed}
\usepackage{microtype}
\usepackage{bm}
\usepackage{silence}
\WarningFilter{mdframed}{You got a bad break}
%}}}
%{{{ Custom commands
% Partial Derivative command
\newcommand*{\pd}[3][]{\ensuremath{\frac{\partial^{#1} {#2}}{\partial {#3}^{#1}}}}
% Material Derivative command
\newcommand*{\md}[1]{\ensuremath{\frac{D #1}{D t}}}
% Gradient
\newcommand{\grad}{\bigtriangledown}
% Infinitesimal contour segment
\newcommand{\dl}{d\ell}
% Define to equal
\newcommand{\defeq}{:=}
\newcommand{\eqdef}{=:}
% Tensor underline
\newcommand{\tul}[1]{\underline{\underline{#1}}}

% Quick column vector command
\newcount\colveccount
\newcommand*\colvec[1]{
        \global\colveccount#1
        \begin{pmatrix}
        \colvecnext
}
\def\colvecnext#1{
        #1
        \global\advance\colveccount-1
        \ifnum\colveccount>0
                \\
                \expandafter\colvecnext
        \else
                \end{pmatrix}
        \fi
}

% Custom formatting commands
\newcommand{\mv}[1]{\bm{#1}}
\newcommand{\mdf}[1]{{\color{red}#1}}

% Norms
\newcommand{\abs}[1]{\left|#1\right|}

% Spaces
\newcommand{\R}{\mathbb{R}}

%}}}
%{{{ Custom environments
% Proof environments
\renewenvironment{proof}{{\bfseries Proof}}{\qed}
\newtheorem{theorem}{Theorem}[section]
\newtheorem{axiom}{Axiom}[section]
\newtheorem{prop}{Proposition}[section]
\newtheorem{corollary}{Corollary}[section]
\newtheorem{lemma}{Lemma}[prop]

\newenvironment{defin}
	{\begin{mdframed}[backgroundcolor=white, roundcorner=5pt, linewidth=1pt]
		\setlength{\parindent}{0pt}
		}
	{\end{mdframed}}

\definecolor{mylg}{rgb}{0.9,0.9,0.9}
\newenvironment{eg}
{\begin{mdframed}[backgroundcolor=mylg, roundcorner=5pt, linewidth=0pt]\textbf{Example: }\normalfont}
    {\end{mdframed}}
    
\newenvironment{note}
    {\textbf{Note:}\begin{mdframed}[backgroundcolor=white, roundcorner=5pt, linewidth=0pt]}
    {\end{mdframed}}

\newenvironment{formula}
	{\begin{mdframed}[backgroundcolor=white, roundcorner=5pt, linewidth=1pt, linecolor=red]}
	{\end{mdframed}}
%}}}
\begin{document}
\maketitle

\section{Mathematical Modelling of Fluid Flow}
\subsection{Validity of continuum mechanics}
Continuum mechanics is valid when
$$\frac{l}{L}<<1$$
where $l$ and $L$ are length scales characterising molecular motion and flow dimension respectively.
\begin{eg}
\textbf{Typical examples:}
\begin{itemize}
    \item $l_{gas}\sim 100nm=10^{-7}m$ is the mean free path of a typical gas
    \item $l_{liquid}\sim 1nm=10^{-9}m$ is the typical distance between molecules.
\end{itemize}
\end{eg}

\subsection{Lagrangian v.s. Eulerian Description}
A fluid will consist of a number of material points/fluid elements/fluid particles/fluid volumes.
In order to uniquely identify each fluid element we can tag it according to its Eulerian coordinates at some initial condition
$$x_j(t-t_0)=X_j=(X_1,X_2,X_3)$$
Alternatively we allow the coordinates to move and deform with time to get the $X_j$ Largrangian coordinates which follow the fluid.
This allows us to derive useful equation but these equations are usually simpler when transformed back to Eulerian coordinates.

\subsection{Flow Visuation}
\subsubsection{Particle Paths}
Particle paths are intuitively the path a particle will follow if dropped into a flow and depends only on initial conditions:
$$\pd{x_i^p}{t}=u_i \quad\quad\quad \text{for fixed} \quad x_i^p(t=0)=X_i$$
We integrate these equations and eliminate $t$ to get particle paths.
\begin{eg}
To calculate particle paths for the flow $\mv{u}(t)=(u_0,kt,0)$, we integrate to get
$$\mv{x}(t)=(u_0t+a,\frac{kt^2}{2}+b,c)\implies y=\frac{k}{2}\left(\frac{x-a}{u_0}\right)^2+b\implies\text{Parabola}$$
\end{eg}
\subsubsection{Stream lines}
Intuitively, we pause time and then draw lines in the vector field. We parameterise these lines in terms of arc length $s$
$$\pd{x_i^p}{s}=u_i \quad\quad\quad \text{for fixed} \quad t$$
We integrate these equations and eliminate $s$ to get stream lines. Note, for a steady flow (i.e $\pd{\mv{u}}{t}=0$), this is the same as the particle paths. 
\begin{eg}
For the same flow $\mv{u}(t)=(u_0,kt,0)$, suppose at $s=0$ we have $(x,y,z)=(a,b,c)$ then
$$\mv{x}(t)=(u_0s+a,kts+b,c)\implies y=kt\left(\frac{x-a}{u_0}\right)+b\implies\text{Straight lines}$$
\end{eg}

\subsection{Material Derivative}
The \mdf{material derivative} is the rate of change of something which follows the flow. To figure out what this is we consider the derivative of a function dependant on time and the particle paths $f(\mv{x}^p(t),t)$. Then
\begin{align*}
    \pd{f}{t}&=\pd{f}{t}+\sum_i \pd{f}{x_i}\pd{x_i}{t}\\
             &=\pd{f}{t}+\sum_i u_i\pd{f}{x_i}\\
             &=\left(\pd{ }{t} + \sum_i u_i\pd{ }{x_i}\right)f
\end{align*}
We then define the \mdf{material derivative} to be
$$\md{ }\defeq \pd{ }{t}+\sum_i u_i\pd{ }{x_i}=\pd{ }{t}+\mv{u}\cdot\grad$$
The first time is the local rate of change at a fixed Eulerian position whereas the second term represents the convective rate of change caused by driving fluid elements through gradients of $f$.
\begin{eg}
For a concentration of pollutant c=c(x) in a river with steady flow $\mv{u}=(u_0,0,0)$, how does the concetration change in a fluid element that follows the fluid?
\begin{align*}
    \md{c}&=\pd{c}{t}+(\mv{u}\cdot\grad)c\\
          &=u_0\pd{c}{x}+0\pd{c}{y}+0\pd{c}{z}\\
          &=u_0\pd{c}{x}
\end{align*}
\end{eg}
\begin{note}
\begin{itemize}
    \item $\md{f}\equiv 0 \implies$ $f$ is constant in fluid elements but, in general, has different values in different elements.
    \item Parameterising streamlines by $s$, let $\mv{e}_s$ be the unit tangent vector at $s$. Then
    $$(\mv{u}\cdot\grad) f= |u|\mv{e}_s\cdot\grad f=|u|\pd{f}{s}$$
    Hence $(u\cdot\grad)f = 0\implies$ $f$ constant on streamlines
\end{itemize}
\end{note}
We can now define \mdf{acceleation} as
$$\mv{a}=\md{\mv{u}}=\pd{\mv{u}}{t}+(\mv{u}\cdot\grad)\mv{u}.$$
We also say a flow is \mdf{steady} if $\pd{\mv{u}}{t}=0$ (normal partial derivaive).
\begin{eg}
Consider the 1D flow $\mv{u}=(u_1(x),0,0)$, noticing we have steady flow but
\begin{align*}
    a_1&=\pd{u_1}{t}+\sum_j u_j\pd{u_1}{x_j}\\
       &= 0 + u_1 \pd{u_1}{x} = u_1 \pd{u_1}{x}
\end{align*}
individual fluid elements still accelerate. To visualise this consider a flow in a pipe which constricts.
\end{eg}

\subsection{Vorticity and Rate of Strain}
We want to understand how flow deforms fluid elements which will show us that rate of deformations is what generates stress in a fluid.

To do this we Taylor expand velocity to first order, at two points differing by some $\delta_{x_j}:$
\begin{align*}
    u_i(x_j+\delta_{x_j},t)&=u_i(x_j,t)+\sum_j \pd{u_i}{x_j}\delta{x_j}\\
    &=\underbrace{u_i(x_j,t)+\sum_j r_{ij}\delta{x_j}}_{\text{rigid body}}+\underbrace{\sum_j e_{ij}\delta{x_j}}_{\text{Shearing + Extension}}\\
    &\eqdef u_i^T + u_i^R + u_i^S
\end{align*}
\begin{align*}
    \text{where} \quad &r_{ij}=-r_{ji}\defeq\frac{1}{2}\left(\pd{u_i}{x_j}-\pd{u_j}{x_i}\right) & \text{is the \mdf{rate of rotation tensor}}\\
    \quad &e_{ij}=e_{ji}\defeq\frac{1}{2}\left(\pd{u_i}{x_j}+\pd{u_j}{x_i}\right) & \text{is the \mdf{rate of strain tensor}}
\end{align*}
This term has three components of interest which we will now explore.
\subsubsection{Translation}
The first term is certainly translation because
$$u_i(x_j+\delta_{x_j},t)=u_i(x_j,t)\implies\text{velocity is locally constant}$$
This induces no internal stress.
\subsubsection{Rotation}
Because the rate of rotation tensor is skew-symmetric we only get 3 non-zero terms:
\[
(r_{ij})=
\begin{bmatrix}
0 & -c & -b\\
c &  0 & -a\\
b &  a &  0
\end{bmatrix}
\]
so we can rewrite the rotation term as
$$u_i^R=\sum_j r_{ij}\delta_{x_j} = \sum_j \sum_k \epsilon_{ijk}\Omega{j}\delta_{x_k}$$
where $\Omega = (r_{32},r_{13},r_{21})=(a,b,c)$.
We then define \mdf{vertocity} to be $\omega\defeq\grad\times\mv{u}=2\Omega$.
We can view this as the rotation term where $\Omega$ is the local rate of rotation. Rotation induces no internal stress.
\subsubsection{Shearing + Extension}
The strain term $u_i^S$ involves the relative motion of fluid particles. They can be split into two types:
\begin{align*}
    \text{Diagonal terms} &\longleftrightarrow \text{Extensional terms}\\
    \text{Off-diagonal terms} &\longleftrightarrow \text{Sharing terms}
\end{align*}
Read written notes for diagrammatic explanation of these terms.
\begin{note}
Denote by $V$ the volume of a fluid element then
\begin{formula}
$$\frac{1}{V}\md{V} = \grad\cdot\mv{u}$$
\end{formula}
Therefore $\md{V}=0\implies\underbrace{\grad\cdot\mv{u}=0}_{Incompressible flow}\iff\sum_k e_{kk}=0$
\end{note}

\subsection{Conservation of Mass}
We want an equation which encompasses conservation of mass in a fluid. Consider a fluid element cube with (infinetesimal) side length $d$ and centred at $(x_0,y_0,x_0)$.
\begin{align*}
    \text{Rate of increase of fluid mass} &= \text{Volume} \cdot \text{Rate of change of density}\\
    &= d^3 \cdot \pd{\rho}{t}
\end{align*}
Summing over all sides we see
\begin{align*}
d^3\pd{\rho}{t}=d^2[&-\rho(x_0+d,y,z)u(x_0+d,y,z)+\rho(x_0,y,z)u(x_0,y,z)\\
                    &-\rho(x,y,_0+d,z)u(x,y_0+d,z)+\rho(x,y_0,z)u(x,y_0,z)\\
                    &-\rho(x,y,z_0+d)u(x,y,z_0+d)+\rho(x,y,z_0)u(x,y,z_0)]
\end{align*}
Dividing by $d^3$ and taking the limit $d\to 0$ we see
$$\pd{\rho}{t}=-\pd{(\rho u)}{x} - \pd{(\rho v)}{y} - \pd{(\rho w)}{z} = -\grad\cdot(\rho\mv{u})$$
So we have our equation
\begin{formula}
$$\pd{\rho}{t}+\grad\cdot(\rho\,\mv{u})=0$$
\end{formula}
\begin{note}
	We have derived this equation in Cartesian coordinates but obtained a vector equation, so we may apply it to any coordinate system.
\end{note}
\subsection{Conservation of Momentum}
We would like to apply Newton's $2^{nd}$ law to fluids. So far we can write
$$m\mv{a}=\underbrace{ \mv{F}_{body} }_{internal}+\underbrace{\mv{F}_{stress}}_{external}$$
We define \mdf{stress} to be the force per unit area and then the \mdf{stress tensor} $T_{ij}$  is the $i^{th}$ component of stress on the surface with normal $\mv{n}_j$.
We normally the symmetric class of tensors where $T_{ij}=T{ji}$.
We now consider the total force acting on our infinitesimal cube and by a similar calculation as the one in conservation of momentum we find
$$\delta F_i=d^3\sum_j\pd{T_{ij}}{x_j}$$
\subsubsection{Cauchy's Momentum Equation}
We can now substitute all of the formulae into Newton's second law to get:
$$\underbrace{ (\rho d^3) }_{mass}\underbrace{ \md{u_i} }_{acceleration}=\underbrace{d^3\sum_j\pd{T_{ij}}{x_j}}_{internal} -\underbrace{(\rho d^3)}_{mass}\underbrace{g\delta_{i3}}_{gravity}$$
Then we can divide by $d^3$ on both sides to yield the Cauchy Momentum Equation:
\begin{formula}
\textbf{Cauchy's Momentum Equation:}
$$\rho\md{u_i}=\pd{T_{ij}}{x_j}-\rho g \delta_{i3}$$
or in vector form
$$\md{\mv{u}}=\frac{1}{\rho}\grad\cdot\;\tul{\mv{T}}-g\mv{e}_z$$
\end{formula}
This is a very important formula but it requires us to know the stress tensor. We would like to relate $\tul{\mv{T}}$ to velocity $\mv{u}$ and pressure $\mv{P}$. 
\subsubsection{Inviscid Flow}
In inviscid flows, there is no internal friction and so there are no sheering terms to the stress tensor, i.e. all off-diagonal terms are 0.
Therefore, the only stress is the inward pressure acting perpendicular to the sides of the fluid element.
We can write this as the following \mdf{constitutive relation}:
$$T_{ij}=-p\delta{i_j}$$
Substituting $\pd{T_{ij}}{x_j}=-\pd{p}{x_i}$ into Cauchy's Momentum Equation we see
\begin{formula}
	\textbf{Euler's Equation}
	$$\md{\mv{u}}=-\frac{1}{\rho}\grad{p}-g\mv{e}_z$$
\end{formula}
\begin{note}
\begin{itemize}
	\item In the absence of gravitational forces we see that fluid elements are accelerated by pressure gradients from high to low pressure.
	\item If we assume fluid elements are not being accelerated then we recover a hydrostatic balance:
		$$0=-\frac{1}{\rho}\pd{p}{z}-g\implies p=-\rho gz + p_0$$
\end{itemize}
\end{note}
\subsubsection{Stress Tensor for a Viscous Fluid}
This time we have the normal inward pressure as with inviscid flows but we also have an additional \mdf{viscous stress tensor} $\sigma_{ij}$.
$$T_{ij}=-p\delta_{ij}+\sigma_{ij}$$
We need another constitutive relation between the rate of strain tensor $e_{ij}$ and the viscous stress tensor $\sigma_{ij}$.
When this relationship is linear, we call this flow \mdf{Newtonian}.
In the case that the flow is incompressible we get $\sum_k e_{kk}=0$ and hence
$$\sigma_{ij}=2\mu e_{ij}$$
where $\mu$ is the dynamic viscosity given in $kgm^{-1}s^{-1}$. The dynamic viscosity is the coefficient of proportionality between the rate of strain tensor and the stress tensor. This describes how easily a fluid moves under a shear force.
Then we can write
\begin{align*}
	T_{ij}&= -p\delta_{ij}+2\mu e_{ij}\\
		  &= -p\delta_{ij}+\mu\left(\pd{u_i}{x_j}+\pd{u_j}{x_i}\right)
\end{align*}
Now $\sum_i T_{ii} = -3p$ and hence $p=-\frac{1}{3}(\sum_i T_{ii})$ is the mechanical (NOT thermodynamic) pressure.
\begin{note}
The flow of incompressible Newtonian fluids covers a huge range of phenomena.
However, there are many cases in which non-Newtonian behaviour is encountered and hence we need more complex constitutive relations than those above.
\end{note}
We can now substitute these relations into Cauchy's momentum equation to get:
\begin{formula}
\textbf{The incompressible Navier-Stokes equations:}
$$\md{u_i}=-\frac{1}{\rho}\pd{p}{x_i}+\nu\pd[2]{u_i}{x_k}-g\delta_{i3}$$
or in vector form
$$\md{\mv{u}}=\underbrace{-\frac{1}{\rho}\grad{p}}_{Pressure\;gradients}+\underbrace{\nu\grad^2\mv{u}}_{Viscous\;forces}\underbrace{-g\mv{e}_z}_{Gravity}$$
where $\nu=\frac{\mu}{\rho}$ is the \mdf{kinematic viscosity coefficient}.
\end{formula}
\subsection{Controlling Flow Parameters}
We have two key questions to answer before we can start using these formulae:
\begin{itemize}
	\item When is the inviscid assumption valid?
	\item When is the incompressible assumption valid?
\end{itemize}
To this end, we shall define two \mdf{dimensionless parameters}, namely the Reynold's Number and Mach Number.
\subsubsection{Reynold's Number}
Our aim is to determine when it is safe to ignore the viscous forces in NS equations.
To do this we compare the approximate sizes of typical terms within the viscous and acceleration term.
Suppose we have a flow with characteristic length $L$ and characteristic speed $u$ with a kinematic viscosity $\nu$.
\begin{align*}
	\text{A typical viscosity term is} &\left|\nu\pd[2]{u}{x}\right|\sim\frac{\nu u}{L^2}\\
	\text{A typical acceleration term is} &\left|u\pd{u}{x}\right|\sim\frac{u^2}{L}
\end{align*}
Dividing these two terms we should get (we can check a dimensional parameter. We call this the \mdf{Reynold's number}.
$$\frac{|(u\cdot\grad)u|}{|\nu\grad^2 u|}\sim \frac{u^2}{L}\cdot\frac{L^2}{\nu u}=\underbrace{\mdf{\frac{uL}{\nu}}}_{\mdf{Reynold's\;Number}}$$
We now have two cases:
\begin{itemize}
	\item $Re >> 1 \implies \frac{Inertia}{Viscosity} >> 1 \implies$ Viscosity is negligible so we can use the Euler equations.
	\item $Re << 1 \implies \frac{Inertia}{Viscosity} << 1 \implies$ Inertia is negligible so we can use the Stoke's equation (where we remove inertia term).
	\item $Re \approx 1 \implies$ We have to use the NS equations
\end{itemize}
\begin{formula}
	\textbf{Stokes's Equation:}
	$$0=-\frac{1}{\rho}\grad{p}+\nu\grad^2\mv{u}-g\mv{e}_z$$
\end{formula}
\begin{note}
Even if $Re >> 1$ based on the scale of the flow on a macro scale, we may see complex behaviour locally where the length scale can be much lower, thus reducing the Reynold's number. This is typical for thin layers of flow close to solid boundaries.
\end{note}

\subsubsection{Mach Number}
Our aim now is to determine when it is safe to neglect compressibility of the flow. We must construct an estimate for relative changes of density $(\delta\rho / \rho)$ in the fluid.
We assume that the dominant terms in the Navier-Stokes equations are the inertial terms and pressure terms, i.e.
$$(\mv{u}\cdot\grad)\mv{u}\sim \frac{\grad p}{\rho}$$
But we can rewrite the right hand side to get
$$(\mv{u}\cdot\grad)\mv{u}\sim \frac{\grad \rho}{\rho}\pd{p}{\rho}$$
On a small scale we can approximate the gradient of $\rho$ as $(\delta\rho/L)$ and then we recognise that $\pd{p}{\rho}=c_s^2$ where $c_s$ is the speed of sound. Substituting in the approximate size of our typical inertial term we get:
$$\frac{U^2}{L}\sim \frac{\delta\rho}{L\rho}c_s^2\implies \frac{U^2}{c_s^2}\sim\frac{\delta\rho}{\rho}$$
We now define the \mdf{Mach number} to be $Ma\defeq\frac{U}{c_s}$.
When $Ma << 1 $, incompressibility is a safe assumption.
This is OK for the majority of cases and will be assumed for the remainder of this course.
(NOTE: We are talking about flow compressibility NOT fluid compressibility).
\subsubsection{Similarity}
It can be shown that the only dimensionless parameter arising from the Navier-Stokes equations is the Reynold's Number.
As such, flows with similar geometries and boundary conditions, whose Reynolds numbers agree will have identical flows.
This can be used to create small scale models for testing purposes.
For example, to decrease the length scale by a factor of 10, we need only increase the characteristic speed of flow by a factor of 10 in order to compensate.

\subsection{Boundary and Initial Conditions}
We have some PDEs and so in order to find an initial condition we must additionally specify some boundary and initial conditions.

\subsubsection{Initial Conditions}
We must specify all variables for which we have taken time derivatives, at time $t=0$, for all points $x$ occupied by the fluid $V\subseteq\mathbb{R}^d$. Note we have no time derivatives on density $\rho$ and hence we only have to specify
$$\mv{u}(\mv{x},t=0)\defeq\mv{u}_0\quad\quad\forall\mv{x}\in V$$
This \mdf{initial velocity field} $\mv{u}_0$ must satisfy the upcoming boundary conditions and be divergence free.

\subsubsection{Boundary Conditions}
The number of boundary conditions depends on the bulk equations (i.e. Euler or NS equations) and their form depends on the physics at the boundaries.

\textbf{Viscous fluids:}

At solid, impenetrable boundaries at has been shown empirically that the fluid velocity $\mv{u}$ at the boundary $\partial V$ of the retaining volume is equal to the boundary's velocity $\mv{U}_b$. This is the \mdf{no-slip boundary condition}:
$$\mv{u}(\mv{x},t)=\mv{U}_b(\mv{x},t)\quad\quad\forall\mv{x}\in\partial V$$

\textbf{Inviscid fluids:}

In inviscid fluids there is no $\grad^2 \mv{u}$ term and so we only have terms in the first derivatives of the velocity.
We can only be sure that the boundary is impenetrable.
This can be expressed by saying that the normal component of the fluid velocity at the boundary is equal to the normal component of the boundary velocity.
The component parallel to the boundary has no restriction.
This means that the fluid can slip freely past the boundary, giving rise to the \mdf{free-slip boundary condition}:
$$\mv{u}(\mv{x},t)\cdot\mv{n}=\mv{U}_b(\mv{x},t)\cdot\mv{n}\quad\quad\forall\mv{x}\in\partial V$$
where $n$ is the unit normal to the boundary.

\textbf{Free surfaces:}

So far we have only considered hard, fixed boundaries. Sometimes a liquid will have its boundary where it meets a passive gas (e.g. at the surface of the pond) in a \mdf{free surface} or \mdf{free boundary}.
The location of this boundary must then be found as part of the solution.
To summarise this, we assume the free-slip boundary conditions and that the pressure of the liquid at the boundary is equal to the atmospheric pressure $p_a$ in the gas.
$$p(\mv{x},t)=p_a\quad\quad\forall\mv{x}\in\partial V$$

\subsection{Summary}
The \mdf{material derivative} is
$$\md{}\defeq\pd{}{t}+\sum_i \pd{}{x_i}=\pd{}{t}+\mv{u}\cdot\grad$$

A flow is \mdf{steady} if $\pd{\mv{u}}{t}=0$.

The \mdf{rate of rotation tensor} is 
$$r_{ij}\defeq\frac{1}{2}\left(\pd{u_i}{x_j}-\pd{u_j}{x_i}\right)$$

The \mdf{rate of strain tensor} is 
$$r_{ij}\defeq\frac{1}{2}\left(\pd{u_i}{x_j}+\pd{u_j}{x_i}\right)$$

The \mdf{vorticity} is $\omega\defeq\grad\times\mv{u}=2(r_{32},r_{13},r_{21})$.

We have \mdf{incompressible flow if} $\grad\cdot\mv{u}=0$.

The \mdf{stress tensor} $T_{ij}$ is the $i^{th}$ component of stress (force per unit area) on the surface with normal $\mv{n}_j$.

\mdf{Cauchy's Momentum equation} is
$$\rho\md{u_i}=d^3\pd{T_{ij}}{x_j}-\rho g\delta_{i3}\;\iff\;\md{\mv{u}}=\frac{1}{\rho}\grad\cdot\tul{\mv{T}}-g\mv{e}_z$$

For inviscid flows we can use \mdf{Euler's Equation}
$$\md{\mv{u}}=-1\frac{1}{\rho}\grad p-g\mv{e}_z$$

For Newtonian flows the relationship between the viscous stress and strain tensors is linear $\sigma_{ij}=2\mu e_{ij}$ where $\mu$ is the dynamic viscosity.

For incompressible Newtonian fluids we have the \mdf{incompressible Navier-Stokes equations}
$$\md{u_i}=\frac{1}{\rho}\pd{p}{x_i}+\nu\pd[2]{u_i}{x_k}-g\delta_{i3}$$
or in vector form
$$\md{\mv{u}}=\frac{1}{\rho}\grad p + \nu \grad^2\mv{u} - g\mv{e}_z$$
where $\nu=\frac{\mu}{\rho}$ is the \mdf{kinematic viscosity coefficient}.

\mdf{Reynold's number} measures the negligibility of viscosity. When large we can use Euler's equations, when low we can use Stoke's equations as approximations.
$$Re\defeq\frac{uL}{\nu}$$

The \mdf{Mach number} measures the importance of compressibility. At low Mach numbers, incompressibility is a safe assumption.
$$Ma\defeq\frac{u}{c_s}$$

When solving these equations we must specify a divergence free \mdf{initial velocity field} which satisfies the boundary conditions.

In viscous fluids we may assume the \mdf{no-slip boundary condition}:
$$\mv{u}(\mv{x},t)=\mv{U}_b(\mv{x},t)\quad\quad\forall\mv{x}\in\partial V$$

In inviscid fluids we may assume the \mdf{free-slip boundary condition}:
$$\mv{u}(\mv{x},t)\cdot\mv{n}=\mv{U}_b(\mv{x},t)\cdot\mv{n}\quad\quad\forall\mv{x}\in\partial V$$

At free surfaces we can assume that surface pressure is equal to atmospheric pressure:
$$p(\mv{x},t)=p_a\quad\quad\forall\mv{x}\in\partial V$$
\section{Additional Conservation Laws}
Now that we have the Navier-Stokes equations we will consider what further conservation laws result from these equations. We consider special cases in which we can make analytic progress.

\subsection{Velocity-Vorticity Form of the NS equations}
When flow in incompressible $\grad\cdot\mv{u}=0$ and we have the following vector identity
$$(\mv{u}\cdot\grad)\mv{u}=\omega\times\mv{u}+\grad\left(\frac{u^2}{2}\right)$$
where we write $u^2=\mv{u}\cdot\mv{u}$. We can sub this into the NS equations and writing $g\mv{e}_z=\grad(gz)$ we get
$$\pd{\mv{u}}{t}+\omega\times\mv{u}=-\grad B+\nu\grad^2\mv{u}$$
where $B$ is the \mdf{Bernoulli Potential}
$$B=\frac{p}{\rho}+\frac{u^2}{2}+gz$$
\subsection{Bernoulli Theorems}
We will now consider two cases:
\begin{itemize}
	\item The flow is irrotational $\omega = 0$.
	\item The flow is steady $\pd{\mv{u}}{t}=0$.
\end{itemize}
\subsubsection{Unsteady Irrotational Flow}
The Helmholtz decomposition tehorem tells us we can split up $\mv{u}$ into a rotational part and an irrotational part $\grad\phi$ where $\phi$ is a velocity potential. So in the irrotational case, velocity can be written
$$\mv{u}=\grad\phi$$
We will also use this identity
$$\grad\times(\grad\times\mv{u})=\grad(\grad\cdot\mv{u})-\grad^2\mv{u}$$
For incompressible irrotational flows this implies $\grad^2\mv{u}=0$.
We substitute all of this into the velocity-vorticity form and get
$$\grad\pd{\phi}{t}=-\grad B$$
which we can over a path $\mv{x}$ to get the following
\begin{theorem}[Bernoulli theorem for steady irrotational flow]
	$$\pd{\phi}{t}+B=C(t), \quad \text{where} \quad B=\frac{p}{\rho}+\frac{\abs{\grad\phi}^2}{2}+gz$$
	where $C(t)$ is a function of time only. The incompressibility condition becomes
	$$\grad\cdot\mv{u}=\grad^2\phi=0.$$
\end{theorem}

\begin{eg}
Do U-bend example here!
\end{eg}

\subsubsection{Steady Irrotational Flow}
If we further assume that flow is steady then the theorem tells us that Bernoulli's potential is constant throughout the flow volume.
$$B\equiv \frac{p}{\rho}+\frac{u^2}{2}+gz$$
\begin{eg}
Pitot Tube example here!
\end{eg}
\subsubsection{Steady Inviscid Rotational Flow}
We again start with the velocity-vorticity form and dot the entire equation with $\mv{u}$. Using $\pd{\mv{u}}{t}=0$ and $\mv{u}\cdot(\omega\times\mv{u})=0$ we arrive at the equation
$$\underbrace{\md{B}=\mv{u}\cdot\grad B}_{\text{for steady flow}} \equiv 0$$
so B follows the fluid. That is B is constant on each streamline, although this constant may differ between stream lines. Moreover, remember steady flow implies that particle paths and steam lines coincide.
\subsection{Global Conservation Laws}
So far all of our conservation laws have been local, i.e. they refer to conservation of a field along fluid trajectories. However, we can also apply similar laws to arbitrary volumes within our fluid by integrating the local equations. This arbitrary volume could be the entire fluid volume.
\subsection{Mass Conservation Over an Arbitrary Volume}
We assume incompressibility $\pd{u_i}{x_i}=0$ and integrate over a finite fixed volume V:
$$0=\int_V \pd{u_i}{x}dV=\int_{\partial V}u_in_idA=\int_{\partial V}\mv{u}\cdot\mv{n}\;dA$$
Intuitively, this means that mass flux through all side of our volume $V$ must sum to zero. In a compressible fluid, where the density inside $V$ can change, this is not necessarily true.
\begin{eg}
Venturi Pipe!
\end{eg}
\subsection{Momentum Conservation}
Recall that the main NS equation is the local momentum balance equation. That is, it was obtained by applying Newton's second law to a moving infinitesimal fluid parcel.
\subsubsection{Over an arbitrary volume}
Consider an arbitrary fixed volume V within the fluid through which the fluid flows.
The total momentum in this volume is given by
$$M_i=\int_V \rho u_i\;dV.$$
Integrating the NS equation over this volume we get
$$\pd{M_i}{t}=-\rho\int_V u_j\pd{u_i}{x_j}dV-\int_V\pd{p_\rho gz}{x_i}dV + \nu\rho\int_V \pd[2]{u_u}{x_j}dV$$
Using incompressibility and the product rule we see that
$$\pd{u_iu_j}{x_j}=u_j\pd{u_i}{x_j}+u_i\underbrace{\pd{u_j}{x_j}}_{=0 \;\text{by incompressibility}}$$
Then using the divergence theorem, we have
$$\pd{M_i}{t}=-\underbrace{\rho\int_{\partial V}u_iu_jn_j\;dA}_{\text{Inertial terms at boundary}}-\underbrace{\int_{\partial V}(p+\rho gz)n_i\;dA}_{\text{Pressure and gravity}}+\underbrace{\mu\int_{\partial V}\pd{u_i}{x_j}n_j\;dA}_{\text{Viscous stress}}$$
Finally, for steady inviscid flows we can conclude
$$\underbrace{\rho\int_{\partial V}u_iu_jn_j\;dA}_{\text{Momentum flux into V}}=\underbrace{-\int_{\partial V}(p + \rho gz)n_i\;dA}_{\text{Boundary forces}}$$
Note, we have information about the balance of forces at the boundary, without needing explicit knowledge of the flow inside our fixed volume V.
\begin{eg}
Jammed garden hose!
\end{eg}
\subsubsection{Over the entire volume}
Suppose we have a fluid contained in a finite volume $V$ whose bounding surface consist of impenetrable, stationary walls. The boundary conditions tell us the following.
\begin{itemize}
	\item For non-zero finite viscosity $\nu$ the normal and parall component of $u_i$ are zero at the boundary
		$$u_i=0\quad\text{on}\;\partial V$$
	\item For zero viscosity $\nu$ we only have the free-slip condition
		$$u_in_i=0\quad\text{on}\;\partial V$$
\end{itemize}
In either case notice that the inertial terms become 0.
If $V$ is the entire volume then momentum will certainly be conserved, i.e. $\pd{M_i}{t}=0$.
So at the boundary we find
$$\int_{\partial V}(p+\rho g z)n_i\;dA=\nu\int_{\partial V}\pd{u_i}{x_j}n_j\;dA$$
so the net effect of the pressure and gravity forces, together with viscous stress at the boundary must be zero.
\subsection{Energy Conservation}
We consider the conservation of energy over the \emph{entire volume} and suppose that the fluid is contained by in a finite volume $V$ bounded by a surface of impenetrable, stationary walls $\partial V$.
We start with a number of notes that will help later
\begin{itemize}
	\item Using integration by parts, followed by the impenetrability of $\partial V$ and incompressibility
		$$\int_V u_i\pd{B}{x_i}dV=\int_V\left(\pd{}{x_i}(u_iB)-B\underbrace{\pd{u_i}{x_i}}_{=0}\right)dV=\int_{\partial V} Bu_in_i\; dA = 0$$
	\item Using integration by parts we see
		\begin{align*}
			\nu\int_V u_i\pd{}{x_j}\pd{u_i}{x_j}dV &= \nu \int_V \left[\pd{}{x_j}\left(u_i\pd{u_i}{x_j}\right)-\pd{u_i}{x_j}\pd{u_i}{x_j}\right]dV\\
												   &= \underbrace{\nu \int_{\partial V} u_i\pd{u_i}{x_j}n_j\;dA}_{=0\;\text{in both conditions}} - \nu \int_{V} \left(\pd{u_i}{x_j}\right)^2 dV
		\end{align*}
	Note we are summing over both indices in this identity.
\item $u_i\pd{u_i}{t}=\frac{1}{2}\pd{u_iu_i}{t}=\frac{1}{2}\pd{u_i^2}{t}$ where $u_i^2=\mv{u}\cdot{\mv{u}}$.
\end{itemize}
Now we dot multiply the velocity-vorticity form with $\mv{u}$ and integrate over the entire volume $V$.
$$\int_V u_i\pd{u_i}{t}dV = - \int_V u_i\pd{B}{x_i}dV + \nu \int_V u_i\pd[2]{u_i}{x_j}dV$$
Substituting in what we found earlier we obtain the relation
$$\pd{}{t}\int_V\frac{1}{2}u^2\;dV=-\nu\int_v\left(\pd{u_i}{x_j}\right)^2dV$$
Note we are summing over two indices (there are no free indices) and hence this is a scalar expression, as we would expect for an energy equation.
We now define the \mdf{total kinetic energy} as
$$E\defeq \frac{\rho}{2}\int_V u^2\;dV$$
and we achieve our energy balance equation.
\begin{formula}
	\textbf{Energy balance equation:}(over the entire volume $V$)
	$$\pd{E}{t}=-\nu\int_v\left(\pd{u_i}{x_j}\right)^2dV$$
\end{formula}
In physicsy terms this means that kinetic is lost by viscous dissipation (i.e. internal friction), which will result in the generation of heat (and hence conservation of total energy).
We see that in the inviscid case, this becomes
$$\pd{E}{t}=0$$
This energy conservation law is \emph{global}.
\subsection{Summary}
We introduced a velocity potential $\mv{u}=\grad\phi$ which, for incompressible flow, yields three different Bernoulli theorems depending on flow characteristics. Recall the \mdf{Bernoulli potential} has now become $B=\frac{p}{\rho}+\frac{\abs{\grad\phi}^2}{2}+gz$.
We have the following theorems:
\begin{enumerate}
	\item \textbf{Unsteady Irrotational Flow ($\omega=0$)}
		$$\pd{\phi}{t}+B=C(t)$$
	\item \textbf{Steady Irrotational Flow ($\omega=\pd{\mv{u}}{t}=0$)}
		$$B=\text{fixed constant}$$
	\item \textbf{Steady Inviscid Rotational Flow ($\pd{\mv{u}}{t}=\nu=0$)}
		$$B=\text{constant on each streamline(= particle paths)}$$
\end{enumerate}
\subsubsection{Fixed Volumes}
Given an arbitrary fixed volume $V$, we have mass conservation
$$\int_{\partial V}u_in_i\;dA=0$$
The \mdf{momentum} $M_i=\int_V \rho u_i\;dV$ satisfies
$$\pd{M_i}{t}=-\rho\int_{\partial V}u_iu_jn_j\;dA-\int_{\partial V}(p+\rho gz)n_i\;dA+\nu\rho\int_{\partial V}\pd{u_i}{x_j}n_j\;dA$$
with the left side zero for steady flow.
\subsubsection{Entire Volume}
Taking $V$ to be the entire volume which is bounded by impenetrable, stationary walls, we have momentum conservation
$$\int_{\partial V}(p+\rho gz)n_idA = \mu\int_{\partial V}\pd{u_i}{x_j}n_j\;dA$$
which is a balance of forces on the boundary.

Considering the \mdf{kinetic energy} $E=\frac{\rho}{2}\int_V u^2\;dV$ we obtain
$$\pd{E}{t}=-\nu\int_V \left(\pd{u_i}{x_j}\right)^2dV$$
so that in the absence of viscosity kinetic energy is conserved globally.

\section{Vortex Dynamics}
To save chalk we define an \mdf{ideal fluid} to be what that is incompressible ($\grad\cdot\mv{u}=0$) and inviscid ($\nu=0$).

\subsection{The Vorticity Form of the Navier-Stokes Equations}
Recall that we found the velocity-vorticity form of the Navier-Stokes equations:
$$\pd{\mv{u}}{t}+\omega\times\mv{u}=-\grad B+\nu\grad^2\mv{u}\quad\text{where}\quad B=\frac{p}{\rho}+\frac{u^2}{2}+gz$$
If we take the curl of this equation, assuming $\rho=\text{const}$ and noting $\grad\times(\grad B)=0$ we have the following vorticity equation:
$$\pd{\omega}{t}+\grad\times(\omega\times\mv{u})=\nu\grad^2\omega$$
For incompressible flows we have the following vector identity
$$\grad\times(\omega\times\mv{u})=(\mv{u}\cdot\grad)\omega-(\omega\cdot\grad)\mv{u}$$
which yields the following formula.
\begin{formula}
	\textbf{Vorticity form of the flow equation}
	$$\md{\omega}\equiv\pd{\omega}{t}+\underbrace{(\mv{u}\cdot\grad)\omega}_{\text{Advection}}=\underbrace{(\omega\cdot\grad)\mv{u}}_{\text{Stretching}}+\underbrace{\nu\grad^2\omega}_{\text{Diffusion}}$$
\end{formula}
\subsection{Vorticity Invariants in 2D flows}
\begin{eg}
Vorticity conservation!
\end{eg}
In fact, given any function $f(\omega)$ the following quantity will be conserved over a \emph{fixed} volume $V$ (finite or infinite)
$$\int_v f(\omega)\;dV$$
provided that this integral converges and the conditions at the boundary of V are appropriate. We can show this by using the above example ($\md{\omega}$=0).
$$\md{f}=\pd{f}{t}+u_j\pd{f}{x_j}=\pd{f}{\omega}\underbrace{\left(\pd{\omega}{t}+u_j\pd{\omega}{x_j}\right)}_{=D_t\omega}=0$$
$$0=\int_V\left(\pd{f}{t}+u_j\pd{f}{x_j}\right)dV$$
so that (using integration by parts)
$$0=\pd{}{t}\int_V f\;dV - \int_V f\pd{u_j}{x_j}dV+\int_{\partial V} fu_jn_j\;dA$$
where the middle term is $0$ by incompressibility and the last term is $0$ if
\begin{itemize}
	\item The volume $V$ is impermeable (e.g. solid walls) so $u_jn_j=0$
	\item $f=0$
	\item $fu_jn_j$ decays fast enough for an infinite volume
\end{itemize}
In any of this cases we have
$$\pd{}{t}\int_V f\;dV=0$$
so that the quantity $\int_V f(\omega)\;dV$ is conserved over time.

In particular we can choose $f(\omega)=\abs{f}^n$ for natural $n$ in which case we get the \mdf{enstrophy series of invariants}:
$$I_n\defeq\int_V \abs{\omega}^n\;dV$$
The most important of these, in particular for 2D turbulence is $n=2$ which is called the \mdf{enstropy}
$$Z\defeq I_2 = \int_V \omega^2\;dV$$

\subsection{Circulation}
We define the \mdf{velocity circulation} $\Gamma$ over an evolving closed contour $C=C(t)$ via the following contour integral
$$\gamma=\oint_{C(t)}\mv{u}\cdot\dl$$
where $\dl$ is an infinitesimal segment of the contour $C$.
By Stokes' theorem we can rewrite this a surface integral of the vorticity over the surface $A=A(t)$ spanned by the contour C,
$$\Gamma=\oint_{C(t)} \mv{u}\cdot\dl=\int_{A(t)}(\grad\times\mv{u})\cdot\mv{n}\;dA=\int_{A(t)}\omega\cdot\mv{n}\;dA$$
where $\mv{n}$ is the normal to $A$.
\subsection{Motion of a Material Line Element}
For the next statement we need to know how to describe the motion of line elements that `follow the fluid'. We consider a \emph{material} contour $C$ that `follows the fluid', so that the points of $C$ move together with the fluid particles.

We consider the evolution of an infinitesimal segment of the contour
\[
	d\mv{l}= \mv{x}_2 - \mv{x}_1
\]
Both ends move with the flow, i.e. $D_t{\mv{x}_i}(t)= \mv{u}(\mv{x}_i(t), t)$ because the $\mv{x}_i$ are a unction of $t$ only and hence $D_t=\partial_t$.
So we can write
\[
	\frac{D(d\mv{l})}{Dt}= \frac{D}{Dt}(\mv{x}_2 - \mv{x}_1) = \mv{u}(\mv{x}_2(t), t) - \mv{u}(\mv{x}_1(t), t) = (d\mv{l} \cdot\grad)\mv{u}
\]
where we have only keep the leading term in the Taylor expansion of $d\mv{l}$.

\subsection{Kelvin's Circulation Theorem}
\begin{theorem}
In an ideal fluid flow (inviscid and incompressible), where all external forces are conservative (e.g. gravity), the circulation $\Gamma$ over any material contour $C=C(t)$ is conserved, i.e.
\[
	\frac{D\Gamma}{Dt}=0
\]
\end{theorem}

\begin{proof}
If we differentiate the circulation we get
\[
	\md{\Gamma}=\oint_{C(t)}\md{\mv{u}} \cdot d\mv{l} + \oint_{C(t)} \mv{u} \cdot \md{(d\mv{l})}
\]
Then in an ideal flow we have the Euler equation
\[
	\md{\mv{u}}=\grad\left( \frac{-p}{\rho}-gz\right)
\]
and using the expression for $D_t(d\mv{l})$ we just figured out we get
\begin{align*}
	\md{\Gamma} & = \oint_{C(t)}\left[ \grad\left( \frac{-p}{\rho}-gz\right)\right]\cdot d\mv{l} + \oint_{C(t)} \mv{u} \cdot \left[ (d\mv{l}\cdot\cdot)\mv{u}\right] \\
				& = \oint_{C(t)} \left[ \grad \left( \frac{-p}{\rho}-gz - \frac{u^2}{2}\right)\right]\cdot d\mv{l} \\
				& = \left[ \frac{-p}{\rho}-gz+\frac{u^2}{2}\right]_{C(t)}
\end{align*}
The expression in the brackets is the increment in the internal value after going once round $C$.
This expression is $0$ because the function is single-valued.
\end{proof}

\begin{note}
	\begin{enumerate}
		\item $C$ must be a material contour. The theorem would not work for a fixed contour.
		\item Kelvin's Theorem also works for certain classes of compressible flows.
		\item Kelvin's Theorem does not require $C$ to be simply connected.
		\item The inviscid equations were only used on $C$.
			Viscosity can happen in the flow so long as it does not take place on $C$.
	\end{enumerate}
\end{note}

\begin{eg}
	\textbf{Starting vortices}
\end{eg}

\subsection{Cauchy-Lagrange Theorem}
\begin{theorem}
In an ideal flow, in the presence of conservative forces, if a portion of fluid is initially in irrotational motion then it will remain irrotational at any time.
\end{theorem}

\begin{proof}
For contradiction suppose that the flow starts irrotational but at a later time some material element ceases to be irrotational.
At $t_0 >0$ choose a small contour $C(t_0)$ within this material element of fluid such that the surface $A$ enclosed by the contour has a non-zero flux of vorticity through it.
\[
	\int_A \mv{\omega}\cdot\mv{n} \; dA\neq 0
\]
Therefore, the circulation around $C(t_0)$ is non-zero.
Tracing this contour back to $t=0$ we see that the circulation around $C(0)$ is non-zero by Kelvin's Theorem
Therefore the flow cannot be irrotational at $t=0$.
\end{proof}

\subsection{Vortex Lines, Surfaces and Tubes}
\begin{defin}
	A \mdf{vortex line} is a field line of vorticity. That is a curve $\mv{x}(s, t)$ to which $\mv{\omega}=(\omega_x, \omega_y, \omega_z)$ is tangential at every point
	\[
		\frac{\pd{x}{s}}{\omega_x}=\frac{\pd{y}{s}}{\omega_y}=\frac{\pd{z}{s}}{\omega_z}
	\]
	Here $s$ is the distance measured along the vortex line and derivatives are taken at a fixed time.
	Vortex lines are to the vorticity as streamlines are to the vorticity.

	A \mdf{vortex surface} is formed by a set of vortex lines passing through some curve which is not a vortex line.
	Note that the vorticity is tangent to the vortex surface at every point and a vortex line is the intersection of two vortex surfaces.

	Finally, a \mdf{vortex tube} is a vortex surface formed by a closed curve.
\end{defin}

\subsection{Helmholtz Theorems}
In an ideal flow in the presence of conservative forces we have the following theorems:

\begin{theorem}
\begin{enumerate}
	\item Vortex lines cannot start or end in the fluid. 
		They either form a closed path, end at a boundary or go to infinity.
	\item Fluid elements on a vortex line will remain on a vortex line (i.e. vortex lines follow the fluid).
	\item Circulation is the same for any cross-section of a vortex tube.
		Therefore the \mdf{strength of a vortex tube}, defined by its circulation about any cross-section, is constant in time.
\end{enumerate}
\end{theorem}

\begin{eg}
	\textbf{Tornadoes}

Consider a tornado.
One can think of a tornado as a vortex tube with vorticity $\omega$ constant across cross sections $\delta A$.
Then the circulation (the strength of the vortex tube) is fixed by Kelvin's Theorem at
\[
\Gamma = \omega \delta A
\]
\textbf{Question: }What happens to the vorticity as it travels over a hump in the train.

\textbf{Answer: }The volume of the tube is approximately $V= h \delta A$ where $h$ is its height.
$V$ remains the same as it travels over the hump and no fluid can enter or leave the tube.
Therefore the reduction in heigh leads to an increase in cross-sectional area $\delta A$.
The circulation is unchanged so $\omega$ must decrease.

The vortex surface forming the boundary of the tube moves with the fluid.
Hence when there is a velocity shear at the top of the tornado, the fluid inside the tornado is stretched.
It cross-section must decrease because the fluid is incompressible so the vorticity gets magnified.
\end{eg}

\subsection{Vortex Rings}
Consider a joined up vortex tube which is axisymmetric.
That is, in polar coordinates, the velocity has no dependence on $\theta$:
\[
	\mv{u}=u_r(r, z, t)\mv{e}_r + u_z(r, z, t)\mv{e}_z
\]
Then the vorticity is $\mv{\omega}=\omega \mv{e}_\theta$ where
\[
	\omega = \pd{u_r}{z}-\pd{u_z}{r}
\]
The tubes are ring-shaped and must move with the fluid.

If the fluid is incompressible then the volume inside a thin vortex tube of radius $r$ and cross-section $\delta A$ will be unchanged at $2\pi r \delta A$.
Helmholtz tells us that $\omega \delta A$ is constant.
Dividing these two suggests that $\omega / r$ is constant.
More precisely
\[
	\frac{D}{Dt}\left( \frac{\omega}{r}\right)=0
\]
and so the vorticity of a fluid element changes in proportion to $r$.

\section{2D Flows}
In this section we focus our attention to flows of the form
\[
	\mv{u} = (u (x, y, t), v(x, y, t), 0)
\]
in which the vorticity component looks like
\[
	\mv{\omega}=(0, 0, \omega(x, y, t))
\]
The vorticity equation then becomes
\[
	\md{\omega} = \left( \pd{}{t}+ (\mv{u} \cdot \grad) \right)\omega = \nu \grad^2 \omega
\]

\subsection{The Stream Function}
In 2D incompressible flows, we can represent the velocity field in terms of the \mdf{stream function} as
\[
	\mv{u}=\grad\psi \times \mv{e}_z
\]
or in component form
\[
	u=\pd{\psi}{y}, \quad \quad v = - \pd{\psi}{x}
\]
which automatically satisfies the condition of incompressibility.
Then we can find the vorticity by
\[
	\omega = \pd{v}{x} - \pd{u}{y} = - \grad^2 \psi
\]

\subsection{Streamlines}
We can see from the component form that
\[
	\mv{u}\cdot \grad \psi = 0
\]
So in steady flows where $\partial_t\psi =0$ we can see
\[
	\mv{u}\cdot\grad\psi=D_t\psi = 0
\]
Therefore $\psi$ is conserved along fluid paths.
So in steady flows the fluid particles move along lines of constant $\psi$ so the streamlines are $\psi=c$ for some $c$.

\subsection{Fluid Flux}
The \mdf{volume flux of fluid}, per unit length (since we're in 2D) across a curve between points $a$ and $b$ is
\[
	Q = \int_a^b \mv{u}\cdot\mv{n} ds
\]
where $s$ is the arc length along a curve between $a$ and $b$ and $\mv{n}=(\partial_s y, -\partial_s x)$ is the normal to the curve.
Substituting in the stream function we get
\[
	Q = \int_a^b \left( \pd{\psi}{y}\pd{y}{s} + \pd{\psi}{x}+ \pd{x}{s}\right)ds = \int_a^v \frac{d\psi}{ds}ds = [\psi]_a^b = \psi(b) - \psi(a)
\]
So the difference of the stream function at the two end points tells us the value of $Q$.

\subsection{Plane-Parallel Shear Flow}
This is a flow of the form
\[
	\mv{u}(\mv{x}, t) = ( u(y, t), 0, 0)
\]
Then the stream function is
\[
	\psi(y, t) = \int_{-h}^y u(y' , t) \;dy'
\]
where we assume that there is a boundary at $y=-h$.
Incompressibility is automatically satisfied and the $(\mv{u}\cdot\grad)\mv{u}$ term is automatically $0$.
Writing NS in component form we get

\begin{align*}
	\pd{u}{t}&= - \frac{1}{\rho}\pd{p}{x} + \nu \pd{^2 u}{y^2} \\
	0 &= -\frac{1}{\rho}\pd{p}{y}
\end{align*}

The second equation tells us that pressure is independent of $y$ and so $p$ must have the form $p=p(x, t)$.
In the first equation we see that the dependence on $x$ is linear. So we can write
\[
	\frac{-\grad p}{\rho}= f(t) \mv{e}_x
\]
which leads us to the equation
\begin{formula}
	\[
		\pd{u}{t} = f + \nu \pd{^2 u}{y^2}
	\]
\end{formula}

\subsubsection{Inviscid and Steady Plane-Parallel Flow}
Consider the same flow with the additional assumptions that $\pd{\mv{u}}{t}=0$ and $\nu=0$. 
Then we get
\[
	0 = - \frac{1}{\rho}\pd{p}{x}, \quad \quad 0 = -\frac{1}{p}\pd{p}{y}
\]
The equations are therefore satisfied by constant pressure and $\mv{u}=(u(y), 0, 0)$.

\subsubsection{Couette Flow (surface driven)}
Consider a steady but viscous plane shear flow
\[
	\mv{u} = (u(y), 0 ,0)
\]
between two infinite plates at $y=h$ and $y=-h$ with are moving in the $x$-direction at speeds $U$ and $-U$.
Assume that pressure in the flow is uniform in space.
What is the velocity profile?

Since we are assuming that pressure is constant our equation reduces to
\[
	\partial_{yy}u(y)=0
\]
which is solved by
\[
	u(y) = Ay + B
\]
The boundary conditions tell us $B=0$ and $A=\frac{U}{h}$ and hence
\[
	u(y) = \frac{U}{h}y
\]
Using this we can calculate the stream function
\[
	\psi(y) = \int_{-h}^\psi \frac{Us}{h}ds = \frac{U}{2h}(y^2 - h^2)
\]
so the volume flux at a channel cross section is $Q = \psi(h) - \psi(-h) = 0$, as one might expect.

We can calculate the friction at the bottom plate using the formula
\begin{align*}
	D & = \nu\rho\cdot\abs{\pd{u}{y}}_{\partial V} \\
	  & = \nu\rho\cdot\abs{\pd{u}{y}}_{y=-h} = \frac{U\nu\rho}{h}=\frac{\mu U}{H}
\end{align*}

\subsubsection{Poiseuille Flow (pressure driven)}
We now consider the opposite situation where we still have steady viscous plane parallel flow but the boundaries are stationary and the flow is driven by a pressure fradient.
\[
	\mv{u}=(u(y), 0, 0)
\]
with infinite, fixed plates at $y=\pm h$.
The flow is driven by a uniform pressure gradient in the $x$-direction
\[
	-\frac{\grad p}{\rho}=f\mv{e}_x
\]
where $f$ is some constant.
Solving the usual equation yields a quadratic in $y$
\[
	u(y)= \frac{f}{2\nu}(h^2 = y^2)
\]
and the drag at the top and bottom plates is given by $D=\rho f h$.
One can show that the stream function is cubic and the volume flux is proportional to the cube of the channel radius $h$.

\subsection{Round vortices}
In a similar vein we consider flow whose vorticity distributions depend only on the radial coordinate and not on angle.
Consider a vorticity field for a steady inviscid 2D flow $\mv{\omega}= (0, 0, \omega)$ so that, in polar coordinates, $\omega=\omega(r)$.
From the incompressibility condition with $\mv{u}=\mv{u}(r)=u_r(r)\mv{e}_r + u_\theta(r)\mv{e}_\theta$ we get
\[
	\grad\cdot\mv{u} = \frac{1}{r}\pd{(ru_r)}{r}+\frac{1}{r}\pd{u_\theta}{\theta}=0
\]
Which leaves us with either $u_r=A/r$ or $u_r\equiv 0$.
We see that it must be the latter because the former would imply a mass source at $r=0$.
So we can conclude
\[
	\mv{u} = u_\theta(r) \mv{e}_\theta
\]
and hence the velocity and vorticity gradient are perpendicular.
So $\mv{u}\cdot\grad\omega = 0$ and hence $D_t\omega = 0$ is satisfied.
Such a vorticity distribution must therefore be a steady-state solution although this solution is not necessarily a realisable solution.

To find the stream function we have to use the cylindrical coordinate equivalent
\begin{formula}
	\[
		u_\theta(r, t) = - \pd{\psi(r, t)}{r}
	\]
	which gives us that
	\[
		\psi(r, t) = - \int_0^r u_\theta(r', t) \;dr'
	\]
\end{formula}
To get a relation between velocity and vorticity we us Stokes' Theorem.
Consider a contour $C$, a circle of radius $r$ centred at the origin.
\[
	\Gamma = \oint_c \mv{u}\cdot d\mv{l} \equiv 2\pi r u_\theta (r) = \int(A) \mv{\omega}\cdot\mv{} \; dA \equiv 2\pi \int_0^r \omega r'\;dr'
\]
which means
\[
	u_\theta(r) = \frac{1}{r}\int_0^r \omega(r') r' \; dr'
\]
\subsubsection{Solid Body Rotation}
We consider the velocity field associated to
\[
	u_\theta(r) = \Omega r
\]
The fluid moves as a rigid body so there must be local rotation.
This means the vorticity is non-zero.
\[
	\omega = \frac{1}{r}\pd{(r u_\theta)}{r}-\frac{1}{r}\pd{u_r}{\theta} = 2\Omega
\]
So fluid particles on the circular streamlines rotate to maintain a rigid body.
We can calculate circulation at a given radius $R$ to be
\[
	\Gamma = 2\pi R u\theta(R) = 2\pi \Omega R^2
\]
Pressure will be purely radius due to the axisymmetric and can be calculated from the radial steady-state Euler equations which reduce to
\[
	-\frac{u_\theta^2}{r}=-\frac{1}{\rho}\pd{p}{r} \implies p(r) = \frac{\rho\Omega^2 r^2}{2} +C
\]
for some constant $C$.

\subsubsection{Point Vortices}
A point vertex is an idealisation which we will often use.
We assume that all of the vorticity is concentrated at a point.
The velocity field is
\[
	u_\theta(r) = \frac{a}{r}
\]
Outside the singularity at $r=0$ we see that the flow is irrotational so there is no local rotation away from the origin.

Then the circulation around a circular contour of radius $R$ will be
\[
	\Gamma = 2\pi R u_\theta(R) = 2\pi a
\]
and is thus independent of radius.
So we commonly represent a point vortex as $u_\theta(r) = \frac{\Gamma}{2\pi r}$.
All of the vorticity is concentrated at the origin so $\omega = \Gamma \delta(\mv{x})$.

Finally, the pressure distribution satisfies the $r$-component of the steady state Euler equation

\[
	\pd{p}{r}= \frac{\rho u_\theta^2}{r} = \frac{\rho a^2}{r^3} \implies p = p_\infty - \frac{\rho a^2}{2r^2}	
\]

\section{Irrotational 2D Flow and Aerofoils}
\subsection{Velocity Potential for Irrotational Flow}
If we have irrotational flow then we get a velocity potential satisfying $\mv{u}=\grad\phi$ and the viscous term is then automatically so as incompressibility says $\grad^2\phi = 0$.
From $\mv{u}= \grad \phi$ we get
\[
	\phi(\mv{x}) = \int_0^{\mv{x}} \mv{u}(\mv{x}') \cdot d\mv{l}
\]
where the integral is taken over some path joining 0 and $\mv{x}$.
If the irrotational flow is living in a simply connected domain then this integral is independent of path (e.g. use Stokes theorem).
If the domain is multiply connected then $\phi$ may be multivalued.
That is $\mv{\omega}\equiv 0$ does not mean that all contours have $0$ circulation.
In this case
\[
	\Gamma = \oint_C \mv{u}(\mv{x}')\cdot d\mv{l} = \oint_C \grad\phi(\mv{x}')\cdot d\mv{l} = \left[ \phi\right]_C
\]
i.e. we see how $\phi$ increments going once around $C$.

\begin{note}
In an irrotational flow then we can compute velocity components by
\[
	u = \pd{\phi}{x}\quad\quad v = \pd{\phi}{y}
\]
\end{note}

\begin{eg}
	\textbf{Velocity Potentials: }
	\begin{itemize}
		\item Uniform flow at an angle of $\alpha$ to the $x$-axis has velocity potential $\phi = U(x\cos\alpha + y \sin\alpha)$.
		\item Pure strain flow has velocity potential $\phi = \frac{\alpha}{2}(x^2 - y^2)$.
		\item Point vortex has the multivalued function $\phi = \frac{\Gamma}{2 \pi}\theta$ where $\theta$ is the polar angle.
	\end{itemize}
\end{eg}

\subsection{The Complex Potential}
Recall the stream function had the properties
\[
	u = \pd{\psi}{y} \quad\quad v = - \pd{\psi}{x}
\]
and hence combining this with relations for the velocity potential we arrive at the Cauchy-Riemann equations
\[
	\pd{\phi}{x}=\pd{\psi}{y}\quad\quad\pd{\phi}{y}=-\pd{\psi}{x}
\]
which then tells us that the \mdf{complex potential} $\mdf{w}=\phi + i \psi$ is in fact holomorphic.
\begin{note}
Given the complex potential $w$ we can then recover the flow velocity profile by
\[
	\frac{dw}{dz}(z) = w'(z) = u(x, y) - i v(x, y)\quad\quad\text{where }z=x+iy
\]
Moreover, we can compute the speed by $\abs{\mv{u}} = \abs{\partial_z w}$.
\end{note}

\begin{eg}
	\textbf{Example complex potentials are}
	\begin{itemize}
		\item Pure strain flow $w = \frac{\alpha}{2}z^2$.
		\item The point vortex has complex potential $w=-\frac{i\Gamma}{2\pi}\log z$.
	\end{itemize}
\end{eg}

Note that any holomorphic function $w$ in fact corresponds to an \emph{ideal} 2D irrotational flow.
Once $u$ and $v$ have been found from the complex potential, the pressure can be obtained from the time-dependent Bernoulli equation
\[
	\pd{\phi}{t} + B = C(t) \quad\text{where}\quad B = \frac{p}{\rho}+\frac{\abs{\grad\phi}^2}{2}+gz	
\]

\subsection{Flow Past a Cylinder}
Consider the ideal flow with the complex potential
\[
	w= U\left( z + \frac{a^2}{z}\right)
\]
where $U$ and $a$ are positive constant.
Note that as $\abs{z}\to\infty$ the first term becomes dominant an we obtain uniform flow of velocity $U$ in the $x$-direction.
One can show that on the circle of radius $a$ the stream function is identically $0$ and hence certain constant.
So the circle is a streamline and the free-slip boundary condition is satisfied.
Therefore, the potential describes a flow past a cylindrical boundary which is uniform at a distance from the cylinder.

\subsubsection{Non-Uniqueness of the Solution}
The problem with this is that the free-slip boundary condition on the cylinder is not sufficient to give a unique solution.
One can add a point vortex with which is irrotational away from origin with arbitrary circulation $\Gamma$.
Note that this is not a real vortex because it is inside the solid
\[
	w = U \left( z + \frac{a^2}{z}\right)-\frac{i\Gamma}{2\pi}\log(z)
\]
Note that the first term is still dominant as $\abs{z}\to\infty$.
Moreover, the point vortex does not have a radial velocity component so we still ratify the free-slip condition at radius $a$.

Which $\Gamma$ is correct is a tough question to answer.
One could guess it deepens on the rotation speed of the solid cylinder.
However, this may only matter when we consider a thin viscous boundary layer and use the non-slip boundary condition.
Based on an ideal flow, there is no way to fix this.

\subsubsection{Velocity and Force}
We can take the imaginary part of the complex potential to find the stream function and then in polar coordinate we have
\begin{align*}
	u_r = \frac{1}{r} \pd{\psi}{\theta} &= U \left( 1- \frac{a^2}{r^2}\right)\cos\theta \\
	u_\theta = - \pd{\psi}{r} &= -U \left( 1 + \frac{a^2}{r^2}\right)\sin\theta + \frac{\Gamma}{2\pi r}
	= -U \left[ \left( 1 + \frac{a^2}{r^2}\right)\sin\theta + \frac{Ba}{r}\right]
\end{align*}
where $B = \frac{-\Gamma}{2\pi U a}$.

Then denoting by $p_0$ the pressure at infinity we can find pressure $p$ on the circle at $\abs{z} = a$ by using Bernoulli:
\[
	p= p_0 + \frac{\rho U^2}{2}- \frac{\rho u_\theta^2}{2} = p_0 + \frac{\rho U^2}{2} - \frac{\rho}{2}\left[ 2U \sin\theta - \frac{\Gamma}{2\pi a}\right]^2
\]


\subsection{Basils' Theorem}
\begin{theorem}[Blasius' Theorem]
Consider a steady ideal flow past a 2D body with a boundary given by a closed contour $C$ having complex potential $w(z)$ then
\[
	F_x - i F_y = \frac{i\rho}{2}\oint_C \left( \pd{w}{z}\right)^2 dz
\]
\end{theorem}

\section{Boundary Layer Theory}
Before we have been assuming the fluid is everywhere inviscid but in relating, near to the boundary there is a thin layer of fluid in which viscous forces can no longer be neglected.
This layer satisfies the no-slip boundary condition and is called the \mdf{boundary layer}.
In the outer flow we can ignore the boundary layer because it has an insignificant thickness $\delta << L$.
Inside the boundary layer we have free-slip velocity at the top and no-slip at the bottom.
When the layer becomes thin, the viscous forces become comparable to other terms.

For high $Re$ we assume the boundary layer is thinner than the curvature radius of the surface and is under plane-parallel shear flow.

\subsection{Thickness}
The boundary layer exists when viscous force becomes comparable to inertia's ones
\[
\abs{\nu\grad^2\mv{u}} \sim \abs{\mv{u}\cdot\grad)\mv{u}}
\]
Assuming the boundary layer is plane-parallel with $x$-axis along the solid with characteristic length scales $L$ and $\delta$ to be found.
Viscous terms are
\[
	\nu\partial_{yy}u \sim \nu U / \delta^2
\]
as the largest gradients in $\mv{u}$ are perpendicular to the solid.
The inertial terms are given by $U^2/L$.
Balancing these we see
\[
	\delta \sim \frac{L}{Re^\frac{1}{2}}
\]
where $Re = UL/\nu$ is based on the outer flow scales.

\subsection{Non-Dimensionalisation}
Often we wish to rephrase equations in terms of dimensionless parameters which we can then vary to see how the flow changes.
We will create dimensionless equations for the Navier-Stokes and see which terms dominate as $Re\to\infty$.

Parallel to the solid we have characteristic scales $U$ and $L$.
In the $y$-direction we have strong gradients in velocity across the boundary layer of length $\delta$.
Using incompressibility we see $U/L \sim v/ \delta$ and hence $v \sim U\delta /L$.
So we make the following substitutions into the NS equations.
\[
	x = Lx', \quad y=\delta y=L(Re)^{-\frac{1}{2}}y', \quad u=Uu', \quad v=\frac{U\delta v'}{L}=\frac{Uv'}{Re^{1/2}}, \quad p=\rho U^2 p'
\]
The pressure scale was chosen as $\rho U^2$ because the Bernoulli equations says that $p+\frac{\rho}{2}u^2$ is constant.
We can then make these substitutions and take the limit as $Re\to\infty$ to obtain a limiting system which we can subsequently return to the dimensional equations.
Doing so leads to the boundary layer equations
\begin{formula}
	\textbf{Boundary Layer Equations: }
	\begin{align*}
		u\pd{u}{x} + v\pd{u}{y} &= - \frac{1}{\rho}\pd{p}{x}+\nu\pd[2]{u}{y}\\
		p & \equiv p(x) \quad\quad\quad \text{independent of }y \\
		\pd{u}{x}+ \pd{v}{y}&=0
	\end{align*}
\end{formula}

\section{Waves}
The basic element of wave dynamics is a \mdf{monochromatic wave}
\[
	a(\mv{x}, t) = A \cos\left( \mv{k}\cdot\mv{x} - \omega t + \varphi\right)
\]
where
\begin{itemize}
	\item $A$ is the \mdf{wave amplitude}
	\item $\mv{k}\in \R^d$ is the \mdf{wave vector} (referred to as the \mdf{wave number} when $d=1$).
	\item $\omega$ is the \mdf{angular wave frequency} (measured in radians per second)
	\item $\varphi$ is the \mdf{wave frequency}
\end{itemize}

The \mdf{wavelength} $\lambda$ is the distance between adjacent wave crests and \mdf{period} $T$ is the time it takes to perform one oscillation.
They can be computed by
\[
	\lambda = \frac{2\pi}{\abs{\mv{k}}} \quad \quad \text{and} \quad \quad T= \frac{2\pi}{\omega}
\]
Note that
\[
	\mv{k}\cdot\mv{x} - wt = \mv{k} \cdot \left(\mv{x} - \frac{\omega\mv{k}}{\abs{k}^2}t\right)
	=\mv{k} \cdot (\mv{x} - \mv{c}_{ph} t) \quad \text{where} \quad
	\mdf{\mv{c}_{ph}}\defeq \frac{\omega\mv{k}}{\abs{\mv{k}}^2}
\]
We call $\mv{c}_{ph}$ the \mdf{phase speed} and it is the speed at which the peaks move.

\subsection{The Dispersion Relation}
The frequency is related to $\mv{k}$ by a \mdf{dispersion relation} $\omega=\omega(\mv{k})$.
The form of the dispersion relation will govern the behaviour of the wave.
\begin{itemize}
	\item We consider \mdf{isotropic systems} where $\omega=\omega(\abs{\mv{k}})$.
	\item For isotropic systems if $\omega$ is a linear function of $\mv{k}$ then we say the waves are \mdf{non-dispersive} (e.g. sound waves).
\end{itemize}

\subsection{Wave Packets and Group Velocity}
In reality tend to move in packets whose amplitude decay with the distance from the centre of the packet.
These packs move with \mdf{group velocity}
\[
	\mdf{\mv{c}_g} \defeq \grad_{\mv{k}}\omega \quad\quad\quad
	\mdf{\mv{c}_{ph}}\defeq \frac{\omega\mv{k}}{\abs{\mv{k}}^2,}
\]
whereas individual wave crests move with phase velocity.

\begin{defin}
	In general we see that waves are \mdf{dispersive} if phase and group speed are not equal.
\end{defin}

\subsection{Free surface flows}
In general a free surface between two viscous fluids gives us 4 boundary conditions for the 4 unknowns (3 velocities and the surface shape).
To obtain these we consider a free surface given in the form $f(\mv{x}, t)=0$ with $f<0$ and $f>0$ referring to fluid 1 and 2 respectively.
The surface's unit normal from 1 to 2 will be
\[
	\mv{n} = \frac{\grad{f}}{\abs{\grad{f}}}
\]

\noindent\textbf{Kinematic Boundary Condition}

The free surface moves with the velocity of the fluid and so particles on the free surface remain there, so we can conclude

\[
	\md{f}{t} =0
\]
Introducing $\mv{n}$ as define above we get that
\[
	\frac{1}{\abs{\grad{f}}}\pd{f}{t}+ \mv{u}\cdot \mv{n} = 0,
\]
For steady problems this reduces to impermeability $\mv{u}\cdot\mv{n}=0$.

\noindent\textbf{Dynamic Boundary Conditions}

To get this condition we apply Newton's 2nd law to the free surface.
There is no mass to the surface and hence the forces acting on either side must balance with forces in the interface.
We can then see that the stress tensors from above and below must balance as
\[
	\mv{T}^+ \cdot \mv{n} = \mv{T}^- \cdot \mv{n} \quad i.e. \quad T^+_{ij}n_j = T^-_{ij}n_j
\]
on the surface $f=0$.
Recall that we have
\begin{itemize}
	\item \textbf{Inviscid Flow: } $T_{ij}=-p\delta_{ij}$ which then implies continuity of pressure $p^+=p^-$ at the boundary $f=0$.
	\item \textbf{Viscous Flow: } $T_{ij}= -p\delta_{ij} + 2\pi e_{ij}
		= -p\delta_{ij} + \mu \left( \partial_{x_j} u_i + \partial_{x_i} u_j \right)$
\end{itemize}

\subsection{Linearising Boundary Conditions}
We often linearise these boundary conditions which take place on a surface $y=h(x, t)$ to obtain an analytically tractable problem.
This is done by simply remove any terms above first order in our equations.
However, we must also make sure that we apply these new equations at $y=0$ because for small $h$ we can Taylor expand
\[
	\phi(x, h, t) = \phi(x, 0 , t) + h \pd{\phi(x, 0 , t)}{y}+ \dots \approx \phi(x, 0, t)
\]

\subsection{Surface Tensions and Capillary Flows}
As we pass to smaller length scales, surface forces dominate volume forces.
In particular, surface tension dominates gravity.

The surface tension force $\sigma$ (constant here), per unit length of line, is directed tangentially along the surface.
Suppose the surface is parametrised with coordinates $(s_1, s_2)$ and has tangent vectors $(\mv{t}_1, \mv{t}_2)$.
A balance of forces on the free surface $f=0$ gives
\[
	(\mv{T}^+ - \mv{T}^-) \cdot \mv{n} = - \sigma \left( \pd{\mv{t}_1}{s_1}+\pd{\mv{t_2}}{s_2}\right)\equiv \sigma\kappa \mv{n}
\]
where we have defined curvature of the interface $\kappa$ in an appropriate way.
In terms of the principle radii of curvature
\[
	\kappa=\frac{1}{R_1} + \frac{1}{R_2}
\]
where $R-i$ is positive when the circle of radius $R-i$ lies within the $(-)$ phase.
Recall the $(-)$ phase is where $f\defeq y - h(x, t) < 0$.

As a boundary condition, we see stresses tangential to the interface are continuous across the surface, e.g. $\mv{t}_1\cdot\mv{T}^+ \cdot \mv{n} = \mv{t}_1 \cdot \mv{T}^- \cdot \mv{n}$.
In the inviscid or static viscous case we get the \mdf{Young-Laplace equation}
\[
	p^- - p^+ = \sigma \grad\cdot\mv{n} \quad \text{at} \quad f=0
\]
which says that the pressure jump across the interface is caused by the surface tension force.

\begin{note}
Surface tension is of the order $\sigma / L$ where $L$ see approximately the radius of curvature.
So capillary force will be important when $\sigma / L$ is comparable to pressure due to gravitational forces, i.e.
\[
	\mdf{Bo} = \frac{\rho g L^2}{\sigma}=1
\]
We call $Bo$ the \mdf{Bond number} and $L_\sigma \defeq \sqrt{\sigma/(\rho g)}$ is the \mdf{capillary length}.
\end{note}

\section{Instabilities}
\begin{theorem}[Rayleigh Criterion]
For an ideal plane-parallel flow between two plates (with free-slip boundary condition) to be linearly unstable, the velocity profile $U(y)$ must have an inflection point.
\end{theorem}

\section{Good Numbers to know!}
\begin{itemize}
	\item The characteristic length in a gas is mean free path $l_{gas}=100nm$.
	\item The characteristic length in a liquid is interatomic spacing $l_{liquid} = 1nm$.
	\item Density of water is about $1000kg\;m^{-3}$.
	\item The dynamic and kinematic viscosity of water at 20C is
		\[
			\mu = 10^{-3} \quad\quad \nu = 10^{-6}
		\]
	\item Surface tension between water and air is $0.07Nm^{-1}$ and $L_\sigma = 3mm$.
\end{itemize}

\end{document}
