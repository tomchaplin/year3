\documentclass[11pt]{article}

%{{{ Document metadata
\title{Fluid Dynamics Notes}
\author{Thomas Chaplin}
\date{}
\setlength{\parindent}{0pt}
%\vfuzz=30pt
%\hfuzz=30pt
%}}}
%{{{ Packages
\usepackage[margin=1in]{geometry}
\usepackage{enumitem}
\usepackage{amsfonts}
\usepackage{amssymb}
\usepackage{amsmath}
\usepackage{amsthm}
\usepackage{mathdots}
\usepackage[dvipsnames]{xcolor}
\usepackage[framemethod=TikZ]{mdframed}
\usepackage{microtype}
%}}}
%{{{ Custom commands
% Partial Derivative command
\newcommand*{\pd}[3][]{\ensuremath{\frac{\partial^{#1} {#2}}{\partial {#3}^{#1}}}}
% Material Derivative command
\newcommand*{\md}[1]{\ensuremath{\frac{D #1}{D t}}}
% Gradient
\newcommand{\grad}{\bigtriangledown}
% Define to equal
\newcommand{\defeq}{:=}
\newcommand{\eqdef}{=:}
% Tensor underline
\newcommand{\tul}[1]{\underline{\underline{#1}}}

% Quick column vector command
\newcount\colveccount
\newcommand*\colvec[1]{
        \global\colveccount#1
        \begin{pmatrix}
        \colvecnext
}
\def\colvecnext#1{
        #1
        \global\advance\colveccount-1
        \ifnum\colveccount>0
                \\
                \expandafter\colvecnext
        \else
                \end{pmatrix}
        \fi
}

% Custom formatting commands
\newcommand{\mv}[1]{\textbf{#1}}
\newcommand{\mdf}[1]{{\color{red}#1}}

% Norms
\newcommand{\abs}[1]{|#1|}

%}}}
%{{{ Custom environments
% Proof environments
\renewenvironment{proof}{{\bfseries Proof}}{\qed}
\newtheorem{theorem}{Theorem}[section]
\newtheorem{axiom}{Axiom}[section]
\newtheorem{prop}{Proposition}[section]
\newtheorem{corollary}{Corollary}[section]
\newtheorem{lemma}{Lemma}[prop]
\newtheorem{defin}{Definition}[section]

\definecolor{mylg}{rgb}{0.9,0.9,0.9}
\newenvironment{eg}
    {\begin{mdframed}[backgroundcolor=mylg, roundcorner=5pt, linewidth=0pt]}
    {\end{mdframed}}
    
\newenvironment{note}
    {\textbf{Note:}\begin{mdframed}[backgroundcolor=white, roundcorner=5pt, linewidth=0pt]}
    {\end{mdframed}}

\newenvironment{formula}
	{\begin{mdframed}[backgroundcolor=white, roundcorner=5pt, linewidth=1pt, linecolor=red]}
	{\end{mdframed}}
%}}}
\begin{document}
\maketitle
\section{Mathematical Modelling of Fluid Flow}
\subsection{Validity of continuum mechanics}
Continuum mechanics is valid when
$$\frac{l}{L}<<1$$
where $l$ and $L$ are length scales characterising molecular motion and flow dimension respectively.
\begin{eg}
\textbf{Typical examples:}
\begin{itemize}
    \item $l_{gas}\sim 100nm=10^{-7}m$ is the mean free path of a typical gas
    \item $l_{liquid}\sim 1nm=10^{-9}m$ is the typical distance between molecules.
\end{itemize}
\end{eg}

\subsection{Lagrangian v.s. Eulerian Description}
A fluid will consist of a number of material points/fluid elements/fluid particles/fluid volumes.
In order to uniquely identify each fluid element we can tag it according to its Eulerian coordinates at some initial condition
$$x_j(t-t_0)=X_j=(X_1,X_2,X_3)$$
Alternatively we allow the coordinates to move and deform with time to get the $X_j$ Largrangian coordinates which follow the fluid.
This allows us to derive useful equation but these equations are usually simpler when transformed back to Eulerian coordinates.

\subsection{Flow Visuation}
\subsubsection{Particle Paths}
Particle paths are intuitively the path a particle will follow if dropped into a flow and depends only on initial conditions:
$$\pd{x_i^p}{t}=u_i \quad\quad\quad \text{for fixed} \quad x_i^p(t=0)=X_i$$
We integrate these equations and eliminate $t$ to get particle paths.
\begin{eg}
To calculate particle paths for the flow $\mv{u}(t)=(u_0,kt,0)$, we integrate to get
$$\mv{x}(t)=(u_0t+a,\frac{kt^2}{2}+b,c)\implies y=\frac{k}{2}\left(\frac{x-a}{u_0}\right)^2+b\implies\text{Parabola}$$
\end{eg}
\subsubsection{Stream lines}
Intuitively, we pause time and then draw lines in the vector field. We parameterise these lines in terms of arc length $s$
$$\pd{x_i^p}{s}=u_i \quad\quad\quad \text{for fixed} \quad t$$
We integrate these equations and eliminate $s$ to get stream lines. Note, for a steady flow (i.e $\pd{\mv{u}}{t}=0$), this is the same as the particle paths. 
\begin{eg}
For the same flow $\mv{u}(t)=(u_0,kt,0)$, suppose at $s=0$ we have $(x,y,z)=(a,b,c)$ then
$$\mv{x}(t)=(u_0s+a,kts+b,c)\implies y=kt\left(\frac{x-a}{u_0}\right)+b\implies\text{Straight lines}$$
\end{eg}

\subsection{Material Derivative}
The \mdf{material derivative} is the rate of change of something which follows the flow. To figure out what this is we consider the derivative of a function dependant on time and the particle paths $f(\mv{x}^p(t),t)$. Then
\begin{align*}
    \pd{f}{t}&=\pd{f}{t}+\sum_i \pd{f}{x_i}\pd{x_i}{t}\\
             &=\pd{f}{t}+\sum_i u_i\pd{f}{x_i}\\
             &=\left(\pd{ }{t} + \sum_i u_i\pd{ }{x_i}\right)f
\end{align*}
We then define the \mdf{material derivative} to be
$$\md{ }\defeq \pd{ }{t}+\sum_i u_i\pd{ }{x_i}=\pd{ }{t}+\mv{u}\cdot\grad$$
The first time is the local rate of change at a fixed Eulerian position whereas the second term represents the convective rate of change caused by driving fluid elements through gradients of $f$.
\begin{eg}
For a concentration of pollutant c=c(x) in a river with steady flow $\mv{u}=(u_0,0,0)$, how does the concetration change in a fluid element that follows the fluid?
\begin{align*}
    \md{c}&=\pd{c}{t}+(\mv{u}\cdot\grad)c\\
          &=u_0\pd{c}{x}+0\pd{c}{y}+0\pd{c}{z}\\
          &=u_0\pd{c}{x}
\end{align*}
\end{eg}
\begin{note}
\begin{itemize}
    \item $\md{f}\equiv 0 \implies$ $f$ is constant in fluid elements but, in general, has different values in different elements.
    \item Parameterising streamlines by $s$, let $\mv{e}_s$ be the unit tangent vector at $s$. Then
    $$(\mv{u}\cdot\grad) f= |u|\mv{e}_s\cdot\grad f=|u|\pd{f}{s}$$
    Hence $(u\cdot\grad)f = 0\implies$ $f$ constant on streamlines
\end{itemize}
\end{note}
We can now define \mdf{acceleation} as
$$\mv{a}=\md{\mv{u}}=\pd{\mv{u}}{t}+(\mv{u}\cdot\grad)\mv{u}.$$
We also say a flow is \mdf{steady} if $\pd{\mv{u}}{t}=0$ (normal partial derivaive).
\begin{eg}
Consider the 1D flow $\mv{u}=(u_1(x),0,0)$, noticing we have steady flow but
\begin{align*}
    a_1&=\pd{u_1}{t}+\sum_j u_j\pd{u_1}{x_j}\\
       &= 0 + u_1 \pd{u_1}{x} = u_1 \pd{u_1}{x}
\end{align*}
individual fluid elements still accelerate. To visualise this consider a flow in a pipe which constricts.
\end{eg}

\subsection{Vorticity and Rate of Strain}
We want to understand how flow deforms fluid elements which will show us that rate of deformations is what generates stress in a fluid.

To do this we Taylor expand velocity to first order, at two points differing by some $\delta_{x_j}:$
\begin{align*}
    u_i(x_j+\delta_{x_j},t)&=u_i(x_j,t)+\sum_j \pd{u_i}{x_j}\delta{x_j}\\
    &=\underbrace{u_i(x_j,t)+\sum_j r_{ij}\delta{x_j}}_{\text{rigid body}}+\underbrace{\sum_j e_{ij}\delta{x_j}}_{\text{Shearing + Extension}}\\
    &\eqdef u_i^T + u_i^R + u_i^S
\end{align*}
\begin{align*}
    \text{where} \quad &r_{ij}=-r_{ji}\defeq\frac{1}{2}\left(\pd{u_i}{x_j}-\pd{u_j}{x_i}\right) & \text{is the \mdf{rate of rotation tensor}}\\
    \quad &e_{ij}=e_{ji}\defeq\frac{1}{2}\left(\pd{u_i}{x_j}+\pd{u_j}{x_i}\right) & \text{is the \mdf{rate of strain tensor}}
\end{align*}
This term has three components of interest which we will now explore.
\subsubsection{Translation}
The first term is certainly translation because
$$u_i(x_j+\delta_{x_j},t)=u_i(x_j,t)\implies\text{velocity is locally constant}$$
This induces no internal stress.
\subsubsection{Rotation}
Because the rate of rotation tensor is skew-symmetric we only get 3 non-zero terms:
\[
(r_{ij})=
\begin{bmatrix}
0 & -c & -b\\
c &  0 & -a\\
b &  a &  0
\end{bmatrix}
\]
so we can rewrite the rotation term as
$$u_i^R=\sum_j r_{ij}\delta_{x_j} = \sum_j \sum_k \epsilon_{ijk}\Omega{j}\delta_{x_k}$$
where $\Omega = (r_{32},r_{13},r_{21})=(a,b,c)$.
We then define \mdf{vertocity} to be $\omega\defeq\grad\times\mv{u}=2\Omega$.
We can view this as the rotation term where $\Omega$ is the local rate of rotation. Rotation induces no internal stress.
\subsubsection{Shearing + Extension}
The strain term $u_i^S$ involves the relative motion of fluid particles. They can be split into two types:
\begin{align*}
    \text{Diagonal terms} &\longleftrightarrow \text{Extensional terms}\\
    \text{Off-diagonal terms} &\longleftrightarrow \text{Sharing terms}
\end{align*}
Read written notes for diagrammatic explanation of these terms.
\begin{note}
Denote by $V$ the volume of a fluid element then
\begin{formula}
$$\frac{1}{V}\md{V} = \grad\cdot\mv{u}$$
\end{formula}
Therefore $\md{V}=0\implies\underbrace{\grad\cdot\mv{u}=0}_{Incompressible flow}\iff\sum_k e_{kk}=0$
\end{note}

\subsection{Conservation of Mass}
We want an equation which encompasses conservation of mass in a fluid. Consider a fluid element cube with (infinetesimal) side length $d$ and centred at $(x_0,y_0,x_0)$.
\begin{align*}
    \text{Rate of increase of fluid mass} &= \text{Volume} \cdot \text{Rate of change of density}\\
    &= d^3 \cdot \pd{\rho}{t}
\end{align*}
Summing over all sides we see
\begin{align*}
d^3\pd{\rho}{t}=d^2[&-\rho(x_0+d,y,z)u(x_0+d,y,z)+\rho(x_0,y,z)u(x_0,y,z)\\
                    &-\rho(x,y,_0+d,z)u(x,y_0+d,z)+\rho(x,y_0,z)u(x,y_0,z)\\
                    &-\rho(x,y,z_0+d)u(x,y,z_0+d)+\rho(x,y,z_0)u(x,y,z_0)]
\end{align*}
Dividing by $d^3$ and taking the limit $d\to 0$ we see
$$\pd{\rho}{t}=-\pd{(\rho u)}{x} - \pd{(\rho v)}{y} - \pd{(\rho w)}{z} = -\grad\cdot(\rho\mv{u})$$
So we have our equation
\begin{formula}
$$\pd{\rho}{t}+\grad\cdot(\rho\,\mv{u})=0$$
\end{formula}
\begin{note}
	We have derived this equation in Cartesian coordinates but obtained a vector equation, so we may apply it to any coordinate system.
\end{note}
\subsection{Conservation of Momentum}
We would like to apply Newton's $2^{nd}$ law to fluids. So far we can write
$$m\mv{a}=\underbrace{ \mv{F}_{body} }_{internal}+\underbrace{\mv{F}_{stress}}_{external}$$
We define \mdf{stress} to be the force per unit area and then the \mdf{stress tensor} $T_{ij}$  is the $i^{th}$ component of stress on the surface with normal $\mv{n}_j$.
We normally the symmetric class of tensors where $T_{ij}=T{ji}$.
We now consider the total force acting on our infinitesimal cube and by a similar calculation as the one in conservation of momentum we find
$$\delta F_i=d^3\sum_j\pd{T_{ij}}{x_j}$$
\subsubsection{Cauchy's Momentum Equation}
We can now substitute all of the formulae into Newton's second law to get:
$$\underbrace{ (\rho d^3) }_{mass}\underbrace{ \md{u_i} }_{acceleration}=\underbrace{d^3\sum_j\pd{T_{ij}}{x_j}}_{internal} -\underbrace{(\rho d^3)}_{mass}\underbrace{g\delta_{i3}}_{gravity}$$
Then we can divide by $d^3$ on both sides to yield the Cauchy Momentum Equation:
\begin{formula}
\textbf{Cauchy's Momentum Equation:}
$$\rho\md{u_i}=\pd{T_{ij}}{x_j}-\rho g \delta_{i3}$$
or in vector form
$$\md{\mv{u}}=\frac{1}{\rho}\grad\cdot\;\tul{\mv{T}}-g\mv{e}_z$$
\end{formula}
This is a very important formula but it requires us to know the stress tensor. We would like to relate $\tul{\mv{T}}$ to velocity $\mv{u}$ and pressure $\mv{P}$. 
\subsubsection{Inviscid Flow}
In inviscid flows, there is no internal friction and so there are no sheering terms to the stress tensor, i.e. all off-diagonal terms are 0.
Therefore, the only stress is the inward pressure acting perpendicular to the sides of the fluid element.
We can write this as the following \mdf{constitutive relation}:
$$T_{ij}=-p\delta{i_j}$$
Substituting $\pd{T_{ij}}{x_j}=-\pd{p}{x_i}$ into Cauchy's Momentum Equation we see
\begin{formula}
	\textbf{Euler's Equation}
	$$\md{\mv{u}}=-\frac{1}{\rho}\grad{p}-g\mv{e}_z$$
\end{formula}
\begin{note}
\begin{itemize}
	\item In the absence of gravitational forces we see that fluid elements are accelerated by pressure gradients from high to low pressure.
	\item If we assume fluid elements are not being accelerated then we recover a hydrostatic balance:
		$$0=-\frac{1}{\rho}\pd{p}{z}-g\implies p=-\rho gz + p_0$$
\end{itemize}
\end{note}
\subsubsection{Stress Tensor for a Viscous Fluid}
This time we have the normal inward pressure as with inviscid flows but we also have an additional \mdf{viscous stress tensor} $\sigma_{ij}$.
$$T_{ij}=-p\delta_{ij}+\sigma_{ij}$$
We need another constitutive relation between the rate of strain tensor $e_{ij}$ and the viscous stress tensor $\sigma_{ij}$.
When this relationship is linear, we call this flow \mdf{Newtonian}.
In the case that the flow is incompressible we get $\sum_k e_{kk}=0$ and hence
$$\sigma_{ij}=2\mu e_{ij}$$
where $\mu$ is the dynamic viscosity given in $kgm^{-1}s^{-1}$. The dynamic viscosity is the coefficient of proportionality between the rate of strain tensor and the stress tensor. This describes how easily a fluid moves under a shear force.
Then we can write
\begin{align*}
	T_{ij}&= -p\delta_{ij}+2\mu e_{ij}\\
		  &= -p\delta_{ij}+\mu\left(\pd{u_i}{x_j}-\pd{u_j}{x_i}\right)
\end{align*}
Now $\sum_i T_{ii} = -3p$ and hence $p=-\frac{1}{3}(\sum_i T_{ii})$ is the mechanical (NOT thermodynamic) pressure.
\begin{note}
The flow of incompressible Newtonian fluids covers a huge range of phenomena.
However, there are many cases in which non-Newtonian behaviour is encountered and hence we need more complex constitutive relations than those above.
\end{note}
We can now substitute these relations into Cauchy's momentum equation to get:
\begin{formula}
\textbf{The incompressible Navier-Stokes equations:}
$$\md{u_i}=-\frac{1}{\rho}\pd{p}{x_i}+\nu\pd[2]{u_i}{x_k}-g\delta_{i3}$$
or in vector form
$$\md{\mv{u}}=\underbrace{-\frac{1}{\rho}\grad{p}}_{Pressure\;gradients}+\underbrace{\nu\grad^2\mv{u}}_{Viscous\;forces}\underbrace{-g\mv{e}_z}_{Gravity}$$
where $\nu=\frac{\mu}{\rho}$ is the \mdf{kinematic viscosity coefficient}.
\end{formula}
\subsection{Controlling Flow Parameters}
We have two key questions to answer before we can start using these formulae:
\begin{itemize}
	\item When is the inviscid assumption valid?
	\item When is the incompressible assumption valid?
\end{itemize}
To this end, we shall define two \mdf{dimensionless parameters}, namely the Reynold's Number and Mach Number.
\subsubsection{Reynold's Number}
Our aim is to determine when it is safe to ignore the viscous forces in NS equations.
To do this we compare the approximate sizes of typical terms within the viscous and acceleration term.
Suppose we have a flow with characteristic length $L$ and characteristic speed $u$ with a kinematic viscosity $\nu$.
\begin{align*}
	\text{A typical viscosity term is} &\left|\nu\pd[2]{u}{x}\right|\sim\frac{\nu u}{L^2}\\
	\text{A typical acceleration term is} &\left|u\pd{u}{x}\right|\sim\frac{u^2}{L}
\end{align*}
Dividing these two terms we should get (we can check a dimensional parameter. We call this the \mdf{Reynold's number}.
$$\frac{|(u\cdot\grad)u|}{|\nu\grad^2 u|}\sim \frac{u^2}{L}\cdot\frac{L^2}{\nu u}=\underbrace{\mdf{\frac{uL}{\nu}}}_{\mdf{Reynold's\;Number}}$$
We now have two cases:
\begin{itemize}
	\item $Re >> 1 \implies \frac{Inertia}{Viscosity} >> 1 \implies$ Viscosity is negligible so we can use the Euler equations.
	\item $Re << 1 \implies \frac{Inertia}{Viscosity} << 1 \implies$ Inertia is negligible so we can use the Stoke's equation (where we remove inertia term).
	\item $Re \approx 1 \implies$ We have to use the NS equations
\end{itemize}
\begin{formula}
	\textbf{Stokes's Equation:}
	$$0=-\frac{1}{\rho}\grad{p}+\nu\grad^2\mv{u}-g\mv{e}_z$$
\end{formula}
\begin{note}
Even if $Re >> 1$ based on the scale of the flow on a macro scale, we may see complex behaviour locally where the length scale can be much lower, thus reducing the Reynold's number. This is typical for thin layers of flow close to solid boundaries.
\end{note}

\subsubsection{Mach Number}
Our aim now is to determine when it is safe to neglect compressibility of the flow. We must construct an estimate for relative changes of density $(\delta\rho / \rho)$ in the fluid.
We assume that the dominant terms in the Navier-Stokes equations are the inertial terms and pressure terms, i.e.
$$(\mv{u}\cdot\grad)\mv{u}\sim \frac{\grad p}{\rho}$$
But we can rewrite the right hand side to get
$$(\mv{u}\cdot\grad)\mv{u}\sim \frac{\grad \rho}{\rho}\pd{p}{\rho}$$
On a small scale we can approximate the gradient of $\rho$ as $(\delta\rho/L)$ and then we recognise that $\pd{p}{\rho}=c_s^2$ where $c_s$ is the speed of sound. Substituting in the approximate size of our typical inertial term we get:
$$\frac{U^2}{L}\sim \frac{\delta\rho}{L\rho}c_s^2\implies \frac{U^2}{c_s^2}\sim\frac{\delta\rho}{\rho}$$
We now define the \mdf{Mach number} to be $Ma\defeq\frac{U}{c_s}$.
When $Ma << 1 $, incompressibility is a safe assumption.
This is OK for the majority of cases and will be assumed for the remainder of this course.
(NOTE: We are talking about flow compressibility NOT fluid compressibility).
\subsubsection{Similarity}
It can be shown that the only dimensionless parameter arising from the Navier-Stokes equations is the Reynold's Number.
As such, flows with similar geometries and boundary conditions, whose Reynolds numbers agree will have identical flows.
This can be used to create small scale models for testing purposes.
For example, to decrease the length scale by a factor of 10, we need only increase the characteristic speed of flow by a factor of 10 in order to compensate.

\subsection{Boundary and Initial Conditions}
We have some PDEs and so in order to find an initial condition we must additionally specify some boundary and initial conditions.

\subsubsection{Initial Conditions}
We must specify all variables for which we have taken time derivatives, at time $t=0$, for all points $x$ occupied by the fluid $V\subseteq\mathbb{R}^d$. Note we have no time derivatives on density $\rho$ and hence we only have to specify
$$\mv{u}(\mv{x},t=0)\defeq\mv{u}_0\quad\quad\forall\mv{x}\in V$$
This \mdf{initial velocity field} $\mv{u}_0$ must satisfy the upcoming boundary conditions and be divergence free.

\subsubsection{Boundary Conditions}
The number of boundary conditions depends on the bulk equations (i.e. Euler or NS equations) and their form depends on the physics at the boundaries.

\textbf{Viscous fluids:}

At solid, impenetrable boundaries at has been shown empirically that the fluid velocity $\mv{u}$ at the boundary $\partial V$ of the retaining volume is equal to the boundary's velocity $\mv{U}_b$. This is the \mdf{no-slip boundary condition}:
$$\mv{u}(\mv{x},t)=\mv{U}_b(\mv{x},t)\quad\quad\forall\mv{x}\in\partial V$$

\textbf{Inviscid fluids:}

In inviscid fluids there is no $\grad^2 \mv{u}$ term and so we only have terms in the first derivatives of the velocity.
We can only be sure that the boundary is impenetrable.
This can be expressed by saying that the normal component of the fluid velocity at the boundary is equal to the normal component of the boundary velocity.
The component parallel to the boundary has no restriction.
This means that the fluid can slip freely past the boundary, giving rise to the \mdf{free-slip boundary condition}:
$$\mv{u}(\mv{x},t)\cdot\mv{n}=\mv{U}_b(\mv{x},t)\cdot\mv{n}\quad\quad\forall\mv{x}\in\partial V$$
where $n$ is the unit normal to the boundary.

\textbf{Free surfaces:}

So far we have only considered hard, fixed boundaries. Sometimes a liquid will have its boundary where it meets a passive gas (e.g. at the surface of the pond) in a \mdf{free surface} or \mdf{free boundary}.
The location of this boundary must then be found as part of the solution.
To summarise this, we assume the free-slip boundary conditions and that the pressure of the liquid at the boundary is equal to the atmospheric pressure $p_a$ in the gas.
$$p(\mv{x},t)=p_a\quad\quad\forall\mv{x}\in\partial V$$

\subsection{Summary}
The \mdf{material derivative} is
$$\md{}\defeq\pd{}{t}+\sum_i \pd{}{x_i}=\pd{}{t}+\mv{u}\cdot\grad$$

A flow is \mdf{steady} if $\pd{\mv{u}}{t}=0$.

The \mdf{rate of rotation tensor} is 
$$r_{ij}\defeq\frac{1}{2}\left(\pd{u_i}{x_j}-\pd{u_j}{x_i}\right)$$

The \mdf{rate of strain tensor} is 
$$r_{ij}\defeq\frac{1}{2}\left(\pd{u_i}{x_j}+\pd{u_j}{x_i}\right)$$

The \mdf{vorticity} is $\omega\defeq\grad\times\mv{u}=2(r_{32},r_{13},r_{21})$.

We have \mdf{incompressible flow if} $\grad\cdot\mv{u}=0$.

The \mdf{stress tensor} $T_{ij}$ is the $i^{th}$ component of stress (force per unit area) on the surface with normal $\mv{n}_j$.

\mdf{Cauchy's Momentum equation} is
$$\rho\md{u_i}=d^3\pd{T_{ij}}{x_j}-\rho g\delta_{i3}\;\iff\;\md{\mv{u}}=\frac{1}{\rho}\grad\cdot\tul{\mv{T}}-g\mv{e}_z$$

For inviscid flows we can use \mdf{Euler's Equation}
$$\md{\mv{u}}=-1\frac{1}{\rho}\grad p-g\mv{e}_z$$

For Newtonian flows the relationship between the viscous stress and strain tensors is linear $\sigma_{ij}=2\mu e_{ij}$ where $\mu$ is the dynamic viscosity.

For incompressible Newtonian fluids we have the \mdf{incompressible Navier-Stokes equations}
$$\md{u_i}=\frac{1}{\rho}\pd{p}{x_i}+\nu\pd[2]{u_i}{x_k}-g\delta_{i3}$$
or in vector form
$$\md{\mv{u}}=\frac{1}{\rho}\grad p + \nu \grad^2\mv{u} - g\mv{e}_z$$
where $\nu=\frac{\mu}{\rho}$ is the \mdf{kinematic viscosity coefficient}.

\mdf{Reynold's number} measures the negligibility of viscosity. When large we can use Euler's equations, when low we can use Stoke's equations as approximations.
$$Re\defeq\frac{uL}{\nu}$$

The \mdf{Mach number} measures the importance of compressibility. At low Mach numbers, incompressibility is a safe assumption.
$$Ma\defeq\frac{u}{c_s}$$

When solving these equations we must specify a divergence free \mdf{initial velocity field} which satisfies the boundary conditions.

In viscous fluids we may assume the \mdf{no-slip boundary condition}:
$$\mv{u}(\mv{x},t)=\mv{U}_b(\mv{x},t)\quad\quad\forall\mv{x}\in\partial V$$

In inviscid fluids we may assume the \mdf{free-slip boundary condition}:
$$\mv{u}(\mv{x},t)\cdot\mv{n}=\mv{U}_b(\mv{x},t)\cdot\mv{n}\quad\quad\forall\mv{x}\in\partial V$$

At free surfaces we can assume that surface pressure is equal to atmospheric pressure:
$$p(\mv{x},t)=p_a\quad\quad\forall\mv{x}\in\partial V$$
\section{Additional Conservation Laws}
Now that we have the Navier-Stokes equations we will consider what further conservation laws result from these equations. We consider special cases in which we can make analytic progress.

\subsection{Velocity-Vorticity Form of the NS equations}
When flow in incompressible $\grad\cdot\mv{u}=0$ and we have the following vector identity
$$(\mv{u}\cdot\grad)\mv{u}=\omega\times\mv{u}+\grad\left(\frac{u^2}{2}\right)$$
where we write $u^2=\mv{u}\cdot\mv{u}$. We can sub this into the NS equations and writing $g\mv{e}_z=\grad(gz)$ we get
$$\pd{\mv{u}}{t}+\omega\times\mv{u}=-\grad B+\nu\grad^2\mv{u}$$
where $B$ is the \mdf{Bernoulli Potential}
$$B=\frac{p}{\rho}+\frac{u^2}{2}+gz$$
\subsection{Bernoulli Theorems}
We will now consider two cases:
\begin{itemize}
	\item The flow is irrotational $\omega = 0$.
	\item The flow is steady $\pd{\mv{u}}{t}=0$.
\end{itemize}
\subsubsection{Unsteady Irrotational Flow}
The Helmholtz decomposition tehorem tells us we can split up $\mv{u}$ into a rotational part and an irrotational part $\grad\phi$ where $\phi$ is a velocity potential. So in the irrotational case, velocity can be written
$$\mv{u}=\grad\phi$$
We will also use this identity
$$\grad\times(\grad\times\mv{u})=\grad(\grad\cdot\mv{u})-\grad^2\mv{u}$$
For incompressible irrotational flows this implies $\grad^2\mv{u}=0$.
We substitute all of this into the velocity-vorticity form and get
$$\grad\pd{\phi}{t}=-\grad B$$
which we can over a path $\mv{x}$ to get the following
\begin{theorem}[Bernoulli theorem for steady irrotational flow]
	$$\pd{\phi}{t}+B=C(t), \quad \text{where} \quad B=\frac{p}{\rho}+\frac{\abs{\grad\phi}^2}{2}+gz$$
	where $C(t)$ is a function of time only. The incompressibility condition becomes
	$$\grad\cdot\mv{u}=\grad^2\phi=0.$$
\end{theorem}

\begin{eg}
Do U-bend example here!
\end{eg}

\subsubsection{Steady Irrotational Flow}
If we further assume that flow is steady then the theorem tells us that Bernoulli's potential is constant throughout the flow volume.
$$B\equiv \frac{p}{\rho}+\frac{u^2}{2}+gz$$
\begin{eg}
Pitot Tube example here!
\end{eg}
\subsubsection{Steady Inviscid Rotational Flow}
We again start with the velocity-vorticity form and dot the entire equation with $\mv{u}$. Using $\pd{\mv{u}}{t}=0$ and $\mv{u}\cdot(\omega\times\mv{u})=0$ we arrive at the equation
$$\underbrace{\md{B}=\mv{u}\cdot\grad B}_{\text{for steady flow}} \equiv 0$$
so B follows the fluid. That is B is constant on each streamline, although this constant may differ between stream lines. Moreover, remember steady flow implies that particle paths and steam lines coincide.
\subsection{Global Conservation Laws}
So far all of our conservation laws have been local, i.e. they refer to conservation of a field along fluid trajectories. However, we can also apply similar laws to arbitrary volumes within our fluid by integrating the local equations. This arbitrary volume could be the entire fluid volume.
\subsection{Mass Conservation Over an Arbitrary Volume}
We assume incompressibility $\pd{u_i}{x_i}=0$ and integrate over a finite fixed volume V:
$$0=\int_V \pd{u_i}{x}dV=\int_{\partial V}u_in_idA=\int_{\partial V}\mv{u}\cdot\mv{n}\;dA$$
Intuitively, this means that mass flux through all side of our volume $V$ must sum to zero. In a compressible fluid, where the density inside $V$ can change, this is not necessarily true.
\begin{eg}
Venturi Pipe!
\end{eg}
\subsection{Momentum Conservation}
Recall that the main NS equation is the local momentum balance equation. That is, it was obtained by applying Newton's second law to a moving infinitesimal fluid parcel.
\subsubsection{Over an arbitrary volume}
Consider an arbitrary fixed volume V within the fluid through which the fluid flows.
The total momentum in this volume is given by
$$M_i=\int_V \rho u_i\;dV.$$
Integrating the NS equation over this volume we get
$$\pd{M_i}{t}=-\rho\int_V u_j\pd{u_i}{x_j}dV-\int_V\pd{p_\rho gz}{x_i}dV + \nu\rho\int_V \pd[2]{u_u}{x_j}dV$$
Using incompressibility and the product rule we see that
$$\pd{u_iu_j}{x_j}=u_j\pd{u_i}{x_j}+u_i\underbrace{\pd{u_j}{x_j}}_{=0 \;\text{by incompressibility}}$$
Then using the divergence theorem, we have
$$\pd{M_i}{t}=-\underbrace{\rho\int_{\partial V}u_iu_jn_j\;dA}_{\text{Inertial terms at boundary}}-\underbrace{\int_{\partial V}(p+\rho gz)n_i\;dA}_{\text{Pressure and gravity}}+\underbrace{\mu\int_{\partial V}\pd{u_i}{x_j}n_j\;dA}_{\text{Viscous stress}}$$
Finally, for steady inviscid flows we can conclude
$$\underbrace{\rho\int_{\partial V}u_iu_jn_j\;dA}_{\text{Momentum flux into V}}=\underbrace{-\int_{\partial V}(p + \rho gz)n_i\;dA}_{\text{Boundary forces}}$$
Note, we have information about the balance of forces at the boundary, without needing explicit knowledge of the flow inside our fixed volume V.
\begin{eg}
Jammed garden hose!
\end{eg}
\subsubsection{Over the entire volume}
Suppose we have a fluid contained in a finite volume $V$ whose bounding surface consist of impenetrable, stationary walls. The boundary conditions tell us the following.
\begin{itemize}
	\item For non-zero finite viscosity $\nu$ the normal and parall component of $u_i$ are zero at the boundary
		$$u_i=0\quad\text{on}\;\partial V$$
	\item For zero viscosity $\nu$ we only have the free-slip condition
		$$u_in_i=0\quad\text{on}\;\partial V$$
\end{itemize}
In either case notice that the inertial terms become 0.
If $V$ is the entire volume then momentum will certainly be conserved, i.e. $\pd{M_i}{t}=0$.
So at the boundary we find
$$\int_{\partial V}(p+\rho g z)n_i\;dA=\nu\int_{\partial V}\pd{u_i}{x_j}n_j\;dA$$
so the net effect of the pressure and gravity forces, together with viscous stress at the boundary must be zero.
\subsection{Energy Conservation}
We consider the conservation of energy over the \emph{entire volume} and suppose that the fluid is contained by in a finite volume $V$ bounded by a surface of impenetrable, stationary walls $\partial V$.
We start with a number of notes that will help later
\begin{itemize}
	\item Using integration by parts, followed by the impenetrability of $\partial V$ and incompressibility
		$$\int_V u_i\pd{B}{x_i}dV=\int_V\left(\pd{}{x_i}(u_iB)-B\underbrace{\pd{u_i}{x_i}}_{=0}\right)dV=\int_{\partial V} Bu_in_i\; dA = 0$$
	\item Using integration by parts we see
		\begin{align*}
			\nu\int_V u_i\pd{}{x_j}\pd{u_i}{x_j}dV &= \nu \int_V \left[\pd{}{x_j}\left(u_i\pd{u_i}{x_j}\right)-\pd{u_i}{x_j}\pd{u_i}{x_j}\right]dV\\
												   &= \underbrace{\nu \int_{\partial V} u_i\pd{u_i}{x_j}n_j\;dA}_{=0\;\text{in both conditions}} - \nu \int_{V} \left(\pd{u_i}{x_j}\right)^2 dV
		\end{align*}
	Note we are summing over both indices in this identity.
\item $u_i\pd{u_i}{t}=\frac{1}{2}\pd{u_iu_i}{t}=\frac{1}{2}\pd{u_i^2}{t}$ where $u_i^2=\mv{u}\cdot{\mv{u}}$.
\end{itemize}
Now we dot multiply the velocity-vorticity form with $\mv{u}$ and integrate over the entire volume $V$.
$$\int_V u_i\pd{u_i}{t}dV = - \int_V u_i\pd{B}{x_i}dV + \nu \int_V u_i\pd[2]{u_i}{x_j}dV$$
Substituting in what we found earlier we obtain the relation
$$\pd{}{t}\int_V\frac{1}{2}u^2\;dV=-\nu\int_v\left(\pd{u_i}{x_j}\right)^2dV$$
Note we are summing over two indices (there are no free indices) and hence this is a scalar expression, as we would expect for an energy equation.
We now define the \mdf{total kinetic energy} as
$$E\defeq \frac{\rho}{2}\int_V u^2\;dV$$
and we achieve our energy balance equation.
\begin{formula}
	\textbf{Energy balance equation:}(over the entire volume $V$)
	$$\pd{E}{t}=-\nu\int_v\left(\pd{u_i}{x_j}\right)^2dV$$
\end{formula}
In physicsy terms this means that kinetic is lost by viscous dissipation (i.e. internal friction), which will result in the generation of heat (and hence conservation of total energy).
We see that in the inviscid case, this becomes
$$\pd{E}{t}=0$$
This energy conservation law is \emph{global}.
\subsection{Summary}
We introduced a velocity potential $\mv{u}=\grad\phi$ which, for incompressible flow, yields three different Bernoulli theorems depending on flow characteristics. Recall the \mdf{Bernoulli potential} has now become $B=\frac{p}{\rho}+\frac{\abs{\grad\phi}^2}{2}+gz$.
We have the following theorems:
\begin{enumerate}
	\item \textbf{Unsteady Irrotational Flow ($\omega=0$)}
		$$\pd{\phi}{t}+B=C(t)$$
	\item \textbf{Steady Irrotational Flow ($\omega=\pd{\mv{u}}{t}=0$)}
		$$B=\text{fixed constant}$$
	\item \textbf{Steady Inviscid Rotational Flow ($\pd{\mv{u}}{t}=\nu=0$)}
		$$B=\text{constant on each streamline(= particle paths)}$$
\end{enumerate}
\subsubsection{Fixed Volumes}
Given an arbitrary fixed volume $V$, we have mass conservation
$$\int_{\partial V}u_in_i\;dA=0$$
The \mdf{momentum} $M_i=\int_V \rho u_i\;dV$ satisfies
$$\pd{M_i}{t}=-\rho\int_{\partial V}u_iu_jn_j\;dA-\int_{\partial V}(p+\rho gz)n_i\;dA+\nu\rho\int_{\partial V}\pd{u_i}{x_j}n_j\;dA$$
with the left side zero for steady flow.
\subsubsection{Entire Volume}
Taking $V$ to be the entire volume which is bounded by impenetrable, stationary walls, we have momentum conservation
$$\int_{\partial V}(p+\rho gz)n_idA = \mu\int_{\partial V}\pd{u_i}{x_j}n_j\;dA$$
which is a balance of forces on the boundary.

Considering the \mdf{kinetic energy} $E=\frac{\rho}{2}\int_V u^2\;dV$ we obtain
$$\pd{E}{t}=-\nu\int_V \left(\pd{u_i}{x_j}\right)^2dV$$
so that in the absence of viscosity kinetic energy is conserved globally.
\end{document}
