\documentclass[11pt]{article}

%{{{ Packages
\usepackage[margin=1in]{geometry}
\usepackage{enumitem}
\usepackage{amsfonts}
\usepackage{amssymb}
\usepackage{amsmath}
\usepackage{bbm}
\usepackage{amsthm}
\usepackage{mathdots}
\usepackage[dvipsnames]{xcolor}
\usepackage[framemethod=TikZ]{mdframed}
\usepackage{microtype}
\setlength{\parindent}{0pt}
%}}}
%{{{ Custom commands
% Nice maths commands
\newcommand{\defeq}{:=}
\newcommand{\abs}[1]{|#1|}
\newcommand{\norm}[1]{||#1||}
%\renewcommand{\dots}{...}
\newcommand{\msrspc}{\ensuremath{(X,\mathcal{B},\mu)}}
\newcommand{\symd}{\triangle}
\newcommand{\indic}[1]{\mathbbm{1}_{#1}}
\newcommand\restr[2]{{% we make the whole thing an ordinary symbol
  \left.\kern-\nulldelimiterspace % automatically resize the bar with \right
  #1 % the function
  \vphantom{\big|} % pretend it's a little taller at normal size
  \right|_{#2} % this is the delimiter
  }}

% Spaces
\newcommand{\ktor}{\mathbb{T}^k}
\newcommand{\R}{\mathbb{R}}
\newcommand{\C}{\mathbb{C}}
\newcommand{\Z}{\mathbb{Z}}

% Ergodic Theory
\newcommand{\gvn}[2]{\ensuremath{\left(#1\;|\;#2\right)}}
\newcommand{\expg}[2]{\ensuremath{\mathbb{E}\gvn{#1}{#2}}}
\newcommand{\infog}[2]{\ensuremath{I\gvn{#1}{#2}}}
\newcommand{\entrg}[2]{\ensuremath{H\gvn{#1}{#2}}}
\newcommand{\probg}[2]{\ensuremath{\mathbb{P}\gvn{#1}{#2}}}
%}}}
%{{{ Environments
% Definitions environment
\newenvironment{defin}
	{\begin{mdframed}[backgroundcolor=white, roundcorner=5pt, linewidth=1pt]}
	{\end{mdframed}}
\newcommand{\mdf}[1]{{\color{red} #1}}

% Important notes environment
\newenvironment{note}
	{\begin{mdframed}[backgroundcolor=white, linecolor=red, roundcorner=5pt, linewidth=1pt]\bfseries{Note:}\normalfont}
	{\end{mdframed}}

% Examples enviornmnet
\definecolor{mylg}{rgb}{0.9,0.9,0.9}
\newenvironment{eg}
	{\begin{mdframed}[backgroundcolor=myld,roundcorner=5pt,linewidth=0pt]\bfseries{Example:}}
	{\end{mdframed}}

% Theorem enviornment
\newtheorem{prop}{Proposition}[section]
\newtheorem{theorem}[prop]{Theorem}
\newtheorem{lemma}[prop]{Lemma}
\newtheorem{cor}[prop]{Corollary}
%}}}
%{{{ Document metadata
\title{Ergodic Theory Notes}
\author{}
\date{}
%}}}

\begin{document}
\maketitle

\section{Basic Definitions}

For this section we fix a probability space \msrspc and we have a transformation $T:X\to X$ which is measurable in our probability space.

\begin{defin}
	
We say $T$ is a \mdf{measure preserving transformation (m.p.t.)} or $\mu$ is a \mdf{$T$-invariant measure} if 
$$\mu(T^{-1}B)=\mu(B)\quad\forall B\in\mathcal{B}$$

The \mdf{push forward of $\mu$ by $T$} is defiend to be
	$$T_*\mu(B)=\mu(T^{-1}B)\quad\forall B \in\mathcal{B}$$

We say a measure $\mu$ is \mdf{regular} if $\forall B\in\mathcal{B}$ we have $\forall\epsilon >0 \exists U\subseteq X$ open such that
$$B\subseteq U \quad \text{and} \quad \mu(U) < \mu(B) + \epsilon$$

An m.p.t $T$ is said to be \mdf{ergodic} if
$$\forall B\in\mathcal{B},\; T^{-1}B=B \implies \mu(B)=0\;\text{or}\;1$$

\end{defin}

\section{Facts on Fourier Series}
Suppose $f\in L_1(\ktor)$ then we can define the \mdf{Fourier coefficients} by
$$\hat{f}(n)=\int_{\ktor}f(x)e^{-2\pi in\cdot x}dx\quad\forall n\in\Z^k$$
\begin{theorem}[Fej\'er's Theorem]
The average of the partial Fourier sums converges uniformly to $f$, i.e.
$$\frac{1}{N}\sum_{k=0}^{N-1}S_kf\to f\quad\text{uniformly}$$
\end{theorem}
\begin{theorem}[Riemann-Lebesgue Lemma]
	For all $f\in L_1(\ktor)$, $$\lim_{\abs{n}\to\infty}\hat{f}(n)=0$$
\end{theorem}
\begin{theorem}[Reisz-Fisher Theorem]
Define $S_Nf(x)=\sum_{\abs{n}\leq N}\hat{f}(n)e^{2\pi i (n\cdot x)}$
then $S_nf\to f$ in $L^2$ for all $f\in L^2(\ktor)$.
\end{theorem}
\begin{cor}
If $f\in L^2(\ktor)$ and $\hat{f}(n)=0\;\forall n\in\Z^k\setminus\{0\}$, then $f$ is constant.
\end{cor}
\begin{theorem}
Given $f\in L^2$ which is $T$-invariant
$$\hat{f}(n)=\lim_{N\to\infty}\int (S_N f)(Tx)e^{-2\pi i n\cdot x}$$
\end{theorem}
\section{Criteria for measure preserving}
\begin{theorem}
Given $T:X\to X$ on a probability space $(X,\mu)$, the following are equivalent:
\begin{enumerate}
	\item $T$ is m.p.t
	\item $\int f\circ T d\mu = \int f d\mu \quad\quad \forall f\in L_1(X)$.\end{enumerate}
\end{theorem}
Recall the space $L_1(X)=\left\{ f:x\to\R : \text{measurable}\quad \norm{f}_1 \defeq\int \abs{f} d\mu < \infty \right\}$

\begin{theorem}
Given $T:X\to X$ on a probability space $(X,\mu)$, the following are equivalent:
\begin{enumerate}
	\item $T$ is m.p.t
	\item $\int f\circ T d\mu = \int f d\mu \quad\quad \forall f\in C(X)$.
\end{enumerate}
\end{theorem}

So we see that in fact it suffices to check that $T$ does not affect the integral of any continuous function $f$.
However, we can extend this further using the density of trigonometric polynomials in the space of continuous functions. 
First, we need to define a trigonometric polynomial in arbitrary dimension on the $k$-torus $X=\ktor$ with $\mu=leb$ and $\mathcal{B}=Borel$.
\begin{defin}
	$P:\ktor\to\ktor$ is a \mdf{trigonometric polynomial} if for some $N\geq 1$ and $c_n\in\C$ we can write
	$$P(x)=\sum_{\abs{n}\leq N} c_n e^{2\pi in\cdot x}$$
	where $n=(n_1,\dots,n_k)\in\Z^k, x=(x_1,\dots,x_k), \abs{n}=\abs{n_1}+\dots+\abs{n_k}$.
\end{defin}
\begin{note}
	\[
		\int_{\ktor} e^{2\pi n\cdot x} dx =
		\begin{cases}
			1 & \text{if} \; n=0\\
			0 & \text{if} \; n\neq 0
		\end{cases}
	\]
	and hence
	$$\int_{\ktor}P = c_0$$
\end{note}

\begin{theorem}
Given $T:\ktor\to \ktor$ continuous and denoting by $\mu$ the Lebesgue measure.
\begin{enumerate}
	\item $T$ is m.p.t
	\item $\int P\circ T d\mu = \int P d\mu \quad\quad \forall\; \text{trigonometric polynomials}\; P$. 
\end{enumerate}
\end{theorem}

\section{Criteria for Ergodicity}
First another few definitions.
\begin{defin}
Given $A,B\subseteq X$, their \mdf{symmetric difference} is
$$A\symd B\defeq (A \setminus B)\cup (B \setminus A)$$

A function $f$ is \mdf{$T$-invariant} if $f \circ T = f$ a.e.

A function $f$ is \mdf{constant} if $\exists c\in\R$ such that $f(x) = c$ almost everywhere.
\end{defin}

\begin{theorem}
Given a measure preserving transformation $T:X\to X$ and some $1\leq p\leq\infty$. TFAE:
\begin{enumerate}
	\item T is ergodic.
	\item For all $f$ measurable $f$ invariant $\iff$ $f$ constant.
	\item For all $f\in L^p(X)$, $f$ invariant $\iff$ $f$ constant.
\end{enumerate}
\end{theorem}
\begin{note}
To check that $T$ is ergodic it suffices to show that all invariant $L^2$ functions have zero Fourier coefficients away from zero.
\end{note}
To this end we present the following formula for computing the Fourier coefficients of invariant $L^2$ functions.
\begin{theorem}
Given $f\in L^2$ which is invariant
$$\hat{f}(n)=\lim_{N\to\infty}\int (S_Nf)(Tx)e^{-2\pi i n\cdot x}dx$$
\end{theorem}

\section{Theorems using Measure Preserving}
\begin{theorem}[Poincar\'e Recurrence Theorem]
	Given a probability space \msrspc and $T:X\to X$ measure preserving. Then
	$$\mu\{x\in B : T^nx\in B \; \text{infinitely often}\,\}=\mu(B)\quad \forall B\in\mathcal{B}$$
\end{theorem}
\section{Theorems using Ergodicity}
\begin{theorem}[Pointwise Ergoic Theorem - Birkhoff 1931]
	Given a measure space \msrspc and a measure preserving transformation $T:X \to X$ and $f\in L^1(X)$. Then $\exists f^*\in L^1(X)$ invariant such that
	$$\frac{1}{n}\sum_{j=0}^{n-1}f\circ T^j \to f^* \; a.e.\quad \text{and} \quad \int f^* = \int f$$
\end{theorem}
Note this does not actually need ergodicity. However, if we additionally assume ergodicity we can prove the following stronger result.
\begin{cor}
Given a probability space \msrspc, $T$ measure preserving and ergodic, $f\in L^1(x)$, then
$$\underbrace{\frac{1}{n}\sum_{j=0}^{n-1}f\circ T^j }_{\text{Time average}}\to \underbrace{\int f d\mu \; a.e.}_{\text{Space average}}$$
\end{cor}
\begin{theorem}[Mean Ergodic Theorems]
	$1\leq p < \infty$, $T$ measure preserving theorem, $f\in L^p(X)$. Define $f^*\defeq \lim_{n\to\infty} \frac{1}{n}\sum_{j=0}^{n-1}f\circ T^j$ almost everywhere. Then
	$$\lim_{n\to\infty}\frac{1}{n}\sum_{j=0}^{n-1}f\circ T^j = f^*$$
	in $L^p$.
\end{theorem}
\section{Examples}
\subsection{Linear toral automorphism}
\begin{defin}
A \mdf{linear toral automorphism} is a map $Tx=Ax (mod 1)$ with $A$ a $k\times k$ matrix with integer entries and $\det(A)\neq 0$.

Such an automorphisms is \mdf{hyperbolic} if all eigenvalue for $A$ have $|\lambda|\neq 1$.
\end{defin}
\begin{theorem}
T ergodic $\iff$ no eigenvalue of $A$ is a root of unity.
\end{theorem}
\subsection{Normality of real numbers}
\begin{defin}
$x\in\R$ is \mdf{normal (base b)} if
\begin{itemize}
	\item $x$ has a  unique expansion in that base.
	\item $\forall k\in\{0,1,\dots,b-1\}$
		$$\frac{1}{n}\#\{1\leq i\leq n | x_i=k\}\to\frac{1}{10}\quad\text{as}\;n\to\infty$$
\end{itemize}

$x\in\R$ is \mdf{absolutely normal} if $x$ is normal base b for all $b\geq 2$.
\end{defin}
\begin{theorem}
Almost every $x\in\R$ is absolutely normal.
\end{theorem}

\section{Existent of invariant/ergodic measures}
\begin{defin}
Let $\mdf{M(X)}$ be the set of all probability measure on X.

We can view measures as linear functionals on the space of continuous functions as such:
$$\forall f\in C(X)\quad\quad \mdf{\mu(f)}\defeq\int_X f d\mu$$

$\mdf{C(X)^*}\defeq\{\text{bounded linear functionals}\quad w:C(X)\to\R\}$

A linear functional is called \mdf{normalised} if $\int 1 d\mu =1$

A linear functional is called \mdf{positive} if $f\geq 0\implies \int f d\mu \geq 0$
\end{defin}
\begin{theorem}
	Every $\mu\in M(X)$ defines a normalised, positive, bounded, linear functional in $C(X)^*$ defined by $\mu(f)=\int_X f d\mu$.
\end{theorem}
\begin{theorem}[Reisz Representation Theorem]
	Let $w\in C(X)*$ be a bounded linear functional. Suppose that $w$ is positive and normalised. Then $\exists !\mu\in M(X)$ such that $w(f)=\mu(f)$ for all $f\in C(X)$.
\end{theorem}

\section{Entropy}
The motivation for a definition of entropy is as a vehicle to distinguish between dynamical systems. First we need to know how tell when two systems are identical.
\begin{defin}
	Two probability spaces with measure preserving transformations, $(X,\mathcal{B},\mu,T),(Y,\mathcal{C}, \nu, S)$ are \mdf{measure-theoretically isomorphic} if there exists a bijection $\pi:B\to C$ where $B\in\mathcal{B}$ and $C\in\mathcal{C}$ such that
	\begin{itemize}
		\item $\mu(B)=\nu(C)=1$
		\item $T(B)\subseteq B, S(C)\subseteq C$
		\item $\pi: B\to C$ and $\pi^{-1}:C\to B$ are measure preserving transformations
		\item $\pi\circ T = S\circ\pi$
	\end{itemize}
Assume $\msrspc$ is a probability space and $\alpha=\{A_i\}$ a countable collection of subsets $A_i\subseteq B$.
\begin{itemize}
	\item We say $\alpha$ is a \mdf{partition} of $X$ if $\cup A_i$ = X and $A_i\cap A_j=\emptyset$ up to measure 0.
	\item The \mdf{join} of two partitions $\alpha,\beta$ is the partition $\alpha\vee\beta$ of all possible intersections $A_i\cap B_j$.
	\item A countable partition $\beta$ is a \mdf{refinement} of $\alpha$ if every element of $\alpha$ is a union of element of $\beta$ and write $\alpha\leq\beta$.
	\item $\alpha,\beta$ are \mdf{independent} if $\mu(A\cap B)=\mu(A)\mu(B)$ for all $A\in\alpha, B\in\beta$.
	\item The \mdf{information of a partition} $\alpha$ is
		$$I(\alpha)\defeq-\sum_{A\in\alpha}\indic{A}\log(\mu(A))$$
		where $I(\alpha):X\to[0,\infty]$.
	\item The \mdf{entropy of a partition} $\alpha$ is
		$$H(\alpha)\defeq\int_X I(\alpha)d\mu=-\sum_{A\in\alpha}\mu(A)\log(\mu(A))$$
		using the convention $0\cdot\log(0)=0$.
	\item The \mdf{expectation given a partition} is
		$$\expg{\cdot}{\alpha}\defeq\expg{\cdot}{\sigma(\alpha)}$$
	\item The \mdf{conditional probability} of $B\in\mathcal{B}$ given $\alpha$ is
		$$\probg{B}{\alpha}\defeq\expg{\indic{B}}{\alpha}$$
\end{itemize}
Suppose that $\mathcal{C}$ is a sub $\sigma$-algebra of $\mathcal{B}$.
\begin{itemize}
	\item The \mdf{conditional information of $\alpha$ given $\mathcal{C}$} is
		$$\infog{\alpha}{\mathcal{C}}\defeq - \sum_{A\in\alpha}\indic{A}\log(\mu\gvn{A}{C})$$
	where $\mu\gvn{A}{\mathcal{C}}\defeq\expg{\indic{A}}{\mathcal{C}}$
	\item The \mdf{conditional entropy of $\alpha$ given $\mathcal{C}$} is
			$$\entrg{\alpha}{\mathcal{C}}\defeq\int_X\infog{\alpha}{\mathcal{C}}d\mu$$	
\end{itemize}
\end{defin}
We have the following desirable properties:
\begin{itemize}
	\item If $\alpha$ and $\beta$ are independent then $I(\alpha\vee\beta)=I(\alpha)+I(\beta)$.
	\item If $\alpha=\{X\}$ then $I(\alpha)=0$ so $H(\alpha)=0$.
	\item If $T$ is a measure preserving transformation then $H(T^{-1}\alpha)=H(\alpha)$.
	\item Given $A\in\alpha$, $\restr{\expg{f}{\alpha}}{A}=\frac{\int_A f \;d\mu}{\mu(A)}$ and hence
		$$\expg{f}{\alpha}=\sum_{A\in\alpha}\indic{A}\frac{\int_A f \;d\mu}{\mu(A)}$$
	\item Conditional probability and expectation are constant on partition elements.
	\item For $A\in\alpha$, 
		$$\restr{\probg{B}{\alpha}}{A}=\restr{\expg{\indic{B}}{\alpha}}{A}=\frac{\int_A \indic{B} d\mu}{\mu(A)}=\frac{\mu(A\cap B}{\mu(A)}$$
	\item If $\mathcal{C}=\{X,\emptyset\}$ then $\infog{\alpha}{\mathcal{C}}=I(\alpha)$ and $\entrg{\alpha}{\mathcal{C}}=H(\alpha)$.
	\item If $g\geq 0$ is $\sigma(\alpha)$-measurable then $\expg{fg}{\sigma(\alpha)}=g\cdot\expg{f}{\sigma(\alpha)}$.
	\item If $T$ is a measure preserving transformation then $\infog{T^{-1}\alpha}{T^{-1}\mathcal{C}}=\infog{\alpha}{\mathcal{C}}\circ T$.
	\item Integrating this gives $\entrg{T^{-1}\alpha}{T^{-1}\mathcal{C}}=\entrg{\alpha}{\mathcal{C}}$.
	\item $\alpha\leq\beta\implies\infog{\alpha}{\beta}=0$.
\end{itemize}
\begin{prop}
	$$\entrg{\alpha}{\mathcal{C}}=-\int_X \sum_{A\in\alpha}\mu\gvn{A}{\mathcal{C}}\log(\mu\gvn{A}{\mathcal{C}})d\mu$$	
\end{prop}
\begin{lemma}[Basic Identity]
	Given $\alpha,\beta,\gamma$ partitions of $X$ then
	$$\infog{\alpha\vee\beta}{\gamma}=\infog{\alpha}{\gamma}+\infog{\beta}{\alpha\vee\gamma}$$
	$$\entrg{\alpha\vee\beta}{\gamma}=\entrg{\alpha}{\gamma}+\entrg{\beta}{\alpha\vee\gamma}$$
\end{lemma}
\begin{cor}
	$$\beta\leq\gamma\implies\infog{\alpha\vee\beta}{\gamma}=\infog{\alpha}{\gamma}$$
\end{cor}
\begin{cor}[Monotonicity of information of entropy]
	$$\alpha\leq\beta\implies\infog{\alpha}{\gamma}\leq\infog{\beta}{\gamma}$$
\end{cor}
\begin{cor}[Anti-monotonicity of entropy]
	$$\beta\leq\gamma\implies\entrg{\alpha}{\beta}\geq\entrg{\alpha}{\gamma}$$
\end{cor}
\begin{cor}
We have the two following properties as well:
\begin{itemize}
	\item $\entrg{\alpha}{\gamma}\leq H(\alpha)$ (because always $\gamma\leq\{X,\emptyset\}$)
	\item $\entrg{\alpha\vee\beta}{\gamma}\leq\entrg{\alpha}{\gamma}+\entrg{\beta}{\gamma}$
\end{itemize}
\end{cor}
So far this does not encapsulate any dynamics of the system and so we must use these concepts to arrive at a definition of entropy which depends on the transformation. For convenience define the following set:
$$\mathcal{P}\defeq\{\alpha\;\text{countable partitions}\;|H(\alpha) < \infty\}$$
Now choose $\alpha\in\mathcal{P}$. Then we define the following:
$$\mdf{H_n(\alpha)}\defeq H(\alpha^n)\quad\text{where}\quad\alpha^n\defeq\bigvee_{j=0}^{n-1}T^{-j}\alpha$$
This has the convenient property that $H_{n+m}(\alpha)\leq H_n(\alpha)+H_m(\alpha)$, i.e. these $H_n$ form a sub-additive sequence $\R$-valued sequence and hence the limit $\mdf{h(T,\alpha)}\defeq\lim_{n\to\infty}\frac{1}{n}H_n(\alpha)$ exists. We call this the \mdf{entropy of T relative to $\alpha$}.
We can then define the \mdf{entropy of T} by taking the supremum:
$$\mdf{h(T)}\defeq\sup_{\alpha\in\mathcal{P}}h(T,\alpha)$$
Having done all this work, this had better be a measure-theoretic isomorphism invariant.
\end{document}
