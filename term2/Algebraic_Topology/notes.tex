\documentclass[11pt]{article}

%{{{ Packages
\usepackage[margin=1in]{geometry}
\usepackage{enumitem}
\usepackage{amsfonts}
\usepackage{amssymb}
\usepackage{amsmath}
\usepackage{amsthm}
\usepackage{amsmath}
\usepackage{mathdots}
\usepackage{float}
\usepackage[thicklines]{cancel}
\renewcommand{\CancelColor}{\color{red}}
\usepackage[dvipsnames]{xcolor}
\usepackage[framemethod=TikZ]{mdframed}
\usepackage{microtype}
\usepackage{tikz-cd}
\setlength{\parindent}{0pt}
%}}}
%{{{ Custom commands
% Nice maths commands
\newcommand{\defeq}{:=}
\newcommand{\eqdef}{=:}
\newcommand{\abs}[1]{|#1|}
\newcommand{\norm}[1]{||#1||}
\DeclareMathOperator{\im}{\mathrm{im}}
%\renewcommand{\dots}{...}
\newcommand{\msrspc}{\ensuremath{(X,\mathcal{B},\mu)}}
\newcommand{\relmiddle}[1]{\mathrel{}\middle#1\mathrel{}}
\newcommand{\rmv}{\relmiddle|}
\newcommand{\stcmp}{^{\mathsf{c}}}
\newcommand\restr[2]{{% we make the whole thing an ordinary symbol
  \left.\kern-\nulldelimiterspace % automatically resize the bar with \right
  #1 % the function
  \vphantom{\big|} % pretend it's a little taller at normal size
  \right|_{#2} % this is the delimiter
  }}
\newcommand{\contr}{\Rightarrow\Leftarrow}
\newcommand{\interior}[1]{%
  {\kern0pt#1}^{\mathrm{o}}%
}

% Spaces
\newcommand{\ktor}{\mathbb{T}^k}
\newcommand{\R}{\mathbb{R}}
\newcommand{\C}{\mathbb{C}}
\newcommand{\Z}{\mathbb{Z}}
\newcommand{\N}{\mathbb{N}}

% Derivatives
\newcommand*{\pd}[3][]{\ensuremath{\frac{\partial^{#1} {#2}}{\partial {#3}^{#1}}}}
\newcommand{\grad}{\bigtriangledown}

% Vectors
\newcommand{\mv}[1]{\textbf{#1}}

%}}}
%{{{ Enviornments
% Definitions environment
\newenvironment{defin}
	{\begin{mdframed}[backgroundcolor=white, roundcorner=5pt, linewidth=1pt]}
	{\end{mdframed}}
\newcommand{\mdf}[1]{{\color{red} #1}}

% Important notes environment
\newenvironment{note}
	{\begin{mdframed}[backgroundcolor=white, linecolor=red, roundcorner=5pt, linewidth=1pt]\bfseries{Note:}\normalfont}
	{\end{mdframed}}

% Examples enviornmnet
\definecolor{mylg}{rgb}{0.9,0.9,0.9}
\newenvironment{eg}
	{\begin{mdframed}[backgroundcolor=myld,roundcorner=5pt,linewidth=0pt]\bfseries{Example:}}
	{\end{mdframed}}

% Theorem environment
\newtheorem{theorem}{Theorem}[section]
\newtheorem{cor}[theorem]{Corollary}
\newtheorem{lemma}[theorem]{Lemma}
\newtheorem{prop}[theorem]{Proposition}
%}}}
%{{{ Document metadata
\title{Algebraic Topology Notes}
\author{Thomas Chaplin}
\date{}
%}}}

\begin{document}
\maketitle

\section{Simplicial Homology}

The \mdf{standard $k$-simplex} is	
\[
	\Delta^k\defeq\left\{(x_0, \dots x_k)\in \R^{k+1} \rmv \sum_{i=0}^{k}x_i = 1, \quad x_i\geq 0\right\}
\]

Given $v_0, \dots, v_k\in \R^N$ we define the \mdf{simplex} by
\[
	[v_0, \dots v_k] \defeq \left\{\sum_{i=0}^{k}x_i v_i \rmv \sum_{i=0}^{k }x_i = 1, \quad x_i \geq 0\right\}
\]
so than $\Delta^k=[e_0, \dots, e_k]$.

This yields an obvious map $\sigma:\Delta^k\to [ v_0, \dots v_k]$
\[
	\sigma(x_0, \dots x_k)=\sum_{x_i v_i}
\]
We will often, confusingly, denote this map $[v_0, \dots v_k]$.

We say $v_0, \dots, v_k$ are \mdf{in general position} if they do not lie on any $(k-1)$-dimensional affine subspace.

\begin{prop}
$v_0, \dots, v_k$ are in general position $\iff$ $\sigma$ is a homeomorphism.
\end{prop}

We can then define the map $i_j:\Delta^{k-1}\to\Delta^k$ to be the map $[e_0, \dots, e_{j-1}, e_{j+1}, \dots, e_k]$.
This map then parametrises the face opposite the vector $e_j$.
The union of the $k-1$ dimensional faces of $[v_0, \dots, v_k]$ is its \mdf{boundary} and its \mdf{interior} is $[v_0, \dots, v_k]$.

\begin{defin}
	A \mdf{$\Delta$-complex structure} on a space $X$ is a collection of maps $\sigma_\alpha:\Delta^k\to X$ for varying $k$ such that
	\begin{enumerate}
		\item $\sigma_\alpha:\interior{\left( \Delta^k \right)}\to X$ is injective and each point of $X$ lies in the image of exactly one interior.
		\item If $\sigma_\alpha:\Delta^k\to X$ is in the collection then $\sigma_\alpha\circ i_j:\Delta^{k-1}\to X$ is also for $j=0, \dots, k$.
		\item $U\subseteq X$ is open $\iff$ $\sigma_\alpha^{-1}(U)$ is open in $\Delta^k$ for all $\alpha$.
	\end{enumerate}
\end{defin}


We then define the spaces of formal sum of simplicies
\[
	\Delta_n(X)\defeq\left\{\sum m_\alpha\sigma_\alpha^n \rmv \text{formal sums with integer coeffs and }\\ \sigma_\alpha^n \text{ are n-simplicies}\right\}
\]
and then we can define the boundary operator $\partial_n:\Delta_n(X)\to\Delta_{n-1}(X)$ by 
\[
	\partial_n(\sigma_\alpha^n)=\sum_{j=0}^{n}(-1)^j(\sigma_\alpha^n \circ i_j) = \sum_{j=0}^{n}(-1)^j\restr{\sigma_\alpha^n}{[e_0, \dots, \hat{e_j}, \dots , e_n]}
\]
where $\hat{e_j}$ means that we omit the $j$'th vertex.

Then the loops are elements in $\ker\partial_n$ but we don't care about loops that are themselves the boundaries of higher order simplicies since these loops can be contracted through the higher-order simplicies.

\begin{defin}
	We define the $n$'th homology of the $\Delta$-complex structure to be
	\[
		\mdf{H_n^\Delta}\defeq\frac{\ker(\partial_n)}{\im(\partial_{n+1})}
	\]
\end{defin}

\section{Singular Homology}
Simplicial homology is very computable but has a number of problems that prevent from having far reaching consequences to topology.
\begin{enumerate}
	\item $\Delta_n(X)$ depends on the choice of $\Delta$ complex structure so perhaps $H_n^\Delta$ does too?
	\item We would like functoriality.
		If $X, Y$ have $\Delta$-complex structures and $f:X\to Y$ is continuous, how do we then define a homomorphism $H_n^\Delta(X)\to H_n^\Delta(Y)$?
\end{enumerate}

These problems can be remedied by studying the more complicated and seemingly incomputable singular homology.
Instead of considering only simplicies $\Delta^n\to X$ in the $\Delta$ complex structure we allow all continuous maps $\Delta^n\to X$ which we call \mdf{singular $n$-simplicies}.
\[
	C_n(X)\defeq\left\{\text{finite formal sums of singular n-simplicies}\right\}=\text{Free AbGr on singular n-simplicies}
\]
We can then define $\partial_n(X)\to C_{n-1}(X)$ in exactly the same way as before and then define \mdf{singular homology}
\[
	\mdf{H_n(x)}\defeq\frac{\ker\partial_n}{\im\partial_{n+1}}
\]
\begin{note}
	\[
		\partial_n\circ\partial_{n+1}=0
	\]
\end{note}

\subsection{Functoriality}
If we have a continuous map $f:X\to Y$ then we can define
\[
	f_\#:C_n(X)\to C_n(Y),\quad \sigma\mapsto f\circ\sigma
\]
on the $n$-simplicies and then extend it to all of $C_n(X)$ linearly so that
\[
	f_\#\left(\sum_{\alpha}m_\alpha \sigma_\alpha\right)=\sum_{\alpha}m_\alpha \left(f \circ \sigma_\alpha\right)
\]

\begin{lemma}
\[
	\partial(f_\#\sigma)=f_\#(\partial\sigma)
\]
\end{lemma}

\begin{proof}
\begin{align*}
	\partial(f_\#\sigma)&=\partial((f\circ\sigma))=\sum_{j=0}^{n}(-1)^j(f\circ\sigma)\circ i_j \\
	f_\#(\partial\sigma)&=f_\#\left(\sum_{j=0}^{n}(-1)^j(\sigma\circ i_j)\right)=\sum_{j=0}^{n}(-1)^j f\circ (\sigma \circ i_j)
\end{align*}
\end{proof}

We can also extend this result to any formal sum of $n$-simplicies in $C_n(X)$.
Hence $f_\#$ induces a \mdf{morphism of chain complexes}.
The morphism arises due to this commutativity and it is of chain complexes because we have $\partial^2=0$.
\begin{figure}[h]
	\centering
\begin{tikzcd}
	\dots  \arrow[r]  & C_{n+1}(X) \arrow[r, "\partial_{n+1}"] \ar[d, "f_\#"] & C_n(X) \arrow[r, "\partial_n"]\ar[d, "f_\#"] &\ar[d, "f_\#"] C_{n-1}(X) \arrow[r] & \dots \\
	\dots \ar[r] & C_{n+1}(X) \arrow[r, "\partial_{n+1}"] & C_n(X) \arrow[r, "\partial_n"] & C_{n-1}(X) \ar[r] & \dots
\end{tikzcd}
\end{figure}

\begin{cor}
From this diagram we can see that
\begin{enumerate}
	\item $f_\#(\underbrace{\ker\partial_n}_{\subseteq C_n(X)})\subseteq\underbrace{\ker\partial_n}_{\subseteq C_n(Y)}$.
	\item $f_\#(\im\partial_n)\subseteq \im\partial_n$.
\end{enumerate}
and hence $f_\#$ induces a homomorphism $f_\ast:H_n(X)\to H_n(Y)$ because if you differ be an element of $\im\partial_{n+1}$ in $H_n(x)$ then you will still differ by an image element in $H_n(Y)$.
\end{cor}

\begin{defin}
	Elements of $\ker\partial$ are called \mdf{cycles} and elements of $\im\partial$ are called \mdf{boundaries}.	
	\begin{align*}
		B_n(X)&\defeq\im\partial_{n+1}\subseteq C_n(X) \\
		Z_n(X)&\defeq\ker\partial_n\subseteq C_n(X)
	\end{align*}
	so the $H_n(X)=\frac{Z_n}{B_n}$ and the induced map $f_\ast$ is given be
	\[
		f_\ast(c+B_n(X))\defeq f_\#(c) + B_n(Y)
	\]
\end{defin}

\begin{theorem}
\[
	(f\circ g)_\ast = f_\ast \circ f_\ast \quad \text{and} \quad \left(id_X\right)_\ast = id_{H_n(X)} \;\; \forall n
\]
and hence singular homology is a functor.
\end{theorem}

This implies the important result that
\[
	X\underbrace{\cong}_{\text{homeo}}Y \implies H_n(X) \underbrace{\cong}_{\text{iso}} H_n(Y) \;\; \forall n
\]

\subsection{Basic computation of singular homology}
Here are some important results.
\begin{theorem}
\[
	H_n(\left\{pt\right\})=
	\begin{cases}
		\Z & \text{if }n=0 \\
		0  & \text{if }n>0
	\end{cases}
\]
\end{theorem}

\begin{note}
$\Delta^n$ is path connected so every continuous map $\sigma:\Delta^n\to X$ falls in one path component.
Hence
\[
	C_n(X)=\bigoplus_{\text{path components }X_i} C_n(X_i)
\]
Moreover $\partial_n (C_n(X_i)\subseteq C_{n-1}(X_i)$ so the chain complexes of the path components are independent of one another and hence we have
\[
	H_n(X)=\bigoplus_{X_i} H_n(X_i)
\]
\end{note}

\begin{theorem}
$X$ path connected $\iff$ $H_0(X)\cong X$.
\end{theorem}

\subsection{Reduced Homology}
It often makes sense to add one extra space to the chain complex under to state results more succinctly, so we modify the chain complex as so

\begin{figure}[ht]
	\centering
	\begin{tikzcd}
		\dots \ar[r] & C_n(X) \ar[r] & \dots \ar[r] & C_1(x) \ar[r, "\partial_1"] & C_0(X) \ar[r, "\epsilon"] & \Z \ar[r] & 0
	\end{tikzcd}
\end{figure}
where we introduce  anew map $\epsilon:C_0(X)\to\Z$ which just sums coefficients:
\[
	\epsilon\left(\sum m_\alpha [p_\alpha]\right)\mapsto \sum m_\alpha
\]
then we define the \mdf{reduced homologies} to be the same $\widetilde{H}_n(X)=H_n(X)$ for $n>0$ but then
\[
	\widetilde{H}_0(X)\defeq\frac{\ker\epsilon}{\im\partial_1}\neq H_0(X)
\]

We can realise the relation of $H_0$ and $\widetilde{H}_0$ by the following commutative diagram:

\begin{figure}[ht]
	\centering
	\begin{tikzcd}
		\ker\epsilon \arrow[r, "i"] \arrow[d, "\frac{}{\im\partial_1}"] & C_0(X) \arrow[d, "\frac{}{\im\partial_1}"]\\
		\widetilde{H}_0(X) \arrow[r, "i"] & H_0(X)
	\end{tikzcd}
\end{figure}

This diagram commutes and hence the map $\epsilon$ passes to the quotient to define
\[
	\overline{\epsilon}:H_0(X)\to\Z
\]
and then $\widetilde{H}_0(X)=\ker\overline{\epsilon}$.

\section{Exact sequences}
A sequence of abelian groups $A_n$ and homomorphisms $\phi_n$ is a \mdf{complex} if $\im\phi_{n+1}\subseteq\ker\phi_n$ for all $n$.
\begin{figure}[H]
	\centering
	\begin{tikzcd}
		\dots \arrow[r, "\phi_{n+2}"] & A_{n+1} \arrow[r, "\phi_{n+1}"] & A_{n} \arrow[r, "\phi_{n}"] & A_{n-1} \arrow[r, "\phi_{n-1}"] & \dots
	\end{tikzcd}
\end{figure}

The complex is \mdf{exact} if $\im\phi_{n+1}=\ker\phi_n$ for all $n$.

\begin{note}
	Whenever we have such a chain complex we can assign the $n$-th homology to be
	\[
		\frac{\ker \phi_n}{\im \phi_{n+1}}
	\]
\end{note}

So assuming that $X$ is non-empty we get the following exact sequence
\begin{figure}[H]
	\centering
	\begin{tikzcd}
		0 \arrow[r] & \widetilde{H}_0(X) \arrow[r, hook] & H_0(X) \arrow[r, "\overline{\epsilon}", two heads] & \Z \arrow[r] & 0
	\end{tikzcd}
\end{figure}

Given $A\subseteq X$ both topological spaces then we say $(X, A)$ is a \mdf{pair}.
Further, we say they are a \mdf{good pair} if $A$ is a deformation retract of some neighbourhood in $X$.

\begin{theorem}
If $(X, A)$ is a good pair then there exists a long exact sequence
\begin{figure}[H]
	\centering
	\begin{tikzcd}
		0 \arrow[r] & \widetilde{H}_n(A)\arrow[r, "i_\ast"] & \widetilde{H}_n(X) \arrow[r, "q_\ast"] & \widetilde{H}_n(\frac{X}{A}) \arrow[lld, "\partial", bend right=5] \\
		\; &\widetilde{H}_{n-1}(A)\arrow[r, "i_\ast"] & \widetilde{H}_{n-1}(X) \arrow[r, "q_\ast"] & \widetilde{H}_{n-1}(\frac{X}{A}) \arrow[lld, "\dots", bend right=5]\\
		\; & \widetilde{H}_0(A)\arrow[r, "i_\ast"] & \widetilde{H}_0(X) \arrow[r, "q_\ast"] & \widetilde{H}_0(\frac{X}{A}) \arrow[r] & 0
	\end{tikzcd}
\end{figure}
\end{theorem}

We can then use this long exact sequence with the pair $(S^{n-1}, D^n)$.
\begin{figure}[H]
	\centering
	\begin{tikzcd}
		0 \arrow[r] & 
		\widetilde{H}_n(S^{n-1})\arrow[r, "i_\ast"] & 
		\cancelto{0}{\widetilde{H}_n(D^n)} \arrow[r, "q_\ast"] & 
		\widetilde{H}_n\left(\frac{D^n}{S^{n-1}}\right) \arrow[lld, "\partial", bend right=5] \\
		\; &
		\widetilde{H}_{n-1}(S^{n-1})\arrow[r, "i_\ast"] & 
		\cancelto{0}{\widetilde{H}_{n-1}(D^n)} \arrow[r, "q_\ast"] & 
		\widetilde{H}_{n-1}\left(\frac{D^n}{S^{n-1}}\right)\arrow[lld, "\dots", bend right=5]\\
		\; & 
		\widetilde{H}_0(S^{n-1})\arrow[r, "i_\ast"] & 
		\cancelto{0}{\widetilde{H}_0(D^n)} \arrow[r, "q_\ast"] & 
		\widetilde{H}_0\left(\frac{D^n}{S^{n-1}}\right) \arrow[r] & 0
	\end{tikzcd}
\end{figure}

Thanks to the exactness of this sequence we see that
\[
	\widetilde{H}_n(S^{n-1})\cong\widetilde{H}_n\left(\frac{D^n}{S^{n-1}}\right)\cong\widetilde{H}_n(S^n)
\]

and hence

\[
	\widetilde{H}_k(S^n)=
	\begin{cases}
		0 & \text{if }k < n \\
		\Z & \text{if }k =n \\
		0 & \text{if }k > n
	\end{cases}
\]

We will often find \mdf{short exact sequences} with are sequences that look like
\begin{figure}[H]
	\centering
	\begin{tikzcd}
		0 \arrow[r] & A \arrow[r, hook] & B \arrow[r, two heads] & C \arrow[r] & 0
	\end{tikzcd}
\end{figure}

in which case the first isomorphism theorem tells us that $C\cong\frac{B}{A}$.

\section{Relative homologies}
Given a pair $(X, A)$ we can define the \mdf{relative chain group}
\[
	C_n(X, A)\defeq\frac{C_n(X)}{C_n(A)}
\]
i.e. we consider two continuous maps into $X$ to be the same if the agree on $X\setminus A$.

Now our maps $\partial_n$ pass to the quotient by
\[
	\overline{\partial}_n(c+C_n(A))\defeq\partial_n(c)+C_{n-1}(A)
\]
which is well defined because the boundary of a chain in $A$ is still a chain in $A$.
We still get $\overline{\partial}_n\circ\overline{\partial}_{n+1}=0$ and hence we can define the \mdf{relative homology groups}
\[
	H_n(X, A)\defeq\frac{\ker\overline{\partial}_n}{\im\overline{\partial}_{n+1}}
\]
Notice that given a pair of spaces $(X, A)$, we get a short exact sequence of chain complexes
\begin{figure}[H]
	\centering
	\begin{tikzcd}
		0 \arrow[r] & C_n(A) \arrow[r, "i_\#"] & C_n(X) \arrow[r, "j_\#"] & C_n(X, A) \arrow[r] & 0
	\end{tikzcd}
\end{figure}

\begin{theorem}
Given any short exact sequence of chain complexes
\begin{figure}[H]
	\centering
	\begin{tikzcd}
		0 \arrow[r] & A_\bullet \arrow[r, "i"] & B_\bullet \arrow[r, "j"] & C_\bullet \arrow[r] & 0
	\end{tikzcd}
\end{figure}
we get a long exact sequence of homology:
\begin{figure}[H]
	\centering
	\begin{tikzcd}
		0 \arrow[r] & H_n(A_\bullet)\arrow[r, "i_\ast"] & H_n(B_\bullet) \arrow[r, "q_\ast"] & H_n(C_\bullet) \arrow[lld, "\partial", bend right=5] \\
		\; &H_{n-1}(A_\bullet)\arrow[r, "i_\ast"] & H_{n-1}(B_\bullet) \arrow[r, "q_\ast"] & H_{n-1}(C_\bullet) \arrow[lld, "\dots", bend right=5]\\
		\; & H_0(A_\bullet)\arrow[r, "i_\ast"] & H_0(B_\bullet) \arrow[r, "q_\ast"] & H_0(C_\bullet) \arrow[r] & 0
	\end{tikzcd}
\end{figure}
where the \mdf{connecting homomorphism} is defined by diagram chasing.
\end{theorem}

\begin{proof}
We construct the connecting homomorphism $\delta: H_n(C_\bullet)\to H_{n-1}(A_\bullet)$ as per the following diagram.
Remember that
\begin{itemize}
	\item Columns are exact sequences.
	\item Rows are complexes.
\end{itemize}

\begin{figure}[H]
	\centering
	\begin{tikzcd}[sep = large]
		\; & 0 \arrow[d] & 0\arrow[d] & 0\arrow[d] & \; \\
		\; \arrow[r] & A_{n+1} \arrow[r] \arrow[d] & A_n \arrow[r] \arrow[d] & A_{n-1} \arrow[r] \arrow[d] &\; \\
		\; \arrow[r] & B_{n+1} \arrow[r] \arrow[d] & B_n \arrow[r] \arrow[d] & B_{n-1} \arrow[r] \arrow[d] &\; \\
		\; \arrow[r] & C_{n+1} \arrow[r] \arrow[d] & C_n \arrow[r] \arrow[d] & C_{n-1} \arrow[r] \arrow[d] &\; \\
		\; & 0 & 0 & 0 & \; 
	\end{tikzcd}
\end{figure}

We need to prove this is well defined, that is:
\begin{enumerate}
	\item Independent of our choice of representative $c_n \in \left[ c_n\right]$
	\item Independent of our choice of $b_n$ such that $j(b_n)=c_n$.
\end{enumerate}

Then we need to prove exactness at each point in the sequence.
\end{proof}

This gives us a long exact sequence of relative homology but unfortunately we cannot yet use it to prove anything about $\widetilde{H_n}\left(\frac{X}{A}\right)$ because the sequence only includes terms like $\widetilde{H_n}(X, A)$.
We would like to show that for a good pair
\[
	\widetilde{H}_n\left(\frac{X}{A}\right)\cong \widetilde{H}_n(X, A)
\]
First homotopy invariance of homology and a relationship between $\pi_1$ and $H_1$.

\section{Homotopy Invariance of Homology}
\begin{theorem}
Homotopy maps $f, g:X \to Y$ induce the same homomorphism of homology, i.e. $f_\ast = g_\ast$.
\end{theorem}
\begin{proof}
This is the proof where we construct the prism operator.
It is very long and complicated but the main ideas should be remembered.
\end{proof}

\begin{cor}
	If $h:X \to Y$ is a homotopy equivalence then $h_\ast: H_n(X) \to H_n(Y)$ is an isomorphism for every $n\in\N$.
\end{cor}
\begin{proof}
This is now a simple application of functoriality.
\end{proof}

\begin{note}
In particular, a contractible space has the same homology as a point.
\end{note}

\begin{defin}
Suppose $p,  q$ are morphisms of chain complexes $A_\bullet\to B_\bullet$.	

\begin{figure}[H]
	\centering
	\begin{tikzcd}[sep = large]
		\dots \arrow[r] & A_{n+1} \arrow[dl, dashed, "P"', red] \arrow[d, bend left, "p"] \arrow[d, bend right, "q"'] \arrow[r]
						& A_n\arrow[dl, dashed, "P"', red] \arrow[d, bend left, "p"] \arrow[d, bend right, "q"'] \arrow[r]
						& A_{n-1}\arrow[dl, dashed, "P"', red] \arrow[d, bend left, "p"] \arrow[d, bend right, "q"']\arrow[r]
						& \dots \arrow[dl, dashed, "P"', red] \\
		\dots \arrow[r] & B_{n+1} \arrow[r] & B_n \arrow[r] & B_{n-1} \arrow[r] & \dots
	\end{tikzcd}
\end{figure}
We say they are \mdf{chain homotopic} if there is a $P:A_n \to B_{n+1}$ for each $n$ such that
\[
p - q = \partial P + P \partial
\]
\end{defin}

\section{Relation between $\pi_1$ and $H_1$}
For this section assume that $X$ is path connected and $x_0\in X$
Consider the map which realises a loop in $X$ based at $x_0$ as a cycle in $Z_1(X)$.
We claim that his passes to the quotient to define
\[
	h: \pi_1(X, x_0) \to Z_1(X)
\]
Moreover, we claim
\begin{enumerate}[label=(\roman*)]
	\item $h$ is a homomorphism.
	\item $h$ is surjective.
	\item $\ker h= \left[ \pi_1(X, x_0), \pi_1(X, x_0)\right]$ is the commutator subgroup.
\end{enumerate}
Hence by the first isomorphism theorem $H_1(X)$ can be seen as the \mdf{abelianisation} of $\pi_1(X, x_0)$
\[
	H_1(X) \cong \frac{\pi_1(X, x_0)}{\left[ \pi_1(X, x_0), \pi_1(X, x_0)\right]} \eqdef \pi_1(X, x_0)_{ab}
\]

\begin{proof}
Again this is very long but an understanding of the main argument would be useful.
Consult Hatcher for more details.
\end{proof}

\section{Homology: Civil War}
\begin{theorem}
The inclusion of $\Delta_\bullet(X) \to C_\bullet(X)$ induces an isomorphism
\[
	H_n^\Delta(X) \to H_n(X)
\]
when $X$ is a $\Delta$-complex.
\end{theorem}

This is quite a long road so we start with the following very important technical result.
The idea is that if we remove chains deep inside $A$ then we do not effect the relative homology because they have already been quotiented out.

\begin{theorem}[Excision]
Given spaces $Z\subseteq A\subseteq X$ such that $\overline{Z}\subseteq\interior{A}$.
Then the inclusion $(X \setminus Z, A\setminus Z) \to (X, A)$ induces an isomorphism of relative homology
\[
	H_n(X \setminus Z, A \setminus Z) \to H_n(X, A)\quad \forall n
\]

\textbf{Version 2: } If $A, B \subseteq X$ such that $\interior{A} \cup \interior{B} \subseteq X$ then the inclusion $(B, A \cap B) \to (X, A)$ induces an isomorphism $H_n(B, A\cap B) \to H_n(X,A)$ for every $n$.
\end{theorem}

We prove venison 2.
Notice that $\left\{ A, B\right\}$ forms a cover of $X$.

Suppose we have a cover $\mathcal{U}=\left\{ U_\alpha\right\}_\alpha$ then we can define
\[
	C_n^\mathcal{U}(X) \defeq\left\{ n\text{-chains in }X\text{ where each simplex lies in some }\alpha\right\}
\]
and we clearly get an inclusion $i: C_\bullet^\mathcal{U}(X) \to C_\bullet(X)$.

\begin{prop}
As defined above $i$ is a chain homotopy equivalent.
That is there is some $S:C_\bullet(X) \to C_\bullet(X)^\mathcal{U}$ such that
\begin{itemize}
	\item $i \circ S: C_\bullet(X) \to C_\bullet(X)$ is chain homotopic to the identity.
	\item $S \circ i: C_\bullet^\mathcal{U}(X) \to C_\bullet^\mathcal{U}(X)$ is chain homotopic to the identity.
\end{itemize}
\end{prop}

\begin{proof}
This was done by barycentric subdivision.
By iterated barycentric subdivision, given any singular $n$-simplex $\sigma:\Delta^n\to X$ we subdivide $\Delta^n$ to define a chain $S(\sigma)$ which lies in $C_n^\mathcal{U}(X)$.
This works because $\partial(subdivision) = subsdivison(\partial)$.
\end{proof}

Out first application of excision gives us a relationship between relative homology and homology of quotient spaces.

\begin{prop}
For a \textbf{good pair} $(X, A)$, the quotient map $q: (X, A) \to \left( \frac{X}{A}, \frac{A}{A}\right)$ induces an isomorphism
\[
H_n(X, A) \to H_n\left( \frac{X}{A}, \frac{A}{A}\right) \cong \widetilde{H}_n\left(\frac{X}{A}\right)
\]
\end{prop}

\section{Useful Results}
\begin{lemma}
A continuous bijection from a compact space to a Hausdorff space is a homeomorphism.
\end{lemma}

\begin{prop}
Given an equivalent relation $\sim$ and a continuous map $f:X\to Y$ that respects the relation, $\frac{f}{\sim}:\frac{X}{\sim}\to Y$ is also continuous.
\end{prop}

\begin{theorem}[First Isomorphism Theorem]
Given a homomorphism $\phi:X\to Y$
\[
	\im(\phi) \cong \frac{X}{\ker(\phi)}
\]
\end{theorem}

\begin{theorem}[Second Isomorphism Theorem]
Let $G$ be a group, $S$ some subgroup and $N$ some normal subgroup of $G$.
Then
\[
	\frac{S\times N}{N}\cong\frac{S}{S\cap N}
\]
\end{theorem}

\end{document}
