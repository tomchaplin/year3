\documentclass[11pt]{article}

%{{{ Packages
\usepackage[margin=1in]{geometry}
\usepackage{enumitem}
\usepackage{amsfonts}
\usepackage{amssymb}
\usepackage{amsmath}
\usepackage{amsthm}
\usepackage{amsmath}
\usepackage{mathdots}
\usepackage{float}
\usepackage[thicklines]{cancel}
\renewcommand{\CancelColor}{\color{red}}
\usepackage[dvipsnames]{xcolor}
\usepackage[framemethod=TikZ]{mdframed}
\usepackage{microtype}
\usepackage{tikz-cd}
%}}}
%{{{ Custom commands
% Nice maths commands
\newcommand{\defeq}{:=}
\newcommand{\eqdef}{+:}
\newcommand{\abs}[1]{|#1|}
\newcommand{\norm}[1]{||#1||}
\DeclareMathOperator{\im}{\mathrm{im}}
%\renewcommand{\dots}{...}
\newcommand{\msrspc}{\ensuremath{(X,\mathcal{B},\mu)}}
\newcommand{\relmiddle}[1]{\mathrel{}\middle#1\mathrel{}}
\newcommand{\rmv}{\relmiddle|}
\newcommand{\stcmp}{^{\mathsf{c}}}
\newcommand\restr[2]{{% we make the whole thing an ordinary symbol
  \left.\kern-\nulldelimiterspace % automatically resize the bar with \right
  #1 % the function
  \vphantom{\big|} % pretend it's a little taller at normal size
  \right|_{#2} % this is the delimiter
  }}
\newcommand{\contr}{\Rightarrow\Leftarrow}
\newcommand{\interior}[1]{%
  {\kern0pt#1}^{\mathrm{o}}%
}
\newcommand{\sm}{\setminus}

% Spaces
\newcommand{\ktor}{\mathbb{T}^k}
\newcommand{\R}{\mathbb{R}}
\newcommand{\C}{\mathbb{C}}
\newcommand{\Z}{\mathbb{Z}}
\newcommand{\N}{\mathbb{N}}

% Derivatives
\newcommand*{\pd}[3][]{\ensuremath{\frac{\partial^{#1} {#2}}{\partial {#3}^{#1}}}}
\newcommand{\grad}{\bigtriangledown}

% Vectors
\newcommand{\mv}[1]{\textbf{#1}}

%}}}
%{{{ Enviornments
% Definitions environment
\newenvironment{defin}
	{\begin{mdframed}[backgroundcolor=white, roundcorner=5pt, linewidth=1pt, linecolor=Green]
		\setlength{\parindent}{0pt}}
	{\end{mdframed}}
	\newcommand{\mdf}[1]{{\color{Green} #1}}

% Important notes environment
\newenvironment{note}
	{\begin{mdframed}[backgroundcolor=white, linecolor=red, roundcorner=5pt, linewidth=1pt]\bfseries{Note:}\normalfont
	\setlength{\parindent}{0pt}}
	{\end{mdframed}}

% Examples enviornmnet
\definecolor{mylg}{rgb}{0.9,0.9,0.9}
\newenvironment{eg}
	{\begin{mdframed}[backgroundcolor=mylg,roundcorner=5pt,linewidth=0pt]\bfseries{Example:}\normalfont
	\setlength{\parindent}{0pt}}
	{\end{mdframed}}

% Theorem environment
\newtheorem{theorem}{Theorem}[section]
\newtheorem{cor}[theorem]{Corollary}
\newtheorem{lemma}[theorem]{Lemma}
\newtheorem{prop}[theorem]{Proposition}
%}}}
%{{{ Document metadata
\title{Manifolds Notes on things I don't understand/remember}
\author{Thomas Chaplin}
\date{}
%}}}

\begin{document}
\maketitle

\section{Orientations}
\section{Abstract Manifolds}
\section{Vector Bundles}
In defining a vector bundle over a manifold we associate a vector space to each point in the manifold and then arrange these together in a smooth way.
Examples of this will include the tangent and normal bundles as defined in times gone past.

Suppose that $M$ is an $m$-manifold.
A \mdf{family of vector spaces} over $M$ is a manifold $E$ together with a smooth map $p:E\to M$ such that for all $x\in M$, the \mdf{fibre} $\mdf{E_x}\defeq p^{-1}(x)$ has a vector space structure.
Given $U\subseteq M$ we write $\restr{E}{U}\defeq p^{-1}U$ which we equip with the restriction of $p$.

Given another family over vector spaces $p':E'\to M$ we define an \mdf{isomorphism} to be a diffeomorphism 
\[
\psi:E\to E'
\]
with the property that $p'\circ\psi = p$ and $\psi:E_x\to E'_x$ restricts to  a linear isomorphism for all $x\in M$.
If such a map exists then $E$ and $E'$ are called \mdf{isomorphic}.
Finally, a family $E$ over $M$ is called \mdf{trivial} if it is isomorphic to $M\times \R^q$ for some $q\in\N$.

\begin{figure}[H]
	\centering
	\begin{tikzcd}
		E' \arrow[rr, "\psi"] \arrow[rd, "p'"']& & E \arrow[dl, "p"] \\
											  & M
	\end{tikzcd}
\end{figure}

\begin{defin}
	A \mdf{vector bundle} over $M$ is a family of vector spaces $p:E\to M$ such that $\forall x\in M$ there is some open $U\subseteq M$ with $x\in U$ such that $\restr{E}{U}$ is trivial.
\end{defin}

Putting this another way, $E$ is a vector bundle over $M$ if we can find an atlas $\left\{\phi_alpha : U_\alpha \to V_\alpha \right\}$ for $M$ such that for each $\alpha$ we get a diffeomorphism
\[
	\psi_\alpha: \restr{E}{U_\alpha}\to V_\alpha\times\R^q\subseteq\R^{m+q}
\]
which restricts to a linear isomorphism on each fibre.
Note than then the maps $\left\{\psi_\alpha\right\}$ form an atlas for $E$ which is called a \mdf{locally trivialising atlas} for $E$.
\begin{note}
	\begin{itemize}
		\item Given a vector bundle, $p:E\to M$ is a surjective smooth submersion.
		\item Each fibre $E_x$ is an embedded sub-manifold and the vector space operations are smooth.
		\item The notion of a vector bundle is preserved under isomorphism.
	\end{itemize}
\end{note}

\begin{eg}
\begin{enumerate}
	\item If $M$ is a smooth manifold then the tangent bundle is a vector bundle over $M$.
		For a manifold in Euclidean space, this follows for the proof that $TM$ was a manifold.
		For an abstract space, we can form a locally trivialising atlas as follows:

		Given a chart $\phi:U\to V\subseteq\R^m$ for $M$ we construct a new map
		\[
			\psi:TU\to V\times\R^m \quad\text{by}\quad\psi(v)=(\phi(x),d_x\phi(v))
		\]
		where $x=pv$.
		These maps then form a locally trivialising atlas because when restricted to some $x$ the map becomes the derivative map
		\[
			\psi_{p^{-1}(x)}:T_x M \to \left\{\phi(x)\right\}\times\R^m \quad \psi(v)=(\phi(x), d_x\phi(v))
		\]
		which is a linear isomorphism.
	\item When can (and perhaps should) check that the atlas given for the normal bundle consists of maps which restrict to linear isomorphisms on each fibre.

	\item The spaces $B_n$ together with the projection to the circle yields a vector bundles.
		In particular, the M\"obius band is a bundle over the circle.
		Note $B_m$ is isomorphic to $B_n$ if $\abs{m-n}$ is even.
		But, $B_0$ cannot be isomorphic to $B_1$ since they are not even diffeomorphic.
		Hence the M\"obius band is non-trivial.
\end{enumerate}
\end{eg}
Given a vector bundle $p:E\to M$ we say that \mdf{section} of $E$ is a smooth map $s:E\to M$ such that $p\circ s$ is the identity on $M$.
This generalises our previous notion of a section of the tangent space which is often referred to as a \mdf{vector field} on $M$.
We say that a section is \mdf{non-vanishing} if $s(x)\neq 0$ for all $x\in M$.
Given any smooth map $f:M\to\R^q$ we get a section of $M\times\R^q$.
\begin{note}
Any section is an embedding of $M$ into $E$.
\end{note}

\begin{theorem}[Hairy Ball Theorem]
Every vector field on $S^2$ has at least one zero.
\end{theorem}

\begin{note}
	It turns out that any odd-dimensional manifold admits a non-vanishing vector field.
	An even-dimensional manifold may or may not.
	The torus does but it is the only compact connected orientable 2-manifold which does.
\end{note}

\begin{lemma}
A vector bundle $E$ is trivial if and only if there are sections $s_1, s_2, \dots, s_q$ such that $\left\{s_1, \dots, s_q(x)\right\}$ forms a basis for $E_x$ for all $x\in M$.
\end{lemma}
\begin{proof}
If $f:M\times \R^q$ is an isomorphism then set $s_i(x)=f(x, e_i)$ for some fixed basis $e_1, \dots , e_q$ for $\R^q$.

Conversely, if we have such sections $s_i$ then we can define a map $f:M\times \R^q \to E$ by
\[
	f(x, (\lambda_1, \dots , \lambda_q))=\sum_{i=1}^{q}\lambda_i s_i(x)
\]
which yields an isomorphism.
\end{proof}
\begin{eg}
	\begin{enumerate}
		\item Using the Intermediate Value Theorem, one can show that the M\"obius has no non-vanishing section and is hence a non-trivial bundle by the above Lemma.
		\item Let $S^n\subseteq\R^{n+1}$ be the unit sphere in $\R^n$ then the normal bundle is trivial because the outward pointing unit normal is a nowhere vanishing section.
	\end{enumerate}
\end{eg}

\begin{defin}
	A manifold $M$ is called \mdf{parallelisable} if the tangent bundle $TM$ is trivial.
\end{defin}

In view of the last lemma we can say that a manifold is parallelisable if and only if $M$ admits a global frame field.
Recall, a frame field is a family of vector fields $v_i$ such that the $v_i(x)$ form a basis for $T_x M$ at every $x\in M$.
We know that frame fields exists locally but not necessarily globally.

\begin{lemma}
Any parallelisable manifold is orientable.
\end{lemma}
\begin{proof}
Exercise.
\end{proof}

\begin{eg}
	\begin{itemize}
		\item The circle $S^1$ is parallelisable.
		\item $S^2$ is not by the Hairy Ball Theorem.
		\item $S^3$ is parallelisable because given $x=(x_1, x_2, x_3, x_4)\in S^3$ we can form a frame field with
			\[
			\begin{array}{rrrr}
				(-x_2,& x_1,& x_4,& -x_3)\\
				(-x_3,&- x_4,& x_1,& x_2)\\
				(-x_4,& x_3,& -x_2,& x_1)
			\end{array}
			\]
		\item $S^5$ is not strangely enough but $S^7$ is.
			It may be worth consulting your written notes from lectures to figure out how this frame was calculated.
	\end{itemize}
\end{eg}

\subsection{Vector Fields}

\subsection{Whitney Sums}

\subsection{Cotangent Bundle}

\subsection{1-forms}


\section{Smooth Function Extension}
\section{Differential Forms}

\end{document}
