\documentclass[11pt]{article}

%{{{ Packages
\usepackage[margin=1in]{geometry}
\usepackage{enumitem}
\usepackage{amsfonts}
\usepackage{amssymb}
\usepackage{amsmath}
\usepackage{amsthm}
\usepackage{amsmath}
\usepackage{mathdots}
\usepackage{float}
\usepackage[thicklines]{cancel}
\renewcommand{\CancelColor}{\color{red}}
\usepackage[dvipsnames]{xcolor}
\usepackage[framemethod=TikZ]{mdframed}
\usepackage{microtype}
\usepackage{tikz-cd}
\usepackage{silence}
\WarningFilter{mdframed}{You got a bad break}
%}}}
%{{{ Custom commands 
% p-forms
\DeclareFontFamily{U}{skulls}{}
\DeclareFontShape{U}{skulls}{m}{n}{ <-> skull }{}
\newcommand{\skull}{\text{\usefont{U}{skulls}{m}{n}\symbol{'101}}}

% Nice maths commands
\newcommand{\defeq}{:=}
\newcommand{\eqdef}{=:}
\newcommand{\abs}[1]{|#1|}
\newcommand{\norm}[1]{||#1||}
\DeclareMathOperator{\im}{\mathrm{im}}
\DeclareMathOperator{\supp}{\mathrm{supp}}
\DeclareMathOperator{\vol}{\mathrm{vol}}
\newcommand{\Lp}{\Lambda^p}
%\renewcommand{\dots}{...}
\newcommand{\msrspc}{\ensuremath{(X,\mathcal{B},\mu)}}
\newcommand{\relmiddle}[1]{\mathrel{}\middle#1\mathrel{}}
\newcommand{\rmv}{\relmiddle|}
\newcommand{\stcmp}{^{\mathsf{c}}}
\newcommand\restr[2]{{% we make the whole thing an ordinary symbol
  \left.\kern-\nulldelimiterspace % automatically resize the bar with \right
  #1 % the function
  \vphantom{\big|} % pretend it's a little taller at normal size
  \right|_{#2} % this is the delimiter
  }}
\newcommand{\contr}{\Rightarrow\Leftarrow}
\newcommand{\interior}[1]{%
  {\kern0pt#1}^{\mathrm{o}}%
}
\newcommand{\sm}{\setminus}

% Spaces
\newcommand{\ktor}{\mathbb{T}^k}
\newcommand{\R}{\mathbb{R}}
\newcommand{\C}{\mathbb{C}}
\newcommand{\Z}{\mathbb{Z}}
\newcommand{\N}{\mathbb{N}}

% Derivatives
\newcommand*{\pd}[3][]{\ensuremath{\frac{\partial^{#1} {#2}}{\partial {#3}^{#1}}}}
\newcommand{\grad}{\bigtriangledown}

% Vectors
\newcommand{\mv}[1]{\textbf{#1}}

%}}}
%{{{ Enviornments
% Definitions environment
\newenvironment{defin}
	{\begin{mdframed}[backgroundcolor=white, roundcorner=5pt, linewidth=1pt, linecolor=RoyalBlue]
		\setlength{\parindent}{0pt}}
	{\end{mdframed}}
	\newcommand{\mdf}[1]{{\color{RoyalBlue} #1}}

% Important notes environment
\newenvironment{note}
	{\begin{mdframed}[backgroundcolor=white, linecolor=RubineRed, roundcorner=5pt, linewidth=1pt]\bfseries{Note:}\normalfont
	\setlength{\parindent}{0pt}}
	{\end{mdframed}}

% Examples enviornmnet
\definecolor{mylg}{rgb}{0.9,0.9,0.9}
\newenvironment{eg}
	{\begin{mdframed}[backgroundcolor=mylg,roundcorner=5pt,linewidth=0pt]\bfseries{Example:}\normalfont
	\setlength{\parindent}{0pt}}
	{\end{mdframed}}

% Theorem environment
\newtheorem{theorem}{Theorem}[section]
\newtheorem{cor}[theorem]{Corollary}
\newtheorem{lemma}[theorem]{Lemma}
\newtheorem{prop}[theorem]{Proposition}
%}}}
%{{{ Document metadata
\title{Manifolds Notes on things I don't understand/remember}
\author{Thomas Chaplin, Nikki Easton}
\date{}
%}}}

\begin{document}
\maketitle
\tableofcontents
\newpage

\section{The Inverse Function Theorem}

\begin{theorem}[for $\R^n$]
Let $f:\R^n \to \R^n$ and $x\in \R^n$, then if $d_xf:\R^n \to \R^n$ has rank $n$ (i.e. $d_xf$ is invertible) then there are open subsets $U, V \subseteq \R^n$ such that $x\in U$ and
\[
	\restr{f}{U}: U \to V \text{ is a diffeomorphism}
\]
\end{theorem}

\begin{theorem}[for manifolds]
Assume that $M$ and $N$ are $n$-manifolds and $f:M \to N$ is a smooth map.
Suppose that $x\in M$ and $d_xf:T_xM \to T_{fx}N$ is invertible.
Then there are open sets $U_1\subseteq M$, $U_2\subseteq N$ with $x\in U_1$ such that
\[
	\restr{f}{U_1}:U_1 \to U_2 \text{ is a diffeomorphism}
\]
\label{th:iftmani}
\end{theorem}

\begin{proof}
Firstly, take $x\in M$ and write $y\defeq f(x)$.
Then we can take charts $\phi_1: U_1 \to V_1$ and $\phi_2: U_2 \to V_2$ where $x\in U_1$ and $y\in U_2$.
After restricting $U_1$ to $U_1\cap f^{-1}U_2$ we may assume that $f(U_1) \subseteq U_2$.
Write $a\defeq \phi_1(x)$.
We then construct the map
\[
	\psi \defeq \phi_2 \circ f \circ \phi_1^{-1}: V_1 \to V_2
\]
By the chain rule we have
\[
	d_a\psi = d_y\phi_2 \circ d_x f \circ (d_x\phi_1)^{-1}
\]
We have already seen that the derivatives of charts are isomorphism so we just have to show that $d_a\psi$ and $d_x f$ are similar and thus have the same rank.
To see this, note that $d_a\psi$ is invertible so we can apply the old AFT.

This gives us open sets $V_1' \subseteq V_1$ and $V_2' \subseteq V_2$ with $a\in V_1'$ so that
\[
	\restr{\psi}{V_1'}:V_1' \to V_2' \text{ is a diffeomorphism}
\]
Let $U_1'\defeq \phi_1^{-1}(V_1')$ and $U_2'\defeq \phi_2^{-1}(V_2')$ so then $\restr{f}{U_1'}:U_1'\to U_2'$ is a diffeomorphism.
\end{proof}

\begin{lemma}
\label{le:f2id}
Under the hypothesis of Theorem \ref{th:iftmani} there are charts $\phi_1: U_1 \to V$ and $\phi_2: U_2 \to V$ with $v\subseteq \R^n$ such that $x\in U$, $f(x)\in U_2$ and $\phi_1 = \phi_2 \circ f$
\end{lemma}

\begin{proof}
Note that $\psi\circ \phi_1 : U_1 \to V_2$ is a chart and $\psi \circ \phi_1 = \phi_2 \circ f$.
So $\psi \circ \phi_1$ and $\phi_2$ do the job.
\end{proof}

\subsection{Immersions and Submersions}

Note that Lemma \ref{le:f2id} says that if you choose the correct coordinates then $f$ is the identity.
However, we can also use this for maps between manifolds of the same dimension.
To generalise we introduce the following:

\begin{defin}
Suppose $m \leq n$ then
\begin{itemize}
	\item the \mdf{standard immersion} $\iota:\R^m \to \R^n$ is given by
		\[
			\iota(x_1, \dots , x_m) = (x_1, \dots, x_m, 0, \dots, ))
		\]
	\item the \mdf{standard submersion} $\sigma:\R^n \to \R^m$ is given by
		\[
			\sigma(x_1, \dots , x_n) = (x_1, \dots , x_m)
		\]
\end{itemize}
\end{defin}

Check your handwritten notes for the following:

\begin{lemma}
Let $L:V\to W$ be a linear map of maximal rank.
Then there are invertible linear maps $P:V\to \R^m$ and $Q:W\to \R^n$ such that $Q \circ L \circ P^{-1}$ is  the standard immersion or submersion.
\end{lemma}

\begin{theorem}[General IFT for maps on $\R^n$]
Suppose $U\subseteq \R^m$ is open and $f:U\to \R^n$ is a smooth map.
Suppose $c\in u$ and $d_c f$ has maximal rank.
Then there are open sets $U_1, V_1 \subseteq \R^m$ and $U_2, V_2 \subseteq \R^n$ with $c\in U_1\subseteq U$ and $f(U_1)\subseteq U_2$.
Then there are diffeomorphism $\theta_1: U_1 \to V_1$ and $\theta_2:U_2\to V_2$ such that $\theta_a(c)=0$ and
\[
	\theta_2 \circ f \circ \theta_1^{-1} : V_1 \to V_2
\]
is the restriction of the standard immersion/submersion of $\R^m$ into $\R^n$.
\end{theorem}

\begin{theorem}[Genreal IFT for maps betwen manifolds]
Suppose that $M$ and $N$ are manifolds and $f:M \to N$ is smooth.
Suppose $c\in M$ and $d_c f$ has maximal rank.
Then there are charts $\phi_1: U_1 \to V_1$ and $\phi_2:U_2 \to V_2$ around $c$ and $f(c)$ respectively such that $\phi_1(c)=0$ and
\[
	\phi_2 \circ f \circ \phi_1^{-1}: V_1 \to V_2
\]
is the restriction of the standard submersion/immersion.
\label{th:iftmanig}
\end{theorem}

\begin{defin}
We say a map $f:M\to N$ is a
\begin{itemize}
	\item \mdf{immersion} if $d_xf$ is injective for every $x\in M$.
	\item \mdf{submersion} if $d_xf$ is surjective for every $x\in M$.
	\item \mdf{embedding} if it is smooth and $f:M \to f(M)$ is a diffeomorphism.
		In this cases $f(M)$ is a \mdf{submanifold} of $M$.
\end{itemize}
A map $f:X \to Y$ between topological spaces is \mdf{proper} if
\[
	\forall K\subseteq Y\text{ compact} \quad f^{-1}(K) \text{ is compact}
\]
The image of a proper embedding a \mdf{proper submanifold}

Let $f:M \to N$ be a smooth map between manifolds.
We say that $x\in M$ is a \mdf{regular point} if $d_x f$ is surjective and a \mdf{critical point} otherwise.
A point $y\in N$ is a \mdf{regular value} if every point in $f^{-1}(y)$ is a regular points.
Otherwise $y$ is a \mdf{critical value}.
\end{defin}

\begin{lemma}
A proper, injective immersion is an embedding.
\end{lemma}

\begin{theorem}[Level Set Theorem]
If $f:M \to N$ is smooth and $y\in f(M)\subseteq N$ is a regular value then $f^{-1}(y)$ is a proper submanifold of dimension $(m-n)=\dim(M)-\dim(N)$.
In this case
\[
	T_x(f^{-1}(y)) = \ker d_xf
\]
\end{theorem}

\begin{proof}
Given $x\in f^{-1}(y)$ we need to find it a chart.
Choose charts as in Theorem \ref{th:iftmanig}.
Define $U\defeq f^{-1}(y) \cap U_1$.
This is set of points in $U_1$ whose first $n$ coordinates are $0$.
Let $\pi:\R^m \to \R^{m-n}$ be defined by $\pi_1(x_1, \dots, x_m) = (0, \dots, x_{n+1}, \dots , x_m)$.
Then set $\phi\defeq \pi\circ\phi_1:U\to\R^{m-n}$ which is our chart.

A vector in the tangent space is given by $\gamma'(0)$ for some $\gamma:I \to f^{-1}(y)$ with $\gamma(0)=x$.
Now $f\circ \gamma$ is constant so
\[
	d_xf(\gamma'(0)) = (f\circ \gamma)'(0) = 0 \implies \gamma'(0) \in \ker d_xf
\]
Then both spaces have dimension $(m-n)$ so they must be the same.
\end{proof}

\section{Tangent and Normal Bundles}

\begin{defin}
	The \mdf{tangent bundle} to an $m$-manifold $M\subseteq\R^n$ is
	\[
		\mdf{TM}\defeq \left\{ (x, v)\in M\times \R^n \rmv v \in T_xM\right\}
	\]
	We write $p:TM \to M$ for the projection sending $p(x, v)=x$.
	Note that $p^{-1}(x)=T_xM$.

	A \mdf{section} of $TM$ is a smooth map $s:M \to TM$ such that $p\circ s = id_M$.

	A \mdf{frame field} on $M$ is a family of $m$ sections $v_1, \dots , v_m $ such that $\left\{ v_1(x), \dots , v_m(x)\right\}$ forms a basis for $T_xM$ at each $x\in M$.
	Such a field is \mdf{orthonormal} if $\left\{ v_1(x) , \dots, v_m(x)\right\}$ forms an orthonormal basis for each $x\in M$ with respect to the dot product induced by embedding $T_xM$ in $\R^n$

	The \mdf{normal bundle} to $M$  is
	\[
		\nu(M , \R^n) \defeq \left\{ (x, v) \in M\times \R^n \rmv v \in (T_xM)^\perp\right\}
	\]
	We use the same projection map as with $TM$.
\end{defin}

Note that we can always find a local frame field by pulling back the standard basis from $\R^m$ using $(d_x\phi)^{-1}$.
Also, using the Gram-Schmidt process, if we have a global frame field then we can always find an orthonormal one.

\begin{prop}
$TM$ is a $2m$-manifold and $p$ is a submersion.
\end{prop}

\begin{prop}
$\nu(M, \R^n)$ is an $n$-manifold and $p$ is a submersion.
\end{prop}

\section{Orientations}
Let $V$ be a vector space of dimension $m> 0$.
Let $I(V)$ be the set of linear isomorphism $V\to \R^m$.
Given $\rho, \sigma \in I(V)$ we get a linear automorphism
\[
	(\sigma \circ \rho^{-1}) : \R^m \to \R^m
\]
Note that that that the determinant of $(\sigma \circ \rho^{-1})$ is non-zero.
We can define an equivalence relation $I(V)$ by
\[
	\sigma \sim \rho \iff \det(\sigma\circ \rho^{-1}) > 0
\]
We write $\mdf{Or(V)}\defeq I(V)/\sim$.
Thus $\abs{Or(V)}=2$.
In the special case $\dim (V) = 0$ we set $Or(V)\defeq \left\{ -1, +1\right\}$.
An \mdf{orientation} on $V$ is an element of $Or(V)$.
When equipped with an orientation, a vector space obtains the adjective \mdf{oriented}.

If we let $V$ be an oriented vector space of positive dimension.
A basis $v_1, \dots , v_m$ of $V$ determines an element of $I(V)$ by sending the basis to a standard basis of $\R^m$.
We then say $\left\{ v_i\right\}$ is \mdf{positively oriented} if this map lies in the class of the orientation.

If $V$ and $W$ are oriented vector spaces of the same dimension, then we say a linear isomorphism $L: V \to W$ is \mdf{orientation preserving} if it sends some positively oriented basis of $V$ to some positively oriented basis for $W$.
Otherwise we say $L$ is \mdf{orientation reversing}.

$\R^m$ is said to have the \mdf{standard orientation} if its orientation class contains the identity.
If $L$ is an automorphism of $\R^m$ with the standard orientation then $L$ is orientation preserving if and only if $\det(L) >0$.

\begin{note}
If $V$ and $W$ are vector spaces and $E= V \oplus W$ then orientations on $V$ and $W$ determine an orientation on $E$.
Likewise, orientation on any two of $\left\{ E, V, W\right\}$ determine an orientation on the remaining space.
\end{note}

\begin{defin}
Let $M$ be an $m$-manifold where $m>0$.
An \mdf{orientation} on $M$ is an assignment of an orientation to each tangent space $T_xM$ such that there is an atlas of charts $\left\{ \phi_\alpha : U_\alpha \to \R^m\right\}$ such that
\[
	\forall \alpha \quad \forall x\in U_\alpha \quad d_x\phi_\alpha:T_xM \to \R^m\text{ is orientation preserving.}
\]
When equipped with such an assignment, $M$ is called an \mdf{oriented manifold}.
We say $M$ is \mdf{orientable} if it admits an orientation.
\end{defin}

\begin{note}
We can obtain an `opposite' orientation on $M$ by post-composing every chart with a transposition.
\end{note}

Not every manifold is orientable however every manifold is locally orientable.
It is also easily seen that orientability is invariant under diffeomorphism.

\begin{eg}
	\begin{enumerate}
		\item $S^n \subseteq \R^{n+1}$ is orientable.
			Take the atlas given by stereographic projection.
			Then the transition function reverses orientation so we just post-compose one of the charts with any orientation-reversing diffeomorphism of $\R^n$.
		\item The directed product of any two orientable manifolds is orientable.
			For example, the Torus or cylinder.
		\item The M\"obius band is not orientable.
			In particular, it is not diffeomorphic to the cylinder.
	\end{enumerate}
\end{eg}

\begin{lemma}
A connected orientable manifold has exactly two orientations.
\end{lemma}

\subsection{Criterion for Orientability}
Suppose that $M\subseteq \R^n$ is a smooth manifold.
Given $x\in M$, 
\[
\R^n\cong T_x\R^n = T_xM \times T_xM^\perp
\]
Since $\R^n$ comes with the standard orientation, an orientation on $T_xM$ determines one on $T_xM^\perp$ (and vice versa).

Consider the case where $n=m+1$.
Then $T_xM^\perp\cong \R$.
So an orientation on $T_xM$, which gives an orientation on $T_xM^\perp$, gives us a unique $\kappa(x)\in T_xM^\perp$ with 
\begin{enumerate}
	\item $\norm{\kappa(x)}=1$
	\item $\left\{ \kappa(x)\right\}$ forms a positively oriented basis for $T_xM^\perp$.
\end{enumerate}

We can think of $\kappa(x)$ as the `outward' unit normal vector.
If $M$ is oriented then $\kappa(x)$ is defined everywhere.
Also $[x\mapsto \kappa(x)]$ is smooth and so $\kappa(x)$ is a normal field.

Conversely, if $\kappa$ is a global nowhere-vanishing normal field on $M$, then $\kappa$ gives rise to an orientation on $T_xM$ for all $x\in M$.
From this we can construct an orientable atlas through great pain.

\begin{theorem}
If $M\subseteq\R^{m+1}$ is an $m$-manifold, then the following are equivalent:
\begin{enumerate}[label=(\alph*)]
	\item $M$ is orientable.
	\item $M$ admits a nowhere-vanishing normal field.
	\item $M$ admits a unit normal field.
\end{enumerate}
\end{theorem}

\begin{defin}
Suppose that $M$ and $N$ or oriented manifolds and $f:M\to N$ is a diffeomorphism.
We say that $f$ \mdf{preserves orientation} if $d_xf:T_xM \to T_{fx}N$ respects the given orientation for all $x\in M$.
It \mdf{reverse orientation} if $d_xf$ reverses orientation for all $x\in M$.
\end{defin}

\section{Abstract Manifolds}

\begin{defin}
	\begin{itemize}
	\item A \mdf{topological $m$-manifold} $M$ is a non-empty, second countable, Hausdorff topological space such that every point has an open neighbourhood which is homeomorphic to some open set of $\R^m$.
	\item A \mdf{topological chart} for $M$ is a homeomorphism $\phi: U \to V$.
	\item A \mdf{topological atlas} for $M$ is a collection of charts whose domains cover $M$.
	\item A topological atlas is \mdf{smooth} if the transition functions are smooth.
	\item A \mdf{smooth structure} on a topological manifold is an equivalence class of smooth atlases where two atlases are equivalent if their union is also a smooth atlas.
	\item A \mdf{smooth manifold} is a topological manifold equipped with a smooth structure.
	\end{itemize}

	Let $M$ and $N$ be smooth manifolds.
	Then a map $f : M \to N$ is called \mdf{smooooooth} if 
	\[
\psi_\beta \circ g \circ \phi_\alpha^{-1}:\phi_\alpha(U_\alpha\cap f^{-1}U_\beta)\to \R^m
	\]
	is smooth for all combinations of charts $\alpha, \beta$.
\end{defin}

\begin{prop}
Two atlases are equivalent if and only if they determine the same set of smooth functions $M\to \R$.
\end{prop}

\subsection{Equivalent Definitions}
\begin{defin}
An \mdf{open cover} of a space $X$ is a collection $\mathcal{U}$ of open subsets of $X$ that union to the whole space.

A \mdf{refinement} of $\mathcal{U}$ is an open cover $\mathcal{V}$ of $X$ such that $\forall V\in \mathcal{V}$ there is a $U\mathcal{U}$ such that $V\subseteq U$.

We say that a cover $\left\{ U_\alpha\right\}$ of a Hausdorff space $X$ is \mdf{locally finite} if
\[
	\forall x \in X \quad \exists O\subseteq X\text{ open, s.t. }x\in O \; \text{ and } \; \left|\left\{ \alpha \rmv O\cap U_\alpha\neq\emptyset\right\}\right|< \infty
\]
that is around every point there is an open set which meets at most finitely many members of the cover.

We say $X$ is \mdf{paracompact} if every open cover has a locally finite refinement.
\end{defin}

\begin{lemma}
Any (abstract) manifold is paracompact.
\end{lemma}

\subsection{Tangent Spaces}

\begin{defin}
	Given $M$ a smooth $m$-manifold and $x\in M$, the \mdf{tangent space} to $M$ at $x$ is
	\[
		\mdf{T_xM}\defeq\left\{ v: \mathcal{G}_x(M) \to \R \text{ linear} \rmv v\text{ satisfies (L)}\right\}
	\]
	where $(L)$ is the Leibniz Rule:
	\[
		v\cdot(fg) = f(x) \; v \cdot g + g(x) \;v \cdot f \quad \forall f, g\in \mathcal{G}_x(M)
	\]
\end{defin}

\begin{lemma}[Hadamard's Lemma]
Given $a\in \R^m$ then every $f\in C^\infty_a (\R^m)$ has the form
\[
	f(x) = f(a) + \sum_{i=1}^{m}(x_i- a_i)g_i(x)
\]
for some locally defined functions $g_i\in C^\infty_a(\R^m)$.
\end{lemma}

We would like to show that this coincides with our previous definition.
To do this, we aim to find a basis of $T_a\R^m$.
Note, in Hadamard's Lemma, we necessarily have $g_i(a) = \pd{f}{x_i}(a)$.
Now, for each $i$, we have an element $d_i\in T_a\R^m$ defined by $d_i \cdot f = \pd{f}{x_i}(a)$ for all $f\in C^\infty_a\R^m$.
This gives us a linear map $\R^m \to T_a\R^m$ sending the standard basis to $\left\{ d_i\right\}$.

\begin{lemma}
$\left\{ d_1, \dots , d_m\right\}$ form a basis for $T_a\R^m$.
\end{lemma}

\begin{proof}
We can see that the $d_i$ are linearly independent as follows.
Let $\pi_i: \R^m \to \R$ be projection to the $i^{th}$ coordinate.
Then $d_i \cdot \pi_j = \delta_{ij}$.

To see that they span, choose some $v\in T_a\R^m$.
Then we can write $\lambda_i \defeq v \cdot \pi_i$ for each $i$.
First we consider constant functions on any open set $U\subseteq \R^m$.
We have that $v\cdot 1 = v\cdot (1\times 1) = 1(v\cdot 1) + 1(v \cdot 1) = 2(v\cdot 1)$ and hence $v\cdot 1$ = 0.
So by linearity, $v$ kills any constant function.

Now suppose that $f\in C_a^\infty \R^m$, then by Hadamard's Lemma we can write
\[
	f(x) = f(a) + \sum_{i=1}^m (\pi_i(x) - \pi_i(a)) g_i(x)
\]
so by the Leibniz rule we get
\[
	v \cdot f = 0 + \sum_{i=1}^m\left( (\pi_i(a) - \pi_i(a)) \; v\cdot g_i + g_i(a) \; v\cdot \pi_i\right) = \sum_{i=1}^m \lambda_i g_i(a) = \sum_{i=1}^m \lambda_i \pd{f}{x_i}(a) = \sum_{i=1}^m \lambda_i (d_i \cdot f)
\]
Thus $v=\sum_{i}\lambda_i d_i$.
\end{proof}

To apply this to smooth manifolds we need a derivative map.
If $M$ and $N$ are smooth manifolds, $x\in M$ and $\theta: M \to N$ is a diffeomorphism then we can construct a linear map.
\[
	\mdf{d_x\theta}:T_xM \to T_{\theta x}N \quad \quad \mdf{(d_x\theta)(v)}\cdot f \defeq v \cdot (f\circ \theta)
\]
for all $f\in\mathcal{G}_{\theta x}(N)$.
One needs to check that $(d_x\theta)(v)$ satisfies the Leibniz rule and so does indeed lie in $T_{\theta x}N$

\begin{proof}
\begin{align*}
	(d_x\theta)(v) \cdot (fg) &= v \cdot \left[ fg \circ \theta\right] \\
							  &= v\cdot \left[ (f \circ \theta) (g\circ \theta)\right] \\
							  & = f(\theta(x)) \; v\cdot (g\circ \theta) + g(\theta(x)) \; v \cdot (f\circ \theta)\\
							  & = f(\theta(x)) \; (d_x\theta)(v) \cdot g + g(\theta(x)) \; (d_x\theta)(v) \cdot f
\end{align*}
\end{proof}

Now since $\theta$ is a diffeomorphism, $d_x\theta$ is a linear isomorphism.
Then by choosing some chart $\phi : U \to \R^m$ we get a linear isomorphism $d_x\phi: T_xU \to T_a\R^m$ where $a=\phi x$.
But $T_xU$ is naturally identified with $T_xM$ and we have seen that $T_a\R^m\cong \R^m$.

\begin{theorem}
If $M$ is a smooth manifold and $x\in M$ then
\begin{enumerate}
	\item $\dim (T_xM)=m$.
	\item If $v\in T_xM$ then there is a smooth curve $\gamma:I \to M$ such that $\gamma(0) = x$ and $\hat{\gamma}'(0)=v$ where
		\[
			\hat{\gamma}'(0) \cdot f \defeq (f\circ \gamma)'(0).
		\]
	\item If $M$ is a smooth submanifold of $\R^m$ then there is a natural identification between this definition of $T_xM$ and our previous one such that if $\gamma: I \to M$ is a smooth curve with $\gamma(0)=x$ then $\gamma'(0)$ gets identified with $\hat{\gamma}'(0)$.
\end{enumerate}
\end{theorem}

\begin{proof}
\begin{enumerate}
	\item Obvious by the above linear isomorphism $d_x\phi$.
	\item Let $\phi: U \to V$ be a chart with $x\in U$ such that $d_x\phi:T_xM \to \R^m$ is a linear isomorphism.
		Let $\delta: I \to V$ be a curve with $\hat{\delta}'(0) = d_x\phi(v)$.
		Then set $\gamma \defeq \phi^{-1} \circ \delta$.
	\item The only thing to prove is that $\hat{\gamma}'(0)$ does not change if we switch from a curve $\gamma$ to some other curve $\sigma$ such that $\sigma'(0) = \gamma'(0)$ (and vice versa).
\end{enumerate}
\end{proof}

\section{Vector Bundles}
In defining a vector bundle over a manifold we associate a vector space to each point in the manifold and then arrange these together in a smooth way.
Examples of this will include the tangent and normal bundles as defined in times gone past.

Suppose that $M$ is an $m$-manifold.
A \mdf{family of vector spaces} over $M$ is a manifold $E$ together with a smooth map $p:E\to M$ such that for all $x\in M$, the \mdf{fibre} $\mdf{E_x}\defeq p^{-1}(x)$ has a vector space structure.
Given $U\subseteq M$ we write $\restr{E}{U}\defeq p^{-1}U$ which we equip with the restriction of $p$.

Given another family over vector spaces $p':E'\to M$ we define an \mdf{isomorphism} to be a diffeomorphism 
\[
\psi:E\to E'
\]
with the property that $p'\circ\psi = p$ and $\psi:E_x\to E'_x$ restricts to  a linear isomorphism for all $x\in M$.
If such a map exists then $E$ and $E'$ are called \mdf{isomorphic}.
Finally, a family $E$ over $M$ is called \mdf{trivial} if it is isomorphic to $M\times \R^q$ for some $q\in\N$.

\begin{figure}[H]
	\centering
	\begin{tikzcd}
		E' \arrow[rr, "\psi"] \arrow[rd, "p'"']& & E \arrow[dl, "p"] \\
											  & M
	\end{tikzcd}
\end{figure}

\begin{defin}
	A \mdf{vector bundle} over $M$ is a family of vector spaces $p:E\to M$ such that $\forall x\in M$ there is some open $U\subseteq M$ with $x\in U$ such that $\restr{E}{U}$ is trivial.
\end{defin}

Putting this another way, $E$ is a vector bundle over $M$ if we can find an atlas $\left\{\phi_\alpha : U_\alpha \to V_\alpha \right\}$ for $M$ such that for each $\alpha$ we get a diffeomorphism
\[
	\psi_\alpha: \restr{E}{U_\alpha}\to V_\alpha\times\R^q\subseteq\R^{m+q}
\]
which restricts to a linear isomorphism on each fibre.
Note than then the maps $\left\{\psi_\alpha\right\}$ form an atlas for $E$ which is called a \mdf{locally trivialising atlas} for $E$.
\begin{note}
	\begin{itemize}
		\item Given a vector bundle, $p:E\to M$ is a surjective smooth submersion.
		\item Each fibre $E_x$ is an embedded sub-manifold and the vector space operations are smooth.
		\item The notion of a vector bundle is preserved under isomorphism.
	\end{itemize}
\end{note}

\begin{eg}
\begin{enumerate}
	\item If $M$ is a smooth manifold then the tangent bundle is a vector bundle over $M$.
		For a manifold in Euclidean space, this follows for the proof that $TM$ was a manifold.
		For an abstract space, we can form a locally trivialising atlas as follows:

		Given a chart $\phi:U\to V\subseteq\R^m$ for $M$ we construct a new map
		\[
			\psi:TU\to V\times\R^m \quad\text{by}\quad\psi(v)=(\phi(x),d_x\phi(v))
		\]
		where $x=pv$.
		These maps then form a locally trivialising atlas because when restricted to some $x$ the map becomes the derivative map
		\[
			\psi_{p^{-1}(x)}:T_x M \to \left\{\phi(x)\right\}\times\R^m \quad \psi(v)=(\phi(x), d_x\phi(v))
		\]
		which is a linear isomorphism.
	\item One can (and perhaps should) check that the atlas given for the normal bundle consists of maps which restrict to linear isomorphisms on each fibre.

	\item The spaces $B_n$ together with the projection to the circle yields a vector bundles.
		In particular, the M\"obius band is a bundle over the circle.
		Note $B_m$ is isomorphic to $B_n$ if $\abs{m-n}$ is even.
		But, $B_0$ cannot be isomorphic to $B_1$ since they are not even diffeomorphic.
		Hence the M\"obius band is non-trivial.
\end{enumerate}
\end{eg}
Given a vector bundle $p:E\to M$ we say that \mdf{section} of $E$ is a smooth map $s:E\to M$ such that $p\circ s$ is the identity on $M$.
This generalises our previous notion of a section of the tangent space which is often referred to as a \mdf{vector field} on $M$.
We say that a section is \mdf{non-vanishing} if $s(x)\neq 0$ for all $x\in M$.
Given any smooth map $f:M\to\R^q$ we get a section of $M\times\R^q$.
\begin{note}
Any section is an embedding of $M$ into $E$.
\end{note}

\begin{theorem}[Hairy Ball Theorem]
Every vector field on $S^2$ has at least one zero.
\end{theorem}

\begin{note}
	It turns out that any odd-dimensional manifold admits a non-vanishing vector field.
	An even-dimensional manifold may or may not.
	The torus does but it is the only compact connected orientable 2-manifold which does.
\end{note}

\begin{lemma}
A vector bundle $E$ is trivial if and only if there are sections $s_1, s_2, \dots, s_q$ such that $\left\{s_1, \dots, s_q(x)\right\}$ forms a basis for $E_x$ for all $x\in M$.
\end{lemma}
\begin{proof}
If $f:M\times \R^q$ is an isomorphism then set $s_i(x)=f(x, e_i)$ for some fixed basis $e_1, \dots , e_q$ for $\R^q$.

Conversely, if we have such sections $s_i$ then we can define a map $f:M\times \R^q \to E$ by
\[
	f(x, (\lambda_1, \dots , \lambda_q))=\sum_{i=1}^{q}\lambda_i s_i(x)
\]
which yields an isomorphism.
\end{proof}
\begin{eg}
	\begin{enumerate}
		\item Using the Intermediate Value Theorem, one can show that the M\"obius has no non-vanishing section and is hence a non-trivial bundle by the above Lemma.
		\item Let $S^n\subseteq\R^{n+1}$ be the unit sphere in $\R^n$ then the normal bundle is trivial because the outward pointing unit normal is a nowhere vanishing section.
	\end{enumerate}
\end{eg}

\begin{defin}
	A manifold $M$ is called \mdf{parallelisable} if the tangent bundle $TM$ is trivial.
\end{defin}

In view of the last lemma we can say that a manifold is parallelisable if and only if $M$ admits a global frame field.
Recall, a frame field is a family of vector fields $v_i$ such that the $v_i(x)$ form a basis for $T_x M$ at every $x\in M$.
We know that frame fields exists locally but not necessarily globally.

\begin{lemma}
Any parallelisable manifold is orientable.
\end{lemma}
\begin{proof}
Exercise.
\end{proof}

\begin{eg}
	\begin{itemize}
		\item The circle $S^1$ is parallelisable.
		\item $S^2$ is not by the Hairy Ball Theorem.
		\item $S^3$ is parallelisable because given $x=(x_1, x_2, x_3, x_4)\in S^3$ we can form a frame field with
			\[
			\begin{array}{rrrr}
				(-x_2,& x_1,& x_4,& -x_3)\\
				(-x_3,&- x_4,& x_1,& x_2)\\
				(-x_4,& x_3,& -x_2,& x_1)
			\end{array}
			\]
		\item $S^5$ is not strangely enough but $S^7$ is.
			It may be worth consulting your written notes from lectures to figure out how this frame was calculated.
	\end{itemize}
\end{eg}

\subsection{Vector Fields}

Let $X$ be a vector field on $M$ (that is a section of $TM$).
Suppose that $f\in C^\infty(M)$.
We can define a function $Xf:M\to \R$ as
\[
	(Xf)(x)\defeq X\cdot [f]
\]
where $[f]\in\mathcal{G}_x(M)$.

\noindent\textbf{Claim: }Xf as defined above is a smooth function

Let $x_i$ be local coordinates of a neighbourhood $U\subseteq M$ of $x\in M$.
Then we write $X=\sum_{i}\lambda_i \pd{}{x_i}$ where $\lambda_i: U \to \R$ are smooth functions.
So we have $Xf=\sum_{i}\lambda_i \pd{f}{x_i}$ which is smooth.

One can also see that $[f \mapsto Xf]$ is a linear map $C^\infty(M)\to C^\infty (M)$.
Directly from our definition of $T_xM$ we have that
\[
	X(fg) = fXg + gXf
\]
for all $f,g \in C^\infty(M)$.

\begin{prop}
\label{pr:LtoX}
Suppose $L:C^\infty(M)\to C^\infty(M)$ is a linear map satisfying
\[
	L(fg) = fL(g) + gL(f)
\]
then there is a unique vector field $X$ on $M$ such that $L(f)=Xf$ for all $f\in C^\infty(M)$.
\end{prop}

\begin{proof}
Probably worth going through this proof.
\end{proof}

\begin{defin}
	Suppose $X, Y$ are vector fields on $M$.	
	One can check that
	\[
		[f \mapsto X(Yf) - Y(Xf)]
	\]
	satisfies the conditions of Proposition \ref{pr:LtoX}.
	It follows that there is a unique vector field, denoted $\mdf{[X, Y]}$, on $M$ such that
	\[
		[X, Y]f = X(Yf) - Y(Xf)
	\]
	for all $f\in C^\infty(M)$.
	This vector field is called the \mdf{Lie bracket}.
	Manifolds is full of lies. $\qed$ 
\end{defin}

\subsection{Whitney Sums}
Suppose we have two vector bundles
$p:E \to M$ and $p':E'\to M$.
We'd like to sum these vector bundles.
Each fibre is a vector space so we can take their sums and stitch the fibres back together
\[
	\hat{E} \defeq \bigsqcup_{x\in M}(E_x \oplus E_x')\quad\quad \hat{p}:\hat{E}\to M
\]
where $\hat{p}$ is the obvious map.
We need to give this a manifold structure.

Let $\left\{ \phi_\alpha: U_\alpha \to V_\alpha\right\}_{\alpha\in\mathcal{A}}$ be an atlas for $M$ which gives rise to locally trivialising atlases\[
	\left\{ \psi_\alpha : p^{-1}U_\alpha \to V_\alpha \times \R^q\right\}_\alpha \quad \text{and} \quad \left\{ \psi_\alpha':p'^{-1}U_\alpha \to V_\alpha \times \R^r\right\}_\alpha
\]
for $E$ and $E'$ respectively.
The first coordinate of these charts is always $\phi_\alpha(x)$.
If we want to stitch two charts $\psi_\alpha$ and $\psi_\alpha'$ together, we only need one copy of this first coordinate.

Given $v\in E_x \oplus E_x'$ write $v=u+u'$ for $u\in E_x$ and $u'\in E_x'$.
Let $w\in \R^q$ and $w'\in\R^r$ be the second coordinates of $\psi_\alpha(u)$ and $\psi_\alpha'(u')$ respectively.
Note we can be sure this is the same $\alpha$ because it only depends on $x$.
We then form a chart for the Whitney sum as follows
\[
	\theta_\alpha:(\hat{p})^{-1}U_\alpha \to V_\alpha \times \R^q \times \R^r \quad \quad \theta_\alpha(v)=(\phi_\alpha(x), w, w')
\]
\noindent\textbf{Topology: }

We deem a set $O\subseteq\hat{E}$ to be open if $\theta_\alpha(O)$ is open in $V_\alpha\times\R^{q+r}$ for every chart $\alpha$.

\begin{note}
	One can check this does indeed give us a topology for $\hat{E}$, endows it with the structure of a vector bundle, and the resulting structure is independent of all the atlases we chose.
\end{note}

\begin{defin}
	The resulting vector bundle $\hat{E}$ is called the \mdf{Whitney sum} of the vector bundles $E$ and $E'$.	
	We often denote this $E\oplus E'$.
\end{defin}

\subsubsection{Alternative approach}
Up to isomorphism, we can construct the Whitney sum in an alternative fashion.
This time we start with the full direct product $E\times E'$ and consider the submanifold
\[
	F \defeq \left\{ (v, w) \in E\times E' \rmv pv = p'w\right\}
\]
There is a map $\widetilde{p}:F\to M$ where $(v,w)\mapsto pv=p'w$.
Under this map $\widetilde{p}^{-1}(x) = E_x \times E_x'$
Since this is just a direct product of vector spaces, there is a natural identification $E_x\oplus E_x' \to E_x \times E_x'$.
Combining these identifications across the Whitney sum we get a bijection $E\oplus E'\to F \subseteq E\times E'$.
This can be shown to be an embedding of manifolds and that these constructions are isomorphic as vector bundles.

\begin{note}
While $E_x\times E_x'$ and $E_x \oplus E_x'$ are essentially the same, $E\times E'$ and $E\oplus E'$ are vastly different!
\end{note}

It might be worth be looking at how we stitch together dual spaces.

\subsection{Cotangent Bundle}
\begin{defin}
We will often be interested in the dual to the tangent bundle $(TM)^\ast$ which is more usually denoted $T^\ast M$.
Its fibres are of the form $(T_xM)^\ast$ but more commonly written $T_x^\ast M$.
An element of $T_x^\ast M$ is called a \mdf{covector} at $x$.
A section of $T^\ast M$ is called a \mdf{covector field}, but more frequently referred to as a \mdf{1-form}.
\end{defin}

Suppose that we have a smooth function $f\in C^\infty(M)$.
Then $f$ determines an element of $T_x^\ast M$ given by the linear map $[v \mapsto v \cdot f]$.
This element is denoted $\mdf{df(x)}\in T_x^\ast M$.
So the map $[f\mapsto df(x)]:M\to T^\ast M$ is a section of $T^\ast M$, i.e. a 1-form.
Note that we get the product rule
\[
	d(fg) = f(dg) + g(df)
\]

To better understand $T_x^\ast M$ we find a basis.
Take a chart $\phi:U \to V$ on the underlying manifold such that $x\in U$.
We get 1-forms $dx_1, \dots , dx_m$ defined by
\[
	dx_i \left(\pd{}{x_j}\right)=\delta_{ij}
\]
These $dx_i$ form a basis for $T_x^\ast M$.
This is, in fact, the dual basis to $\pd{}{x_1}, \dots, \pd{}{x_m}$ for $T_xM$.
Note that, if $f$ is a smooth function then we have
\[
	df = \sum_{i}\pd{f}{x_i}dx_i
\]
Suppose $\left\{ \phi_\alpha: U_\alpha \to V_\alpha\right\}_\alpha$ is a chart for $M$.
Then for each $\alpha$ we get 1-forms $dx_1^\alpha, \dots, dx_m^\alpha$.
On the overlap $U_\alpha \cap U_\beta$, these 1-forms transform according to the rule
\[
	dx_i^\alpha = \sum_{j=1}^m \pd{x_i^\alpha}{x_j^\beta}dx_j^\beta
\]
Note that $\left( \pd{x_i^\alpha}{x_j^\beta}\right)_{i, j}$ is the Jacobian of the transition function $\phi_\alpha \circ \phi_\beta^{-1}$.

\noindent\textbf{Integration: }

1-forms can be used to integrate along curves.
Suppose that $\gamma:[a, b] \to M$ is a smooth curve and $\omega$ is a 1-form on $M$.
Then we get a smooth function $[t \mapsto \omega(\gamma(t))(\gamma'(t))]:[a, b] \to \R$.
Then we write
\[
	\mdf{\int_\gamma \omega}\defeq\int_a^b \omega(\gamma(t))(\gamma'(t))\; dt
\]

\subsection{Pull-backs}
Suppose we have a smooth function of manifolds $f:M \to N$.
Given a 1-form $\omega$ on $N$ we can define a 1-form $\eta$ on $M$ as follows:
Given $x\in M$, write $\eta(x)(v)\defeq\omega(f(x))(f_\ast(v))$ where $f_\ast(v)\defeq d_xf (v)$.
Thus $\eta(x)\in T_x^\ast M$.

This gives us a map $\eta: M \to T^\ast M$ which is smooth and hence a section to $T^\ast M$.
In other words, $\eta$ is a 1-form on $T^\ast M$ which is called the \mdf{pull-back} of $\omega$.

\section{Smooth Function Extension}

\begin{lemma}
There is a smooth function $\theta_0: \R^n \to [0, 1]$ such that
\begin{enumerate}
	\item $\theta_0(x) = 1$ whenever $\norm{x} \leq 1$.
	\item $\theta_0(x) = 0$ whenever $\norm{x} \geq 2$.
\end{enumerate}
\end{lemma}

\begin{proof}
This is done in a series of steps:
\begin{enumerate}
	\item Define $\theta_1: \R \to \R$ by
		\[
			\theta_1(t) \defeq
			\begin{cases}
				e^{-1/t} & \text{for }t\geq 0 \\
				0 		& \text{for } t< 0
			\end{cases}
		\]
		which is smooth.

	\item Set $\theta_2(t) \defeq \frac{\theta_1(t)}{\theta_1(t) + \theta_1(1-t)}$.
		So $\restr{\theta_2}{(1, \infty)} \equiv 1$ and $\restr{\theta_2}{(-\infty, 0]}\equiv 0$.
	\item Define $\theta_0(x) \defeq \theta_2(2- \norm{x})$.
\end{enumerate}
\end{proof}

\begin{lemma}
Let $M$ be an $m$-manifold and $W\subseteq M$ open such that $x\in W$.
Then there is a smooth function $\theta: M\to [0, 1]$ such that $\restr{\theta}{M\setminus W}\equiv 0$ and $\restr{\theta}{U}\equiv 1$ for some open neighbourhood $U$ of $x$.
\end{lemma}

\begin{proof}
Let $\phi:U \to V$ with $x\in U$.
After pre-composing with a translation we can assume $0\in\phi (W\cap U)$.
After post-composing with a dilation we can assume $\mathbb{B}_2(0)\subseteq \phi(W\cap U)$.
Now set 
\[
\theta(x) \defeq 
\begin{cases}
	\theta_0(\phi(x)) & \text{for }x\in U \\
	0 				  & \text{otherwise}
\end{cases}
\]
\end{proof}

\begin{cor}
Suppose $f:W \to \R$ is a smooth function on some open $W\subseteq M$ with $x\in W$.
Then there is a smooth function $g:M \to \R$ which agree with $f$ on some neighbourhood of $x$ in $W$.
\end{cor}
\begin{proof}
\[
	g(x)\defeq 
\begin{cases}
	f(x)\cdot\theta(x) & \text{for }x\in W \\
	0 & \text{otherwise}
\end{cases}
\]
\end{proof}



\section{Differential Forms}
\begin{defin}
Given vector spaces $E, F$ and $p\in \N$ we denote
\[
	\mdf{A(E^p, F)}\defeq \left\{p-\text{linear alternating maps }E^p\to F\right\}
\]
where by alternating we mean that swapping any two coordinates negates the output.
Equivalently, if two coordinates are the same then the output is $0$.
\end{defin}

\begin{lemma}
Given $E$ and $p$ there is a vector space $V$ together with a surjective map $\mu\in A(E^p, V)$ with the property that if $\theta\in A(E^p, F)$ then there is a linear map $\hat{\theta}:V\to F$ such that $\theta = \hat{\theta}\circ\mu$
\end{lemma}

\begin{proof}
Go through this!
\end{proof}

\begin{figure}[H]
	\centering
	\begin{tikzcd}
		\; & V \arrow[rd, "\hat{\theta}", dashed] & \; \\
		E^p \arrow[ru, "\mu", two heads] \arrow[rr, "\theta"']& \; & F
	\end{tikzcd}
\end{figure}

\begin{note}
	The $\hat{\theta}$ is unique given $\theta$ and $V$.
	The $V$ is unique up to isomorphism.
\end{note}

\begin{defin}
We write $V = \Lambda^p E$ and given $v_1, \dots , v_p\in E$ we write
\[
	v_1 \wedge \dots \wedge v_p \defeq \mu(v_1, \dots ,v_p)
\]
We say $\Lambda^p E$ is the \mdf{$p$-th exterior power of $E$}.
\end{defin}

\subsection{Basis for $\Lambda^p E$}
Let $e_1, \dots , e_m$ be a basis for $E$.
Since $\mu$ is surjective $\Lambda^p E$ is spanned by
\[
	\left\{ e_{i_1} \wedge \dots \wedge e_{i_p} \rmv i_k\in I(m) \right\}
\]
where we can assume that the $i_k$ are distinct else their image would be null.
We can also assume that the indices are in order up to sign.

\begin{lemma}
These elements are linearly independent and hence form a basis.
\end{lemma}

Therefore we can say $\dim(\Lambda^p E) = \binom{m}{p}$.

\subsection{Wedge Product}
Given $p, q\in\N$ with $p, q \geq 1$ we can define the bilinear wedge product
\[
	\cdot \wedge \cdot : (\Lambda^p E \times \Lambda^q E) \to \Lambda^{p+q}E
\]
First we define on it on a basis.
So take a basis $e_1, \dots , e_m$ of $E$ and then define
\[
	(e_{i_1} \wedge \dots \wedge e_{i_p}) \wedge (e_{j_1} \wedge \dots \wedge e_{j_q}) = e_{i_1} \wedge \dots \wedge e_{i_p} \wedge e_{j_1} \wedge \dots \wedge e_{j_q}
\]
This can then be extended linearly to arbitrary elements and hence doesn't depend on our initial choice of basis.

\subsection{Induced maps}
Suppose we have a linear map between finite dimensional vector spaces
\[
\phi : E \to F
\]
then we get a multi linear map in the natural way
\[
\phi^p : E^p \to F^p
\]
By composing with the surjective map $\mu_F$ we get an alternating map
\[
\mu_F \circ \phi^p : E^p \to \Lambda^p F
\]
Hence by the defining property of $\Lambda^P E$ we get a linear map
\[
\Lambda^p \phi : \Lambda^p E \to \Lambda^p F
\]
with the property that the outer diamond in the below diagram commutes.

\begin{figure}[H]
	\centering
	\begin{tikzcd}
		\; & \Lambda^p E \arrow[rd, dashed, "\Lambda^p \phi"] & \; \\
		E^p \arrow[rr, "\mu_f \circ \phi^p"'] \arrow[rd, "\phi^p"'] \arrow[ru, "\mu_E"] & \; & \Lambda^p F \\
		\; & F^p \arrow[ru, "\mu_F"'] & \;
	\end{tikzcd}
\end{figure}

Essentially, if $e_1, \dots , e_m$ is a basis for $E$, then we can describe $\Lambda^p\phi$ by
\[
	(\Lambda^p\phi)( e_{i_1}\wedge \dots \wedge e_{i_p}) = (\phi e_{i_1}) \wedge \dots \wedge (\phi e_{i_p}).
\]

\subsection{$\skull$ The dreaded p-form $\skull$}

\begin{defin}
Let $M$ be an $m$-manifold.
Given $x\in M$ we can form the $p$-th exterior power of the cotangent space
\[
	\Lambda^p (T_x^\ast M)
\]
We can assemble these together into a vector bundle $\Lambda^p(T^\ast M)$.
Subsequently, a \mdf{$p$-form} on $M$ is define to be a section of the bundle $\Lambda^p(T^\ast M)$	
\end{defin}

\noindent\textbf{What on earth does this mean???}

A more natural way to think about $p$-forms is to take local coordinates.
Let $\phi: U \to \R^m$ be a chart yielding local coordinates $x_1, \dots , x_m$.
We have locally defined $1$-forms $dx_1, \dots , dx_m$ which form a basis for the cotangent space
\[
	dx_i\left(\pd{}{x_j}\right)=\delta_{ij}
\]
Then given $I\in\mathcal{I}(m, p)$ we write $\mv{d}x_I\defeq d{x_{i_1}} \wedge \dots \wedge d{x_{i_p}}$.
Thus $\left\{ \mv{d}x_I \rmv I \in\mathcal{I}(m, p)\right\}$ forms a basis for $\Lambda^P(T^\ast M)$.
It follows that any $p$-form $\omega$ on $U$ can be uniquely written in the form
\[
	\omega = \sum_{I\in\mathcal{I}(m,p)}\lambda_I\mv{d}x_I
\]
where each $\lambda_I:U \to \R$ is a locally-defined smooth function.
\begin{note}
	This is all we really need from the bundle structure of $\Lambda^p(T^\ast M)$.
\end{note}

In particular, if $p=m$ then an $m$-form locally looks like
\[
	\lambda \; (dx_1 \wedge \dots \wedge dx_m)
\]
for some smooth function $\lambda: U \to \R$.

\subsection{Pull-backs}
Suppose we have a smooth function between manifolds
\[
f:M \to N
\]
Given a $p$-form $\omega$ on $N$ we can define a \mdf{pull-back $p$-form $f^\ast\omega$} on $M$ as follows. 
Given $x\in M$ we have the derivative map $d_xf$ and hence a dual map
\[
	(d_x f)^\ast : T^\ast_{f(x)}N \to T^\ast_x M , \quad \quad \eta \mapsto \eta \circ d_xf \quad \text{where }\eta:T_{f(x)}N \to \R\text{ is linear}
\]
This in turn gives rise to a linear map
\[
	\Lp(d_x\phi)^\ast : \Lambda^pT_{fx}^\ast N \to \Lp T_x^\ast M
\]
Then our pull-back is defined by
\[
	(f^\ast\omega)(x) \defeq (\Lp(d_x\phi)^\ast)\left[\omega(f(x))\right]
\]
One takes on blind faith that this is smooth and hence a $p$-form.
In particular, we can pull back $p$-forms to any manifold embedded within a larger manifold (such as $\R^n$).
This is a load of gobbledygook so let's go step by step.
\begin{enumerate}
	\item $x\in M$
	\item $f(x)\in N$
	\item $\omega$ is a $p$-form on $N$ so we get some linear maps $\eta_i:T_{fx}N\to\R$, then
		\[
			\omega(f(x))=\eta_1 \wedge \dots \wedge \eta_p
		\]
	\item Then we take the induced $p$'th exterior power map which just does $(d_x\phi)^\ast$ on each of the $\eta_i$
	\item Hence we can write
		\[
			(f^\ast\omega)(x) = (\eta_1 \circ d_xf) \wedge \dots \wedge (\eta_p \circ d_xf)
		\]
\end{enumerate}

\begin{eg}
	\begin{enumerate}
		\item Consider $S^1$; the unit circle in $\R^2$.
			Let $\theta$ by the angle coordinate so that $x=\cos\theta$ and $y=\sin\theta$.
			Then the pull back of $dx$, $dy$ is obtained by differentiating these formulae:
			\[
				-\sin\theta d\theta \quad\text{and}\quad cos\theta d\theta
			\]
			From this we can pull back an arbitrary 1-form by linear extension.
		\item Consider $S^2$; the unit 2-sphere in $\R^3$.
			Consider spherical polar coordinates $\theta, \phi$ away from the poles.
			\begin{align*}
				x &= \sin\theta\cos\phi \\
				y &= \sin\theta\sin\phi \\
				z &= \cos\theta
			\end{align*}
			Then the pull backs of $dx$, $dy$ and $dz$ repressively are
			\begin{align*}
				&\cos\theta\cos\phi d\theta - \sin\theta\sin\phi d\phi\\
				&\cos\theta\sin\phi d\theta + \sin\theta\cos\phi d\phi \\
				& -\sin\theta d\theta
			\end{align*}
			One can see this by writing out the Jacobian and then composing on the left with $dx$ which is $(0 , 0, 1)$ and remembering that $d\theta=\binom{1}{0}$ and $d\phi=\binom{0}{1}$.

			So then the pull back of $dx \wedge dy = (\cos\theta\sin\theta)(d\theta \wedge d\phi)$.
			We can see this by writing out the full expression, using multi linearity and alternating-ness of the wedge product and then trigonometric identities.
	\end{enumerate}
\end{eg}

\subsection{Integration of m-forms}
Take an atlas $\left\{ \phi_\alpha: U_\alpha \to V_\alpha\right\}_{\alpha\in\mathcal{A}}$.
Given $\alpha, \beta\in\mathcal{A}$, then on the overlap $U_\alpha \cap U_\beta$ we get
\[
	d{x_1}^\alpha \wedge \dots \wedge d{x_m}^\alpha = \Delta_{\alpha\beta}(x) d{x_1}^\beta \wedge \dots \wedge d{x_m}^\beta
\]
where $\Delta_{\alpha\beta}$ is the determinant of the Jacobian of the transition function $\phi_\alpha \circ \phi_\beta^{-1}$.
Note if our atlas is oriented then $\Delta_{\alpha\beta}(x) > 0$.
Hence we have the following result.

\begin{theorem}
An $m$-manifold is orientable if and only if it admits a nowhere vanishing $m$-form.
\end{theorem}

\begin{proof}
Worth looking over.
\end{proof}

\begin{defin}
Let $M$ be an oriented manifold.
Given an $m$-form $\omega$ on $M$ we define
\[
\mdf{\supp(\omega)}\defeq\overline{\left\{ x\in M \rmv \omega(x) \neq 0\right\}}
\]
We say that a cover $\left\{ U_\alpha\right\}$ of a Hausdorff space $X$ is \mdf{locally finite} if
\[
	\forall x \in X \quad \exists O\subseteq X\text{ open, s.t. }x\in O \; \text{ and } \; \left|\left\{ \alpha \rmv O\cap U_\alpha\neq\emptyset\right\}\right|< \infty
\]
that is around every point there is an open set which meets at most finitely many members of the cover.
\end{defin}

For now, suppose that $\supp(\omega)$ is compact.
Let $\left\{ \phi_\alpha: U_\alpha \to V_\alpha\right\}_{\alpha\in\mathcal{A}}$ be a locally finite, oriented atlas.
Suppose that $\eta$ is an n-form such that $\supp(\eta)\subseteq U_\alpha$ for some $\alpha\in\mathcal{A}$.
Then write in local coordinates $\eta = \lambda_\alpha(dx_1^\alpha \wedge \dots \wedge dx_m^\alpha)$ where $\lambda_\alpha:U_\alpha\to\R$ is smooth and compactly supported.
Then we set
\[
	I_\alpha(\eta)\defeq \int_{V_\alpha} \lambda_\alpha \circ \phi_\alpha^{-1}(x) \; dx
\]

\begin{defin}
	A \mdf{partition of unity subordinate to} $\left\{ U_\alpha\right\}$ is a collection of smooth functions $\left\{ \rho_\alpha: M \to [0, 1]\right\}$ such that
	\begin{enumerate}
		\item $\supp(\rho_\alpha)\subseteq U_\alpha$ for all $\alpha$,
		\item $\sum_{\alpha}\rho_\alpha(x)=1$ for all $x\in M$.
	\end{enumerate}
\end{defin}

\begin{theorem}
Any locally finite open cover of $M$ has a subordinate partition of unity.
\end{theorem}

So choose a partition of unity $\left\{ \rho_\alpha\right\}$ subordinate to $\left\{ U_\alpha\right\}$ and set
\[
	\mdf{\int_M \omega} \defeq \sum_{\alpha\in\mathcal{A}} I_\alpha(\rho_\alpha \omega)
\]
\begin{note}
	This is a finite sum because only finitely many $U_\alpha$ meet the support of $\omega$.
\end{note}

\begin{lemma}
This integral is well-defined.
That is, its independent of choice of atlas and partition.
\end{lemma}

\section{Riemannian Manifolds}
Using the integral defined in the previous section we can define the volume of a compact, orientable Riemannian manifold.
Choose any orientation and let $\omega$ be the volume form (that is any $m$-form, I think).
Then the volume is
\[
	\vol(M)\defeq\int_M \omega
\]
In fact, if $f:M \to \R$ is any smooth function then we can integrate $f$ with respect to volume.
That is, integrate the $m$-form $f\omega$.
The result $\int_M f\omega$ is often denoted informally as $\int_M f \; dV$.
We shouldn't use this notation because exterior derivatives will confuse things.

\begin{defin}
A \mdf{Riemannian metric} on $M$ is a smooth map $f: TM \oplus TM \to \R$ such that
\[
	\forall x \in M \quad \restr{f}{T_xM\oplus T_xM}\text{ is an inner product on }T_xM
\]
\end{defin}
Note that $f$ is not a metric in the usual sense.
Given $v, w\in T_xM$ we usually denote $f(v, w)\eqdef \langle v, w\rangle$.
If $v\in T_x M$ we define $\norm{v}\defeq \sqrt{\langle v, v\rangle}$.
In particular, a Riemannian metric gives us a way to measure norms of tangent vectors in a nice smooth way.

\begin{theorem}
Every manifold admits a Riemannian metric.
\end{theorem}

\begin{proof}
Let $\left\{ \phi_\alpha\right\}$ be an atlas for $M$ which gives rise to a trivialising atlas $\left\{ \psi_\alpha\right\}$ for $TM$.
Then, by the paracompactness of $M$, we can assume the cover $\left\{ U_\alpha\right\}$ is locally finite.
Let $\rho_\alpha$ be a partition of unity subordinate to the $U_\alpha$.
Given $\alpha$, and $v, w\in T_xM$ with $x\in U_\alpha$, we set
\[
	\langle v, w\rangle_\alpha \defeq (\psi_\alpha v)\cdot (\psi_\alpha w)
\]
Now we can set
\[
	\langle v, w\rangle\defeq\sum_{\alpha}\rho_\alpha(x) \langle v, w\rangle_\alpha
\]
This is smooth and its restriction to each tangent space is an inner product.
\end{proof}

\begin{defin}
	Given $\gamma:[a, b]\to M$ a smooth curve we define its \mdf{Riemannian length} as
	\[
		\int_a^b \norm{\gamma'(t)}\; dt
	\]
	where $\gamma'(t) \in T_{\gamma(t)}M$ is the tangent as previously defined.
\end{defin}

\section{Manifolds with Boundary}
Define the following sets
\[
	H^m\defeq \left\{ (x_1, \dots , x_m) \in \R^m \rmv x_m \geq 0\right\} \quad \quad \partial H^m \defeq \R^{m-1} \times \left\{ 0\right\}
\]
A \mdf{manifold with boundary} is essentially a manifold but where the charts go into some open set of $H^m$ in place of $\R^m$.

\begin{lemma}
Suppose that $x\in U_\alpha \cap U_\beta$, then
\[
	\phi_\alpha(x)\in \partial H^m \iff \phi_\beta(x)\in\partial H^m
\]
\end{lemma}

This boils down to proving the following 

\begin{lemma}
Let $\theta: U \to V$ be a diffeomorphism between open subsets of $H^m$.
Then 
\[
\theta(U\cap \partial H^m) = V\cap \partial H^m
\]
\end{lemma}

\begin{proof}
Exercise, using the Inverse Function Theorem.
\end{proof}

\begin{defin}
	Suppose $M$ is a manifold with boundary, then the \mdf{boundary of $M$}, $\mdf{\partial M}$, is
	\[
		\partial M \defeq \left\{ x\in M \rmv \phi_\alpha(x) \in H^m \text{ for some } \alpha\right\}
	\]
	The \mdf{interior of $M$} is $M\setminus\partial M$.
\end{defin}

\begin{prop}
The interior of $M$ is canonically an $m$-manifold and the boundary an $( m-1 )$-manifold
\end{prop}

\begin{note}
If $M$ is oriented then we can get an orientation on $\partial M$ simply by restricting the charts in an oriented atlas to $\partial M$.
We adopt the convection, that the orientation induced on $\partial M$ is given by a such restriction in the case where $m$ is even and the opposite orientation when $m$ is odd.
\end{note}

\section{The Exterior Derivative}
Let $M$ be an $m$-manifold.
We write
\[
	\Omega^p(M)\defeq\left\{ p\text{-forms}\right\}
\]
which we view as a real vector space.
We identity $\Omega^0(M)\defeq C^\infty(M)$.

\noindent\textbf{Claim: }We can consistently define a family of linear functions $d:\Omega^p(M)\to \Omega^{p+1}(M)$ satisfying:

\begin{enumerate}
	\item If $f\in\Omega^0(M)$ then $df$ has its original meaning as a 1-form.
	\item If $f\in\Omega^0(M)$ and $\omega\in\Omega^p(M)$ then
		\[
			d(f\omega) = df \wedge \omega + f \; d\omega
		\]
	\item If $\omega\in\Omega^p(M)$ then $dd\omega =0$.
	\item If $U\subseteq M$ is open then $d(\restr{\omega}{U})=\restr{( d\omega )}{U}$.
\end{enumerate}

\begin{note}
The last point tells us that $d$ is a local operation.
So if $x\in M$ then $d\omega(x)$ only depends on $\omega$ on an arbitrarily small neighbourhood of $x$.
\end{note}

\noindent\textbf{Claim: }If such system of maps exists then it is unique.

Firstly we show that if $f_1, \dots , f_n\in\Omega^0(M)$ then
\[
	d(df_1 \wedge \dots \wedge df_n) = 0
\]

\begin{proof}
If there is just one $f_1$ then the result is obvious.
Inductively, set $f\defeq f_1$ and $\omega\defeq df_2 \wedge \dots \wedge df_n$.
By number 2, we can write
\[
	df_1 \wedge \dots \wedge df_n = d(f\omega) - f \; d\omega
\]
Our inductive hypothesis tells us that $d\omega = 0$.
Hence we see
\[
	d(df_1 \wedge \dots \wedge df_n) = dd(fw) = 0
\]
by number 3.
\end{proof}

By number 4, it is enough to check uniqueness for forms defined on the domain of some chart $\phi: U \to V$.
As usual we write, $x_1, \dots , x_m$ for local coordinates.
We give $dx_i$ the usual meaning by property 1.
Then any $\omega\in\Omega^p(U)$ has the form
\[
	\omega = \sum_{I\in\mathcal{I}(m, p)}\lambda_I \mv{d}x_i
\]
for some $\lambda_I\in \Omega^0(U)$.
What we showed earlier was precisely $d(\mv{d}x_I)=0$ for any $I\in\mathcal{I}(m, p)$.

Now we apply property 2 and linearity we get
\[
	d\omega = \sum_{I\in\mathcal{I}(m, p)}d\lambda_I \wedge \mv{d}x_I
\]
We have no choice in how we defined $d\omega$ and hence it is unique.

\begin{note}
Given $I=\left\{ i_1, \dots, i_p\right\}$ with $i_1 < i_2 < \dots < i_p$ we write $\mv{d}x_I\defeq dx_{i_1} \wedge \dots dx_{i_p}$.
\end{note}

It remains to prove that this definition satisfies the properties we wanted.

\begin{enumerate}
	\item If $\omega = \lambda\in\Omega^0(U)$ then we clearly get $d\omega = d\lambda$.
	\item Note that, 
		\[
 d((f\lambda_I)\mv{d}x_I)=d(f\lambda_I) \wedge \mv{d}x_I = (\lambda_I\; df) \wedge \mv{d}x_I + f(d\lambda_I) \wedge \mv{d}x_I
		\]
		Summing over $I\in\mathcal{I}(m, p)$ we see
		\begin{align*}
			d(f\omega) &= \sum_{I\in\mathcal{I}(m, p)}(\lambda_I df) \wedge \mv{d}x_I + \sum_{I\in\mathcal{I}(m, p)}f(d\lambda_I)\wedge \mv{d}x_I\\
					   & = \sum_{I\in\mathcal{I}(m, p)}df \wedge (\lambda_I\mv{d}x_I) + f\sum_{I\in\mathcal{I}(m, p)}(d\lambda_I)\wedge \mv{d}x_I\\ 
					   & = df \wedge \omega + f (d\omega)
		\end{align*}
	\item We first note that $d\lambda_I=\sum_{i=1}^m \pd{\lambda_I}{x_i}dx_i$.
		Hencethusly,
		\begin{align*}
			dd(\lambda_I\mv{d}x_I)&=d\left( \sum_{j}\pd{\lambda_I}{x_j}dx_j \wedge \mv{d}x_I\right) =  \sum_{i, j}\pd{^2 \lambda_I}{x_i\partial x_j}dx_i \wedge dx_j \wedge \mv{d}x_I \\
								  &=\sum_{i < j}\left( \pd{^2\lambda_I}{x_i\partial x_j}- \pd{^2\lambda_I}{x_j \partial x_i} \right)dx_i \wedge dx_j \wedge \mv{d}x_I = 0
		\end{align*}
\end{enumerate}

Note that we have only shown these properties hold locally.
However, we only needed the first 3 properties for uniqueness.
So if the forth property holds then these properties must hold globally.

More rigorously, we want to define $d:\Omega^p(W)\to \Omega^{p+1}(W)$ for arbitrary open sets $W\subseteq M$.
If $x\in W$, choose any chart $\phi: U \to V$ such that $x\in U\subseteq W$.
Given any $\omega\in\Omega^p(U)$, the above gives $d(\restr{\omega}{U})\in \Omega^{p+1}(U)$ and we set $d\omega(x) = d(\restr{\omega}{U})(x)$.
To see that this is well-defined consider some other chart $\phi':U' \to V'$ with $x\in U'$.
Since we only needed 1-3 to prove uniqueness, the two constructions must agree on $U \cap U'$.

\begin{defin}
	Given $\omega\in\Omega^p(M)$ we call $d\omega\in\Omega^{p+1}(M)$ the \mdf{exterior derivative of $\omega$}.	
\end{defin}

\subsection{Stokes' Theorem}
We can define $p$-forms on a manifold with boundary in much the same way.
Having done this, one can construct a natural linear map $\iota : \Omega^p(M) \to \Omega^p(\partial M)$ as follows.

Given local coordinates write $\omega=\sum_{I\in\mathcal{I}(m, p)}\lambda_I\mv{d}x_I$.
Then we define
\[
	\iota(\omega) \defeq \sum_{I\in\mathcal{I}(m-1, p)} \lambda_I \mv{d}x_I
\]
where we have identified $\mathcal{I}(m-1, p)\subseteq \mathcal{I}(m, p)$.
That is, we just throw away any terms that include $dx_m$.
Since $\iota\omega(x)$ does not depend on the choice of chart containing $x$ we see that $\iota\omega$is defined on all of $\partial M$.
In fact $\iota\omega$ is often denoted by $\omega$ and is think of as the restriction of $\omega$ to $\partial M$.

\begin{theorem}[Stokes' Theorem]
Let $M$ be a compact oriented $m$-manifold with boundary.
Let $\omega$ be an $(m-1)$-form on $M$.
Then
\[
	\int_M d\omega = \int_{\partial M}\omega
\]
\end{theorem}

\begin{proof}
Maybe look at this ???
\end{proof}

\begin{cor}
If $\partial M = \emptyset$ then $\int_M d\omega =0$.
\end{cor}

\end{document}
