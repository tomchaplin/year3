\documentclass[11pt]{article}

%{{{ Packages
\usepackage[margin=1in]{geometry}
\usepackage{enumitem}
\usepackage{amsfonts}
\usepackage{amssymb}
\usepackage{amsmath}
\usepackage{amsthm}
\usepackage{amsmath}
\usepackage{mathdots}
\usepackage{float}
\usepackage[thicklines]{cancel}
\renewcommand{\CancelColor}{\color{red}}
\usepackage[dvipsnames]{xcolor}
\usepackage[framemethod=TikZ]{mdframed}
\usepackage{microtype}
\usepackage{tikz-cd}
%}}}
%{{{ Custom commands
% p-forms
\DeclareFontFamily{U}{skulls}{}
\DeclareFontShape{U}{skulls}{m}{n}{ <-> skull }{}
\newcommand{\skull}{\text{\usefont{U}{skulls}{m}{n}\symbol{'101}}}

% Nice maths commands
\newcommand{\defeq}{:=}
\newcommand{\eqdef}{+:}
\newcommand{\abs}[1]{|#1|}
\newcommand{\norm}[1]{||#1||}
\DeclareMathOperator{\im}{\mathrm{im}}
\DeclareMathOperator{\supp}{\mathrm{supp}}
\DeclareMathOperator{\vol}{\mathrm{vol}}
\newcommand{\Lp}{\Lambda^p}
%\renewcommand{\dots}{...}
\newcommand{\msrspc}{\ensuremath{(X,\mathcal{B},\mu)}}
\newcommand{\relmiddle}[1]{\mathrel{}\middle#1\mathrel{}}
\newcommand{\rmv}{\relmiddle|}
\newcommand{\stcmp}{^{\mathsf{c}}}
\newcommand\restr[2]{{% we make the whole thing an ordinary symbol
  \left.\kern-\nulldelimiterspace % automatically resize the bar with \right
  #1 % the function
  \vphantom{\big|} % pretend it's a little taller at normal size
  \right|_{#2} % this is the delimiter
  }}
\newcommand{\contr}{\Rightarrow\Leftarrow}
\newcommand{\interior}[1]{%
  {\kern0pt#1}^{\mathrm{o}}%
}
\newcommand{\sm}{\setminus}

% Spaces
\newcommand{\ktor}{\mathbb{T}^k}
\newcommand{\R}{\mathbb{R}}
\newcommand{\C}{\mathbb{C}}
\newcommand{\Z}{\mathbb{Z}}
\newcommand{\N}{\mathbb{N}}

% Derivatives
\newcommand*{\pd}[3][]{\ensuremath{\frac{\partial^{#1} {#2}}{\partial {#3}^{#1}}}}
\newcommand{\grad}{\bigtriangledown}

% Vectors
\newcommand{\mv}[1]{\textbf{#1}}

%}}}
%{{{ Enviornments
% Definitions environment
\newenvironment{defin}
	{\begin{mdframed}[backgroundcolor=white, roundcorner=5pt, linewidth=1pt, linecolor=Green]
		\setlength{\parindent}{0pt}}
	{\end{mdframed}}
	\newcommand{\mdf}[1]{{\color{Green} #1}}

% Important notes environment
\newenvironment{note}
	{\begin{mdframed}[backgroundcolor=white, linecolor=red, roundcorner=5pt, linewidth=1pt]\bfseries{Note:}\normalfont
	\setlength{\parindent}{0pt}}
	{\end{mdframed}}

% Examples enviornmnet
\definecolor{mylg}{rgb}{0.9,0.9,0.9}
\newenvironment{eg}
	{\begin{mdframed}[backgroundcolor=mylg,roundcorner=5pt,linewidth=0pt]\bfseries{Example:}\normalfont
	\setlength{\parindent}{0pt}}
	{\end{mdframed}}

% Theorem environment
\newtheorem{theorem}{Theorem}[section]
\newtheorem{cor}[theorem]{Corollary}
\newtheorem{lemma}[theorem]{Lemma}
\newtheorem{prop}[theorem]{Proposition}
%}}}
%{{{ Document metadata
\title{Manifolds Notes on things I don't understand/remember}
\author{Thomas Chaplin}
\date{}
%}}}

\begin{document}
\maketitle

\section{Orientations}
\section{Abstract Manifolds}
\section{Vector Bundles}
In defining a vector bundle over a manifold we associate a vector space to each point in the manifold and then arrange these together in a smooth way.
Examples of this will include the tangent and normal bundles as defined in times gone past.

Suppose that $M$ is an $m$-manifold.
A \mdf{family of vector spaces} over $M$ is a manifold $E$ together with a smooth map $p:E\to M$ such that for all $x\in M$, the \mdf{fibre} $\mdf{E_x}\defeq p^{-1}(x)$ has a vector space structure.
Given $U\subseteq M$ we write $\restr{E}{U}\defeq p^{-1}U$ which we equip with the restriction of $p$.

Given another family over vector spaces $p':E'\to M$ we define an \mdf{isomorphism} to be a diffeomorphism 
\[
\psi:E\to E'
\]
with the property that $p'\circ\psi = p$ and $\psi:E_x\to E'_x$ restricts to  a linear isomorphism for all $x\in M$.
If such a map exists then $E$ and $E'$ are called \mdf{isomorphic}.
Finally, a family $E$ over $M$ is called \mdf{trivial} if it is isomorphic to $M\times \R^q$ for some $q\in\N$.

\begin{figure}[H]
	\centering
	\begin{tikzcd}
		E' \arrow[rr, "\psi"] \arrow[rd, "p'"']& & E \arrow[dl, "p"] \\
											  & M
	\end{tikzcd}
\end{figure}

\begin{defin}
	A \mdf{vector bundle} over $M$ is a family of vector spaces $p:E\to M$ such that $\forall x\in M$ there is some open $U\subseteq M$ with $x\in U$ such that $\restr{E}{U}$ is trivial.
\end{defin}

Putting this another way, $E$ is a vector bundle over $M$ if we can find an atlas $\left\{\phi_alpha : U_\alpha \to V_\alpha \right\}$ for $M$ such that for each $\alpha$ we get a diffeomorphism
\[
	\psi_\alpha: \restr{E}{U_\alpha}\to V_\alpha\times\R^q\subseteq\R^{m+q}
\]
which restricts to a linear isomorphism on each fibre.
Note than then the maps $\left\{\psi_\alpha\right\}$ form an atlas for $E$ which is called a \mdf{locally trivialising atlas} for $E$.
\begin{note}
	\begin{itemize}
		\item Given a vector bundle, $p:E\to M$ is a surjective smooth submersion.
		\item Each fibre $E_x$ is an embedded sub-manifold and the vector space operations are smooth.
		\item The notion of a vector bundle is preserved under isomorphism.
	\end{itemize}
\end{note}

\begin{eg}
\begin{enumerate}
	\item If $M$ is a smooth manifold then the tangent bundle is a vector bundle over $M$.
		For a manifold in Euclidean space, this follows for the proof that $TM$ was a manifold.
		For an abstract space, we can form a locally trivialising atlas as follows:

		Given a chart $\phi:U\to V\subseteq\R^m$ for $M$ we construct a new map
		\[
			\psi:TU\to V\times\R^m \quad\text{by}\quad\psi(v)=(\phi(x),d_x\phi(v))
		\]
		where $x=pv$.
		These maps then form a locally trivialising atlas because when restricted to some $x$ the map becomes the derivative map
		\[
			\psi_{p^{-1}(x)}:T_x M \to \left\{\phi(x)\right\}\times\R^m \quad \psi(v)=(\phi(x), d_x\phi(v))
		\]
		which is a linear isomorphism.
	\item When can (and perhaps should) check that the atlas given for the normal bundle consists of maps which restrict to linear isomorphisms on each fibre.

	\item The spaces $B_n$ together with the projection to the circle yields a vector bundles.
		In particular, the M\"obius band is a bundle over the circle.
		Note $B_m$ is isomorphic to $B_n$ if $\abs{m-n}$ is even.
		But, $B_0$ cannot be isomorphic to $B_1$ since they are not even diffeomorphic.
		Hence the M\"obius band is non-trivial.
\end{enumerate}
\end{eg}
Given a vector bundle $p:E\to M$ we say that \mdf{section} of $E$ is a smooth map $s:E\to M$ such that $p\circ s$ is the identity on $M$.
This generalises our previous notion of a section of the tangent space which is often referred to as a \mdf{vector field} on $M$.
We say that a section is \mdf{non-vanishing} if $s(x)\neq 0$ for all $x\in M$.
Given any smooth map $f:M\to\R^q$ we get a section of $M\times\R^q$.
\begin{note}
Any section is an embedding of $M$ into $E$.
\end{note}

\begin{theorem}[Hairy Ball Theorem]
Every vector field on $S^2$ has at least one zero.
\end{theorem}

\begin{note}
	It turns out that any odd-dimensional manifold admits a non-vanishing vector field.
	An even-dimensional manifold may or may not.
	The torus does but it is the only compact connected orientable 2-manifold which does.
\end{note}

\begin{lemma}
A vector bundle $E$ is trivial if and only if there are sections $s_1, s_2, \dots, s_q$ such that $\left\{s_1, \dots, s_q(x)\right\}$ forms a basis for $E_x$ for all $x\in M$.
\end{lemma}
\begin{proof}
If $f:M\times \R^q$ is an isomorphism then set $s_i(x)=f(x, e_i)$ for some fixed basis $e_1, \dots , e_q$ for $\R^q$.

Conversely, if we have such sections $s_i$ then we can define a map $f:M\times \R^q \to E$ by
\[
	f(x, (\lambda_1, \dots , \lambda_q))=\sum_{i=1}^{q}\lambda_i s_i(x)
\]
which yields an isomorphism.
\end{proof}
\begin{eg}
	\begin{enumerate}
		\item Using the Intermediate Value Theorem, one can show that the M\"obius has no non-vanishing section and is hence a non-trivial bundle by the above Lemma.
		\item Let $S^n\subseteq\R^{n+1}$ be the unit sphere in $\R^n$ then the normal bundle is trivial because the outward pointing unit normal is a nowhere vanishing section.
	\end{enumerate}
\end{eg}

\begin{defin}
	A manifold $M$ is called \mdf{parallelisable} if the tangent bundle $TM$ is trivial.
\end{defin}

In view of the last lemma we can say that a manifold is parallelisable if and only if $M$ admits a global frame field.
Recall, a frame field is a family of vector fields $v_i$ such that the $v_i(x)$ form a basis for $T_x M$ at every $x\in M$.
We know that frame fields exists locally but not necessarily globally.

\begin{lemma}
Any parallelisable manifold is orientable.
\end{lemma}
\begin{proof}
Exercise.
\end{proof}

\begin{eg}
	\begin{itemize}
		\item The circle $S^1$ is parallelisable.
		\item $S^2$ is not by the Hairy Ball Theorem.
		\item $S^3$ is parallelisable because given $x=(x_1, x_2, x_3, x_4)\in S^3$ we can form a frame field with
			\[
			\begin{array}{rrrr}
				(-x_2,& x_1,& x_4,& -x_3)\\
				(-x_3,&- x_4,& x_1,& x_2)\\
				(-x_4,& x_3,& -x_2,& x_1)
			\end{array}
			\]
		\item $S^5$ is not strangely enough but $S^7$ is.
			It may be worth consulting your written notes from lectures to figure out how this frame was calculated.
	\end{itemize}
\end{eg}

\subsection{Vector Fields}

\subsection{Whitney Sums}

\subsection{Cotangent Bundle}

\subsection{1-forms}


\section{Smooth Function Extension}

\section{Differential Forms}
\begin{defin}
Given vector spaces $E, F$ and $p\in \N$ we denoted
\[
	A(E^p, F)\defeq p-\text{linear alternating maps }E^p\to F
\]
where by alternating we mean that swapping any two coordinates negates the output.
Equivalently, if two coordinates are the same then the output is $0$.
\end{defin}

\begin{lemma}
Given $E$ and $p$ there is a vector space $V$ together with a surjective map $\mu\in A(E^p, V)$ with the property that if $\theta\in A(E^p, F)$ then there is a linear map $\hat{\theta}:V\to F$ such that $\theta = \hat{\theta}\circ\mu$
\end{lemma}

\begin{figure}[H]
	\centering
	\begin{tikzcd}
		\; & V \arrow[rd, "\hat{\theta}", dashed] & \; \\
		E^p \arrow[ru, "\mu", two heads] \arrow[rr, "\theta"']& \; & F
	\end{tikzcd}
\end{figure}

\begin{note}
	The $\hat{\theta}$ is unique given $\theta$ and $V$.
	The $V$ is unique up to isomorphism.
\end{note}

\begin{defin}
We write $V = \Lambda^p E$ and given $v_1, \dots , v_p\in E$ we write
\[
	v_1 \wedge \dots \wedge v_p \defeq \mu(v_1, \dots ,v_p)
\]
We say $\Lambda^p E$ is the \mdf{$p$-th exterior power of $E$}.
\end{defin}

\subsection{Basis for $\Lambda^p E$}
Let $e_1, \dots , e_m$ be a basis for $E$.
Since $\mu$ is surjective $\Lambda^p E$ is spanned by
\[
	\left\{ e_{i_1} \wedge \dots \wedge e_{i_p} \rmv i_k\in I(m) \right\}
\]
where we can assume that the $i_k$ are distinct else their image would be null.
We can also assume that the indices are in order up to sign.

\begin{lemma}
These elements are linearly independent and hence form a basis.
\end{lemma}

Therefore we can say $\dim(\Lambda^p E) = \binom{m}{p}$.

\subsection{Wedge Product}
Given $p, q\in\N$ with $p, q \geq 1$ we can define the bilinear wedge product
\[
	\cdot \wedge \cdot : (\Lambda^p E \times \Lambda^q E) \to \Lambda^{p+q}E
\]
First we define on it on a basis.
So take a basis $e_1, \dots , e_m$ of $E$ and then define
\[
	(e_{i_1} \wedge \dots \wedge e_{i_p}) \wedge (e_{j_1} \wedge \dots \wedge e_{j_q}) = e_{i_1} \wedge \dots \wedge e_{i_p} \wedge e_{j_1} \wedge \dots \wedge e_{j_q}
\]
This can then be extended linearly to arbitrary elements and hence doesn't depend on our initial choice of basis.

\subsection{Induced maps}
Suppose we have a linear map between finite dimensional vector spaces
\[
\phi : E \to F
\]
then we get a multi linear map in the natural way
\[
\phi^p : E^p \to F^p
\]
By composing with the surjective map $\mu_F$ we get an alternating map
\[
\mu_F \circ \phi^p : E^p \to \Lambda^p F
\]
Hence by the defining property of $\Lambda^P E$ we get a linear map
\[
\Lambda^p \phi : \Lambda^p E \to \Lambda^p F
\]
with the property that the outer diamond in the below diagram commutes.

\begin{figure}[H]
	\centering
	\begin{tikzcd}
		\; & \Lambda^p E \arrow[rd, dashed, "\Lambda^p \phi"] & \; \\
		E^p \arrow[rr, "\mu_f \circ \phi^p"'] \arrow[rd, "\phi^p"'] \arrow[ru, "\mu_E"] & \; & \Lambda^p F \\
		\; & F^p \arrow[ru, "\mu_F"'] & \;
	\end{tikzcd}
\end{figure}

Essentially, if $e_1, \dots , e_m$ is a basis for $E$, then we can describe $\Lambda^p\phi$ by
\[
	(\Lambda^p\phi)( e_{i_1}\wedge \dots \wedge e_{i_p}) = (\phi e_{i_1}) \wedge \dots \wedge (\phi e_{i_p}).
\]

\subsection{$\skull$ The dreaded p-form $\skull$}

\begin{defin}
Let $M$ be an $m$-manifold.
Given $x\in M$ we can form the $p$-th exterior power of the cotangent space
\[
	\Lambda^p (T_x^\ast M)
\]
We can assemble these together into a vector bundle $\Lambda^p(T^\ast M)$.
Subsequently, a \mdf{$p$-form} on $M$ is define to be a section of the bundle $\Lambda^p(T^\ast M)$	
\end{defin}

\noindent\textbf{What the fuck does this mean???}

A more natural way to think about $p$-forms is to take local coordinates.
Let $\phi: U \to \R^m$ be a chart yielding local coordinates $x_1, \dots , x_m$.
We have locally defined $1$-forms $dx_1, \dots , dx_m$ which form a basis for the cotangent space
\[
	dx_i\left(\pd{}{x_j}\right)=\delta_{ij}
\]
Then given $I\in\mathcal{I}(m, p)$ we write $\mv{d}x_I\defeq d{x_{i_1}} \wedge \dots \wedge d{x_{i_p}}$.
Thus $\left\{ \mv{d}x_I \rmv I \in\mathcal{I}(m, p)\right\}$ forms a basis for $\Lambda^P(T^\ast M)$.
It follows that any $p$-form $\omega$ on $U$ can be uniquely written in the form
\[
	\omega = \sum_{I\in\mathcal{I}(m,p)}\lambda_I\mv{d}x_I
\]
where each $\lambda_I:U \to \R$ is a locally-defined smooth function.
\begin{note}
	This is all we really need from the bundle structure of $\Lambda^p(T^\ast M)$.
\end{note}

In particular, if $p=m$ then an $m$-form locally looks like
\[
	\lambda \; (dx_1 \wedge \dots \wedge dx_m)
\]
for some smooth function $\lambda: U \to \R$.

\subsection{Pull-backs}
Suppose we have a smooth function between manifolds
\[
f:M \to N
\]
Given a $p$-form $\omega$ on $N$ we can define a \mdf{pull-back $p$-form $f^\ast\omega$} on $M$ as follows. 
Given $x\in M$ we have the derivative map $d_xf$ and hence a dual map
\[
	(d_x f)^\ast : T^\ast_{f(x)}N \to T^\ast_x M , \quad \quad \eta \mapsto \eta \circ d_xf \quad \text{where }\eta:T_{f(x)}N \to \R\text{ is linear}
\]
This in turn gives rise to a linear map
\[
	\Lp(d_x\phi)^\ast : \Lambda^pT_{fx}^\ast N \to \Lp T_x^\ast M
\]
Then our pull-back is defined by
\[
	(f^\ast\omega)(x) \defeq (\Lp(d_x\phi)^\ast)\left[\omega(f(x))\right]
\]
One takes on blind faith that this is smooth and hence a $p$-form.
In particular, we can pull back $p$-forms to any manifold embedded within a larger manifold (such as $\R^n$).
This is a load of gobbledygook so let's go step by step.
\begin{enumerate}
	\item $x\in M$
	\item $f(x)\in N$
	\item $\omega$ is a $p$-form on $N$ so we get some linear maps $\eta_i:T_{fx}N\to\R$, then
		\[
			\omega(f(x))=\eta_1 \wedge \dots \wedge \eta_p
		\]
	\item Then we take the induced $p$'th exterior power map which just does $(d_x\phi)^\ast$ on each of the $\eta_i$
	\item Hence we can write
		\[
			(f^\ast\omega)(x) = (\eta_1 \circ d_xf) \wedge \dots \wedge (\eta_p \circ d_xf)
		\]
\end{enumerate}

\begin{eg}
	\begin{enumerate}
		\item Consider $S^1$; the unit circle in $\R^2$.
			Let $\theta$ by the angle coordinate so that $x=\cos\theta$ and $y=\sin\theta$.
			Then the pull back of $dx$, $dy$ is obtained by differentiating these formulae:
			\[
				-\sin\theta d\theta \quad\text{and}\quad cos\theta d\theta
			\]
			From this we can pull back an arbitrary 1-form by linear extension.
		\item Consider $S^2$; the unit 2-sphere in $\R^3$.
			Consider spherical polar coordinates $\theta, \phi$ away from the poles.
			\begin{align*}
				x &= \sin\theta\cos\phi \\
				y &= \sin\theta\sin\phi \\
				z &= \cos\theta
			\end{align*}
			Then the pull backs of $dx$, $dy$ and $dz$ repressively are
			\begin{align*}
				&\cos\theta\cos\phi d\theta - \sin\theta\sin\phi d\phi\\
				&\cos\theta\sin\phi d\theta + \sin\theta\cos\phi d\phi \\
				& -\sin\theta d\theta
			\end{align*}
			One can see this by writing out the Jacobian and then composing on the left with $dx$ which is $(0 , 0, 1)$ and remembering that $d\theta=\binom{1}{0}$ and $d\phi=\binom{0}{1}$.

			So then the pull back of $dx \wedge dy = (\cos\theta\sin\theta)(d\theta \wedge d\phi)$.
			We can see this by writing out the full expression, using multi linearity and alternating-ness of the wedge product and then trigonometric identities.
	\end{enumerate}
\end{eg}

\subsection{Integration of m-forms}
Take an atlas $\left\{ \phi_\alpha: U_\alpha \to V_\alpha\right\}_{\alpha\in\mathcal{A}}$.
Given $\alpha, \beta\in\mathcal{A}$, then on the overlap $U_\alpha \cap U_\beta$ we get
\[
	d{x_1}^\alpha \wedge \dots \wedge d{x_m}^\alpha = \Delta_{\alpha\beta}(x) d{x_1}^\beta \wedge \dots \wedge d{x_m}^\beta
\]
where $\Delta_{\alpha\beta}$ is the determinant of the Jacobian of the transition function $\phi_\alpha \circ \phi_\beta^{-1}$.
Note if our atlas is oriented then $\Delta_{\alpha\beta}(x) > 0$.
Hence we have the following result.

\begin{theorem}
An $m$-manifold is orientable if and only if it admits a nowhere vanishing $m$-form.
\end{theorem}

\begin{proof}
Worth looking over.
\end{proof}

\begin{defin}
Let $M$ be an oriented manifold.
Given an $m$-form $\omega$ on $M$ we define
\[
\mdf{\supp(\omega)}\defeq\overline{\left\{ x\in M \rmv \omega(x) \neq 0\right\}}
\]
We say that a cover $\left\{ U_\alpha\right\}$ of a Hausdorff space $X$ is \mdf{locally finite} if
\[
	\forall x \in X \quad \exists O\subseteq X\text{ open, s.t. }x\in O \; \text{ and } \; \left|\left\{ \alpha \rmv O\cap U_\alpha\neq\emptyset\right\}\right|< \infty
\]
that is around every point there is an open set which meets at most finitely many members of the cover.
\end{defin}

For now, suppose that $\supp(\omega)$ is compact.
Let $\left\{ \phi_\alpha: U_\alpha \to V_\alpha\right\}_{\alpha\in\mathcal{A}}$ be a locally finite, oriented atlas.
Suppose that $\eta$ is an n-form such that $\supp(\eta)\subseteq U_\alpha$ for some $\alpha\in\mathcal{A}$.
Then write in local coordinates $\eta = \lambda_\alpha(dx_1^\alpha \wedge \dots \wedge dx_m^\alpha)$ where $\lambda_\alpha:U_\alpha\to\R$ is smooth and compactly supported.
Then we set
\[
	I_\alpha(\eta)\defeq \int_{V_\alpha} \lambda_\alpha \circ \phi_\alpha^{-1}(x) \; dx
\]

\begin{defin}
	A \mdf{partition of unity subordinate to} $\left\{ U_\alpha\right\}$ is a collection of smooth functions $\left\{ \rho_\alpha: M \to [0, 1]\right\}$ such that
	\begin{enumerate}
		\item $\supp(\rho_\alpha)\subseteq U_\alpha$ for all $\alpha$,
		\item $\sum_{\alpha}\rho_\alpha(x)=1$ for all $x\in M$.
	\end{enumerate}
\end{defin}

So choose a partition of unity $\left\{ \rho_\alpha\right\}$ subordinate to $\left\{ U_\alpha\right\}$ and set
\[
	\mdf{\int_M \omega} \defeq \sum_{\alpha\in\mathcal{A}} I_\alpha(\rho_\alpha \omega)
\]
\begin{note}
	This is a finite sum because only finitely many $U_\alpha$ meet the support of $\omega$.
\end{note}

\begin{lemma}
This integral is well-defined.
That is, its independent of choice of atlas and partition.
\end{lemma}

Using this we can define the volume of a compact, orientable Riemannian manifold.
Choose any orientation and let $\omega$ be the volume form (that is any $m$-form, I think).
Then the volume is
\[
	\vol(M)\defeq\int_M \omega
\]
In fact, if $f:M \to \R$ is any smooth function then we can integrate $f$ with respect to volume.
That is, integrate the $m$-form $f\omega$.
The result $\int_m f\omega$ is often denoted informally as $\int_M f \; dV$.
We shouldn't use this notation because exterior derivatives will confuse things.

\end{document}
\end{document}
