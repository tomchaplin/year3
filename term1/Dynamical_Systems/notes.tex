\documentclass[11pt]{article}

%{{{ Packages
\usepackage[margin=1in]{geometry}
\usepackage{enumitem}
\usepackage{amsfonts}
\usepackage{amssymb}
\usepackage{amsmath}
\usepackage{amsthm}
\usepackage{amsmath}
\usepackage{mathdots}
\usepackage{float}
\usepackage[thicklines]{cancel}
\renewcommand{\CancelColor}{\color{red}}
\usepackage[dvipsnames]{xcolor}
\usepackage[framemethod=TikZ]{mdframed}
\usepackage{microtype}
\usepackage{tikz-cd}
\usetikzlibrary{decorations.pathmorphing}
\setlength{\parindent}{0pt}
%}}}
%{{{ Custom commands
% Nice maths commands
\newcommand{\defeq}{:=}
\newcommand{\eqdef}{+:}
\newcommand{\abs}[1]{|#1|}
\newcommand{\norm}[1]{||#1||}
\DeclareMathOperator{\im}{\mathrm{im}}
\DeclareMathOperator{\intr}{\mathrm{int}}
%\renewcommand{\dots}{...}
\newcommand{\msrspc}{\ensuremath{(X,\mathcal{B},\mu)}}
\newcommand{\relmiddle}[1]{\mathrel{}\middle#1\mathrel{}}
\newcommand{\rmv}{\relmiddle|}
\newcommand{\stcmp}{^{\mathsf{c}}}
\newcommand\restr[2]{{% we make the whole thing an ordinary symbol
  \left.\kern-\nulldelimiterspace % automatically resize the bar with \right
  #1 % the function
  \vphantom{\big|} % pretend it's a little taller at normal size
  \right|_{#2} % this is the delimiter
  }}
\newcommand{\contr}{\Rightarrow\Leftarrow}
\newcommand{\interior}[1]{%
  {\kern0pt#1}^{\mathrm{o}}%
}
\newcommand{\sm}{\setminus}

% Spaces
\newcommand{\ktor}{\mathbb{T}^k}
\newcommand{\R}{\mathbb{R}}
\newcommand{\C}{\mathbb{C}}
\newcommand{\Z}{\mathbb{Z}}
\newcommand{\N}{\mathbb{N}}
\newcommand{\Q}{\mathbb{Q}}

% Derivatives
\newcommand*{\pd}[3][]{\ensuremath{\frac{\partial^{#1} {#2}}{\partial {#3}^{#1}}}}
\newcommand{\grad}{\bigtriangledown}

% Vectors
\newcommand{\mv}[1]{\textbf{#1}}

%}}}
%{{{ Enviornments
% Definitions environment
\newenvironment{defin}
	{\begin{mdframed}[backgroundcolor=white, roundcorner=5pt, linewidth=1pt, linecolor=Green]
		\setlength{\parindent}{0pt}}
	{\end{mdframed}}
	\newcommand{\mdf}[1]{{\color{Green} #1}}

% Important notes environment
\newenvironment{note}
	{\begin{mdframed}[backgroundcolor=white, linecolor=red, roundcorner=5pt, linewidth=1pt]\bfseries{Note:}\normalfont
	\setlength{\parindent}{0pt}}
	{\end{mdframed}}

% Examples enviornmnet
\definecolor{mylg}{rgb}{0.9,0.9,0.9}
\newenvironment{eg}
	{\begin{mdframed}[backgroundcolor=mylg,roundcorner=5pt,linewidth=0pt]\bfseries{Example:}\normalfont
	\setlength{\parindent}{0pt}}
	{\end{mdframed}}

% Theorem environment
\newtheorem{theorem}{Theorem}[section]
\newtheorem{cor}[theorem]{Corollary}
\newtheorem{lemma}[theorem]{Lemma}
\newtheorem{prop}[theorem]{Proposition}
%}}}
%{{{ Document metadata
\title{Dynamical Systems - Proofs to Remember}
\author{Thomas Chaplin}
\date{}
%}}}

\begin{document}
\maketitle

\section{Sharkovskii's Theorem}
\begin{theorem}[Sharkovskii's Theorem]
\label{thrm:shark}
If $f:I \to I $ is continuous and there is a point of prime period 3.
Then for each $n\in \N$ there is a periodic point of prime period $n$.
\end{theorem}

The proof proceeds by a number of lemmata.

\begin{lemma}
Given $I\subseteq[0 ,1 ]$ a closed interval ,if $f(I) \supseteq I$ or $f(I) \subseteq I$ then $I$ contains a fixed point for $f$.
\end{lemma}

\begin{proof}
Use the ITV on $g(x)=f(x)-x$ and consider the endpoints.
\end{proof}

\begin{lemma}[Whittling down intervals]
If $I, I' \subseteq [0,1]$ are closed intervals and $f(I)=I'$, then $\exists$ a closed interval $I_0\subseteq I$ such that $f(I_0)=I'$.
\end{lemma}
\begin{proof}
Suppose $I'=[a, b]$ then let
\begin{align*}
	A & \defeq f^{-1}(a)\cap I \\
	B & \defeq f^{-1}(b)\cap I
\end{align*}
then take $x_0=\sup(A)$ and $y_0=\inf(B)$.
Then $I_0\defeq [x_0, y_0]$ will do the job.
\end{proof}

\begin{lemma}
Assume that we have closed intervals $I_1, \dots, I_n \subseteq [0, 1]$ such that
\begin{itemize}
	\item $f(I_n)\supseteq I_1$,
	\item $f(I_j)\supseteq I_{j+1}$ for all appropriate $j$,
\end{itemize}
then there is a fixed point $x$ for $f^n$ such that
\[
	x\in I_1, f(x) \in I_2, \dots , f^{n-1}\in I_n
\]
\end{lemma}

\begin{proof}
We can just apply the whittling lemma to the intervals in reverse order so
\[
\begin{array}{lcl}
	\exists I_n' \subseteq I_n & s.t. & f(I_n') = I_1\\
	\exists I_{n-1}' \subseteq I_{n-1} & s.t. & f(I_{n-1}') = I_n'\\
	 \; & \vdots & \; \\
	\exists I_1' \subseteq I_1 & s.t. & f(I_1)' = I_2'
\end{array}
\]
In particular we have that $f^n(I_1')=I_1\supseteq I_1'$ and hence the first lemma gives us the desired fixed point.
\end{proof}

\begin{proof}
\textit{of Theorem \ref{thrm:shark}.}

Let $f^3(x)=x$ be our point of prime period 3.
For now we will assume that 
\[
	\left\{x, f(x), f^2(x)\right\} = \left\{x_1, x_2, x_3\right\}
\]
where $0\leq x_1 < x_2 < x_3 \leq 1$.
We also assume $f(x_1)=x_2$, $f(x_2)=x_3$ and $f(x_3)=x_1$.
Other cases are similar.
Let $I_0\defeq[x_1, x_2]$ and $I_1\defeq[x_2, x_3]$.

Observe that
\begin{enumerate}[label=(\alph*)]
	\item $f(I_0)\supseteq I_1$, and
	\item $f(I_1)\supseteq I_0\cup I_1$.
\end{enumerate}

\noindent We now split the proof into a number of cases:

\noindent\textbf{Case 1: }($n=3$) This follows from the assumption.

\noindent\textbf{Case 2: }($n=1$) This follows from the first lemma thanks to \textit{(b)}.

\noindent\textbf{Case 3: }($n=2$ or $n\geq 4$)

Note that we can make a chain of intervals as in the last Lemma.
\begin{figure}[H]
	\centering
	\begin{tikzcd}
		I_0 \arrow[r, rightsquigarrow, "f"]\arrow[rrrrr, bend right=15, dashrightarrow, "n-1 \text{ times}"'] & I_1 \arrow[r, rightsquigarrow, "f"] & I_1 \arrow[r, rightsquigarrow, "f"] & \dots \arrow[r, rightsquigarrow, "f"] & I_1 \arrow[r, rightsquigarrow, "f"] & I_0
	\end{tikzcd}
\end{figure}
where $A\rightsquigarrow B$ means $f(A) \supseteq B$.
Hence there is a fixed point for $f^n$ which starts in $I_0$ spends $n-1$ in $I_1$ and then returns to $I_0$.
Because the earliest return is at time $n$ we can be sure that this is our prime period.
\end{proof}

\section{Independence of Lifts}

\section{Dense Irrational Orbits}
\begin{theorem}
If $\alpha\in\R\setminus\Q$ then for any $z\in\mathcal{K}$ we have
\[
	\left\{R_\alpha^n(x) \rmv n\in \N\right\}
\]
is a dense set in the circle $\mathcal{K}$.
\end{theorem}

\begin{proof}
Get yourself some pigeons.
If you put a pigeon on the circle for every point in the orbit up to time $\frac{1}{\epsilon}+1$ then two pigeons, Kenny $k$ and Lenny $l$, must be $\epsilon$ close.
\[
	d(R_\alpha^l(p), R_\alpha^k(p)) < \epsilon
\]
Without loss of generality, assume that Kenny is further along the orbit then Lenny so that 
\[m\defeq k-l > 0.\]
Then for any $x\in\mathcal{K}$ we have $d(R_\alpha^m(x), x < \epsilon)$.
Hence the orbit $\left\{x, R_\alpha^m(x), R_\alpha^{2m}(x), R_\alpha^{3m}(x) , \dots\right\}$ is $\epsilon$ dense in the circle.
\end{proof}

\section{Rational Points and Periodic Points}
\begin{theorem}
If $f:\mathcal{K}\to\mathcal{K}$ has a periodic point $x_0$ of period $m$ then $\alpha(f)\in\Q$.
\end{theorem}
\begin{proof}
Let $F:\R\to\R$ be some lift then we must have
\[
	F^m(x)-x = k\in \Z
\]
where $\rho(x)=x_0$.
Then we can write any integer as $n=pm+r$ where $p\geq 0$ and $r\in[0, m)$.
Hence
\[
	F^n(x)=F^{pm+r}(x)=F^r(x)+pk
\]
Then we can conclude
\[
	\lim_{n\to\infty}\frac{1}{n}F^n(x)=\lim_{p\to\infty}\frac{1}{pm+r}\left(F^r(x)+pk\right)=\frac{k}{m}\in\Q
\]
\end{proof}

\begin{theorem}
If $f:\mathcal{K}\to\mathcal{K}$ has $0$ rotation number then $f$ has a fixed point.
\end{theorem}

\begin{proof}
\begin{itemize}
	\item Take a lift $\widetilde{F}$ that gives $\lim_{n\to\infty}\frac{\widetilde{F}^n(x)}{n}=m$.
	\item Create a nicer lift $F\defeq\widetilde{F}-m$ so that the limit becomes $0$.
	\item Assume $d$ has no fixed point then by the IVT we can assume WLOG $F(y) > y$ for all $y\in\R$.
	\item Hence $\left(F^n(0)\right)_{n\in\N}$ is increasing so we just need to show boundedness.
	\item Suppose unbounded then $\abs{F^{n_0}(0)}>1$ and hence for all $m$ we have $\abs{F^{mn_0}(0)} > m$.
		\[
			\frac{\abs{F^{mn_0}(0)}}{mn_0} > \frac{m}{mn_0} = \frac{1}{n_0}
		\]
		which cause a contradiction with the earlier $0$ limit.
	\item It can be seen that the limit of this sequence is a fixed point.
\end{itemize}
\end{proof}

\begin{note}
	As a corollary if the rotation number is $\frac{a}{b}\in\Q$ then $f^b$ has $0$ rotation number and hence fixed point.
	Therefore, $f$ has a periodic point.
\end{note}

\subsection{Tending to periodic orbits}
\begin{theorem}
Let $f$ be a circle homeomorphism.
Prove that if its rotation number $\rho\in\Q$ is rational then any point $x\in\mathcal{K}$ is either period or converges to some periodic orbit.
More succinctly there is a point $p\in\mathcal{K}$ such that
\[
	\lim_{n\to\infty}d(f^n(x), f^n(p))=0
\]
\end{theorem}

\begin{proof}
We have seen that circle homeomorphisms of rational degree certainly have periodic orbits.
We can therefore partition the circle using the periodic points of some period $m$.
For simplicity let's assume $m=1$.

By slicing the circle into arcs between fixed points we can assume that on each arc $f(z)>z$ or $f(z)<z$.
Without loss of generality lets assume the former.

\textbf{Claim: }The iterates $(f^n(z))_{n=0}^\infty$ form a bounded, increasing sequence.

This follows because $f$ is a circle homeomorphism.
This is thanks to injectivity which prevents iterated from "jumping over" a fixed point into the next arc where we might have $f(z) < z$.

So the sequence must have a limit $x_\ast$.
Moreover, this limit can easily be shown to be a fixed point and therefore (thanks to our previous division of the circle) must be the fixed point at the end of the arc.

\textbf{What about $m> 1$?}

We can certainly get the iterated $f^{nm}(x)$ and $f^{nm}(p)$ to tend to one another as $n\to\infty$, but what about the points in between?
Since $\mathcal{K}$ is compact there is a fixed $\delta$ such that points $\delta$ close will stay $\epsilon$ close over the next $m-1$ iterates.
This gives convergence of the entire sequence.
\end{proof}

\section{Poincar\'e's Theorem and Minimality}

\begin{defin}
	A homeomorphism is called \mdf{minimal} if every orbit is dense.	
\end{defin}

\begin{eg}
Any irrational rotation $R_\alpha$ is minimal.
\end{eg}

\begin{theorem}[Poincar\'e's Theorem]
Any minimal circle homeomorphism is topologically conjugate to an irrational rotation.
\end{theorem}

Given a circle homeomorphism $f:\mathcal{K}\to\mathcal{K}$ and some lift $F$ we define the following countable sets
\begin{align*}
	\Lambda_{x_0}&\defeq\left\{F^n(x_0)+m \rmv m, n\in \Z\right\}\\
	\Omega &\defeq \left\{n \rho +m \rmv m, n \in \Z\right\}
\end{align*}
for some fixed $x_0\in\R$ and where $\rho = \rho(f)$ is the rotation number.
Note that $\Lambda_{x_0}=\pi^{-1}\left\{f^n(\pi x_0)\right\}$ and $\Omega=\pi^{-1}\left\{R_\rho ^n (0)\right\}$ where $\pi$ is the usual projection.

\begin{lemma}
Let $f$ be a circle homeomorphism and $x_0\in\mathcal{K}$.
If the rotation number $\rho$ is irrational then the map $T:\Lambda_{x_0}\to\Omega$ given be
\[
	T(F^n(x_0)+m)=n\rho +m
\]
is a bijection.
Moreover,
\begin{enumerate}
	\item $T$ is strictly increasing
	\item $T(x+1) = T(x) + 1$
	\item $T(F(x))= T(x) + \rho$ for all $x\in\Lambda_{x_0}$.
\end{enumerate}
\end{lemma}

\begin{proof}
This is omitted but might be worth glancing over.
\end{proof}

\begin{proof}
\textit{of Poincar\'e's Theorem}
Since $f$ is minimal, it has no periodic points because their orbits would be finite and hence not dense.
So the rotation number $\rho$ is irrational.


Take a lift $F$ of $f$ and $x_0\in \R$ and write $\Lambda=\Lambda_{x_0}$.
The sets $\Omega$ and $\Lambda$ are dense in $\R$ due to the minimality of $R_\rho$ and $f$ respectively.

Thus $\pi(\Omega)$ and $\pi(\Lambda)$ must be dense in $\mathcal{K}$.
Moreover, the Lemma tells us that $T:\Lambda\to\Omega$ is strictly increasing.
Consequently, we can extend to a unique continuous function $H:\R \to \R$ (which restricts to $T$ on $\Lambda$).
Moreover, $H$ is strictly increasing, $H$ is continuous and so is its inverse.

\begin{note}
This is non-trivial.
It is an exercise to show that given dense sets $X,Y\subseteq \R$ and $f:X\to Y$ a bijection, there exists a unique homeomorphism extension to $\R$.
\end{note}

By continuity $H$ inherits the properties \textit{(2)} and \textit{(3)} in the previous Lemma.
The first says that $H$ is a lift of circle homeomorphism $h$.
The second say that $h\circ f= R_\rho \circ h$.
\end{proof}

So we now know that if $f$ is a circle homeomorphism then there is a unique homeomorphism $h$ satisfying
\[
	h(f(x))= h(x) + \rho \quad (mod 1) \quad \forall x\in\mathcal{K}
\]
Note that this is a linear equation on $h$.
We can conclude that a solution to this equation is unique up to adding a constant corresponding to choosing with point in $\mathcal{K}$ is sent to zero.
For a hand-wavey explanation of this, see the lecture notes.

\section{Expanding Maps}

\subsection{Fixed Points}
\begin{theorem}
If $f:\mathcal{K}\to\mathcal{K}$ is an expanding, orientation preserving map and $d=\deg(f)$, then there are exactly $d^n-1$ points $p\in\mathcal{K}$ such that $f^n(p)=p$.
\end{theorem}

\begin{proof}
We'll do $n=1$ then $\deg(f^n)=\deg(f)^n$ implies the rest.
Take a lift $F:\R\to\R$ and recall that $F(1)=F(0)+d$.

\begin{align*}
	\#\text{fixed points for }f & = \#\left\{x\in[0,1) \rmv x=F(x) \mod 1\right\} \\
								& = \#\text{integer values assumed by }g(x)\defeq F(x) -x \text{ in the range }[0,1)
\end{align*}
But $g$ is monotone increasing (take derivative) and $g(1)-g(0)= F(1)- F(0) - 1 = d-1$.
Therefore $g$ assumes $d-1$ integer values on the range $[0,1)$.
\end{proof}

\begin{theorem}
Let $f:\frac{\R}{\Z}\to\frac{\R}{\Z}$ be an expanding map, than there exists a dense $G_\delta$ set of points whose orbits are dense.
(Recall that a dense $G_\delta$ set is a countable intersection of open dense sets).
\end{theorem}

\begin{proof}
Choose a countable dense set of points $\left\{x_n\right\}$ and for each natural $m\geq 1$ consider the ball $B\left(x_n, \frac{1}{m}\right)$.
A point $x$ has a dense orbit if and only if it intersects every one of these balls.
That is for all $n$ and $m$ there is a $k$ such that $f^k(x)\in B\left(x_n, \frac{1}{m}\right)$ or more precisely
\[
	x\in \bigcap_n \bigcap_m \bigcup_k f^{-k} B\left(x_n, \frac{1}{m}\right)
\]
Note that the $\cup_k f^{-k}B\left(x_n, \frac{1}{m}\right)$ are open and dense since any expanding map is mixing and so at least transitive.
\end{proof}

\subsection{Conjugacy to shift maps}
\begin{theorem}
If $f:\mathcal{K}\to\mathcal{K}$ is an expanding map, preserves orientation and has degree 2 then there is a semi-conjugacy $h:\Sigma\to\mathcal{K}$ to the full shift on two symbols.

\begin{figure}[H]
	\centering
	\begin{tikzcd}
		\Sigma \arrow[r, "\sigma"] \arrow[d, "h"'] & \Sigma \arrow[d, "h"]\\
		\mathcal{K} \arrow[r, "f"] & \mathcal{K}
	\end{tikzcd}
\end{figure}
\end{theorem}

\begin{proof}
Take any $n\in\N$.
Then $\deg f^n = (\deg f)^n$ so there are $w^n$ pre-images of $p$ under $f^n$.
These are numbered $p_j$ starting with $p_0=p$ and number consecutively anticlockwise.
These points define intervals which we denote $A_{\omega_0\dots\omega_{n-1}}$ where the sequence of $\omega_i$ is just the binary representation of the position in the circle.

Let $K$ denote the uniform bound away from $1$ of the derivative.
We have a number of results:
\begin{enumerate}
	\item $f^n(A_{\omega_0\dots\omega_{n-1}}^\circ)=\mathcal{K}\setminus\left\{p\right\}$
	\item $A_{\omega_0\dots\omega_{n-1}}$ is a closed interval of length $<K^{-n}$.
	\item $A_{\omega_0\dots\omega_{n-1}\omega_n}\subseteq A_{\omega_0\dots\omega_{n-1}}$.
	\item $f^n(A_{\omega_0\dots\omega_{n}})=A_{\omega_n}$.
	\item $f(A_{\omega_0\dots\omega_{n}})=A_{\omega_1\dots\omega_{n}}$.
\end{enumerate}

Now we can define our conjugacy $h:\Sigma\to\mathcal{K}$.
Given $\omega=(\omega_k)_{k=0}^\infty\in\Sigma$ let $B_n(\omega)=A_{\omega_0\dots\omega_{n-1}}$.
These are the points in the circle that start in the $\omega_0$ interval then go to $\omega_1$, then to $\omega_2$ and after $f^{n-1}$ are in the $\omega_{n-1}$ interval.
The properties implies that $B_{n+1}(\omega) \subseteq B_n(\omega)$.
The sets are also closed and their diameters go to $0$.
Hence their infinite intersection is a single points which we define to be $h(\omega)$.
The proof of their desired properties is discussed below in vague detail but is written in the lecture notes with more rigour.
\end{proof}

\section{Finding semi-conjugacies/conjugacies}
If you can partition your space $X$ into $n$ subsets $I_1, \dots, I_n$ where one could conceivably go from any partition element $I_a$ to any other $I_b$, then you might be able to find a semi-conjugacy to the full shift on $n$ symbols.

The trick is to define a map $\pi:\Sigma\to X$ by
\[
	\pi(\mv{x})=\bigcap_{n=1}^\infty T^{-n}I_{x_n}
\]
If the sets $I(x_0, \dots, x_n)\defeq\cap_{k=0}^n T^{-k}I_{x_k}$ are closed and nested and their diameter tends to zero as $n\to \infty$ then this map is well-defined because the infinite intersection contains one point.
Moreover, it is continuous because if $\mv{x}$ and $\mv{y}$ agree up to $N$ places then they both lie in $I(x_0, \dots x_{N-1})$ whose diameters goes to $0$ as $N\to\infty$.

The commutative relationship $T\circ\pi=\pi\circ\sigma$ then follows rather quickly.
To get surjectivity, it suffices to show that the image of $\Sigma$ is dense.
This usually involves taking in point $x\in X$ such that no $T^n x$ lies on the boundary between any $I_j$ for some $n\geq 0$ and then this points orbit will describe its pre-image in $\Sigma$.

\begin{note}
	Shift spaces are \mdf{totally disconnected}, i.e. the connected components are one-point sets.
	In particular, they are disconnected and so this can often be used to rule out the existence of conjugacies to more familiar sets.
\end{note}



\section{Transitivity and Mixing}
\begin{note}
A compact metric space has a countable dense set of points!
\end{note}

\begin{theorem}[Baire's Theorem]
Given a compact metric space $X$, the intersection of countably many open, dense subsets of $X$ is itself dense in $X$.
\end{theorem}

\begin{theorem}
If a map $T:X\to X$ on a compact metric space $X$ is topologically transitive then there exists a dense orbit.
\end{theorem}

\begin{proof}
There is a countable dense set of points $\left\{x_k\right\}$ so if we can find an orbit the gets $\epsilon$ close to every $x_k$ for arbitrary $\epsilon$ then we are done.
So we want $x$ such that for every $x_k$ and $m\geq 1$ there is an $n\in\Z$ such that
\[
	x\in T^{-n}\mathbb{B}\left(x_k,\frac{1}{m}\right)
\]
or equivalently we want to find
\[
	x\in\bigcap_{k, m}\bigcup_{n\in\Z}T^{-n}\mathbb{B}\left(x_k,\frac{1}{m}\right)
\]
which is a countable intersection (over $m$) of open dense sets.
By Baire's Theorem our desired point exists.
\end{proof}

\begin{proof}
\textit{(Alternate).}
Since we're in a compact space, there is a countable dense set.
Then we can choose a sequence of open discs around these dense set $(U_n)_{n=1}^\infty$.
Choose $N_1$ such that $T^{-N_1}(U_2)\cap U_1 \neq \emptyset$.
Then choose an open disc $V_1$ of radius less than a half such that 
\[
	V_1 \subseteq \overline{V_1}\subseteq U_1 \cap T^{-N_1}(U_2)
\]
Then choose $N_2$ such that $T^{-N_2}(U_3)\cap V_1\neq\emptyset$ and subsequently choose an open disc $V_2$ of radius less that $\frac{1}{4}$ such that
\[
	V_2\subseteq \overline{V_2} \subseteq V_1 \cap f^{-N_2}(U_3)
\]
By induction we get a sequence of discs
\[
V_1 \supseteq V_2 \supseteq V_3 \supseteq \dots
\]
such that $diam(V_n)\leq\frac{1}{2^n}$.
So choose the point $x$ in $\cap_{n}V_n$ then $T^{N_{n-1}}(x)\in U_n$ for each $n\geq 1$.
Therefore $(T ^n(x))$ forms a dense orbit.
\end{proof}
\subsection{Shift Spaces}
\begin{theorem}
The shift map $\sigma: \Sigma_A \to \Sigma_A$ is mixing if and only if the matrix $A$ is aperiodic.
\end{theorem}

\begin{proof}
Suppose the matrix $A$ is aperiodic.
Then it suffices to show that any two cylinder sets $U, V$ of the same length have the mixing property, i.e. there is an $N$ such that for all $n\geq N$ we have $T^{-n}U\cap V\neq \emptyset$.
Write $U\defeq [u_0, \dots u_n]$ and $V\defeq [v_0, \dots, v_n]$.
Then there is an $N$ such that we can go from any symbol to any other symbol in $N$ or more steps.
So for all $m\geq N$ we can find a point in $U$ that looks like
\[
	u_0, \dots, u_n, \underbrace{\dots, \dots, \dots}_{\text{length }m}, v_0, \dots, v_n, \dots
\]
Hence for all $m\geq N$ we have that $\sigma^{m+n+1}(U)\cap V\neq\emptyset$ and hence $\sigma$ is mixing.

Conversely, suppose that $\sigma$ is mixing then the cylinder sets are all open so there is a common $m\geq 1$ such that
\[
	\sigma^m [i] \cap [j]\neq\emptyset
\]
So then given any pair $(i,j)$ there is a sequence $\omega$ such that $\omega_0=i$ and $\omega_m=j$, and hence $(A^m)_{i,j}\geq 1$ because there is at least one path of length $m$ from $i$ to $j$.
\end{proof}

\begin{theorem}
The shift map $\sigma: \Sigma_A \to \Sigma_A$ is transitive if and only if the matrix $A$ is irreducible.
\end{theorem}

\begin{proof}
Suppose $A$ is irreducible then given any two open sets we can find cylinders $U$ and $V$ inside them.
Write
\[
	U=[ u_0, \dots, u_n] \quad V=[v_0, \dots, v_n]
\]
then $A$ is transitive so there exists an admissible path $p_0, \dots, p_k$ of some length from $u_n$ to $v_0$.
Then $(u_0, \dots, u_n, p_1, \dots, p_{k-1}, v_0, \dots, v_n, \dots)\in U \cap \sigma^{-(n+k)} V$ where we just fill the rest of the sequence out with random junk.

Now suppose that $\sigma$ is transitive.
Then the cylinder sets $U\defeq [i]$ and $V\defeq [j]$ are open and hence there is an $n$ such that $\sigma^{-n}U \cap V\neq\emptyset$, i.e. there is a sequence which starts at $j$ and after $n$ arrives at $i$.
Therefore $(A^n)_{i,j} \geq 1$ and hence $A$ is transitive.

\end{proof}

\section{Arithmetic Progressions}
\begin{defin}
	We say a subset $C\subseteq \Z$ \mdf{contains arithmetic progressions of arbitrary length} if
\[
	\forall k \geq 1 \quad \exists c\in \Z \;\text{ and }\; d\in\N \quad \text{such that}
\]\[
	c, c+d, c+ 2d, \dots, c+ (k-1)d\in C
\]
Similarly we say a map $T:X\to X$ is \mdf{multiple mixing} if for any non-empty open set $U\subseteq X$ and $k\geq 1$ there exists $d\geq 1$ such that
\[
	U\cap T^{-d}U \cap T^{-2d}U \cap \dots \cap T^{-(k-1)d}U\neq \emptyset
\]
\end{defin}

\begin{theorem}[van der Waerden's Theorem]
Given any finite integer partition $\Z=\cup_{i=1}^M C_i$ there is an $i$ such that $C_i$ contains arithmetic progressions of arbitrary length.
\end{theorem}

To prove this via a dynamical approach we must create a dynamical formulation.
To a partition of $\Z$ we associate  a single infinite sequence $\mv{x}=(x_n)\in\left\{1, \dots, M\right\}^\Z$ defined by
\[
	x_n=i\quad\text{if}\quad n\in C_i
\]
Let $X=\overline{\cup_{n\in\Z}\sigma^n\mv{x}}$ be the closure of the orbit of $\mv{x}$ where $\sigma$ is the shift map.

\begin{lemma}[Dynamic Formulation]
Assume that for some $[i]$ (cylinder set) we have that
\[
	X\cap [i] \cap \sigma^{-d}[i]\cap \sigma^{-2d}[i] \cap \dots \sigma^{-(k-1)d}[i]\neq\emptyset
\]
for some $k,d\geq 1$ then $C_i$ contains an arithmetic progression of length $k$.
\end{lemma}

\begin{proof}
The space is the closure of the orbit of $\mv{x}$ and this set is non-empty and open.
The orbit itself is dense in $X$ and hence intersects our open set.
So there is $n\in\Z$ such that $\sigma^n x$ is in our set.
This means that $x_{n+jd}=i$ for $j=0, \dots, k-1$ and hence $n+jd\in C_i$ for these $j$.
\end{proof}

\begin{prop}[Multiple Recurrence]
The shift map is multiple mixing when restricted to a minimal subset $Y\subseteq X$.
\end{prop}

\begin{proof}
\textit{of van der Waerden's Theorem}
Take a minimal subset $Y\subseteq X$.
Taking $U=[i]$ where $i$ is chosen such that $[i]\cap Y \neq \emptyset$, we see the set from the dynamical formulation is open and hence non-empty by multiple recurrence so we have arbitrary arithmetic progressions.
\end{proof}

\section{Hyperbolic Toral Automorphisms}

\begin{defin}
Given a $2\times 2$ matrix $A=
\begin{pmatrix}
	a & b \\
	c & d
\end{pmatrix}$ such that $ad-bc=1$ we can associate an automorphism $f:\mathbb{T}^2\to\mathbb{T}^2$ by
\[
	f(x,y)\defeq (ax + by, cx + dy) \quad \mod 1
\]
We say this is \mdf{hyperbolic} if no eigenvalue of $A$ lives on the unit circle.
\end{defin}

\begin{theorem}
The fixed points of a hyperbolic toral automorphism are precisely those $(x_1,x_2)\in T^2$ such that $x_1, x_2\in\Q$.
\end{theorem}

\begin{proof}
Take $(x_1, x_2)\in\mathbb{T}^2$ such that $x_1, x_2\in\Q$.
We can therefore write $x_1=\frac{m_1}{d}$ and $x_2=\frac{m_2}{d}$ for some integers $m_1, m_2\in\Z$.
Write $m\defeq(m_1, m_2)$ then
\[
	A^k x^T = \frac{1}{d} A^k m^T\quad\forall k\in\Z
\]
But by the pigeonhole principle, since we are only looking for a fixed point $\mod 1$, there are only $l^2$ distinct pairs of values that $A^km^T$ can assume.
Hence there is $k_1 < k_2$ such that
\[
	A^{k_1}m^T = A^{k_2}m^t \mod 1
\]
Set $n\defeq k_2 - k_1 > 0 $.
Then $A^n x^t = x^t \mod 1$.

Conversely, suppose that $x$ is a periodic point of $f$.
Then $A^n x = x \mod 1$ or, more succinctly, there exists integers $k_1, k_2$ such that
\[
	(A^n - I) x =
	\begin{pmatrix}
	k_1 \\
	k_2
	\end{pmatrix}
\]
But then $A^n - I$ is an integer matrix and hence its inverse has rational entries, so $x$ must have rational entries.
\end{proof}

\begin{theorem}
The number of fixed points for $f^n$ where $f$ is a hyperbolic toral automorphism is $\abs{tr(A^n)-2}$.
\end{theorem}

\begin{proof}
Note that the number of fixed points for $f^n$ precisely the number of $x\in\Delta\defeq[0, 1) \times [0, 1)$ such that
$(A^n - I)x\in\Z^2$.
But the number of lattice points in $(A^n - I)(\Delta)$ is equal to the area of the parallelogram $(A^n -I)(\Delta)$.
But $\Delta$ has unit areas so the parallelogram has area $\abs{\det(A^n - I)}$.
Then
\[
	\abs{\det(A^n-I)} = \abs{(1-\lambda_+^n)(1-\lambda_-^n)} = \abs{2 - (\lambda_+^n + \lambda_-^n)} = \abs{tr(A^n) - 2}
\]
\end{proof}

\begin{theorem}
Hyperbolic toral automorphisms are topologically mixing.
\end{theorem}

\begin{proof}
\textit{(Sketch).}
\begin{itemize}
	\item Take $U, V\subseteq \mathbb{T}^2$ open and non-empty.
	\item Let $l_\pm$ be the lines spanned by the eigenvectors. 
	\item These lines have irrational slope and hence their projection to the torus is dense. (Why!?)
	\item Take small parallelograms $U'\subseteq U$ and $V'\subseteq V$ with sides parallel to $l_\pm$.
	\item Density implies that the projected lines intersect $U'$ and $V'$.
	\item Then as we take $f^n$ on $U'$ for larger and larger $n$ we stretch along $W_+$ and shrink along $W_-$.
	\item Eventually $f^n(U')$ will reach the part of $W_+$ which intersects $V'$ and continue to intersect for all future $n$.
\end{itemize}
\end{proof}

\subsection{Markov Partitions}
We wish to divide the torus up into a partition $\mathcal{P}\defeq\left\{P_0, \dots, P_{k-1}\right\}$ with the properties
\begin{itemize}
	\item $\cup_i P_i = \mathbb{T}^2$
	\item $\intr(P_i)\cap \intr(P_j)=\emptyset$.
	\item The \mdf{Markov property} if there are points $x,y\in\mathbb{T}^2$ and a sequence $(\omega_n)_{n\in\Z}$ such that
		\begin{align*}
		T_A^n(x)\in\intr(P_{\omega_n})\quad \forall n\geq 0 \\
		T_A^n(y)\in\intr(P_{\omega_n})\quad \forall n\leq 0 
		\end{align*}
		then there is a $z\in\mathbb{T}^2$ such that $\:T_A^n(z)\in\intr(P_{\omega_n})\quad \forall n\in \Z$.
\end{itemize}
Such a partition is called a \mdf{Markov partition}.

\begin{theorem}
We can divide the torus up into a Markov partition for any linear hyperbolic toral automorphisms $f:\mathbb{T}^2\to\mathbb{T}^2$.

Once we have this partition we can create a semi conjugacy to a subshift of finite type $\pi:\Sigma_b\to\mathbb{T}^2$.
\end{theorem}

\begin{proof}
\textit{(Sketch).}
We divide the torus up by extending the eigenvectors sufficiently far and making sure that when we finish we don't leave any dangling ends.
Having obtained the Markov partition $\mathcal{P}=\left\{P_1, \dots, P_k\right\}$ we define a matrix $B$ by
\[
	B(i,j)\defeq
	\begin{cases}
		1 & \text{if }f(P_i^\circ)\cap P_j\neq \emptyset \\
		0 & \text{otherwise}
	\end{cases}
\]
Thanks to the Markov property we have the following property.
If $i_{-n}, \dots, i_n$ satisfy
\[
	\cap_{k=-n}^n f^{-k}(P_{i_k}^\circ)\neq\emptyset
\]
and $i_{-(n+1)}, i_n$ and $i_n, i_{n+1}$ are admissible, then
\[
	\cap_{k=-(n+1)}^{n+1} f^{-k}(P_{i_k}^\circ)\neq\emptyset
\]
That is, along as we take admissible steps, we can be sure that an admissible sequence is non-empty.
Hence given a sequence $\omega=(\omega_n)_{n=-\infty}^\infty$ we may conclude that
\[
	\cap_{k=-\infty}^{\infty} f^{-k}(P_{i_k}^\circ)\neq\emptyset
\]
Moreover, since the diameters of the finite intersections are decreasing, it is a single point which we denote by $\pi(\omega)$.
This is a semi-conjugacy, similar to the proof for circle maps we've seen before.
\end{proof}

\section{Entropy}

\begin{theorem}
We can calculate entropy through minimal spanning sets $S(n,\epsilon)$ and maximal separated sets $N(n,\epsilon)$.
\end{theorem}

\begin{proof}
\[
	S(n,\epsilon) \leq N(n,\epsilon) \leq S\left(n, \frac{\epsilon}{2}\right)
\]
For the first inequality, show that an $(n,\epsilon)$ separated set is an $(n,\epsilon)$ spanning set.
For the second, take an $(n, \frac{\epsilon}{2})$ spanning set and then any $(n,\epsilon)$ separated set would contain at most one point from each $(n, \frac{\epsilon}{2})$ ball.
Moreover, every element of a separated set would fall in at least one ball.
Hence $N(n, \epsilon) \leq S\left(n, \frac{\epsilon}{2}\right)$.
\end{proof}

\subsection{of Shift Maps}

\begin{theorem}[Gelfand's theorem]
Let $\norm{A}$ be a norm of $A$ and $\lambda_1$ a maximal, positive, real eigenvalue. Then
\[
	\lambda_1=\lim_{k\to\infty}\norm{A^k}^{1/k}
\]
\end{theorem}

\begin{theorem}
If the transition matrix $A$ is aperiodic, then the topological entropy is
\[
	h\left(\sigma_A\right)=\log\rho(A)
\]
where $\rho(A)$ is the spectral radius of the matrix $A$.
\end{theorem}

\begin{proof}
\begin{enumerate}
	\item \textbf{Get your head around the balls}
		
		Take a point $\alpha\in\Sigma_A$ then
		\[
			B(\alpha, n, 2^{-k})=[a_0, \dots, \alpha_{n+k}]
		\]
	\item \textbf{Show that admissible cylinders are non-empty}

		Let $W_m(A)$ denote the set of admissible strings of length $m$.
		Since $A$ is aperiodic (although irreducible will do), given any admissible $\alpha_0\dots\alpha_m$ the cylinder $\left[\alpha_0\dots\alpha_m\right]$ is non-empty.
		This is because we can keep adding rubbish on the end.
	\item \textbf{Relate admissible cylinders to separated and spanning sets}

		Note that admissible cylinders of length $m$ are pairwise disjoint and union to $\Sigma_A$.
		Hence
		\[
			S(n, 2^{-k}) \leq \#W_{n+k+1} \leq N(n, 2^{-k})
		\]
	\item \textbf{Compute the entropy}

		When taking the limit we can get rid of the $k$.
		\[
			h\left(\sigma_A\right)= \limsup_{n\to\infty}\frac{\log(\#W_{n+1}(A))}{n}
		\]
	\item \textbf{Relate \#admissible cylinders to the spectral radius}

		We can choose the norm $\norm{A^n}=\sum_{i,j =1}^N \abs{A^n_{i,j}}$.
		The $(i,j)$th entry of $A^n$ tells us how many admissible words of length $n$ start at $i$ and end at $j$.
		Hence $\norm{A^n}=\#W_{n+1}(A)$.
		Then, using Gelfand's Theorem we have that
		\[
			h\left(\sigma_A\right)=\lim_{n\to\infty}\frac{\log(\#W_{n+1}(A))}{n}=\lim_{n\to\infty}\frac{\log\norm{A^n}}{n}=\log\lambda_1
		\]
\end{enumerate}
\end{proof}

\subsection{of Toral Automorphisms}
\begin{theorem}
Given a hyperbolic toral automorphism with eigenvalues $\lambda_+ > 1 > \lambda_->0$ then
\[
	h(f)=\log \lambda_+
\]
\end{theorem}

\section{Preserved Quantities}

\subsection{Semi-Conjugacies}

\begin{defin}
Given continuous maps $T:X\to X$ and $S:Y\to Y$, a semi-conjugacy from $T$ to $S$ is a continuous surjective map $\pi:Y\to X$ such that
\[
T\circ \pi = \pi \circ S
\]

\begin{figure}[H]
	\centering
	\begin{tikzcd}
		X \arrow[r, "T"] & X  \\
		Y \arrow[r, "S"'] \arrow[u, "\pi"] & Y \arrow[u, "\pi"']
	\end{tikzcd}
\end{figure}
\end{defin}

\begin{theorem}
Let $T:X\to X$ and $S:Y\to Y$ be continuous maps on compact metric spaces and $\pi:Y\to X$ a semi-conjugacy then $h(S)\geq h(T)$.
\begin{figure}[H]
	\centering
	\begin{tikzcd}
		X \arrow[r, "T"] & X  \\
		Y \arrow[r, "S"'] \arrow[u, "\pi"] & Y \arrow[u, "\pi"']
	\end{tikzcd}
\end{figure}

This makes sense because the dynamics of $T$ are contained in the dynamics of $S$ and hence $S$ must be at least as "complex" as $T$.

\end{theorem}

\subsection{Conjugacies}

\begin{itemize}
	\item Rotation number of circle homeomorphisms.
	\item Transitivity and mixing.
	\item Topological entropy.
\end{itemize}

\section{Known Conjugacies}

\subsection{Semi-Conjugacies}

\begin{itemize}
	\item $\pi:\Sigma\to\mathcal{K}$ from full one-sided shift on two symbols to the doubling map.
\end{itemize}

\subsection{Conjugacies}

\begin{itemize}
	\item Expanding maps of the same degree are conjugate (through the linear map of the same degree).
	\item Smale Horseshoe and full two-sided shift on two symbols.
\end{itemize}

\begin{theorem}[Poincar\'es Theorem]
A minimal circle homeomorphism with irrational ration number is conjugate to $R_\alpha$.
\end{theorem}

\begin{theorem}[Denjoy's Theorem]
If $f:\mathcal{K}\to\mathcal{K}$ is a homeomorphism with irrational rotation, $f\in\mathcal{C}^1$ and $w\defeq \log\abs{f'}$ has bounded variation then $f$ is conjugate to $R_\alpha$.
\end{theorem}

\end{document}
