\documentclass[11pt]{article}

%{{{ Packages
\usepackage[margin=1in]{geometry}
\usepackage{enumitem}
\usepackage{amsfonts}
\usepackage{amssymb}
\usepackage{amsmath}
\usepackage{amsthm}
\usepackage{amsmath}
\usepackage{mathdots}
\usepackage{float}
\usepackage[thicklines]{cancel}
\renewcommand{\CancelColor}{\color{red}}
\usepackage[dvipsnames]{xcolor}
\usepackage[framemethod=TikZ]{mdframed}
\usepackage{microtype}
\usepackage{tikz-cd}
\usetikzlibrary{decorations.pathmorphing}
\setlength{\parindent}{0pt}
%}}}
%{{{ Custom commands
% Nice maths commands
\newcommand{\defeq}{:=}
\newcommand{\eqdef}{+:}
\newcommand{\abs}[1]{|#1|}
\newcommand{\norm}[1]{||#1||}
\DeclareMathOperator{\im}{\mathrm{im}}
%\renewcommand{\dots}{...}
\newcommand{\msrspc}{\ensuremath{(X,\mathcal{B},\mu)}}
\newcommand{\relmiddle}[1]{\mathrel{}\middle#1\mathrel{}}
\newcommand{\rmv}{\relmiddle|}
\newcommand{\stcmp}{^{\mathsf{c}}}
\newcommand\restr[2]{{% we make the whole thing an ordinary symbol
  \left.\kern-\nulldelimiterspace % automatically resize the bar with \right
  #1 % the function
  \vphantom{\big|} % pretend it's a little taller at normal size
  \right|_{#2} % this is the delimiter
  }}
\newcommand{\contr}{\Rightarrow\Leftarrow}
\newcommand{\interior}[1]{%
  {\kern0pt#1}^{\mathrm{o}}%
}
\newcommand{\sm}{\setminus}

% Spaces
\newcommand{\ktor}{\mathbb{T}^k}
\newcommand{\R}{\mathbb{R}}
\newcommand{\C}{\mathbb{C}}
\newcommand{\Z}{\mathbb{Z}}
\newcommand{\N}{\mathbb{N}}
\newcommand{\Q}{\mathbb{Q}}

% Derivatives
\newcommand*{\pd}[3][]{\ensuremath{\frac{\partial^{#1} {#2}}{\partial {#3}^{#1}}}}
\newcommand{\grad}{\bigtriangledown}

% Vectors
\newcommand{\mv}[1]{\textbf{#1}}

%}}}
%{{{ Enviornments
% Definitions environment
\newenvironment{defin}
	{\begin{mdframed}[backgroundcolor=white, roundcorner=5pt, linewidth=1pt, linecolor=Green]
		\setlength{\parindent}{0pt}}
	{\end{mdframed}}
	\newcommand{\mdf}[1]{{\color{Green} #1}}

% Important notes environment
\newenvironment{note}
	{\begin{mdframed}[backgroundcolor=white, linecolor=red, roundcorner=5pt, linewidth=1pt]\bfseries{Note:}\normalfont
	\setlength{\parindent}{0pt}}
	{\end{mdframed}}

% Examples enviornmnet
\definecolor{mylg}{rgb}{0.9,0.9,0.9}
\newenvironment{eg}
	{\begin{mdframed}[backgroundcolor=mylg,roundcorner=5pt,linewidth=0pt]\bfseries{Example:}\normalfont
	\setlength{\parindent}{0pt}}
	{\end{mdframed}}

% Theorem environment
\newtheorem{theorem}{Theorem}[section]
\newtheorem{cor}[theorem]{Corollary}
\newtheorem{lemma}[theorem]{Lemma}
\newtheorem{prop}[theorem]{Proposition}
%}}}
%{{{ Document metadata
\title{Dynamical Notes - Proofs to Remember}
\author{Thomas Chaplin}
\date{}
%}}}

\begin{document}
\maketitle

\section{Sharkovskii's Theorem}
\begin{theorem}[Sharkovskii's Theorem]
\label{thrm:shark}
If $f:I \to I $ is continuous and there is a point of prime period 3.
Then for each $n\in \N$ there is a periodic point of prime period $n$.
\end{theorem}

The proof proceeds by a number of lemmata.

\begin{lemma}
Given $I\subseteq[0 ,1 ]$ a closed interval ,if $f(I) \supseteq I$ or $f(I) \subseteq I$ then $I$ contains a fixed point for $f$.
\end{lemma}

\begin{proof}
Use the ITV on $g(x)=f(x)-x$ and consider the endpoints.
\end{proof}

\begin{lemma}[Whittling down intervals]
If $I, I' \subseteq [0,1]$ are closed intervals and $f(I)=I'$, then $\exists$ a closed interval $I_0\subseteq I$ such that $f(I_0)=I'$.
\end{lemma}
\begin{proof}
Suppose $I'=[a, b]$ then let
\begin{align*}
	A & \defeq f^{-1}(a)\cap I \\
	B & \defeq f^{-1}(b)\cap I
\end{align*}
then take $x_0=\sup(A)$ and $y_0=\inf(B)$.
Then $I_0\defeq [x_0, y_0]$ will do the job.
\end{proof}

\begin{lemma}
Assume that we have closed intervals $I_1, \dots, I_n \subseteq [0, 1]$ such that
\begin{itemize}
	\item $f(I_n)\supseteq I_1$,
	\item $f(I_j)\supseteq I_{j+1}$ for all appropriate $j$,
\end{itemize}
then there is a fixed point $x$ for $f^n$ such that
\[
	x\in I_1, f(x) \in I_2, \dots , f^{n-1}\in I_n
\]
\end{lemma}

\begin{proof}
We can just apply the whittling lemma to the intervals in reverse order so
\[
\begin{array}{lcl}
	\exists I_n' \subseteq I_n & s.t. & f(I_n') = I_1\\
	\exists I_{n-1}' \subseteq I_{n-1} & s.t. & f(I_{n-1}') = I_n'\\
	 \; & \vdots & \; \\
	\exists I_1' \subseteq I_1 & s.t. & f(I_1)' = I_2'
\end{array}
\]
In particular we have that $f^n(I_1')=I_1\supseteq I_1'$ and hence the first lemma gives us the desired fixed point.
\end{proof}

\begin{proof}
\textit{of Theorem \ref{thrm:shark}.}

Let $f^3(x)=x$ be our point of prime period 3.
For now we will assume that 
\[
	\left\{x, f(x), f^2(x)\right\} = \left\{x_1, x_2, x_3\right\}
\]
where $0\leq x_1 < x_2 < x_3 \leq 1$.
We also assume $f(x_1)=x_2$, $f(x_2)=x_3$ and $f(x_3)=x_1$.
Other cases are similar.
Let $I_0\defeq[x_1, x_2]$ and $I_1\defeq[x_2, x_3]$.

Observe that
\begin{enumerate}[label=(\alph*)]
	\item $f(I_0)\supseteq I_1$, and
	\item $f(I_1)\supseteq I_0\cup I_1$.
\end{enumerate}

\noindent We now split the proof into a number of cases:

\noindent\textbf{Case 1: }($n=3$) This follows from the assumption.

\noindent\textbf{Case 2: }($n=1$) This follows from the first lemma thanks to \textit{(b)}.

\noindent\textbf{Case 3: }($n=2$ or $n\geq 4$)

Note that we can make a chain of intervals as in the last Lemma.
\begin{figure}[H]
	\centering
	\begin{tikzcd}
		I_0 \arrow[r, rightsquigarrow, "f"]\arrow[rrrrr, bend right=15, dashrightarrow, "n-1 \text{ times}"'] & I_1 \arrow[r, rightsquigarrow, "f"] & I_1 \arrow[r, rightsquigarrow, "f"] & \dots \arrow[r, rightsquigarrow, "f"] & I_1 \arrow[r, rightsquigarrow, "f"] & I_0
	\end{tikzcd}
\end{figure}
where $A\rightsquigarrow B$ means $f(A) \supseteq B$.
Hence there is a fixed point for $f^n$ which starts in $I_0$ spends $n-1$ in $I_1$ and then returns to $I_0$.
Because the earliest return is at time $n$ we can be sure that this is our prime period.
\end{proof}

\section{Dense Irrational Orbits}
\begin{theorem}
If $\alpha\in\R\setminus\Q$ then for any $z\in\mathcal{K}$ we have
\[
	\left\{R_\alpha^n(x) \rmv n\in \N\right\}
\]
is a dense set in the circle $\mathcal{K}$.
\end{theorem}

\begin{proof}
Get yourself some pigeons.
If you put a pigeon on the circle for every point in the orbit up to time $\frac{1}{\epsilon}+1$ then two pigeons, Kenny $k$ and Lenny $l$, must be $\epsilon$ close.
\[
	d(R_\alpha^l(p), R_\alpha^k(p)) < \epsilon
\]
Without loss of generality, assume that Kenny is further along the orbit then Lenny so that 
\[m\defeq k-l > 0.\]
Then for any $x\in\mathcal{K}$ we have $d(R_\alpha^m(x), x < \epsilon)$.
Hence the orbit $\left\{x, R_\alpha^m(x), R_\alpha^{2m}(x), R_\alpha^{3m}(x) , \dots\right\}$ is $\epsilon$ dense in the circle.
\end{proof}

\section{Rational Points and Periodic Points}
\begin{theorem}
If $f:\mathcal{K}\to\mathcal{K}$ has a periodic point $x_0$ of period $m$ then $\alpha(f)\in\Q$.
\end{theorem}
\begin{proof}
Let $F:\R\to\R$ be some lift then we must have
\[
	F^m(x)-x = k\in \Z
\]
where $\rho(x)=x_0$.
Then we can write any integer as $n=pm+r$ where $p\geq 0$ and $r\in[0, m)$.
Hence
\[
	F^n(x)=F^{pm+r}(x)=F^r(x)+pk
\]
Then we can conclude
\[
	\lim_{n\to\infty}\frac{1}{n}F^n(x)=\lim_{p\to\infty}\frac{1}{pm+r}\left(F^r(x)+pk\right)=\frac{k}{m}\in\Q
\]
\end{proof}

\begin{theorem}
If $f:\mathcal{K}\to\mathcal{K}$ has $0$ rotation number then $f$ has a fixed point.
\end{theorem}

\begin{proof}
\begin{itemize}
	\item Take a lift $\widetilde{F}$ that gives $\lim_{n\to\infty}\frac{\widetilde{F}^n(x)}{n}=m$.
	\item Create a nicer lift $F\defeq\widetilde{F}-m$ so that the limit becomes $0$.
	\item Assume $d$ has no fixed point then by the IVT we can assume WLOG $F(y) > y$ for all $y\in\R$.
	\item Hence $\left(F^n(0)\right)_{n\in\N}$ is increasing so we just need to show boundedness.
	\item Suppose unbounded then $\abs{F^{n_0}(0)}>1$ and hence for all $m$ we have $\abs{F^{mn_0}(0)} > m$.
		\[
			\frac{\abs{F^{mn_0}(0)}}{mn_0} > \frac{m}{mn_0} = \frac{1}{n_0}
		\]
		which cause a contradiction with the earlier $0$ limit.
	\item It can be seen that the limit of this sequence is a fixed point.
\end{itemize}
\end{proof}

\begin{note}
	As a corollary if the rotation number is $\frac{a}{b}\in\Q$ then $f^b$ has $0$ rotation number and hence fixed point.
	Therefore, $f$ has a periodic point.
\end{note}

\end{document}
