\documentclass[11pt]{article}

%{{{ Packages
\usepackage[margin=1in]{geometry}
\usepackage{enumitem}
\usepackage{amsfonts}
\usepackage{amssymb}
\usepackage{amsmath}
\usepackage{amsthm}
\usepackage{amsmath}
\usepackage{mathdots}
\usepackage{float}
\usepackage[thicklines]{cancel}
\renewcommand{\CancelColor}{\color{red}}
\usepackage[dvipsnames]{xcolor}
\usepackage[framemethod=TikZ]{mdframed}
\usepackage{microtype}
\usepackage{tikz-cd}
\usetikzlibrary{decorations.pathmorphing}
\setlength{\parindent}{0pt}
%}}}
%{{{ Custom commands
% Nice maths commands
\newcommand{\defeq}{:=}
\newcommand{\eqdef}{+:}
\newcommand{\abs}[1]{|#1|}
\newcommand{\norm}[1]{||#1||}
\DeclareMathOperator{\im}{\mathrm{im}}
%\renewcommand{\dots}{...}
\newcommand{\msrspc}{\ensuremath{(X,\mathcal{B},\mu)}}
\newcommand{\relmiddle}[1]{\mathrel{}\middle#1\mathrel{}}
\newcommand{\rmv}{\relmiddle|}
\newcommand{\stcmp}{^{\mathsf{c}}}
\newcommand\restr[2]{{% we make the whole thing an ordinary symbol
  \left.\kern-\nulldelimiterspace % automatically resize the bar with \right
  #1 % the function
  \vphantom{\big|} % pretend it's a little taller at normal size
  \right|_{#2} % this is the delimiter
  }}
\newcommand{\contr}{\Rightarrow\Leftarrow}
\newcommand{\interior}[1]{%
  {\kern0pt#1}^{\mathrm{o}}%
}
\newcommand{\sm}{\setminus}

% Spaces
\newcommand{\ktor}{\mathbb{T}^k}
\newcommand{\R}{\mathbb{R}}
\newcommand{\C}{\mathbb{C}}
\newcommand{\Z}{\mathbb{Z}}
\newcommand{\N}{\mathbb{N}}
\newcommand{\Q}{\mathbb{Q}}

% Derivatives
\newcommand*{\pd}[3][]{\ensuremath{\frac{\partial^{#1} {#2}}{\partial {#3}^{#1}}}}
\newcommand{\grad}{\bigtriangledown}

% Vectors
\newcommand{\mv}[1]{\textbf{#1}}

%}}}
%{{{ Enviornments
% Definitions environment
\newenvironment{defin}
	{\begin{mdframed}[backgroundcolor=white, roundcorner=5pt, linewidth=1pt, linecolor=Green]
		\setlength{\parindent}{0pt}}
	{\end{mdframed}}
	\newcommand{\mdf}[1]{{\color{Green} #1}}

% Important notes environment
\newenvironment{note}
	{\begin{mdframed}[backgroundcolor=white, linecolor=red, roundcorner=5pt, linewidth=1pt]\bfseries{Note:}\normalfont
	\setlength{\parindent}{0pt}}
	{\end{mdframed}}

% Examples enviornmnet
\definecolor{mylg}{rgb}{0.9,0.9,0.9}
\newenvironment{eg}
	{\begin{mdframed}[backgroundcolor=mylg,roundcorner=5pt,linewidth=0pt]\bfseries{Example:}\normalfont
	\setlength{\parindent}{0pt}}
	{\end{mdframed}}

% Theorem environment
\newtheorem{theorem}{Theorem}[section]
\newtheorem{cor}[theorem]{Corollary}
\newtheorem{lemma}[theorem]{Lemma}
\newtheorem{prop}[theorem]{Proposition}
%}}}
%{{{ Document metadata
\title{Dynamical Notes - Proofs to Remember}
\author{Thomas Chaplin}
\date{}
%}}}

\begin{document}
\maketitle

\section{Sharkovskii's Theorem}
\begin{theorem}[Sharkovskii's Theorem]
\label{thrm:shark}
If $f:I \to I $ is continuous and there is a point of prime period 3.
Then for each $n\in \N$ there is a periodic point of prime period $n$.
\end{theorem}

The proof proceeds by a number of lemmata.

\begin{lemma}
Given $I\subseteq[0 ,1 ]$ a closed interval ,if $f(I) \supseteq I$ or $f(I) \subseteq I$ then $I$ contains a fixed point for $f$.
\end{lemma}

\begin{proof}
Use the ITV on $g(x)=f(x)-x$ and consider the endpoints.
\end{proof}

\begin{lemma}[Whittling down intervals]
If $I, I' \subseteq [0,1]$ are closed intervals and $f(I)=I'$, then $\exists$ a closed interval $I_0\subseteq I$ such that $f(I_0)=I'$.
\end{lemma}
\begin{proof}
Suppose $I'=[a, b]$ then let
\begin{align*}
	A & \defeq f^{-1}(a)\cap I \\
	B & \defeq f^{-1}(b)\cap I
\end{align*}
then take $x_0=\sup(A)$ and $y_0=\inf(B)$.
Then $I_0\defeq [x_0, y_0]$ will do the job.
\end{proof}

\begin{lemma}
Assume that we have closed intervals $I_1, \dots, I_n \subseteq [0, 1]$ such that
\begin{itemize}
	\item $f(I_n)\supseteq I_1$,
	\item $f(I_j)\supseteq I_{j+1}$ for all appropriate $j$,
\end{itemize}
then there is a fixed point $x$ for $f^n$ such that
\[
	x\in I_1, f(x) \in I_2, \dots , f^{n-1}\in I_n
\]
\end{lemma}

\begin{proof}
We can just apply the whittling lemma to the intervals in reverse order so
\[
\begin{array}{lcl}
	\exists I_n' \subseteq I_n & s.t. & f(I_n') = I_1\\
	\exists I_{n-1}' \subseteq I_{n-1} & s.t. & f(I_{n-1}') = I_n'\\
	 \; & \vdots & \; \\
	\exists I_1' \subseteq I_1 & s.t. & f(I_1)' = I_2'
\end{array}
\]
In particular we have that $f^n(I_1')=I_1\supseteq I_1'$ and hence the first lemma gives us the desired fixed point.
\end{proof}

\begin{proof}
\textit{of Theorem \ref{thrm:shark}.}

Let $f^3(x)=x$ be our point of prime period 3.
For now we will assume that 
\[
	\left\{x, f(x), f^2(x)\right\} = \left\{x_1, x_2, x_3\right\}
\]
where $0\leq x_1 < x_2 < x_3 \leq 1$.
We also assume $f(x_1)=x_2$, $f(x_2)=x_3$ and $f(x_3)=x_1$.
Other cases are similar.
Let $I_0\defeq[x_1, x_2]$ and $I_1\defeq[x_2, x_3]$.

Observe that
\begin{enumerate}[label=(\alph*)]
	\item $f(I_0)\supseteq I_1$, and
	\item $f(I_1)\supseteq I_0\cup I_1$.
\end{enumerate}

\noindent We now split the proof into a number of cases:

\noindent\textbf{Case 1: }($n=3$) This follows from the assumption.

\noindent\textbf{Case 2: }($n=1$) This follows from the first lemma thanks to \textit{(b)}.

\noindent\textbf{Case 3: }($n=2$ or $n\geq 4$)

Note that we can make a chain of intervals as in the last Lemma.
\begin{figure}[H]
	\centering
	\begin{tikzcd}
		I_0 \arrow[r, rightsquigarrow, "f"]\arrow[rrrrr, bend right=15, dashrightarrow, "n-1 \text{ times}"'] & I_1 \arrow[r, rightsquigarrow, "f"] & I_1 \arrow[r, rightsquigarrow, "f"] & \dots \arrow[r, rightsquigarrow, "f"] & I_1 \arrow[r, rightsquigarrow, "f"] & I_0
	\end{tikzcd}
\end{figure}
where $A\rightsquigarrow B$ means $f(A) \supseteq B$.
Hence there is a fixed point for $f^n$ which starts in $I_0$ spends $n-1$ in $I_1$ and then returns to $I_0$.
Because the earliest return is at time $n$ we can be sure that this is our prime period.
\end{proof}

\section{Independence of Lifts}

\section{Dense Irrational Orbits}
\begin{theorem}
If $\alpha\in\R\setminus\Q$ then for any $z\in\mathcal{K}$ we have
\[
	\left\{R_\alpha^n(x) \rmv n\in \N\right\}
\]
is a dense set in the circle $\mathcal{K}$.
\end{theorem}

\begin{proof}
Get yourself some pigeons.
If you put a pigeon on the circle for every point in the orbit up to time $\frac{1}{\epsilon}+1$ then two pigeons, Kenny $k$ and Lenny $l$, must be $\epsilon$ close.
\[
	d(R_\alpha^l(p), R_\alpha^k(p)) < \epsilon
\]
Without loss of generality, assume that Kenny is further along the orbit then Lenny so that 
\[m\defeq k-l > 0.\]
Then for any $x\in\mathcal{K}$ we have $d(R_\alpha^m(x), x < \epsilon)$.
Hence the orbit $\left\{x, R_\alpha^m(x), R_\alpha^{2m}(x), R_\alpha^{3m}(x) , \dots\right\}$ is $\epsilon$ dense in the circle.
\end{proof}

\section{Rational Points and Periodic Points}
\begin{theorem}
If $f:\mathcal{K}\to\mathcal{K}$ has a periodic point $x_0$ of period $m$ then $\alpha(f)\in\Q$.
\end{theorem}
\begin{proof}
Let $F:\R\to\R$ be some lift then we must have
\[
	F^m(x)-x = k\in \Z
\]
where $\rho(x)=x_0$.
Then we can write any integer as $n=pm+r$ where $p\geq 0$ and $r\in[0, m)$.
Hence
\[
	F^n(x)=F^{pm+r}(x)=F^r(x)+pk
\]
Then we can conclude
\[
	\lim_{n\to\infty}\frac{1}{n}F^n(x)=\lim_{p\to\infty}\frac{1}{pm+r}\left(F^r(x)+pk\right)=\frac{k}{m}\in\Q
\]
\end{proof}

\begin{theorem}
If $f:\mathcal{K}\to\mathcal{K}$ has $0$ rotation number then $f$ has a fixed point.
\end{theorem}

\begin{proof}
\begin{itemize}
	\item Take a lift $\widetilde{F}$ that gives $\lim_{n\to\infty}\frac{\widetilde{F}^n(x)}{n}=m$.
	\item Create a nicer lift $F\defeq\widetilde{F}-m$ so that the limit becomes $0$.
	\item Assume $d$ has no fixed point then by the IVT we can assume WLOG $F(y) > y$ for all $y\in\R$.
	\item Hence $\left(F^n(0)\right)_{n\in\N}$ is increasing so we just need to show boundedness.
	\item Suppose unbounded then $\abs{F^{n_0}(0)}>1$ and hence for all $m$ we have $\abs{F^{mn_0}(0)} > m$.
		\[
			\frac{\abs{F^{mn_0}(0)}}{mn_0} > \frac{m}{mn_0} = \frac{1}{n_0}
		\]
		which cause a contradiction with the earlier $0$ limit.
	\item It can be seen that the limit of this sequence is a fixed point.
\end{itemize}
\end{proof}

\begin{note}
	As a corollary if the rotation number is $\frac{a}{b}\in\Q$ then $f^b$ has $0$ rotation number and hence fixed point.
	Therefore, $f$ has a periodic point.
\end{note}

\section{Pointer\'e's Theorem and Minimality}

\begin{defin}
	A homeomorphism is called \mdf{minimal} if every orbit is dense.	
\end{defin}

\begin{eg}
Any irrational rotation $R_\alpha$ is minimal.
\end{eg}

\begin{theorem}[Poincar\'e's Theorem]
Any minimal circle homeomorphism is topologically conjugate to an irrational rotation.
\end{theorem}

Given a circle homeomorphism $f:\mathcal{K}\to\mathcal{K}$ and some lift $F$ we define the following countable sets
\begin{align*}
	\Lambda_{x_0}&\defeq\left\{F^n(x_0)+m \rmv m, n\in \Z\right\}\\
	\Omega &\defeq \left\{n \rho +m \rmv m, n \in \Z\right\}
\end{align*}
for some fixed $x_0\in\R$ and where $\rho = \rho(f)$ is the rotation number.
Note that $\Lambda_{x_0}=\pi^{-1}\left\{f^n(\pi x_0)\right\}$ and $\Omega=\pi^{-1}\left\{R_\rho ^n (0)\right\}$ where $\pi$ is the usual projection.

\begin{lemma}
Let $f$ be a circle homeomorphism and $x_0\in\mathcal{K}$.
If the rotation number $\rho$ is irrational then the map $T:\Lambda_{x_0}\to\Omega$ given be
\[
	T(F^n(x_0)+m)=n\rho +m
\]
is a bijection.
Moreover,
\begin{enumerate}
	\item $T$ is strictly increasing
	\item $T(x+1) = T(x) + 1$
	\item $T(F(x))= T(x) + \rho$ for all $x\in\Lambda_{x_0}$.
\end{enumerate}
\end{lemma}

\begin{proof}
This is omitted but might be worth glancing over.
\end{proof}

\begin{proof}
\textit{of Poincar\'e's Theorem}
Since $f$ is minimal, it has no periodic points because their orbits would be finite and hence not dense.
So the rotation number $\rho$ is irrational.


Take a lift $F$ of $f$ and $x_0\in \R$ and write $\Lambda=\Lambda_{x_0}$.
The sets $\Omega$ and $\Lambda$ are dense in $\R$ due to the minimality of $R_\rho$ and $f$ respectively.

Thus $\pi(\Omega)$ and $\pi(\Lambda)$ must be dense in $\mathcal{K}$.
Moreover, the Lemma tells us that $T:\Lambda\to\Omega$ is strictly increasing.
Consequently, we can extend to a unique continuous function $H:\R \to \R$ (which restricts to $T$ on $\Lambda$).
Moreover, $H$ is strictly increasing, $H$ is continuous and so is its inverse.

\begin{note}
This is non-trivial.
It is an exercise to show that given dense sets $X,Y\subseteq \R$ and $f:X\to Y$ a bijection, there exists a unique homeomorphism extension to $\R$.
\end{note}

By continuity $H$ inherits the properties \textit{(2)} and \textit{(3)} in the previous Lemma.
The first says that $H$ is a lift of circle homeomorphism $h$.
The second say that $h\circ f= R_\rho \circ h$.
\end{proof}

So we now know that if $f$ is a circle homeomorphism then there is a unique homeomorphism $h$ satisfying
\[
	h(f(x))= h(x) + \rho \quad (mod 1) \quad \forall x\in\mathcal{K}
\]
Note that this is a linear equation on $h$.
We can conclude that a solution to this equation is unique up to adding a constant corresponding to choosing with point in $\mathcal{K}$ is sent to zero.
For a hand-wavey explanation of this, see the lecture notes.

\section{Expanding Maps and Shift Spaces}
\begin{theorem}
If $f:\mathcal{K}\to\mathcal{K}$ is an expanding map, preserves orientation and has degree 2 then there is a semi-conjugacy $h:\Sigma\to\mathcal{K}$ to the full shift on two symbols.

\begin{figure}[H]
	\centering
	\begin{tikzcd}
		\Sigma \arrow[r, "\sigma"] \arrow[d, "h"'] & \Sigma \arrow[d, "h"]\\
		\mathcal{K} \arrow[r, "f"] & \mathcal{K}
	\end{tikzcd}
\end{figure}
\end{theorem}

\begin{proof}
Take any $n\in\N$.
Then $\deg f^n = (\deg f)^n$ so there are $w^n$ pre-images of $p$ under $f^n$.
These are numbered $p_j$ starting with $p_0=p$ and number consecutively anticlockwise.
These points define intervals which we denote $A_{\omega_0\dots\omega_{n-1}}$ where the sequence of $\omega_i$ is just the binary representation of the position in the circle.

Let $K$ denote the uniform bound away from $1$ of the derivative.
We have a number of results:
\begin{enumerate}
	\item $f^n(A_{\omega_0\dots\omega_{n-1}}^\circ)=\mathcal{K}\setminus\left\{p\right\}$
	\item $A_{\omega_0\dots\omega_{n-1}}$ is a closed interval of length $<K^{-n}$.
	\item $A_{\omega_0\dots\omega_{n-1}\omega_n}\subseteq A_{\omega_0\dots\omega_{n-1}}$.
	\item $f^n(A_{\omega_0\dots\omega_{n}})=A_{\omega_n}$.
	\item $f(A_{\omega_0\dots\omega_{n}})=A_{\omega_1\dots\omega_{n}}$.
\end{enumerate}

Now we can define our conjugacy $h:\Sigma\to\mathcal{K}$.
Given $\omega=(\omega_k)_{k=0}^\infty\in\Sigma$ let $B_n(\omega)=A_{\omega_0\dots\omega_{n-1}}$.
These are the points in the circle that start in the $\omega_0$ interval then go to $\omega_1$, then to $\omega_2$ and after $f^{n-1}$ are in the $\omega_{n-1}$ interval.
The properties implies that $B_{n+1}(\omega) \subseteq B_n(\omega)$.
The sets are also closed and their diameters go to $0$.
Hence their infinite intersection is a single points which we define to be $h(\omega)$.
The proof of their desired properties is discussed below in vague detail but is written in the lecture notes with more rigour.
\end{proof}

\section{Finding semi-conjugacies/conjugacies}
If you can partition your space $X$ into $n$ subsets $I_1, \dots, I_n$ where one could conceivably go from any partition element $I_a$ to any other $I_b$, then you might be able to find a semi-conjugacy to the full shift on $n$ symbols.

The trick is to define a map $\pi:\Sigma\to X$ by
\[
	\pi(\mv{x})=\bigcap_{n=1}^\infty T^{-n}I_{x_n}
\]
If the sets $I(x_0, \dots, x_n)\defeq\cap_{k=0}^n T^{-k}I_{x_k}$ are closed and nested and their diameter tends to zero as $n\to \infty$ then this map is well-defined because the infinite intersection contains one point.
Moreover, it is continuous because if $\mv{x}$ and $\mv{y}$ agree up to $N$ places then they both lie in $I(x_0, \dots x_{N-1})$ whose diameters goes to $0$ as $N\to\infty$.

The commutative relationship $T\circ\pi=\pi\circ\sigma$ then follows rather quickly.
To get surjectivity, it suffices to show that the image of $\Sigma$ is dense.
This usually involves taking in point $x\in X$ such that no $T^n x$ lies on the boundary between any $I_j$ for some $n\geq 0$ and then this points orbit will describe its pre-image in $\Sigma$.

\begin{note}
	Shift spaces are \mdf{totally disconnected}, i.e. the connected components are one-point sets.
	In particular, they are disconnected and so this can often be used to rule out the existence of conjugacies to more familiar sets.
\end{note}



\section{Transitivity and Mixing}
\begin{note}
A compact metric space has a countable dense set of points!
\end{note}

\begin{theorem}[Baire's Theorem]
Given a compact metric space $X$, the intersection of countably many open, dense subsets of $X$ is itself dense in $X$.
\end{theorem}

\begin{theorem}
If a map $T:X\to X$ on a compact metric space $X$ is topologically transitive then there exists a dense orbit.
\end{theorem}

\begin{proof}
There is a countable dense set of points $\left\{x_k\right\}$ so if we can find an orbit the gets $\epsilon$ close to every $x_k$ for arbitrary $\epsilon$ then we are done.
So we want $x$ such that for every $x_k$ and $m\geq 1$ there is an $n\in\Z$ such that
\[
	x\in T^{-n}\mathbb{B}\left(x_k,\frac{1}{m}\right)
\]
or equivalently we want to find
\[
	x\in\bigcap_{k, m}\bigcup_{n\in\Z}T^{-n}\mathbb{B}\left(x_k,\frac{1}{m}\right)
\]
which is a countable intersection (over $m$) of open dense sets.
By Baire's Theorem our desired point exists.
\end{proof}


\section{Arithmetic Progressions}
\begin{defin}
	We say a subset $C\subseteq \Z$ \mdf{contains arithmetic progressions of arbitrary length} if
\[
	\forall k \geq 1 \quad \exists c\in \Z \;\text{ and }\; d\in\N \quad \text{such that}
\]\[
	c, c+d, c+ 2d, \dots, c+ (k-1)d\in C
\]
Similarly we say a map $T:X\to X$ is \mdf{multiple mixing} if for any non-empty open set $U\subseteq X$ and $k\geq 1$ there exists $d\geq 1$ such that
\[
	U\cap T^{-d}U \cap T^{-2d}U \cap \dots \cap T^{-(k-1)d}U\neq \emptyset
\]
\end{defin}

\begin{theorem}[van der Waerden's Theorem]
Given any finite integer partition $\Z=\cup_{i=1}^M C_i$ there is an $i$ such that $C_i$ contains arithmetic progressions of arbitrary length.
\end{theorem}

To prove this via a dynamical approach we must create a dynamical formulation.
To a partition of $\Z$ we associate  a single infinite sequence $\mv{x}=(x_n)\in\left\{1, \dots, M\right\}^\Z$ defined by
\[
	x_n=i\quad\text{if}\quad n\in C_i
\]
Let $X=\overline{\cup_{n\in\Z}\sigma^n\mv{x}}$ be the closure of the orbit of $\mv{x}$ where $\sigma$ is the shift map.

\begin{lemma}[Dynamic Formulation]
Assume that for some $[i]$ (cylinder set) we have that
\[
	X\cap [i] \cap \sigma^{-d}[i]\cap \sigma^{-2d}[i] \cap \dots \sigma^{-(k-1)d}[i]\neq\emptyset
\]
for some $k,d\geq 1$ then $C_i$ contains an arithmetic progression of length $k$.
\end{lemma}

\begin{proof}
The space is the closure of the orbit of $\mv{x}$ and this set is non-empty and open.
The orbit itself is dense in $X$ and hence intersects our open set.
So there is $n\in\Z$ such that $\sigma^n x$ is in our set.
This means that $x_{n+jd}=i$ for $j=0, \dots, k-1$ and hence $n+jd\in C_i$ for these $j$.
\end{proof}

\begin{prop}[Multiple Recurrence]
The shift map is multiple mixing when restricted to a minimal subset $Y\subseteq X$.
\end{prop}

\begin{proof}
\textit{of van der Waerden's Theorem}
Take a minimal subset $Y\subseteq X$.
Taking $U=[i]$ where $i$ is chosen such that $[i]\cap Y \neq \emptyset$, we see the set from the dynamical formulation is open and hence non-empty by multiple recurrence so we have arbitrary arithmetic progressions.
\end{proof}

\section{Hyperbolic Toral Automorphisms}
Fixed points and mixing

\section{Entropy}
of shift maps and toral automorphisms.

\end{document}
