\documentclass[11pt]{article}

%{{{ Packages
\usepackage[margin=1in]{geometry}
\usepackage{enumitem}
\usepackage{amsfonts}
\usepackage{amssymb}
\usepackage{amsmath}
\usepackage{amsthm}
\usepackage{mathdots}
\usepackage{mathtools}
\usepackage[dvipsnames]{xcolor}
\usepackage[framemethod=TikZ]{mdframed}
\usepackage{microtype}
\usepackage{witharrows}
\usepackage{float}
\usepackage{silence}
\WarningFilter{mdframed}{You got a bad break}
\usepackage{tikzsymbols}
\setlength{\parindent}{0pt}
%}}}
%{{{ Custom commands
% Nice maths commands
\newcommand{\defeq}{:=}
\newcommand{\eqdef}{+:}
\newcommand{\abs}[1]{|#1|}
\newcommand{\norm}[1]{||#1||}
%\renewcommand{\dots}{...}
\newcommand{\msrspc}{\ensuremath{(X,\mathcal{A},\mu)}}
\newcommand{\sigal}{$\sigma$-algebra}
\newcommand{\relmiddle}[1]{\mathrel{}\middle#1\mathrel{}}
\newcommand{\rmv}{\relmiddle|}
\newcommand{\dm}{\ensuremath{\,d\mu}}
\DeclareMathOperator\vol{vol}
\newcommand{\stcmp}{^{\mathsf{c}}}
\newcommand\restr[2]{{% we make the whole thing an ordinary symbol
  \left.\kern-\nulldelimiterspace % automatically resize the bar with \right
  #1 % the function
  \vphantom{\big|} % pretend it's a little taller at normal size
  \right|_{#2} % this is the delimiter
  }}

% Spaces
\newcommand{\pow}[1]{\mathcal{P}(#1)}
\newcommand{\ktor}{\mathbb{T}^k}
\newcommand{\R}{\mathbb{R}}
\newcommand{\Rb}{\overline{\R}}
\newcommand{\C}{\mathbb{C}}
\newcommand{\Z}{\mathbb{Z}}
\newcommand{\N}{\mathbb{N}}
\newcommand{\Q}{\mathbb{Q}}

% Derivatives
\newcommand*{\pd}[3][]{\ensuremath{\frac{\partial^{#1} {#2}}{\partial {#3}^{#1}}}}
\newcommand{\grad}{\bigtriangledown}

% Vectors
\newcommand{\mv}[1]{\textbf{#1}}

%}}}
%{{{ Enviornments
% Definitions environment
\newenvironment{defin}
	{\begin{mdframed}[backgroundcolor=white, roundcorner=5pt, linewidth=1pt]}
	{\end{mdframed}}
\newcommand{\mdf}[1]{{\color{red} #1}}

% Important notes environment
\newenvironment{note}
	{\begin{mdframed}[backgroundcolor=white, linecolor=red, roundcorner=5pt, linewidth=1pt]\bfseries{Note:}\normalfont}
	{\end{mdframed}}

% Examples enviornmnet
\definecolor{mylg}{rgb}{0.9,0.9,0.9}
\newenvironment{eg}
{\begin{mdframed}[backgroundcolor=mylg,roundcorner=5pt,linewidth=0pt]\bfseries{Example:}\normalfont}
	{\end{mdframed}}

\newcommand{\mem}[1]{{\color{red} #1}}
% Theorem enviornment
\newtheorem{theorem}{Theorem}[section]
\newtheorem{prop}[theorem]{Proposition}
\newtheorem{cor}[theorem]{Corollary}
\newtheorem{lemma}[theorem]{Lemma}
%}}}
%{{{ Document metadata
\title{Radon-Nikodym - Statement and Proof}
\author{}
\date{}
%}}}

\begin{document}
\maketitle
\section{Statement}
\begin{theorem}[Radon-Nikodym Theorem]
Suppose $\mu$ and $\nu$ are $\sigma$-finite measures on $(X, \mathcal{A})$ such that $\nu << \mu$.
Then $\exists$ a measurable function $f: X \to [0, +\infty)$ such that
\[
	\forall A \in \mathcal{A}, \quad \nu(A)=\int_A f \dm
\]
Moreover, for all such functions h, we have that $h=f$ $\mu$-almost everywhere.
We call such function the \mdf{Radon-Nikodym derivative} which we denote $\frac{d\nu}{d\mu}$.
\end{theorem}

\section{Existence}
\begin{proof}
Suppose $\mu(x), \nu(X) < \infty$ then let
\[
	\mathcal{F}\defeq\left\{f:X\to[0, + \infty) \text{ measurable} \rmv \forall A \in \mathcal{A} \quad \int_A f\dm \leq \nu (A)\right\}
\]
Note $\mathcal{F}\neq\emptyset$ because it contains the $0$-function.

\textbf{Claim: }Given $f_1, f_2\in\mathcal{F}$ then $f_1 \vee f_2\in\mathcal{F}$.

Choose a countable sequence $f_1, f_2, \dots \in \mathcal{F}$ such that
\[
	\lim_{n\to\infty}\int f_n \dm = \sup \left\{\int f \rmv f\in \mathcal{F}\right\}
\]
By replacing $f_i$ with $f_1 \vee \dots \vee f_i$ we can assume that this sequence is increasing.
Let $g\defeq\lim_{n\to\infty}f_n$.
Then for all measurable $A\in\mathcal{A}$ we have
\[
	\int_A g \dm = \lim_{n\to\infty}\int_A f_n \dm \leq \nu(A)
\]
by the monotone convergence theorem.
Therefore we also have $g\in\mathcal{F}$.

We're going to define a new measure $\rho$ by
\[
	\rho(A)\defeq \nu(A)-\int_A g\dm \geq 0
\]
We aim to show that this is the 0 measure.
Suppose for contradiction that $\rho(X) >0$ then we can find an $\epsilon >0$ such that $\rho(X) > \epsilon \mu(X)$ because $\mu(X)$ is finite.
Now we take a Hahn decomposition $X=P \sqcup N$ of the signed measure $\rho - \epsilon\mu$.
\[
	\begin{WithArrows}
		\forall A \in\mathcal{A} \quad\quad \nu(A)&= \int_A g\dm + \rho(A) \Arrow{smaller set}\\
				 & \geq \int_A g \dm + \rho (A\cap P) \Arrow{since P positive set for $\rho - \epsilon\mu$} \\
				 & \geq \int_A g \dm + \epsilon\mu (A\cap P)\\
				 & \geq \int_A ( g+ \epsilon \chi_P) \dm
	\end{WithArrows}
\]
This is clearly a problem because we've found an integral larger than the largest integral however $g+\epsilon\chi_P$ is the same as $g$ up to measure $0$ if $\mu(P)=0$.
Therefore we split into two cases:

\textbf{Case 1: }If $\mu(P)=0$ then $\rho(P)=\nu(P)-\int_P g\dm = 0 - 0$ since $\nu < < \mu$.
Hence, $\rho(X) - \epsilon\mu(X) = \rho(N) - \epsilon\mu(N)$ since both measure vanish on $P$.
But then $N$ is a negative set so $\rho(X) - \epsilon\mu(X) \leq 0$ which contradicts our definition of $\epsilon$.


\textbf{Case 2: }If $\mu(P)\neq 0$ then $g+ \epsilon\chi_P \in \mathcal{F}$ but if we integrate over $X$ we get
\[
	\int (g+ \epsilon\chi_P)\dm > \int g \dm
\]
which is a contradiction because $g$ was chosen to maximise this integral subject to $\int_A g \dm \leq \nu (A)$.

This proves that $\rho(X)=0$ and so $\rho\equiv 0$, i.e.
\[
	\forall A \in \mathcal{A} \quad \nu(A) = \int_A g \dm
\]

\textbf{Problem: }What if $g(x) = \infty$ for some $x\in X$?

Well we certainly have $\int g \dm \leq \nu (X) < +\infty$ and hence $\mu\left(\left\{g = +\infty\right\}\right)=0$.
We can just redefine $g$ to be $0$ on the set where it was previously infinity and this won't change any integrals.
\end{proof}

\subsection{General Case}
Previously we assumed that $\mu$ and $\nu$ were finite, but actually we can do the proof when they are only $\sigma$-finite.

\begin{proof}
Decompose the space $X=\sqcup_n B_n$ where the $B_n$ are some measurable sets such that $\mu(B_n),\nu(B_n)< \infty$.
Then for each $n$ we can define new finite measures by 
\[
	\mu_n(A)= \mu(A\cap B_n)\quad\text{and}\quad\nu_n(A)=\nu(A\cap B_n)
\]
Note that we still maintain absolute continuity $\nu_n << \mu_n$ for each $n$.
We can now apply the first existence proof to each $\mu_n$ and $\nu_n$ to find Radon-Nikodym derivatives $g_n:X\to[0, +\infty)$.
Moreover, we can choose these such that $g_n\equiv 0$ on $B_n\stcmp$ because this won't change anything with respect to $\mu_n$ and $\nu_n$.

We now define our global derivative as
\[
	g=\sum_{n}g_n
\]
which is measurable and non-negative because it is the sum of measurable and non-negative functions.
Moreover, $g$ is finite because the $g_n$ are finite and only assume non-zero values on their respective $B_n$.
Finally given any $A\in\mathcal{A}$ we have
\[
	\int_A g \dm \underbrace{=}_{\text{Levi}} \sum_n\int g_n\dm = \sum_n \nu_n(A) = \sum _n \nu(A\cap B_n) = \nu(A)
\]
which is what we wanted.
Woo hoo! \Cooley
\end{proof}

\section{Uniqueness}
\begin{proof}
Suppose $h,g : X \to [0, \infty)$ satisfy $\forall A \in \mathcal{A} \nu(A) = \int_A g \dm = \int_A h \dm$. 
Suppose that $\nu (X) < +\infty$.
Then $g-h$ is integrable with respect to $\mu$ and $\int_{\left\{g>f\right\}} g-h \dm = 0 = \int_X (g-h)^+ \dm$. 
This implies that $(g-h)^+ = 0$ $ \mu$ a.e.
We have a similar argument for the negative part.
Thus $g=h$ a.e.

In the $\sigma$- finite case, write $X = \sqcup_n B_n$ with $\nu(B_n) < \infty$.
So $g=h$ on $B_n$ $\mu-$ a.e.
Thus $g = h$ $\mu-$ a.e. 
\end{proof}

\end{document}
